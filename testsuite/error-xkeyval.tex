\documentclass{article}
\usepackage{xkeyval}
\usepackage[ factor = 1500 , verbose,expansion = false]{microtype}
\SetProtrusion
   [ name = test  ,
     factor = 750 ]
   { font = * }
   { - = {1000,1000},
    A = {1000,1000},
    l = {1000,1000} }
\textwidth=10cm

\begin{document}
 `Margin kerning is the adjustments of the characters at the margins of a
  typeset text. A simplified employment of margin kerning is hanging
  punctuation. Margin kerning is needed for optical alignment of the margins
  of a typeset text, because mechanical justification of the margins makes
  them look rather ragged. Some characters can make a line appear shorter to
  the human eye than others. Shifting such characters by an appropriate
  amount into the margins would greatly improve the appearance of a typeset
  text.

  Composing with font expansion is the method to use a wider or narrower
  variant of a font to make interword spacing more even. A font in a loose
  line can be substituted by a wider variant so the interword spaces are
  stretched by a smaller amount. Similarly, a font in a tight line can be
  replaced by a narrower variant to reduce the amount that the interword
  spaces are shrunk by. There is certainly a potential danger of font
  distortion when using such manipulations, thus they must be used with
  extreme care. The potentiality to adjust a line width by font expansion can
  be taken into consideration while a paragraph is being broken into lines,
  in order to choose better breakpoints.'

\end{document}
