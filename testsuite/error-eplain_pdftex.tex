\input eplain
\beginpackages
\tracingassigns1
\tracingifs1
   \usepackage[letterspace=250]{letterspace}
   \usepackage{soul}
\endpackages
\input protcode
\setprotcode\font
\hsize=30em

Here: $2_{\textls{AAA}}$

 `Margin kerning is the \textls[200]{adjustments of \tenit\lsstyle the \lslig{ch}aracters at the margins} of a
  typeset text. A simplified employment of margin kerning is \textls*{hanging
  punctuation}. Margin kerning is needed for \textls*[150]{optical alignment} of the margins
  of a typeset text, because mechanical justification of the margins makes
  them look rather ragged. Some characters can make a line appear shorter to
  the human eye than others. Shifting \textls{such} characters by an appropriate
  amount into the margins would greatly improve the appearance of a typeset
  text.

A\lslig{fi} \textls*[1200]{This} \lslig{fi} \textls{whatever or \lslig{fi} that},

\vfill \eject

 `Margin kerning is the \so{adjustments of the characters at the margins} of a
  typeset text. A simplified employment of margin kerning is \so{hanging
  punctuation}. Margin kerning is needed for \so{optical alignment} of the margins
  of a typeset text, because mechanical justification of the margins makes
  them look rather ragged. Some characters can make a line appear shorter to
  the human eye than others. Shifting such characters by an appropriate
  amount into the margins would greatly improve the appearance of a typeset
  text.

\bye
