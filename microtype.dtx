%\iffalse meta-comment
% -*- TeX:DTX:UK -*-
% smartQuote: english
% $Id: microtype.dtx,v 1.25 2005-10-28 16:31:54+02 schlicht Exp schlicht $
% -----------------------------------------------------------------------
%                      The `microtype' package
%       An interface to the micro-typographic extensions of pdfTeX
%          Copyright (c) 2004, 2005 R Schlicht <w.m.l@gmx.net>
%
% This work  may be  distributed and/or modified  under the conditions of
% the LaTeX Project Public License, either version 1.3 of this license or
% (at your option) any later version.  The latest version of this license
% is in:  http://www.latex-project.org/lppl.txt, and version 1.3 or later
% is part of all distributions of LaTeX version 2003/12/01 or later.
%
% This work has the LPPL maintenance status `author-maintained'.
%
% This work consists of the files microtype.dtx and microtype.ins and the
% derived file microtype.sty.
%
% Modified versions of the configuration files (*.cfg) may be distributed
% provided that  (1) the original copyright statement is not removed, and
% (2) the identification string is changed.
% -----------------------------------------------------------------------
%
%<*driver>
\expandafter\newif\csname ifbeta\endcsname
%\betatrue
\ProvidesFile{microtype.dtx}[2005/10/28 v\ifbeta 2.0-beta \else 1.9 \fi]
%</driver>
%<package>\NeedsTeXFormat{LaTeX2e}
%<package>\ProvidesPackage{microtype}
%<package&beta>    [2005/10/28 v2.0-beta Micro-typography with pdfTeX (RS)]
%<package&!beta>   [2005/10/28 v1.9 Micro-typography with pdfTeX (RS)]
%<*config|cfg-u>
%<config|cfg-u>\ProvidesFile
%<m-t>   {microtype.cfg}[2005/08/16 v1.6 microtype main configuration file (RS)]
%<bch>   {mt-bch.cfg}[2005/03/15 v1.3 microtype config. file: Bitstream Charter (RS)]
%<cmr>   {mt-cmr.cfg}[2005/08/16 v1.8 microtype config. file: Computer Modern (RS)]
%<euf>   {mt-euf.cfg}[2005/06/01 v1.0 microtype config. file: AMS Euler Fraktur (RS)]
%<eus>   {mt-eus.cfg}[2005/06/01 v1.0 microtype config. file: AMS Euler script (RS)]
%<msa>   {mt-msa.cfg}[2005/06/01 v1.0 microtype config. file: AMS (a) (RS)]
%<msb>   {mt-msb.cfg}[2005/06/01 v1.0 microtype config. file: AMS (b) (RS)]
%<pad>   {mt-pad.cfg}[2005/03/15 v1.4a microtype config. file: Adobe Garamond (RS)]
%<pmn>   {mt-pmn.cfg}[2004/03/15 v1.1b microtype config. file: Adobe Minion (HH)]
%<ppl>   {mt-ppl.cfg}[2005/08/16 v1.5 microtype config. file: Palatino (RS)]
%<ptm>   {mt-ptm.cfg}[2005/08/16 v1.5 microtype config. file: Times (RS)]
%</config|cfg-u>
%
%<*driver>
\documentclass[10pt,a4paper]{ltxdoc}
\usepackage[latin1]{inputenc}
\usepackage{textcomp}
\usepackage{array}
 \newcolumntype{L}[1]{p{#1}<{\raggedright}}
\expandafter\newif\csname ifbooktabs\endcsname
\IfFileExists{booktabs.sty}
 {\usepackage{booktabs}\booktabstrue}{}
\IfFileExists{url.sty}
 {\usepackage{url}}
 {\def\url##1{\texttt{##1}}}
% Let's abolish CM! We use Charter and Letter Gothic
% (for the pre-built documentation on CTAN):
% \def\rmdefault{bch}
% \def\ttdefault{blg}
 \usepackage[T1]{fontenc}
\ifbeta
 \PassOptionsToPackage{kerning=true}{microtype}
\fi
 \usepackage[expansion=false]{microtype} % it's a web document
 \DeclareMicrotypeSet*
   [ protrusion ]*
   { encoding = {*, TS1, OMS},
     family = {rm*, tt*, cmsy},
     size = {footnotesize,small,normalsize}
   }
\ifbeta
 \DeclareMicrotypeSet*
   [ kerning ]
   { lettergothic }
   { encoding = T1,
     family   = blg }
 \SetExtraKerning
   { encoding = T1,
     family   = blg }
   {
      _ = {100,100},
   }
\else
 \newcommand\textls[2][]{#2}
 \let\lsstyle\relax
\fi
\usepackage{color}
 \definecolor{theblue}{rgb}{0.02,0.04,0.48}
 \definecolor{thered}{rgb}{0.65,0.04,0.07}
 \definecolor{thegreen}{rgb}{0.06,0.44,0.08}
 \definecolor{thegrey}{gray}{0.5}
 \definecolor{theshade}{gray}{0.92}
\makeatletter
\ifbeta
 \usepackage[color]{changebar}
 \cbcolor{thered}
 \def\betastart{\ifbeta\cbstart}
 \def\betaend{\cbend\fi}
\else
 \let\betastart\ifbeta
 \let\betaend\fi
\fi
% layout
\frenchspacing
\expandafter\newif\csname ifcharter\endcsname
\def\@tempa{bch}
\ifx\rmdefault\@tempa
  \chartertrue
  \def\bfdefault{b}
  \def\PackageFont{\ttfamily}
  \def\match{{\large\raisebox{-.15em}{\textbullet}}}
  \def\Module#1{{\color{theblue}$\langle$\textit{#1}$\rangle$}}
  \DeclareRobustCommand{\LaTeX}{L\kern-.26em{\sbox\z@ T\vbox to\ht\z@{%
     \hbox{\check@mathfonts\fontsize\sf@size\z@\math@fontsfalse\selectfont A}%
    \vss}}\kern-.1em\TeX}
  \newcommand*\textgeq{% being picky...
    \raisebox{.2ex}{\kern1pt\underline
    {\kern-1pt\textgreater\kern-1pt}\kern1pt}}
  \linespread{1.07}
\else
  \def\PackageFont{\sffamily}
  \def\match{\textbullet}
  \newcommand*\textgeq{$\geq$}
  \linespread{1}
\fi
\normalfont
\setlength\textheight{\topskip}
\addtolength\textheight{44\baselineskip}
\def\paragraph{\@startsection
  {paragraph}{4}{0pt}%
  {3.25ex plus 1ex minus .2ex}{-1em}%
  {\normalfont\normalsize\itshape}}
% macrocode
\MacroTopsep=0pt
\MacrocodeTopsep=3pt
\settowidth\MacroIndent{\scriptsize 000\ }
\def\theCodelineNo{\reset@font\color{thegrey}\scriptsize\arabic{CodelineNo}}
\def\MacroFont{\ttfamily\small}
\def\AltMacroFont{\ttfamily\footnotesize}
\tolerance=400
\def\ImplementationSettings{%
  \linespread{1}%
  \tolerance=1000
  \hfuzz=20pt
  \def\MacroFont{\ttfamily\footnotesize}%
  \def\macro@font{\ttfamily\footnotesize}%
}
% headers
\headheight=15pt
\def\ps@MTheadings{%
  \def\@oddhead{%
    \hbox to\textwidth{\vbox{\hbox to\textwidth{%
      \footnotesize{\leftmark\rightmark\strut}\hfill\thepage\strut}%
      \hrule height 0.4pt width\textwidth \vskip-0.4pt
    }}\hss}
  \let\@oddfoot\@empty
  \let\@mkboth\markboth
  \def\sectionmark##1{\markboth{\textls[30]{\MakeUppercase{##1}}}{}}
  \def\subsectionmark##1{\markright{: ##1}}}
\pagestyle{MTheadings}
% toc
\let\l@section@\l@section
\def\l@section{\vskip -.5ex\l@section@}
\let\l@subsection@\l@subsection
\def\l@subsection{\vspace{.25ex}\l@subsection@}
\def\l@subsubsection#1#2{%
  \leftskip 3.8em
  \rightskip 2em plus 1fil
  \parindent 0pt
  {\let\numberline\@gobble{\small[#1~--~#2]}}}
\def\l@table{\@dottedtocline{1}{0pt}{1.5em}}
\def\@pnumwidth{1.7em}
\long\def\tableofcontents{%
  \section*{\contentsname }\ifPDF\pdfbookmark[1]{\contentsname}{section.TOC}\fi
  \@mkboth{\textls[30]{\MakeUppercase{\contentsname}}}{}%
  \@starttoc{toc}}
\long\def\listoftables{%
  \section*{\listtablename }\ifPDF\pdfbookmark[1]{\listtablename}{section.LOT}\fi
  \@mkboth{\textls[30]{\MakeUppercase{\listtablename}}}{}%
  \@starttoc{lot}}
% bibliography
\def\@cite#1#2{#1\if@tempswa, #2\fi}
\def\thebibliography#1{%
  \section{\refname}%
  \list{}{\leftmargin 0pt}%
  \sloppy
  \clubpenalty 4000
  \@clubpenalty \clubpenalty
  \widowpenalty 4000}
\def\@biblabel#1{}
% footnotes
\long\def\@makefntext#1{%
  \leftskip 15pt
  \parindent -15pt
  \everypar{\parindent 0pt}%
  \leavevmode\hbox to 15pt{\@thefnmark\hss}#1}
\skip\@mpfootins=0.5\baselineskip
% lists
\def\descriptionlabel#1{\hspace\labelsep\normalfont#1}
\def\@listi{\leftmargin\leftmargini
            \parsep 4.5pt plus 1pt minus 1pt
            \topsep 4.5pt plus 1pt minus 1pt
            \itemsep 0pt}
\let\@listI\@listi
\setlength\leftmargini{15pt}
\newenvironment{options}
               {\list{}{\leftmargin 0pt
                        \labelwidth 0pt
                        \labelsep 1em
                        \itemindent \labelsep}}
               {\endlist}
% tables
\setlength\tabcolsep{2pt}
%\def\arraystretch{1.2}
\long\def\@makecaption#1#2{%
  \sbox\@tempboxa{\footnotesize\itshape#1: #2}%
  \hb@xt@\hsize{\box\@tempboxa\hfil}%
  \vskip 6pt}
% index and change log
\IndexPrologue{\section{Index}%
  Numbers written in italic refer to the page where the corresponding entry is
  described; numbers underlined refer to the code line of the definition;
  numbers in roman refer to the code lines where the entry is used.}
\IfFileExists{multicol.sty}{
  %\raggedcolumns
  \setcounter{finalcolumnbadness}{100}
  \setcounter{IndexColumns}{2}
  \def\IndexMin{12\baselineskip}
  \def\GlossaryParms{%
    \IndexParms
    \def\@idxitem ####1####2\efill{%
      \end{multicols}
      \begin{multicols}{2}
        [\subsubsection*{\if####1v\relax
            Version ####2 (\csname MTversiondate####2\endcsname)%
          \else ####2\fi}]
        [6\baselineskip]
      \GlossaryParms
      \rightskip 15pt plus 10pt
      \let\item\@idxitem
      \ignorespaces \makeatletter \scan@allowedfalse}%
    \def\subitem{\par\hangindent 25pt}%
    \def\subsubitem{\subitem\hspace*{15pt}}}
  \GlossaryPrologue{%
    \section{Change History}\label{sec:changes}%
    \vspace*{-2\multicolsep}}
}{%
  \GlossaryPrologue{%
    \section{Change History}\label{sec:changes}}
}
% additional bells ...
\def\Describe#1#2#3{\noindent
  \csname Describe#1\endcsname{#2}%
  \DescribeValues{#1}{#3}}
\def\DescribeOption{\leavevmode\@bsphack
  \begingroup\MakePrivateLetters\Describe@Option}
\def\Describe@Option#1{\endgroup
  \marginpar{\raggedleft\PrintDescribeOption{#1}}%
  \SpecialOptionIndex{#1}\@esphack\ignorespaces}
\def\DescribeValues#1#2{%
  \let\@tempa\@empty
  \let\Option@default\@empty
  \@for\@tempb:=#2\do{%
    \expandafter\csname Special#1Value\expandafter\endcsname\@tempb\@nil
    \expandafter\g@addto@macro\expandafter\@tempa
      \expandafter{\csname #1Sep\endcsname}%
    \expandafter\g@addto@macro\expandafter\@tempa
      \expandafter{\@tempb}}%
  \@ifnextchar[\PrintValues{\PrintValues[\Option@default]}}
\def\SpecialOptionValue#1#2\@nil{%
  \if#1:\def\@tempb{\Variable{#2}}\fi             % : = variable
  \if#1!\def\@tempb{#2}\def\Option@default{#2}\fi}% ! = default
\def\SpecialMacroValue#1#2\@nil{%
  \if#1?\def\@tempb{\normalsize[\Variable{#2}]}%  % ? = optional
   \else\def\@tempb{\normalsize\{\Variable{#1#2}\}}\fi}
\def\OptionSep{,\,}
\def\MacroSep{\space}
\def\PrintValues[#1]{{\MacroFont\expandafter\@gobble\@tempa\hfill #1}%
  \\*[.25\baselineskip]}
\def\CatIndex#1#2{\index{#2\actualchar{\protect\ttfamily #2} (#1)\encapchar hyperpage}}
\def\SpecialOptionIndex#1{\@bsphack
  \CatIndex{option}{#1}%
  \index{\quotechar!Options % the `!' will be sorted first
    \actualchar{\protect\bfseries Options:}%
    \levelchar{\protect\ttfamily#1}\encapchar usage}%
  \@esphack}
\def\SpecialUsageIndex#1{\@bsphack
  {\index{\quotechar!User Commands
    \actualchar{\protect\indexspace\protect\bfseries User Commands:}%
    \levelchar\expandafter\@gobble\string#1\actualchar\string\verb
      \quotechar*\verbatimchar\string#1\verbatimchar\encapchar usage}%
   \let\special@index\index\SpecialIndex@{#1}{\encapchar usage}}%
   \@esphack}
\def\IndexNewif#1{\@bsphack
  \expandafter\SpecialMainIndex\expandafter{\csname #1true\endcsname}%
  \expandafter\SpecialMainIndex\expandafter{\csname #1false\endcsname}%
  \@esphack}
\def\PrintDescribeMacro#1{\strut\MacroFont\color{thegreen}\string #1}
\def\PrintDescribeOption#1{\strut\MacroFont\color{thered}#1}
\def\Key#1{{\color{thered}\ttfamily#1}}
\def\Variable#1{$\langle${\rmfamily\itshape\small#1}$\rangle$}
\DeclareRobustCommand\pkg[1]{{\PackageFont#1}\@bsphack\CatIndex{package}{#1}\@esphack}
\DeclareRobustCommand\opt[1]{{\ttfamily#1}\@bsphack\CatIndex{option}{#1}\@esphack}
\DeclareRobustCommand\file[1]{{\ttfamily#1}}
% ... and whistles
\expandafter\newif\csname iflistings\endcsname
\IfFileExists{listings.sty}{
 \usepackage{listings}
 \lstset{%
  gobble=1,
  columns=flexible,
  keepspaces=true,
  basicstyle=\MacroFont,
  keywords=[0]{\microtypesetup,\DeclareMicrotypeSet,\UseMicrotypeSet,
    \DeclareMicrotypeSetDefault,\SetProtrusion,\SetExpansion,\DisableLigatures,
    \DeclareCharacterInheritance,\DeclareMicrotypeAlias,\LoadMicrotypeFile,
    \microtypecontext,\Microtype@Hook},
  keywordstyle=[0]\color{thegreen},
  keywords=[1]{protrusion,expansion,activate,DVIoutput,draft,final,verbose,
    config,factor,auto,stretch,shrink,step,selected,unit},
  keywordstyle=[1]\color{thered},
  comment=[l]\%,
  commentstyle=\color{thegrey},
  escapeinside=<>,
  frame=single,
  framerule=0pt,
  xleftmargin=\fboxsep,
  xrightmargin=\fboxsep,
%  aboveskip=\smallskipamount,
  belowskip=\smallskipamount,
  backgroundcolor=\color{theshade}
 }
 \let\verbatim\relax
 \lstnewenvironment{verbatim}{}{}
 \listingstrue
}{
 \let\lstset\@gobble
}
\ifbeta
 \lstset{
   morekeywords=[0]{\SetExtraSpacing,\SetExtraKerning,\textls,\lsstyle,
     \DeclareMicrotypeBabelHook},
   morekeywords=[1]{spacing,kerning,letterspacing,babel}
 }
\fi
\ifbeta
 \def\todo{\changes{zTo Do}{0000/00/00}}
\else
 \def\todo#1{}
\fi
\def\Indexing{\let\special@index\codeline@wrindex}
\def\NoIndexing{\let\special@index\@gobble}
\def\GeneralChanges#1{\edef\generalname{\if!#1General\else#1\fi}}
% Heiko Oberdiek's workaround from latexbugs (latex/3540):
\begingroup
  \def\x\begingroup#1\@nil{%
    \endgroup
    \def\DoNotIndex{%
      \begingroup
      \@makeother\#%
      \@makeother\$%
      \@makeother\%%
      \@makeother\&%
      \@makeother\^%
      \@makeother\_%
      \@makeother\|%
      \@makeother\~%
      \@makeother\ %
      % \@makeother\{\@makeother\} cannot be used here
      #1}}%
\expandafter\x\DoNotIndex\@nil
% fancy PDF document
\expandafter\newif\csname ifPDF\endcsname
\ifx\pdfoutput\undefined\else\ifx\pdfoutput\relax\else
  \ifcase\pdfoutput\else\PDFtrue\fi\fi\fi
\ifPDF
%  \ifx\eTeXversion\undefined \else
%    \usepackage{pdfcolmk}% errors when not using etex (conflict with multicol)
%  \fi
  \usepackage{graphicx}
  \usepackage[
    bookmarks=true,
    colorlinks=true,
    linkcolor=theblue,
    urlcolor=theblue,
    citecolor=theblue,
    hyperindex=false,
    hyperfootnotes=false,
    pdftitle={The microtype package},
    pdfauthor={R Schlicht <w.m.l@gmx.net>},
    pdfsubject={An interface to the micro-typographic extensions of pdfTeX},
    pdfkeywords={TeX, LaTeX, pdftex, typography, micro-typography,
                 character protrusion, margin kerning, font expansion,
                 kerning, spacing, letterspacing}
    ]{hyperref}
  \def\usage#1{\textbf{\textit{\hyperpage{#1}}}}% for indexing of \DescribeMacro ...
  \def\changes@#1#2#3{% ... the changes ...
    \protected@edef\@tempa{\noexpand\glossary{#1\levelchar
      \ifx\saved@macroname\@empty \space\actualchar\generalname
      \else\expandafter\@gobble\saved@macroname\actualchar
        \string\verb\quotechar*\verbatimchar\saved@macroname\verbatimchar\fi
      :\levelchar #3\encapchar hyperpage}}%
    \@tempa\endgroup\@esphack}
%  \def\theCodelineNo{% % ... and everything else (would double the pdf file size)
%    \reset@font\color{thegrey}\scriptsize
%    \@tempcnta\arabic{CodelineNo}\advance\@tempcnta by\@ne
%    \hypertarget{L:\number\@tempcnta}{\arabic{CodelineNo}}}
%  \def\main#1{\underline{\hyperlink{L:#1}{#1}}}
%  \def\codeline#1{\link@sanitize#1-\@nil{#1}}
%  \def\link@sanitize#1-#2\@nil{\link@@sanitize#1,\@nil}
%  \def\link@@sanitize#1,#2\@nil{\hyperlink{L:#1}}
%  \def\SpecialIndex#1{\@bsphack\special@index{\expandafter\@gobble
%      \string#1\actualchar
%      \string\verb\quotechar*\verbatimchar\string#1\verbatimchar
%      \encapchar codeline}%
%    \@esphack}
  \def\ctanurl#1{\href{http://www.tex.ac.uk/tex-archive#1}{\texttt{#1}}}
  \def\mailto#1{\href{mailto:#1}{\texttt{#1}}}
  \def\mailtoRS{\href{mailto:<w.m.l@gmx.net> R Schlicht?subject=microtype \fileversion}
                {\texttt{w.m.l@gmx.net}}}
\else
  \let\hyperpage\relax
  \def\texorpdfstring#1#2{#1}
  \def\ctanurl#1{\url{#1}}
  \def\mailto#1{\texttt{#1}}
  \def\mailtoRS{\mailto{w.m.l@gmx.net}}
  \let\nolinkurl\texttt
\fi
\hyphenation{Ha-rald Har-ders Ber-nard Ku-char-czyk}
% abbreviations
\def\microtype{{\PackageFont microtype}}
\def\pdftex{\texorpdfstring{pdf\kern.05em\TeX}{pdfTeX}}
\def\etex{\mbox{e-\TeX}}
\DeclareRobustCommand\thanh{H\`an Th\^e\llap{\raisebox{0.5ex}{\'{}}} Th\`anh}
\DeclareRobustCommand\acro[1]{\texorpdfstring
  {{\ifdim\f@size pt=1\@ptsize pt\relax\small\fi
    \ifcharter\textls[30]{#1}\else#1\fi}}{#1}}
\def\spaceout#1{\ifx#1\relax\else#1\kern.06em\expandafter\spaceout\fi}
\def\ams{\acro{AMS}}
\def\ctan{\acro{CTAN}}
\def\dvi{\acro{DVI}}
\def\nfss{\acro{NFSS}}
\def\pdf{\acro{PDF}}
\def\tI{\acro{T1}}
\def\tV{\acro{T5}}
\def\otI{\acro{OT1}}
\def\otIV{\acro{OT4}}
\def\lyI{\acro{LY1}}
\def\tsI{\acro{TS1}}
\def\oml{\acro{OML}}
\def\oms{\acro{OMS}}
\def\U{\acro{U}}
\def\ascii{\acro{ASCII}}
\def\eg{e.\,g.}
\def\ie{i.\,e.}
\makeatother
\CodelineIndex
\EnableCrossrefs
\RecordChanges
%\OnlyDescription
\begin{document}
  \DocInput{microtype.dtx}
\end{document}
%</driver>
% \fi
%
% ^^A -------------------------------------------------------------------------
%\changes{v1.0}{2004/09/11}{Initial version}
%\GeneralChanges{Documentation}
%
% \GetFileInfo{microtype.dtx}
% \title{\IfFileExists{mt-logo.pdf}{\vspace{-2\baselineskip}^^A
%          \includegraphics{mt-logo}\\[3.5\baselineskip]}{}^^A
%        \textls[30]{The \microtype\ package}\\[.3\baselineskip]\large
%        An interface to the micro-typographic extensions of \pdftex}
% \author{R Schlicht\quad\mailtoRS}
% \date{\fileversion\ --- \filedate}
%
% \maketitle
% \thispagestyle{empty}
%
%\begin{abstract}
%\noindent
% The \microtype\ package provides an interface to the micro-typographic
% extensions of \pdftex: most prominently, character protrusion and font
% expansion, furthermore
%\betastart
% the adjustment of interword spacing and additional kerning, as well as
%\betaend
% the possibility to disable all ligatures of a font.\ifbeta\else\footnote{
%   A preview of the next version with support for even more micro-typographical
%   extensions is also included in this package. Footnote~\ref{fn:beta} on
%   page~\pageref{fn:beta} contains the details.}\fi\
% It allows to apply these features to customizable sets of fonts, and to
% configure all micro-typographic aspects of the fonts in a straight-forward
% and flexible way. Settings for various fonts are provided.\footnote{
%   Currently, this package provides settings for Computer Modern Roman,
%   Palatino, Times, Adobe Garamond and Minion, Bitstream Charter, and the
%   \ams\ math fonts, as well as some generic settings for unknown fonts.
%   Contributions are very welcome.}
%
% Note that font expansion and character protrusion will only work with \pdftex,
% at least version 0.14f. Automatic font expansion requires version 1.20 or
% newer. Disabling ligatures require \pdftex\ 1.30^^A
%\betastart,
% adjustment of interword spacing and of kerning requires version 1.3x^^A
%\betaend.
% The package will by default enable the features that can safely be assumed to
% work.
%
%\betastart
%\bigskip\noindent\bfseries
% This is a beta version! New features are marked with a red bar.
%\betaend
%\end{abstract}
%
%\newpage
%\tableofcontents
%\listoftables
%
%\newpage
%\section{Micro-Typography with \pdftex}\label{sec:micro-type}
%
% \pdftex, the \TeX\ extension written by \thanh, introduces two features that
% make it the tool of choice not only for the creation of electronic documents
% but also of works of outstanding time-honoured typography: \textit{character
% protrusion} (also known as margin kerning) and \textit{font expansion}.
% Quoting \thanh's thesis:
%
%\begin{quote}\small
%\ifPDF
%  \dimen0=\lpcode\font`\`em
%  \divide\dimen0 by 500 ^^A double
%  \hskip-\dimen0        ^^A cheat!
%\fi
% `Margin kerning is the adjustments of the characters at the margins of a
%  typeset text. A simplified employment of margin kerning is hanging
%  punctuation. Margin kerning is needed for optical alignment of the margins
%  of a typeset text, because mechanical justification of the margins makes
%  them look rather ragged. Some characters can make a line appear shorter to
%  the human eye than others. Shifting such characters by an appropriate
%  amount into the margins would greatly improve the appearance of a typeset
%  text.
%
%  Composing with font expansion is the method to use a wider or narrower
%  variant of a font to make interword spacing more even. A font in a loose
%  line can be substituted by a wider variant so the interword spaces are
%  stretched by a smaller amount. Similarly, a font in a tight line can be
%  replaced by a narrower variant to reduce the amount that the interword
%  spaces are shrunk by. There is certainly a potential danger of font
%  distortion when using such manipulations, thus they must be used with
%  extreme care. The potentiality to adjust a line width by font expansion can
%  be taken into consideration while a paragraph is being broken into lines,
%  in order to choose better breakpoints.' [\cite[p.~323]{ThanhThesis}]
%\end{quote}
%
% Both these features have been lacking a simple \LaTeX\ user interface for
% quite some time. Then, the \pkg{pdfcprot} package was released [\cite{pdfcprot}],
% which allowed \LaTeX\ users to employ character protrusion without having to
% mess much with the internals.
%
% Font expansion, however, was still most difficult to utilize, since it
% required that the font metrics are available in all levels of expansion.
% Therefore, anybody who wanted to use this feature had to create multiple
% instances of the fonts in advance. Shell scripts to partly relieve the user
% from this burden were available~-- however, it remained a cumbersome task.
% Furthermore, all fonts were still being physically created, thus wasting
% compilation time and disk space.
%
% In the summer of 2004, \thanh\ implemented a feature that can be expected to
% prove as a major facilitation for \TeX\ and \LaTeX\ users: Font expansion can
% now take place automatically. That is, \pdftex\ no longer needs the expanded
% font metrics but will calculate them at run-time, and completely in memory.
%
%\betastart
%\bigskip
% After this great leap in usability had been taken, the development did not
% stop. On the contrary, \pdftex\ was extended with even more features: Version
% 1.3x introduced the \emph{adjustment of interword spacing}, additional
% \emph{kerning settings} and the possibility to \emph{disable all ligatures}
% of a font.
%
% Adjustment of interword spacing is based upon the idea that in order to
% achieve a uniform greyness of the text, the space between words should also
% depend on the surrounding characters. For example, if a words ends with an
% `r', the following space should be a tiny bit smaller than that following an,
% say, `m'. You can think of this concept as an extension to \TeX's `space
% factors'. However, while space factors will influence all three parameters of
% interword space (or glue) by the same amount -- the kerning, the maximum
% amount that the space may be stretched and the maximum amount that it may be
% shrunk~\mbox{--,} \pdftex\ provides the possibility to modify these
% parameters independently from one other. Furthermore, the values may be set
% differently for each font. And, probably most importantly, the parameters may
% not only be increased but also decreased. This feature may enhance the
% appearance of paragraphs even more.
%
% Emphasis in the last sentence is on the word `may': This extension is still
% highly experimental -- in particular, only ending characters will currently
% have an influence on the interword space. Also, the settings that are shipped
% with \microtype\ are but a first approximation, and I would welcome
% corrections and improvements very much. I suggest reading the reasoning
% behind the settings in section~\ref{sub:conf-spacing}.
%
% Setting additional kerning for characters of a font is especially useful for
% languages whose typographical tradition requires certain characters to be
% separated by a space. For example, it is customary in French typography to
% add a small space before question mark, exclamation mark and semi-colon, and
% a bigger space before the colon and the guillemets. Until now, this could
% only be achieved by making these characters active (for example by the
% \pkg{babel} package), which may not always be a robust solution. In contrast
% to the standard kerning that is built into the fonts (and will of course
% apply as usual), this additional kerning is based on single characters, not
% on character pairs.
%
% Another application for this feature is letterspacing, which was even more
% difficult to achieve on the macro level. This package introduces a new
% command \cs{textls}, which can be used much like the normal text commands.
%\betaend
%
% Finally, the possibility to disable all ligatures of a font has been introduced.
% This may be useful when using typewriter fonts.
%
%\ifcharter\else\enlargethispage\baselineskip\fi ^^A layout
%\bigskip\noindent
% The \microtype\ package provides an interface to all these micro-typographic
% extensions.\footnote{
%    Therefore, it is an alternative, not a supplement, to the \pkg{pdfcprot}
%    package, which provides an interface to character protrusion.}
% All micro-typographic aspects may be customized to your taste and needs in a
% straightforward manner. The next chapters will present a survey of all
% options and customization possibilities.
%
%
%\section{Invoking the Package}
%
% There is nothing surprising in loading this package:
%{\lstset{morekeywords=[1]{microtype}}
%\begin{verbatim}
%\usepackage{microtype}
%\end{verbatim}}
%\noindent
% This will be sufficient in most cases, and if you are not interested in
% fine-tuning the micro-typographic appearance of your document (which would
% seem unlikely, since using this package is proof of your interest in
% typographic issues), you may actually skip the rest of this document.
%
%
%\section{Options}
%
% Like many other \LaTeX\ packages, the \microtype\ package accepts options in
% the well known |key=value| syntax. In the following, you'll find a
% description of all \Key{keys} and their possible |values|
% (`|true|' may be omitted; multiple values, where allowed, must be enclosed in
% braces; the default value is shown on the right, preceded by an asterisk
% if it is contingent on the \pdftex\ version).
%
%\subsection{Micro-Typographic Options}\label{sub:options-microtype}
%
%\Describe{Option}{protrusion}
%                 {true,false,compatibility,nocompatibility,:font set name}[*\,true]
%\DescribeOption{expansion}
% These are the main options to control the level of micro-typographic
% refinement, which the fonts in your document should gain. By default, the
% package is moderately greedy: Character protrusion will be enabled, font
% expansion will only be disabled in circumstances where \pdftex\ cannot expand
% the fonts automatically, that is, if it is either too old (versions before
% 1.20) or if the output mode is \dvi\ (see section~\ref{sub:options-misc}).
%
% Protrusion and expansion may be enabled or disabled independently from each
% other by setting the respective key to |true| resp. |false|. The
% \opt{activate} option
%\DescribeOption{activate}
% is a shortcut for setting both options at the same time. Therefore, the
% following lines all have the same effect (when creating \pdf\ files with a
% new \pdftex):
%\begin{verbatim}
%\usepackage[protrusion=true,expansion=true]{microtype}
%\end{verbatim}
%\begin{verbatim}
%\usepackage[protrusion,expansion]{microtype}
%\end{verbatim}
%\begin{verbatim}
%\usepackage[activate={true,nocompatibility}]{microtype}
%\end{verbatim}
%\begin{verbatim}
%\usepackage{microtype}
%\end{verbatim}
%\smallskip
% When \pdftex\ employs font expansion and character protrusion, line breaks
% (and consequently, page breaks) may turn out differently. If that is not
% desired, you may pass the value |compatibility| to the \opt{protrusion}
% and/or \opt{expansion} options. Typographically, however, the results may be
% suboptimal.
%
%\enlargethispage\baselineskip ^^A layout
% Finally, you may also specify the name of a font set to which character
% protrusion and/or font expansion should be restricted. See
% section~\ref{sec:font-sets} for a detailed discussion.
%
%\betastart
%\bigskip
%\Describe{Option}{spacing}{true,!false,:font set name}
%\DescribeOption{kerning}
% The new extensions of interword spacing and additional kerning adjustment do
% not feature a compatibility layer. Therefore, they can only be switched on or
% off, or they may be activated by passing a set name to the option. By
% default, neither extension is enabled.
%\betaend
%
%\bigskip\noindent
% Whether ligatures should be disabled cannot be controlled via a package option
% but by using the \cs{DisableLigatures} command, which is explained in
% section~\ref{sec:disable-ligatures}.
%
%
%\subsection{Options for Character Protrusion}\label{sub:options-protrusion}
%
%\Describe{Option}{factor}{:integer}[1000]
% Using this option, you can globally increase or decrease the amount by which
% the characters will be protruded. While a value of 1000 means that the full
% protrusion as specified in the configuration (see section~\ref{sub:protrusion})
% will be used, a value of 500 would result in halving all protrusion factors of
% the configuration. This might be useful if you are generally satisfied with
% the settings but prefer the margin kerning to be less or more visible
% (\eg, if you are so proud of being able to use this feature that you want
% everybody to see it, or -- to mention a motivation more in compliance with
% typographical correctness -- if you are using a large font that calls for
% more modest protrusion).
%
%\bigskip
%\Describe{Option}{unit}{!character,:dimension}
% This option is described in section~\ref{sub:protrusion}, apropos the command
% \cs{SetProtrusion}. Use with care.
%
%
%\subsection{Options for Font Expansion}\label{sub:options-expansion}
%
%\Describe{Option}{auto}{true,false}[*\,true]
% As noted in chapter~\ref{sec:micro-type}, the expanded versions of the fonts
% may be calculated automatically. This option is true by default provided that
% \pdftex's version is found to be 1.20 or higher and the output mode is \pdf;
% otherwise, it will be disabled.
% If \opt{auto} is set to false, the fonts for all expansion steps must exist
% (with files called \meta{font~name}\texttt{\textpm}\meta{expansion value},
% \eg\ \file{cmr12+10}, as described in the \cite[p.~20]{pdftexman}). If expanded
% instances of the fonts are available, they will be used regardless whether
% \opt{auto} is true or not.
%
%\changes{v1.9}{2005/08/27}{add remark about Type\,1 fonts required for
%                           automatic font expansion}
% Automatic font expansion requires fonts in Type\,1 format. Therefore, if you
% are using the Computer Modern Roman fonts in \tI\ encoding\footnote{
%     En passant, it may be noted that Type\,1 format and \tI\ encoding are in
%     no other way related than that both start with a `T' and end with a `1'.},
% you should either install the \pkg{cm-super} fonts or use the Latin Modern
% fonts (package \pkg{lmodern}).
%
%\bigskip
%\Describe{Option}{stretch}{:integer}[20]
%\DescribeOption{shrink}
% You may specify the stretchability and shrinkability of a font, \ie, the
% maximum amount that a font may be stretched or shrunk. The numbers will be
% divided by 1000, so that a stretch limit of 10 means that the font may be
% expanded by up to~1\%.
% The default stretch limit is 20. The shrink limit will by default be the
% same as the stretch limit.
%
%\pagebreak ^^A layout
%\bigskip
%\Describe{Option}{step}{:integer}[\textrm{min(\opt{stretch},\opt{shrink})/5}]
% Font expansion will be applied in discrete steps. For example, if \opt{step}
% is set to~4 (which it is by default), \pdftex\ will try up to eleven different
% expansion levels of a font (from \textminus20 to +20). If you set
% \opt{stretch} or \opt{shrink} to something other than their default values
% but do not specify \opt{step}, it will be set to 1/5th of the smaller value
% of the two.
% Therefore, the following lines are all equivalent:
%\begin{verbatim}
%\usepackage[stretch=20,shrink=20]{microtype}
%\end{verbatim}
%\begin{verbatim}
%\usepackage[stretch=20,step=4]{microtype}
%\end{verbatim}
%\begin{verbatim}
%\usepackage{microtype}
%\end{verbatim}
%
%\bigskip
%\Describe{Option}{selected}{true,!false}
% When applying font expansion, it is possible to restrict the expansion of some
% characters that are more sensitive to deformation than others (\eg, the `O',
% in contrast to the `I'). This is called \emph{selected expansion}, and its
% usage allows to increase the stretch and shrink limits (to, say, 30 instead
% of 20); however, the gain is limited since at the same time the average
% stretch variance will be decreased.
%
% Beginning with version 1.5, where this option was introduced, it is by default
% set to |false|, so that all characters will be expanded by the same amount.
% See section~\ref{sub:expansion} for a more detailed discussion.
%
%
%\subsection{Miscellaneous Options}\label{sub:options-misc}
%
%\Describe{Option}{DVIoutput}{true,!false}
% \pdftex\ is not only able to generate \pdf\ output but can also spit out \dvi\
% files.\footnote{
%    \TeX\ systems are beginning to switch to pdfe\TeX\ as the default engine
%    even for \dvi\ output.}
% The latter can be ordered with the option \opt{DVIoutput}, which will set
% \cmd{\pdfoutput} to zero.
%
%\changes{v1.5}{2004/12/15}{add note about \opt{DVIoutput} option}
% Note that this will confuse packages that depend on the value of
% \cmd{\pdfoutput} if they were loaded earlier, as they had been made believe
% that they were called to generate \pdf\ output where they actually weren't.
% These packages are, among others: \pkg{graphics}, \pkg{color},
% \pkg{hyperref}, \pkg{pstricks} and, obviously, \pkg{ifpdf}. Either load these
% packages after \microtype\ or else issue the command |\pdfoutput=0| earlier
% -- in the latter case, the \opt{DVIoutput} option is redundant.
%
% When generating \dvi\ files, font expansion has to be enabled explicitly.
% \emph{Automatic} font expansion will not work because dvips (resp. the \dvi\
% viewer) is not able to generate the expanded fonts on the fly.
%
%\bigskip
%\Describe{Option}{draft}{true,!false}
%\DescribeOption{final}
% If the \opt{draft} option is passed to the package, \emph{all
% micro-typographic extensions will be disabled}. The \opt{draft} and
% \opt{final} options may also be inherited from the class options; of course,
% you can override them in the package options.
%
%\bigskip
%\Describe{Option}{verbose}{true,!false,errors}
% Information on the settings used for each font will be written into the log
% file if you enable the \opt{verbose} option, which is disabled by default.
%
% When \microtype\ encounters a problem that is not fatal (\eg, an unknown
% character in the settings, or non-existent settings), it will by default only
% issue a warning and try to continue. Loading the package with |verbose=errors|
% will turn all warnings into errors, so that you can be sure that no problem
% will go unnoticed.
%
%\bigskip
%\Describe{Option}{config}{:file name}[microtype]
% Various settings for this package will be loaded from a main configuration
% file, by default \file{microtype.cfg} (see section~\ref{sub:config-file}).
% You can have a different configuration file loaded instead by specifying its
% name \emph{without the extension}, \eg, |config=mycrotype|.
%
%\betastart
%\bigskip
%\Describe{Option}{letterspacing}{:number}[100]
% This option changes the default amount that the command \cs{textls}, which is
% described below in section~\ref{sec:lettersp}, will space out its argument.
% The amount is specified in thousandth of 1em.
%
%\bigskip
%\Describe{Option}{babel}{true,!false}
% Loading the package with the \opt{babel} option will adjust the typesetting
% according to the respective selected language. Read section~\ref{sec:context}
% for further information.
%\betaend
%
%
%\subsection{Changing Settings Later}\label{sec:options-cmd}
%
%\Describe{Macro}{\microtypesetup}{key\,=\,value list}
% This command may be used in the document body to change the general settings
% of the micro-typographic extensions. It accepts the keys: \Key{expansion},
% \Key{protrusion} and \Key{activate}, which in turn may receive the values
% |true|, |false|, |compatibility| or |nocompatibility| (but not the name of a
% font set).
%\ifbeta
% For \Key{spacing} and \Key{kerning}, there is no compatibility level,
% therefore, only the values |true| and |false| are admissible here.
%\fi
% Using this command, you could for instance temporarily disable font expansion
% by saying:
%\begin{verbatim}
%\microtypesetup{expansion=false}
%\end{verbatim}
%
%
%\section{Declaring Font Sets}\label{sec:font-sets}
%
% By default, character protrusion will be applied to all text fonts that are
% being used in the document, and a basic set of fonts will be expanded. You
% may want to customize the set of fonts that should get the benefit of
% micro-typographic treatment. This can be achieved by specifying attributes of
% the font that have to be matched for them to be taken into account.
%
%\bigskip
%\Describe{Macro}{\DeclareMicrotypeSet}{?features,set name,set of fonts}
%\DescribeMacro{\DeclareMicrotypeSet*}
% This command declares a new set of fonts to which the micro-typographic
% extensions should be applied. The optional argument may contain a
% comma-separated list of features to which this set should be restricted. The
% set can subsequently be activated by calling:
%\iflistings
%\begin{verbatim}
%\UseMicrotypeSet[<\Variable{features}>]{<\Variable{set name}>}
%\end{verbatim}
%\else
%\begin{flushleft}\small
%|\UseMicrotypeSet[|\Variable{features}|]{|\Variable{set name}|}|
%\end{flushleft}
%\fi
%
%\bigskip\noindent
% The starred version of the command declares \emph{and} activates the font
% set at the same time.
%
% \paragraph{The set of fonts} is specified by assigning values to the \nfss\
% font attributes: encoding, family, series, shape, and size (cf.~\cite{fntguide}).
% Let's start with an example. This package defines a font set called
% `|basictext|' in the main configuration file as follows:
%\begin{verbatim}
%\DeclareMicrotypeSet{basictext}
%   { encoding = {T1,OT1,LY1},
%     family   = {rm*,sf*},
%     series   = {m},
%     size     = {normalsize,footnotesize,small,large}
%   }
%\end{verbatim}
% If you now call
%{\lstset{deletekeywords=[1]{expansion}}
%\begin{verbatim}
%\UseMicrotypeSet[expansion]{basictext}
%\end{verbatim}}
%\noindent
% in the document's preamble, only fonts in the text encodings \tI, \otI\ or
% \lyI, roman or sans serif families, normal (or `medium') series, and in
% sizes called by \cmd{\normalsize}, \cmd{\footnotesize}, \cmd{\small} or
% \cmd{\large}, will be expanded. Math fonts, on the other hand, will not,
% since they are in another encoding. Neither will fonts in bold face, or huge
% fonts. Etc.
%
% If an attribute list is empty or missing -- like the `shape' attribute
% in the above example --, it does not constitute a restriction. In other
% words, this is equivalent to specifying \emph{all} possible values for that
% attribute.
% Therefore, the predefined set `|alltext|', which is declared as:
%\begin{verbatim}
%\DeclareMicrotypeSet{alltext}
%   { encoding = {T1,OT1,LY1,TS1} }
%\end{verbatim}
% is far less restrictive. The only condition is that the encoding must match.
%
% If a value is followed by an asterisk (like `|rm*|' and `|sf*|' in the example
% above), it does not designate an \nfss\ code, but will expand to the current
% |\|\meta{value}|default|, \eg\ \cmd{\rmdefault}.\footnote{
%    Note that this expansion will take place immediately, so you should make
%    all relevant changes \emph{before} loading the \microtype\ package.}
% For example, if you want to include the bold font, too, you should say `|bf*|'
% instead of `|b|' (\cmd{\bfdefault} for Computer Modern is `|bx|', while for
% other fonts, it might be `|b|' or even `|sb|'). A single asterisk means
% |\|\meta{characteristic}|default|, \eg\ \cmd{\encodingdefault}, respectively
% \cmd{\normalsize} for the size axis.
%
% Sizes may be either specified as a dimension (`|10|' or `|10pt|'), or as a
% size selection command \emph{without} the backslash. You may also specify
% ranges (\eg, `|small-Large|'); while the lower boundary is included in the
% range, the upper boundary is not. Thus, `|12-16|' would match 12pt, 13.5pt,
% and 15.999pt, \eg, but not 16pt. You are allowed to omit the lower or
% upper bound (`|-10|', `|large-|').
%
%\bigskip\noindent
% Additionally to this declaration scheme, you can add single fonts to
% a set using the `|font|' key, which expects the concatenation of all font
% characteristics, separated by forward slashes, \ie,
%`|font|\,=\,\meta{encoding}|/|\meta{family}|/|\meta{series}|/|\meta{shape}|/|\meta{size}'.
% This allows you to add fonts to the set that are otherwise disjunct from it.
% For instance, if you wanted to have the roman family in all sizes protruded,
% but only the normal sized, possibly italic, typewriter font (in contrast to,
% say, the small one), this is how you could declare the set:
%{\lstset{deletekeywords=[1]{protrusion}}
%\begin{verbatim}
%\DeclareMicrotypeSet*
%   [ protrusion ]
%   { myset }
%   { encoding = T1,
%     family   = rm*,
%     font     = {T1/tt*/m/n/*,
%                 T1/tt*/m/it/*} }
%\end{verbatim}}
%\noindent
% As you can tell from the example, the asterisk notation is also allowed for
% the |font| key. Size selection commands are possible, too, however, ranges are
% not allowed.
%
%\begin{table}\small
%\caption{Predefined font sets}\label{tab:predefined-font-sets}
%\begin{tabular}{@{}L{65pt}L{74pt}*3{L{43pt}}L{67pt}@{}}
%\ifbooktabs\toprule\else\hline\fi
%      \footnotesize Set name
%    & \multicolumn{5}{l}{\footnotesize  Font attributes}\\
%\ifbooktabs\cmidrule{2-6}\fi
%    & \footnotesize Encoding
%    & \footnotesize Family
%    & \footnotesize Series
%    & \footnotesize Shape
%    & \footnotesize Size\\
%\ifbooktabs\cmidrule(r){1-1}\cmidrule(r){2-2}\cmidrule(r){3-3}
%    \cmidrule(r){4-4}\cmidrule(r){5-5}\cmidrule{6-6}\else\hline\fi
% |all|
%    & --
%    & --
%    & --
%    & --
%    & -- \\
%\ifbooktabs\cmidrule(r){1-1}\fi
% |alltext|\linebreak\linebreak(|allmath|)
%    & \fontsize{8}{11}\selectfont \otI, \tI, \lyI,\linebreak \otIV, \tV, \tsI\linebreak (\oml, \oms, \U)
%    & --
%    & --
%    & --
%    & -- \\
%\ifbooktabs\cmidrule(r){1-1}\fi
% |basictext|\linebreak\linebreak(|basicmath|)
%    & \fontsize{8}{11}\selectfont \otI, \tI, \lyI,\linebreak \otIV, \tV\linebreak (\oml, \oms)
%    & |\rm*|,\linebreak |\sf*|
%    & m
%    & --
%    & \cmd{\normalsize}, \cmd{\footnotesize}, \cmd{\small}, \cmd{\large} \\
%\ifbooktabs\cmidrule(r){1-1}\fi
% |normalfont|
%    &  |\encoding*|
%    &  |\family*|
%    &  |\series*|
%    &  |\shape*|
%    &  |\normalsize| \\
%\ifbooktabs\bottomrule\else\hline\fi
%\end{tabular}
% \vskip8pt
% `|*|' = `|default|'
%\end{table}
%
% Table~\ref{tab:predefined-font-sets} lists the six predefined font sets. They
% may also be activated by passing their name to the feature options
% \opt{expansion} and \opt{protrusion}
%\ifbeta
% and \opt{spacing} and \opt{kerning}
%\fi
% when loading the package, for example:
%\begin{verbatim}
%\usepackage[protrusion=allmath,expansion=basicmath]{microtype}
%\end{verbatim}
%
%\bigskip
%\Describe{Macro}{\UseMicrotypeSet}{?features,set name}
% This command activates a font set previously declared by \cs{DeclareMicrotypeSet}.
% Using the optional argument, you can limit the application of the set to
% one or more features.
%
%\bigskip
%\Describe{Macro}{\DeclareMicrotypeSetDefault}{?features,set name}
% If the package has been loaded without activating any font sets in the package
% options, the sets declared by this command will be activated (provisionally).
% By default, the `|basictext|' font set will be used for font expansion, the
% `|alltext|' set for character protrusion.
%
%\bigskip\noindent
% This command will only have an effect inside the main configuration file (cf.
% section~\ref{sub:config-file}). The commands \cs{DeclareMicrotypeSet} and
% \cs{UseMicrotypeSet} may only be used in the preamble or in the main
% configuration file. Their scope is global to the document. Only one set per
% feature may be activated.
%
%
%\section{Micro Fine Tuning}\label{sec:fine-tuning}
%
% Every character asks for a particular amount of protrusion. It may also be
% desirable to restrict the maximum expansion of certain characters.
% Furthermore, since every font looks different, settings have to be specific
% to a font or set of fonts. This package offers flexible and straight-forward
% methods of customizing these finer aspects of micro-typography.
%
%\subsection{Character Protrusion}\label{sub:protrusion}
%
%\Describe{Macro}{\SetProtrusion}{?options,set of fonts,protrusion settings}
% Using this command, you can set the protrusion factors for each character of a
% font or a set of fonts. A very incomplete example would be the following:
%\begin{verbatim}
%\SetProtrusion
%   { encoding = T1,
%     family   = cmr }
%   {  A             = { 50, 50},
%     \textquoteleft = {700,   }  }
%\end{verbatim}
% which would result in the character `A' being protruded by 5\% of its width on
% both sides, and the left quote character by 70\% of its width into the left
% margin. This would apply to all font shapes, series and sizes of the Computer
% Modern Roman family in encoding~\tI.
%
%\paragraph{The protrusion settings}
% consist of \meta{character}\,|=|\,\meta{protrusion factors} pairs.
%
% The \meta{characters} may be specified either as a single character (`|A|'),
% as a text symbol command (`\cmd{\textquoteleft}'), or as a slot number: three
% digits for decimal notation, prefixed with~|"| for hexadecimal, with~|'| for
% octal (\eg, the `fl' ligature in \tI\ encoding:~|029|, |"1D|, |'35|).
% \mbox{8-bit} characters may be entered directly or in the \LaTeX\
% \mbox{7-bit} way of defining them: both |�| and |\"A| are valid, provided the
% character is actually included in the encoding(s). You also have the
% possibility to declare lists of characters that should inherit protrusion or
% expansion factors (see section~\ref{sub:inherit}).
%
% The \meta{protrusion factors} designate the amount that a character should be
% protruded into the left margin (first value) respectively into the right
% margin (second value). By default, the values are relative to the character
% widths, so that a value of 1000 means that the character should be shifted
% fully into the margin, while, for example, with a value of 50 it would
% be protruded by 5\% of its width. Negative values are admitted, as well as
% numbers larger than 1000 (but effectively not more than 1em of the font). You
% can omit either number if the character should not be protruded on that side,
% but must not drop the separating comma.
%
%\paragraph{The set of fonts}
% to which the settings should apply is declared using the same syntax of
% \meta{font axis}\,|=|\,\meta{value list} pairs as for the command
% \cs{DeclareMicrotypeSet}.
%
%\changes{v1.7}{2005/03/23}{remove table of match order}
% To find the matching settings for a given font the package will try all
% combinations of font encoding, family, series, shape and size, with
% decreasing significance in this order. For instance, if both settings for the
% current family (say, |T1/cmr///|) and settings for italic fonts in the normal
% weight (|T1//m/it/|) exist, those for the Computer Modern Roman font would
% apply.\footnote{
%    For the interested, table~\ref{tab:match-order} on
%    page~\pageref{tab:match-order} presents the exact order.}
% The encoding must always match.
%
%\paragraph{Options:}
%\begin{options}
%  \item[\Key{name}] You may assign a name to the protrusion settings, so that you
%       are able to load it by another list.
%  \item[\Key{load}] You can load another list (provided, you previously assigned
%       a name to it) before the current list will be loaded, so that the
%       fonts will inherit the values from the loaded list.\par
%       Thus, the configuration may be simplified considerably. You can for
%       instance create a default list for a font; settings for other shapes
%       or series can then load these settings, and extend or overwrite them
%       (since the value that comes last will take precedence). Font settings
%       will be loaded recursively.\par
%       The following options will affect all loaded lists:
%\changes{v1.8}{2005/06/23}{add example for \texttt{factor} option}
%  \item[\Key{factor}] This option can be used to influence all protrusion
%       factors of the list, overriding any global \opt{factor} setting (see
%       section~\ref{sub:options-protrusion}). For instance, if you want fonts
%       in larger sizes to be protruded less, you could load the normal lists
%       with a different factor applied to them:
%\begin{verbatim}
%\SetProtrusion
%   [ load     = cmr-T1,
%     factor   = 700    ]
%   { encoding = T1,
%     family   = cmr,
%     size     = large- }
%   { }
%\end{verbatim}
%  \item[\Key{unit}] By default, the protrusion factors are relative to the
%       respective character's width. The |unit| option may be used to override
%       this and make \microtype\ regard all values in the list as thousandths
%       of the specified width. Issuing, for instance, `|unit=1em|' would have
%       the effect that a value of, say, 50 now results in the character being
%       protruded by 5\% of an |em| of the font (thus simulating the internal
%       measuring of \pdftex's \cmd{\lpcode} and \cmd{\rpcode} primitives).
%       The default behaviour can be restored with |unit=character|.\footnote{
%          The |unit| option can even be passed globally to the package.
%          However, all provided settings are created under the assumption
%          that the values are relative to the character width. Therefore, you
%          should only ask for a different \opt{unit} if you are certain that
%          none of the default settings will be used in your document.}
%  \item[\Key{preset}] Presets the protrusion codes of all characters to the
%       specified values (|={|\meta{left}|,|\meta{right}|}|), possibly
%       scaled by a |factor|. A |unit| setting will only be taken into account
%       if it is not |=character|.
%  \item[\Key{context}] The scope of the list may be limited to a certain
%       context. For an example application, see section~\ref{sec:context}.
%\end{options}
%
%
%\subsection{Font Expansion}\label{sub:expansion}
%
%\Describe{Macro}{\SetExpansion}{?options,set of fonts,expansion settings}
% By default, all characters of a font are allowed to be stretched or shrunk by
% the same amount. However, it is also possible to limit the expansion of
% certain characters if they are more sensitive to deformation. This is the
% purpose of the \cs{SetExpansion} command. Note that it will only have an effect
% if the package has been loaded with the \opt{selected} option. Otherwise,
% the expansion settings will be ignored.
%
%\paragraph{The expansion settings}
% consist of \meta{character}\,=\,\meta{expansion factor} pairs.
%
% You may specify one number for each character, which determines the amount
% that a character may be expanded. The numbers denominate thousandths of the
% full expansion.
% For example, if you set the expansion factor for the character `O' to 500,
% it will only be expanded or shrunk by one half of the amount that the rest
% of the characters will be expanded or shrunk.
% While the default value for character protrusion is~0~-- that is, if you
% didn't specify any characters, none would be protruded~--, the default value
% for expansion is 1000, which means that all characters would be expanded by
% the same amount.
%
%\paragraph{The set of fonts}
% is declared in the same way as for \cs{SetProtrusion}.
%
%\ifbeta\pagebreak\fi ^^A layout
%\paragraph{Options:}
%\begin{options}
%  \item[\Key{name}, \Key{load}, \Key{preset}, \Key{context}]
%       Analogous to \cs{SetProtrusion}, the optional argument may be used to
%       assign a name to the list, to load another list, to preset all
%       expansion factors, or to determine the context of the list.
%\changes{v1.8}{2005/06/23}{add example of how to get rid of a widow}
%  \item[\Key{auto}, \Key{stretch}, \Key{shrink}, \Key{step}]
%       These keys can be used to override the global settings from the package
%       options (see section~\ref{sub:options-expansion}). If you don't
%       specify either one of |stretch|, |shrink| and |step|, their respective
%       global value will be used (that is, no calculation will take place).
%
%       As a practical example, suppose you have a paragraph containing a widow
%       that could easily be avoided by shrinking the font a little bit more.
%       You could take advantage of the |stretch| and |shrink| options to
%       allow for more expansion in this particular paragraph. There is one
%       problem that has to be worked around, however: \pdftex\ prohibits the
%       use of the same font with different expansion parameters. If you do
%       not want to create a clone of the font setup (this would require
%       duplicating the \file{tfm}/\file{vf} files under a new name, and
%       writing new \file{fd} files and \file{map} entries), you could exploit
%       a dirty trick and load a minimally larger font for the paragraph in
%       question. E.\,g., for a document printed in 10pt:\footnote{
%           Note that the \cmd{\expandpar} command can only be applied to
%           complete paragraphs. If you are using Computer Modern Roman, you
%           have to load the \pkg{fix-cm} package to be able to select fonts
%           in arbitrary sizes. Finally, the reason I suggest to use a larger
%           font, and not a smaller one, is to prevent a different design size
%           being selected.}
%{\lstset{gobble=2}
%\begin{verbatim}
% \SetExpansion
%    [ stretch = 30,
%      shrink  = 60 ]
%    { encoding = *,
%      size = 10.001 }
%    { }
% \newcommand{\expandpar}[1]
%    {{\fontsize{10.001}{\baselineskip}\selectfont #1}}
% % ...
% \expandpar{This paragraph contains an `unnecessary' widow.}
%\end{verbatim}}
%  \item[\Key{factor}]
%       This option provides a different method to alter expansion settings for
%       certain fonts, working around another restriction of \pdftex: It does
%       not allow different expansion limits or steps within one paragraph.
%       The |factor| option influences the expansion factors of all characters
%       (in contrast to the overall stretchability) of the font. For instance,
%       if you want the italic shape to be expanded less, you could declare:
%\begin{verbatim}
%\SetExpansion
%   [ factor   = 500 ]
%   { encoding = *,
%     shape    = it }
%   { }
%\end{verbatim}
%       The |factor| option can only be used to \emph{decrease} the
%       stretchability of the characters, that is, it may only receive values
%       smaller than 1000. Also, it can only be used for single fonts or font
%       sets; setting it globally in the package options wouldn't make much
%       sense -- to this end, you use the package's \opt{stretch} and
%       \opt{shrink} options.
%\end{options}
%
%\medskip\noindent
% These options in the optional first argument will even be taken into account
% if the package has not been loaded with the \opt{selected} option.
%
% If the \opt{selected} option has been passed to the package (cf.
% section~\ref{sub:options-expansion}), and settings for a font don't exist,
% font expansion will not be applied to this font at all. Should the
% extraordinary situation arise that you want to employ selected expansion in
% general but that all characters of a particular font (set) should be expanded
% or shrunk by the same amount, you would have to declare an empty list for
% these fonts.
%
%\betastart
%\subsection{Interword Spacing}\label{sub:interword-spacing}
%
%\Describe{Macro}{\SetExtraSpacing}{?options,set of fonts,spacing settings}
% This command allows you to fine tune the interword spacing (also known as
% glue). A preliminary remark about what a `space' is may be in order: Between
% two words, \TeX\ will insert a so called glue, which is characterized by
% three parameters -- the normal distance between two words, the maximum amount
% that this distance may be stretched, and the maximum amount that it may be
% shrunk. The latter two parameters come into effect whenever \TeX\ tries to
% break a paragraph into lines and does not succeed; it can then stretch or
% shrink the spaces between words. These three parameters are specific to each
% font.
%
% On top of these glue dimensions, \TeX\ has the concept of `space factors'.
% They may be used to increase the space after certain characters, most
% prominently the punctuation characters. If \pdftex's additional spacing
% adjustment is in effect, space factors are ignored, since it may be
% considered an extension to space factors with much finer control.
%
%\paragraph{The spacing settings}
% are declared as pairs of \meta{character}\,=\,\meta{spacing factors}, where
% the latter consist of three numbers: first, the additional kern inserted
% after this character if it appears before an interword space, second, the
% additional stretch amount, and third, the additional shrink amount. All
% values may also be negative, in which case the dimensions will be decreased.
% Not all values have to be specified, however, the settings must contain the
% two separating commas.
%
%\paragraph{The set of fonts} is declared in the usual way.
%
%\paragraph{Options:}
%\begin{options}
% \item[\Key{name}, \Key{load}, \Key{factor}, \Key{context}]
%      These options serve the same function as in the previous configuration
%      commands.
% \item[\Key{unit}] Like in \cs{SetProtrusion}, you can specify the unit by which
%      the specified numbers are measured. Possible values are: |character|, a
%      \meta{dimension} and, additionally, |space|. The latter will measure the
%      values in thousandths of the respective space dimension set by the
%      font. For example, with these settings:
%\begin{verbatim}
%\SetExtraSpacing
%   [ unit = space  ] % default
%   { font = */*/*/*/* }
%   {
%      .  = {1000,1000,1000},
%   }
%\end{verbatim}
%      the space inserted after a full stop would be doubled (technically
%      speaking: 2\,\texttimes\ \cmd{\fontdimen}\,2), as well as the maximum
%      stretch amount and the maximum shrink amount of the interword space
%      (\cmd{\fontdimen}\,3 and \cmd{\fontdimen}\,4). As another example,
%      setting all three value to |-1000| would completely cancel a space after
%      the respective character. By default, the unit is measured by the space
%      dimensions.
%  \item[\Key{preset}] Preset all characters to the specified three values,
%      possibly scaled by a |factor|, and relative to the |unit|.
%\end{options}
%
%
%\subsection{Additional Kerning}\label{sub:kerning}
%
%\Describe{Macro}{\SetExtraKerning}{?options,set of fonts,kerning settings}
% Fine tune the additional kerning. In contrast to standard kerning, which
% is always associated with a \emph{pair} of characters, the additional kerning
% relates to single characters.
%
%\paragraph{The kerning settings}
% are specified as pairs of \meta{character}\,=\,\meta{kerning values}, where
% the latter consist of two values: the kerning added before the character, and
% the kerning appended after the respective character. Once again, either value
% may be omitted, but not the separating comma.
%
%\paragraph{The set of fonts}\dots\ well, you know by now~\dots
%
%\paragraph{Options:}
%\begin{options}
% \item[\Key{name}, \Key{load}, \Key{factor}]
%      These options serve the same function as in the previous configuration
%      commands.
% \item[\Key{unit}] Admissible values are: |space|, |character| and a
%      \meta{dimension}. By default, the values are relative to 1em.
%  \item[\Key{context}] When it comes to kerning settings, this option is
%      especially useful, since it allows to apply settings depending on the
%      current language.
%  \item[\Key{preset}] Preset all characters to the specified values.
%\end{options}
%\betaend
%
%
%\subsection{Character Inheritance}\label{sub:inherit}
%
% \Describe{Macro}{\DeclareCharacterInheritance}
%                 {?features,set of fonts,inheritance lists}
% In most cases, accented characters should inherit the protrusion resp.
% expansion factors from the respective base character. For example, all of the
% characters \`A, \'A, \^A, \~A, \"A, \r{A} and \u{A} should probably be
% protruded by the same (absolute) amount as the character A. Using the command
% \cs{DeclareCharacterInheritance}, you may declare such lists of characters,
% so that you then only have to set up the base characters. With the optional
% argument, which may contain a comma-separated list of features, you may
% confine the scope of the list. The font set can be declared in the usual way,
% with the only exception that you must specify exactly one encoding. The
% inheritance lists are to be declared as pairs of \meta{base
% character}\,|=|\,\meta{list of inheriting characters}. Unless you are using a
% different encoding or a very peculiarly shaped font, there should be no need
% to change the default character inheritance settings.
%
%\bigskip\noindent
% In the main configuration file \file{microtype.cfg} and the other
% font-specific configuration files, you can find examples of all these
% commands.
%
%\subsection{Configuration Files}\label{sub:config-file}
%
% The default configuration, consisting of inheritance settings, declarations of
% font sets and alias fonts, and generic protrusion\ifbeta, \else\space and \fi
% expansion\ifbeta, spacing and kerning\fi\space settings, will be loaded from the
% file \file{microtype.cfg}. You may extend this file with custom settings (or
% load a different configuration file with the `\opt{config}' option, see
% section~\ref{sub:options-misc}).
%
% If you are embarking on creating new expansion and protrusion settings for a
% font family, you should put them into a separate file, whose name must be:
% `|mt-|\meta{font family}|.cfg|' (\eg\ `\file{mt-pad.cfg}'), and may contain
% the commands \cs{SetProtrusion}, \cs{SetExpansion} and
% \cs{DeclareCharacterInheritance}. These files will be loaded automatically if
% you are actually using the respective fonts. If the font name consists of
% four characters, the package will also try to find the file for the base font
% family by removing the suffix denoting the sub-family, so that you may put
% settings for the fonts |padx| (expert set), |padj| (oldstyle numerals) and
% |pad| (plain) into one and the same file.
%
%\changes{v1.6a}{2005/02/02}{add table of fonts with tailored protrusion settings}
%\begin{table}\small
%\begin{minipage}{\textwidth}
%\def\arraystretch{1.2}
%\let\footnoterule\relax
%\def\fnref#1{\textsuperscript{\itshape#1}}
%\settowidth{\dimen0}{(}
%\def\p{\hskip\dimen0}
%\caption{Fonts with tailored protrusion settings}\label{tab:fonts}
%\begin{tabular}{@{}L{170pt}L{31pt}L{75pt}*2{L{31pt}}@{}}
%\ifbooktabs\toprule\else\hline\fi
%       \footnotesize Font family (\nfss\ code)
%      & \multicolumn{4}{l}{\footnotesize Features} \\
%\ifbooktabs\cmidrule{2-5}\fi
%      & \footnotesize Series
%      & \footnotesize Shapes
%      & \footnotesize \tsI
%      & \footnotesize Math \\
%\ifbooktabs\cmidrule(r){1-1}\cmidrule(r){2-2}
%    \cmidrule(r){3-3}\cmidrule(r){4-4}\cmidrule{5-5}\else\hline\fi
% Generic
%      & m
%      & n, (it, sl, sc)\footnote{Incomplete}
%      & (\match)\fnref{a}\\
% Computer Modern Roman (|cmr|)\footnote{Also used for:
%        Latin Modern (|lmr|), \pkg{ae} (|aer|), \pkg{zefonts} (|zer|),
%        \pkg{eco} (|cmor|), \pkg{hfoldsty} (|hfor|)}
%      & m
%      & n, it, sl, sc
%      & \p\match
%      & \p\match \\
% Bitstream Charter (|bch|)
%      & m
%      & n, it, (sl)\footnote{Settings inherited from italic shape}, sc
%      & \p\match \\
% Adobe Garamond (|pad|, |padx|, |padj|)
%      & m
%      & n, it, (sl)\fnref{c}, sc
%      & \p\match \\
% Adobe Minion (|pmnx|, |pmnj|)
%      & m
%      & n, it, (sl)\fnref{c}, sc, si
%      & \p\match \\
% Palatino (|ppl|, |pplx|, |pplj|)\footnote{Also used for: \pkg{pxfonts} (|pxr|),
%                                           \pkg{qfonts}/QuasiPalatino (|qpl|)}
%      & m
%      & n, it, (sl)\fnref{c}, sc
%      & (\match)\fnref{a}  \\
% Times (|ptm|, |ptmx|, |ptmj|)\footnote{Also used for: \pkg{txfonts} (|txr|),
%                                        \pkg{qfonts}/QuasiTimes (|qtm|)}
%      & m
%      & n, it, (sl)\fnref{c}, sc
%      & (\match)\fnref{a}  \\
% \ams\ math fonts (|msa|, |msb|, |euf|, |eus|)
%      & m
%      & n
%      &
%      & \p\match \\
%\ifbooktabs\bottomrule\else\hline\fi
%\end{tabular}
%\end{minipage}
%\end{table}
%
% This package ships with configuration files for the font families Computer
% Modern Roman, Palatino, the inescapable Times, Adobe Garamond and
% Minion\footnote{
%    By courtesy of Harald Harders (\mailto{h.harders@tu-bs.de}).},
% for Bitstream Charter and the \ams\ math fonts.
% Table~\ref{tab:fonts} lists them all.
%
% If you have created a file for another font and you are willing to share,
% don't hesitate to send it to me so that it can be included in future releases
% of this package.
%
%\bigskip
%\Describe{Macro}{\DeclareMicrotypeAlias}{font name,alias font}
% You may use this command for fonts that are very similar, or actually the same
% (for instance if you did not stick to the Berry naming scheme when installing
% the font). An example would be the Latin Modern fonts which are clones of the
% Computer Modern fonts, so that it is not necessary to create new settings for
% them -- you could say:
%\begin{verbatim}
%\DeclareMicrotypeAlias{lmr}{cmr}
%\end{verbatim}
% which would make the package, whenever it encounters the font |lmr| and does
% not find settings for it, also try the font |cmr|. In fact, you will find
% this very line in the default configuration file, along with others for the
% virtual fonts provided by the packages \pkg{ae}, \pkg{zefonts}, \pkg{eco} and
% \pkg{hfoldsty}.
%
%\bigskip
%\Describe{Macro}{\LoadMicrotypeFile}{font name}
% In rare cases, it might be necessary to load a font configuration file
% manually, for instance, from within another configuration file, or to be able
% to extend settings defined in a file that would otherwise not be loaded
% automatically, or would be loaded too late.\footnote{
%    Font package authors might also want to have a look at the hook
%    \cs{Microtype@Hook}, described in the implementation part,
%     section~\ref{sub:hook}.}
% This command will load the file |mt-|\meta{font name}|.cfg|.
%
%
%\pagebreak ^^A layout
%\section{Disabling Ligatures}\label{sec:disable-ligatures}
%
%\Describe{Macro}{\DisableLigatures}{set of fonts}
% A new feature has been introduced with \pdftex\ 1.30: The possibility to
% completely disable all ligatures of a font (which will also switch off
% kerning). While this purposely \textit{lowers} the micro-typographic quality
% instead of raising it, it is especially useful for typewriter fonts, so that,
% \eg, in a \tI\ encoded font, `|\texttt{--}|' will indeed be printed as
% `|--|', not as `\texttt{--}'. \cs{DisableLigatures} may be used to specify,
% in the usual way, a set of fonts for which ligatures should be disabled, for
% example, of the typewriter font in \tI\ encoding:
%\begin{verbatim}
%\DisableLigatures{encoding = T1, family = tt* }
%\end{verbatim}
%
%
%\betastart
%\section{Letterspacing}\label{sec:lettersp}
%
%\Describe{Macro}{\textls}{?amount,general text}
%\DescribeMacro{\lsstyle}
% Until recently, it has been quite difficult to achieve letterspacing with
% \TeX.\footnote{^^A
%   Many renowned typographers would have praised \TeX\ for this difficulty,
%   since `he who would letterspace lower-case text, would steal sheep'
%   (credited, among others, to Eric Gill). To learn more about the pros and
%   cons of letterspacing, I suggest to read the documentation of the
%   \cite{soul} package.}
% The most robust solution is provided by the \cite{soul} package; however, it
% can still fail in certain circumstances. Employing \pdftex's new extension of
% additional kerning, letterspacing is now entirely robust, since the extra
% kerning will be inserted on the typesetting level, not on the macro level. In
% contrast to the \pkg{soul} implementation, there are two major differences:
% Firstly, the downside: hyphenation will be suppressed (this restriction will
% hopefully be lifted some day). Secondly, ligatures will not be broken up,
% which is usually typographically correct. If this is not desired, you can
% split the ligatures with a pair of braces |{}| between them.
%
% By default, the text will be spaced out by 0.1em on both sides of every
% character; this amount may altered in the optional argument to \cs{textls} or
% globally with the \opt{letterspacing} package option (see
% section~\ref{sub:options-misc}). You may also fine tune the kerning for each
% character by defining a list for the |letterspacing| context. The default
% declarations can be found in the main configuration file. Improvements are
% welcome.
%
% Font switches inside \cs{textls} may lead to undesired results, therefore you
% should start a new letterspacing group when changing to a different font.
%\betaend
%
%
%\section{Context-sensitive setup}\label{sec:context}
%
% In previous versions of \microtype, each font was set up exactly once for
% the entire document. Since version 1.9, it is possible to apply different
% settings to a font depending on the context it appears in.
%
%\bigskip
%\Describe{Macro}{\microtypecontext}{key=value list}
% This command may be used anywhere in the document (also in the preamble) to
% change the micro-typographic context. For each feature (\Key{protrusion},
% \Key{expansion}\ifbeta, \Key{spacing} and \Key{kerning}\fi), one context may
% be specified. Only settings which have been specified with the corresponding
% `|context|' keyword will then be applied. This makes it possible to use
% different settings for different parts of the document.\ifbeta\else\footnote{
%   This feature is especially useful for the new experimental extensions of
%   \pdftex: adjustment of interword spacing (glue) and the possibility to
%   specify additional character kerning. The former may improve the appearance
%   of the text even more, the latter allows for instance to insert small
%   spaces before certain characters (\eg, for typesetting in the French
%   tradition) without having to use active characters; also, letterspacing can
%   be implemented in a robust way. Currently, these extensions are only
%   available through patches from \url{http://pdftex.sarovar.org/}. However,
%   if you are adventurous, know how to apply the patches and you are able to
%   compile \pdftex\ yourself, you can easily experiment with them, since
%   \microtype\ already supports these new extensions. To generate the extended
%   version of the \microtype\ package and its documentation, simply remove the
%   comments before `\texttt{\textbackslash betatrue}' near the beginning of
%   \file{microtype.ins} and \file{microtype.dtx}.\label{fn:beta}}\fi
%
%\betastart
% For instance, if you are writing a text in French, you could add
%\begin{verbatim}
%\microtypecontext{kerning=french}
%\end{verbatim}
% to the preamble. This would have the effect that kerning settings for the
% French context would be applied to the document. Should parts of the document
% be in English, you could insert
%\begin{verbatim}
%\microtypecontext{kerning=}
%\end{verbatim}
% to reset the context, so that the punctuation characters in these parts will
% not receive any extra kerning.
%
% Instead of adding these commands manually to your document, you may also load
% \microtype\ with the \opt{babel} option. The current language will then be
% automatically detected and the contexts set accordingly.
%
%\bigskip
%\Describe{Macro}{\DeclareMicrotypeBabelHook}
%   {list of \pkg{babel} languages,context list}
% Naturally, \microtype\ does not know about the typographic specialties of
% every language. This command is a means of teaching it how to adjust the
% context when a particular language is selected. The main configuration file
% contains among others the following declaration:
%\begin{verbatim}
%\DeclareMicrotypeBabelHook
%   {french,francais}
%   {kerning=french, spacing=}
%\end{verbatim}
% Consequently, whenever you switch to the French language, the kerning context
% will be changed to `|french|' and the spacing context will be reset. This
% hook only has an effect if the package has been loaded with the \opt{babel}
% option (see section~\ref{sub:options-misc}).
% Currently, \microtype\ supports French and Turkish kerning, and English
% spacing (aka. \cmd{\nonfrenchspacing}). For unknown languages, all contexts
% will be reset.
%\betaend
%
%
%\section{Hints and Caveats}\label{sec:caveats}
%
%\paragraph{Use settings that match your font.}
% Although the default settings should give reasonable results for most fonts,
% the particular font you happen to be using may have different character
% shapes that necessitate more or less protrusion or expansion. In particular,
% italic letter shapes may differ wildly in different fonts, hence I have
% decided against providing default protrusion settings for them.
%
% The file \file{test-microtype.tex} might be of some help when adjusting the
% protrusion settings for a font.
%
%\paragraph{Don't use too large a value for expansion.}
% Font expansion is a feature that is supposed to enhance the typographic
% quality of your document by producing a more uniform greyness of the text
% block (and potentially reducing the number of necessary hyphenations). When
% expanding or shrinking a font too much, the effect will be turned into the
% opposite. Expanding the fonts by more than 2\%, \ie, setting a \opt{stretch}
% limit of more than 20, should be justified by a typographically trained eye.
% If you are so lucky as to be in the possession of multiple instances of a
% Multiple Master font, you may set expansion limits to up to 4\%.
%
%\paragraph{Don't use font expansion for web documents.}
% Because each expanded instance of the font will be embedded in the \pdf\ file,
% the file size may increase by quite a large factor (depending on expansion
% limits and step). Therefore, courtesy and thriftiness of bandwidth command it
% not to enable font expansion when creating files to be distributed
% electronically.
%
%\betastart
%\paragraph{Only employ kerning adjustment if it is customary in the language's
%typographic tradition.}
% In contrast to protrusion, expansion and spacing adjustment, kerning
% adjustment does not unconditionally improve the micro-typographical quality
% of your document. You should only switch it on if you are writing a document
% in a language, whose typographic tradition warrants such kerning. If you are,
% for example, writing an English text, your readers would probably be rather
% confused by any additional spaces before the punctuation characters.
%\betaend
%
%\changes{v1.7}{2005/03/23}{add hint about compatibility}
%\paragraph{Compatibility.}
% The package should work happily together with all other \LaTeX\ packages
% (except \pkg{pdfcprot}). However, life isn't perfect, so problems are to be
% expected. Currently, you should be aware of the following issues concerning
% the loading order of packages:
%\begin{itemize}
%  \item All packages that change the default fonts and encodings (\eg\
%        \pkg{mathpazo}, \pkg{fontenc}) should be loaded before \microtype, so
%        that variables used in the configuration file (\eg\ `|rm*|' for
%        \cmd{\rmdefault}) don't expand to a different value than in the body of
%        the document (as explained in section~\ref{sec:font-sets}).
%  \item When using 8-bit characters in the configuration, \pkg{inputenc} must
%        be loaded first. Unicode input in the configuration is currently not
%        supported.
%  \item The \pkg{CJK} package must be loaded before \microtype.\footnote{
%          And it might still not work. I simply don't know, since I know
%          nothing about \pkg{CJK}. Feedback on the interaction of both
%          packages -- positive or negative -- would be appreciated.}
%\end{itemize}
%
%\changes{v1.9}{2005/07/10}{add hint about \texttt{verbatim} environment}
%\paragraph{You might want to disable protrusion in \texttt{verbatim} environments.}
% As you know by now, \microtype\ will by default apply character protrusion to
% all fonts part of the font set `|alltext|'. This also includes the typewriter
% font. Although it does make sense to protrude the typewriter font if it
% appears in running text (like, for example, in this manual), this is probably
% not desirable inside the |verbatim| environment. However, \microtype\ has no
% knowledge about the context that a font appears in but will solely decide by
% examining its attributes. Therefore, you have to take of care of disabling
% protrusion in |verbatim| environments for yourself (that is, if you don't
% want to disable protrusion for the typewriter font altogether, by choosing a
% different font set). While the \cs{microtypesetup} command has of course been
% designed for cases like this, you might find it tiring to repeat it every
% time if you are using the |verbatim| environment frequently. The following
% incantation, added to the document's preamble, would serve the same
% purpose:\footnote{
%    If you are using the \pkg{fancyvrb} or the \pkg{listings} package, this is
%    not necessary, since their implementation of the corresponding
%    environments will inhibit protrusion anyway.}
%\begin{verbatim}
%\makeatletter
%\g@addto@macro\verbatim{\pdfprotrudechars=0 \pdfadjustspacing=0\relax}
%\makeatother
%\end{verbatim}
%
%\changes{v1.8}{2005/06/23}{add hint about error messages}
%\paragraph{Possible error messages and how to get rid of them:}
%\begin{itemize}
%\lstset{belowskip=-\smallskipamount,frame=none,xleftmargin=0pt,backgroundcolor=}
%  \item
%\begin{verbatim}
%font ____ cannot be expanded (not an included Type1 font)
%\end{verbatim}
% Font expansion can only be applied if the font is actually embedded in the
% \pdf\ file. If you receive the above error message, your \TeX\ system is not
% set up to embed (or `download') the base PostScript fonts (\eg\ Times,
% Helvetica, Courier). In most \TeX\ distributions, this can be changed in the
% file \file{updmap.cfg} by setting |pdftexDownloadBase14| to |true|.
% Otherwise, consult the local guide of your \TeX\ system.
%  \item
%\begin{verbatim}
%! TeX capacity exceeded, sorry [PDF memory size (pdf_mem_size)=65536].
%\end{verbatim}
% When applying micro-typographic enhancement to a large number of fonts, you
% may be running out of \pdftex\ memory. You can increase it by setting
% |pdf_mem_size| to a larger value (maximum 524\,288). For te\TeX-based
% systems, change the settings in \file{texmf.cnf}, for MiK\TeX, in the file
% \file{miktex.ini}. Beginning with version 1.30 of \pdftex, the memory will
% grow dynamically, so that this problem can no longer arise.
%\end{itemize}
%
%
%\section{Contributions}\label{sec:contrib}
%
% I would be glad to include configuration files for more fonts. Preparing such
% configurations is quite a time-consuming task and requires a lot of patience.
% To alleviate this process, this package also includes a test file that can be
% used to check at least the protrusion settings (\file{test-microtype.tex}).
%
% If you have created a configuration file for another font, or if you have any
% suggestions for enhancements in the default configuration files, I~would
% gratefully accept them: \mailtoRS.\footnote{
%    Should you have lots of \pkg{pdfcprot} configuration files lying around,
%    I can also provide you with a \TeX\ conversion script. Just ask me.}
%
%
%\section{Acknowledgments}
%
%\def\contributor#1 <#2\at#3>{#1 (\mailto{#2@#3})}
%\def\contributor#1 <#2\at#3>{\textit{#1}}
%
% This package would be pointless if \contributor\thanh{} <thanh\at pdftex.org>
% hadn't created the \pdftex\ programme in the first place, which introduced the
% micro-typographic extensions and made them available to the \TeX\ world.
% Furthermore, I thank him for helping me to improve this package, and not
% least for promoting it in [\cite{ThanhPracTeX}].
%
% \contributor Harald Harders <h.harders\at tu-bs.de> has contributed protrusion
% settings for Adobe Minion. I~would also like to thank him for a number of bug
% reports and suggestions he had to make.
% \contributor Andreas B\"uhmann <andreas.buehmann\at web.de> has suggested the
% possibility to specify ranges of font sizes, and resourcefully assisted in
% implementing this. He also came up with some good ideas for the management of
% complex configurations.
%\ifbeta
% \contributor Ulrich Dirr <ud\at art-satz.de> has made numerous suggestion
% especially concerning the new extensions of interword spacing adjustment
% and additional character kerning.
%\fi
%
% I thank \contributor Philipp Lehman <plehman\at gmx.net> for adding to his
% \pkg{csquotes} package the possibility to restore the original meanings of
% all activated characters, thus allowing for these characters to be used in the
% configuration files.
% \contributor Peter Wilson <herries.press\at earthlink.net> kindly provided a
% hook in his \pkg{ledmac}/\pkg{ledpar} packages, so that critical editions can
% finally also benefit from character protrusion.
%
% Additionally, the following people have reported bugs or helped otherwise
% (in chronological order):
%\ifbeta\else
%  \contributor Ulrich Dirr     <ud\at satz-art.de>,
%\fi
%  \contributor Tom Kink        <kink\at hia.rwth-aachen.de>,
%  \contributor Herb Schulz     <herbs\at wideopenwest.com>,
%  \contributor Michael Hoppe   <mh\at michael-hoppe.de>,
%  \contributor Gary~L. Gray    <gray\at engr.psu.edu>,
%  \contributor Georg Verweyen  <Georg.Verweyen\at web.de>,
%  \contributor Christoph Bier  <christoph.bier\at web.de>,
%  \contributor Peter Muthesius <Pemu\at gmx.de>,
%  \contributor Bernard Gaulle  <gaulle\at idris.fr>,
%  \contributor Adam Kucharczyk <adamk\at ijp-pan.krakow.pl>,
%  \contributor Mark Rossi      <mark.rossi\at de.bosch.com>,
%  \contributor Stephan Hennig  <stephanhennig\at arcor.de>,
%  \contributor Michael Zedler  <Michael.Zedler\at tum.de>,
%and
%  \contributor Herbert Voss    <Herbert.Voss\at gmx.net>.
%
%
%\begin{thebibliography}{}
% \bibitem[Th\`anh 2000]{ThanhThesis}
%   \thanh, \emph{Micro-typographic extensions to the \TeX\ typesetting system},
%   \newblock Diss. Masaryk University Brno 2000,
%   \newblock in: \textit{TUGBoat}, vol.~21(2000), no.~4, pp.~317--434.
%   \newblock (Online at \url{http://www.tug.org/TUGboat/Articles/tb21-4/tb69thanh.pdf})
%
%\iffalse
% \bibitem[Th\`anh 2001]{ThanhTUG}
%   \thanh, \emph{Margin Kerning and Font Expansion with \pdftex},
%   \newblock in: \textit{TUGBoat}, vol.~22(2001), no.~3 -- Proceedings of the
%            2001 Annual Meeting, pp.~146--148.
%   \newblock (Online at \url{http://www.tug.org/TUGboat/Articles/tb22-3/tb72thanh.pdf})
%\fi
%
%\bibitem[Th\`anh 2004]{ThanhPracTeX}
%   \thanh, \emph{Micro-typographic Extensions of \pdftex\ in Practice},
%   \newblock in: \textit{TUGBoat}, vol.~25(2004), no.~1 -- Proceedings of the
%            Practical \TeX\ 2004 Conference, pp.~35--38.
%   \newblock (Online at \url{http://www.tug.org/TUGboat/Articles/tb25-1/thanh.pdf})
%
%\iffalse
% \bibitem[Th\`anh 2005]{ThanhEuroTeX}
%   \thanh, \emph{Experiences with micro-typographic extensions of \pdftex\ in practice},
%   \newblock in: Proceedings of Euro\TeX\ 2005, 15\textsuperscript{th}~Annual
%           Meeting of the European \TeX\ Users, March~7 -- March~11, 2005,
%           Abbaye des Pr\'emontr\'es, Pont-\`a-Mousson, pp.~81--88.
%   \newblock (Online at \url{http://www.dante.de/dante/events/eurotex/papers/TUT07.pdf})
%\fi
%
% \bibitem[\pdftex\ manual]{pdftexman}
%   \thanh, Sebastian Rahtz, Hans Hagen, Hartmut Henkel, \emph{The \pdftex\ user manual},
%   \newblock August 21, 2005.
%   \newblock (Available from \ctan\ at \ctanurl{/systems/pdftex/manual};
%              latest version at \url{http://sarovar.org/projects/pdftex/})
%
% \bibitem[\LaTeXe\ font selection]{fntguide}
%   \LaTeX3 Project Team, \emph{\LaTeXe\ font selection},
%   \newblock February 10, 2004.
%   \newblock (Available from \ctan\ at \ctanurl{/macros/latex/doc/fntguide.pdf})
%
% \bibitem[\texttt{pdfcprot}]{pdfcprot}
%   Carsten Schurig, \emph{The |pdfcprot.sty| package},
%   \newblock August 14, 2002.
%   \newblock (Available from \ctan\ at \ctanurl{/macros/latex/contrib/pdfcprot/})
%
%\ifbeta
% \bibitem[\texttt{soul}]{soul}
%   Melchior Franz, \emph{The |soul| package},
%   \newblock November 13, 2003.
%   \newblock (Available from \ctan\ at \ctanurl{/macros/latex/contrib/soul/})
%\fi
%\end{thebibliography}
%
%
%\section{Short History}\label{sec:short-history}
%\changes{v1.5}{2004/12/11}{add short history (section~\ref{sec:short-history})}
%
% The comprehensive list of changes can be
% \expandafter\ifx\csname r@sec:changes\endcsname\relax
%     obtained by running `\texttt{makeindex \mbox{-s gglo.ist}
%                                  \mbox{-o microtype.gls} microtype.glo}'^^A
% \else
%     found in appendix~\ref{sec:changes}^^A
% \fi.
% The following is a list of all changes relevant in the user land; bug fixes
% are swept under the rug.
%
%\newenvironment{History}
%  {\list{}{
%     \leftmargin 15pt
%     \itemindent-15pt
%     \labelwidth 22pt
%     \labelsep 0pt
%     \parsep 0pt
%     \itemsep 0pt plus 2pt
%     \def\makelabel##1{##1\hss}
%     }}
%  {\endlist}
%\def\Version #1 (#2.#3.#4){
%  \vskip 1ex plus 1pt
%  \pagebreak[2]
%  \VersionDate{#1}{#4/#3/#2}
%  \item[#1](\number#2.\,\number#3.\,#4)^^A \number removes leading zeroes
%  \nopagebreak[4]
%}
%\def\VersionDate#1#2{^^A needed in the Change History
%  \global\expandafter\def\csname MTversiondate#1\endcsname{#2}}
%
%\begin{History}
%\betastart
%\Version 2.0 (28.10.2005)
%  \item Support for the new extensions of \pdftex\ version 1.3x: adjustment of
%        interword spacing (glue), and additional kerning
%        (new commands \cs{SetExtraSpacing} and \cs{SetExtraKerning},
%         new options `\opt{spacing}', `\opt{kerning}';
%         see section~\ref{sub:interword-spacing} and~\ref{sub:kerning})
%  \item New option `\opt{babel}' to activate automatic adjustment to the
%        \pkg{babel} package
%        (see sections~\ref{sub:options-misc} and~\ref{sec:context})
%  \item New commands \cs{textls} and \cs{lsstyle} for letterspacing
%        (requires \pdftex\ version 1.3x;
%         see section~\ref{sec:lettersp})
%\betaend
%
%\Version 1.9 (28.10.2005)
%  \item New command \cs{DisableLigatures} to disable ligatures of fonts
%        (requires \pdftex\ version 1.30 or later;
%         see section~\ref{sec:disable-ligatures})
%  \item New command \cs{microtypecontext} to change the configuration context;
%        new key `|context|' for the configuration commands
%        (see section~\ref{sec:context})
%  \item New key `|font|' to add single fonts to the font sets
%        (see section~\ref{sec:font-sets})
%  \item New key `|preset|' to set all characters to the specified value before
%        loading the lists
%  \item Value `|relative|' renamed to `|character|' for `|unit|' keys
%  \item Support for the Polish \otIV\ encoding (protrusion, expansion, inheritance)
%  \item Support for the Vietnamese \tV\ encoding (protrusion, expansion, inheritance)
%  \item `\opt{DVIoutput}' will work with \TeX Live 2004
%
%\Version 1.8 (23.06.2005)
%  \item If font substitution has occurred, the settings for the substitute will
%        be used instead of those for the selected font
%  \item New command \cs{DeclareMicrotypeSetDefault} to declare the default font
%        sets (see section~\ref{sec:font-sets})
%  \item New option `\opt{config}' to load a different configuration file
%        (see section~\ref{sub:options-misc})
%  \item New option `\opt{unit}' to measure protrusion factors relative to a
%        dimension instead of the character width
%        (see section~\ref{sub:protrusion})
%  \item Renamed commands from \cs{..MicroType..} to \cs{..Microtype..}
%  \item Protrusion settings for \ams\ math fonts
%  \item Protrusion settings for Times in \lyI\ encoding completed
%  \item The `|allmath|' font set also includes \U\ encoding
%  \item 8-bit characters in the configuration files finally work as advertised,
%        even if made active by the \pkg{csquotes} package
%  \item When using the \pkg{ledmac} package, character protrusion will work
%        for the first time ever (requires \pdftex\ version 1.30 or later)
%
%\Version 1.7 (23.03.2005)
%  \item Possibility to specify ranges of font sizes in the set declarations
%        and protrusion and expansion settings
%        (see sections~\ref{sec:font-sets} and~\ref{sec:fine-tuning})
%  \item Always take font size into account when trying to find protrusion resp.
%        expansion settings for a given font
%        (see section~\ref{sec:fine-tuning})
%  \item New command \cs{LoadMicrotypeFile} to load a font configuration file
%        manually (see section~\ref{sub:config-file})
%  \item Hook \cs{Microtype@Hook} for font package authors (see section~\ref{sub:hook})
%  \item New option `\opt{verbose}|=errors|' to turn all warnings into errors
%^^A  \item Compatibility with the \pkg{chemsym} and \pkg{statex} packages
%  \item Disable expansion inside \cmd{\showhyphens}
%  \item Warning when running in draft mode
%
%\Version 1.6a (02.02.2005)
%  \item Compatibility with the \pkg{frenchpro} package
%
%\Version 1.6 (24.01.2005)
%  \item New option `\opt{factor}' to influence protrusion resp. expansion of
%        all characters of a font or font set
%        (see sections~\ref{sub:options-protrusion} and~\ref{sec:fine-tuning})
%  \item When \pdftex\ is too old to expand fonts automatically, expansion
%        has to be enabled explicitly, automatic expansion will be disabled
%        (see section~\ref{sub:options-microtype})
%  \item Protrusion settings of digits improved
%  \item Use \etex\ extensions, if available
%
%\Version 1.5 (15.12.2004)
%  \item When output mode is \dvi, font expansion has to be enabled
%        explicitly, automatic expansion will be disabled
%        (see section~\ref{sub:options-microtype})
%  \item New option `\opt{selected}' to enable selected expansion
%        (see sections~\ref{sub:options-expansion} and \ref{sub:expansion});
%        default is: |false|
%  \item New default for expansion option `\opt{step}': 4 (min(\opt{stretch},\opt{shrink})/5)
%        (see section~\ref{sub:options-expansion})
%  \item Protrusion settings for Bitstream Charter
%  \item Compatibility with Turkish \pkg{babel}
%
%\Version 1.4b (26.11.2004)
%  \item \cs{UseMicrotypeSet} requires the set to be declared
%        (see section~\ref{sec:font-sets})
%  \item Internal optimization
%^^A\Version 1.4a (17.11.2004)
%
%^^A  \item Bug fix update (font setup inside |tabular|s)
%\VersionDate{1.4a}{2004/11/17}
%\Version 1.4 (12.11.2004)
%  \item Set up fonts independently from \LaTeX\ font loading
%        (therefore, no risk of overlooking fonts anymore, and the package may
%        be loaded at any time)
%  \item \cs{microtypesetup} now sets the correct level of protrusion
%        (see chapter~\ref{sec:options-cmd})
%  \item New option: `\opt{final}'
%
%\Version 1.3 (27.10.2004)
%  \item Compatibility with the \pkg{german} and \pkg{ngerman} packages
%
%\Version 1.2 (03.10.2004)
%  \item New font sets: `|allmath|' and `|basicmath|'
%        (see section~\ref{sec:font-sets} and table~\ref{tab:predefined-font-sets})
%  \item Protrusion settings for Computer Modern Roman math symbols
%  \item Protrusion settings for \tsI\ encoding completed for Computer Modern
%        Roman and Adobe Garamond
%  \item If an alias font name is specified, it will be used as an alternative,
%        not as a replacement (see section~\ref{sub:config-file})
%  \item More tests for sanity of settings and whether all fonts will be set up
%  \item More robust parsing of sizes in font sets
%
%\Version 1.1 (21.09.2004)
%  \item Protrusion settings for Adobe Minion, contributed by Harald Harders
%  \item New command: \cs{DeclareCharacterInheritance}
%        (see section~\ref{sub:inherit})
%  \item Characters may also be specified as octal or hexadecimal numbers
%        (see section~\ref{sec:fine-tuning})
%  \item Configuration file names in lowercase (see section~\ref{sub:config-file})
%
%\Version 1.0 (11.09.2004)
%  \item First \ctan\ release
%\end{History}
%
% \GeneralChanges!
%
% \StopEventually{
%   \typeout{:?1000} ^^A tell WinEdt not to bother about overfull boxes
%   \appendix
%   \PrintChanges
%   \PrintIndex
%   \typeout{:?1111}
% }
%
%
% ^^A =========================================================================
%
%\DoNotIndex{\!,\",\',\(,\),\*,\+,\,,\-,\.,\/,\:,\;,\<,\=,\>,\?,\[,\\,\],\`,
%  \#,\$,\%,\&,\^,\_,\|,\~,\ } ^^A won't work: \{,\}
%\DoNotIndex{\advance,\aftergroup,\begingroup,\catcode,\char,\chardef, ^^A TeX
%  \csname,\def,\divide,\edef,\else,\endcsname,\endgroup,\endinput,\escapechar,
%  \expandafter,\fi,\gdef,\global,\hbox,\if,\ifcase,\ifdim,\iffalse,\ifhbox,\ifnum,
%  \iftrue,\ifx,\input,\inputlineno,\kern,\let,\meaning,\multiply,\newcount,
%  \newdimen,\newif,\newtoks,\noexpand,\number,\or,\relax,\romannumeral,\setbox,
%  \space,\string,\the,\uppercase,\wd,\xdef}
%^^A\DoNotIndex{\font,\fontdimen,\nullfont}
%\DoNotIndex{\@backslashchar,\@cclv,\@cclvi,\@classoptionslist,\@currext, ^^A LaTeX
%  \@currname,\@defaultunits,\@empty,\@expandtwoargs,\@firstofone,\@firstoftwo,
%  \@gobble,\@ifnextchar,\@ifpackagelater,\@ifpackageloaded,\@ifpackagewith,\@ifstar,
%  \@m,\@makeother,\@nameuse,\@ne,\@nil,\@nnil,\@nodocument,\@onelevel@sanitize,
%  \@onlypreamble,\@ptionlist,\@removeelement,\@secondoftwo,\@spaces,\@tempa,
%  \@tempb,\@tempc,\@tempcnta,\@tempcntb,\@tempdima,\@tempswafalse,\@tempswatrue,
%  \@undefined,\@unprocessedoptions,\@unusedoptionlist,\g@addto@macro,\if@tempswa,
%  \m@ne,\p@,\set@display@protect,\strip@prefix,\strip@pt,\thr@@,\tw@,\z@,\zap@space}
%\DoNotIndex{\active,\makeatletter,\newcommand,\nonfrenchspacing,\normalsize,
%  \renewcommand,\AtBeginDocument,\AtEndOfPackage,\CheckCommand,\CurrentOption,
%  \DeclareRobustCommand,\IfFileExists,\InputIfFileExists,\MessageBreak,
%  \PackageError,\PackageInfo,\PackageWarning,\RequirePackage}
%^^A\DoNotIndex{\@@enc@update,\add@accent,\cf@encoding,\curr@fontshape,
%^^A  \define@newfont,\do@subst@correction,\f@encoding,\f@size,\font@name,
%^^A  \normalfont,\pickup@font,\selectfont,\set@fontsize}
%\DoNotIndex{\detokenize,\dimexpr,\eTeXversion,\ifcsname,\ifdefined,\numexpr} ^^A e-TeX
%^^A\DoNotIndex{\fontcharwd,\iffontchar}
%^^A\DoNotIndex{\{ef,lp,rp}code,\{left,right}marginkern,\{kn{bc,ac},{kn,st,sh}bs}code,\pdf*} ^^A pdfTeX
%\DoNotIndex{\normalpdfoutput,\normalpdftexversion,\normalpdftexrevision} ^^A TeXLive
%\DoNotIndex{\@inpenc@undefined@,\IeC} ^^A inputenc
%\DoNotIndex{\@disablequotes} ^^A csquotes
%\DoNotIndex{\foreign@language,\languagename,\select@language} ^^A babel
%\DoNotIndex{\percentsign} ^^A Spanish babel
%\DoNotIndex{\NoAutoSpaceBeforeFDP} ^^A French babel
%\DoNotIndex{\l@dunhbox@line} ^^A ledmac/ledpar
%\DoNotIndex{\pdfstringdefDisableCommands} ^^A hyperref
%\DoNotIndex{\define@key,\KV@@sp@def,\setkeys} ^^A keyval
%\DoNotIndex{\color@begingroup,\color@endgroup,\everypar,\hbadness,\hsize, ^^A \showhyphens
%  \maxdimen,\parfillskip,\pretolerance,\showboxdepth,\tolerance,\vbox,\z@skip}
%\DoNotIndex{\CHAR,\KVo@tempa,\MT@dinfo,\MT@dinfo@nl,\tracingmicrotype,\x,\y} ^^A microtype
%\DoNotIndex{\DeclareMicroTypeSet,\UseMicroTypeSet,\DeclareMicroTypeAlias, ^^A ... old commands
%  \LoadMicroTypeFile,\MicroType@Hook}
%
% ^^A -------------------------------------------------------------------------
%
%\newpage
%\ImplementationSettings
%
%\section{Implementation}
%
% The \pkg{docstrip} modules in this file are:
%\begin{description}
%  \item[|driver|:]  The documentation driver, only visible in the \file{dtx} file.
%  \item[|package|:] The code for the \microtype\ package (\file{microtype.sty}).
%  \begin{description}
%    \item[|debug|:] Code for additional output in the log file.\\
%                    Used for -- surprise! -- debugging purposes.
%  \end{description}
%  \item[|config|:]  Surrounds all configuration modules.
%  \begin{description}
%    \item[|m-t|:]   The main configuration file (\file{microtype.cfg}).
%    \item[|bch|:]   Settings for Bitstream Charter (\file{mt-bch.cfg}).
%    \item[|cmr|:]   Settings for Computer Modern Roman (\file{mt-cmr.cfg}).
%    \item[|pad|:]   Settings for Adobe Garamond (\file{mt-pad.cfg}).
%    \item[|ppl|:]   Settings for Palatino (\file{mt-ppl.cfg}).
%    \item[|ptm|:]   Settings for Times (\file{mt-ptm.cfg}).
%    \item[|pmn|:]   Settings for Adobe Minion (\file{mt-pmn.cfg}). \\
%                    Contributed by Harald Harders.
%    \item[|cfg-u|:] Surrounds non-text configurations (\U\ encoding).
%    \begin{description}
%      \item[|msa|:] Settings for \ams\ `a' symbol font (\file{mt-msa.cfg}).
%      \item[|msb|:] Settings for \ams\ `b' symbol font (\file{mt-msb.cfg}).
%      \item[|euf|:] Settings for \ams\ Euler Fraktur font (\file{mt-euf.cfg}).
%      \item[|eus|:] Settings for \ams\ Euler script font (\file{mt-eus.cfg}).
%    \end{description}
%  \end{description}
%  \item[|test|:]    A helper file that may be used to create and test
%                    protrusion settings (\file{test-microtype.tex}).
%  \item[|beta|:]    Support for features not yet included in an official
%                    release of \pdftex.
%\end{description}
% And now for something completely different.
%
%    \begin{macrocode}
%<*package>
%    \end{macrocode}
% These are all commands for the outside world. We define them here as dummy
% commands, so that they won't generate an error if we are not running \pdftex.
%    \begin{macrocode}
\newcommand*\DeclareMicrotypeSet[3][]{}
\newcommand*\UseMicrotypeSet[2][]{}
\newcommand*\DeclareMicrotypeSetDefault[2][]{}
\newcommand*\DeclareMicrotypeAlias[2]{}
\newcommand*\SetProtrusion[3][]{}
\newcommand*\SetExpansion[3][]{}
\newcommand*\DisableLigatures[1]{}
\newcommand*\DeclareCharacterInheritance[3][]{}
\newcommand*\LoadMicrotypeFile[1]{}
\newcommand*\microtypesetup[1]{}
\newcommand*\microtypecontext[1]{}
%<*beta>
\newcommand*\SetExtraSpacing[3][]{}
\newcommand*\SetExtraKerning[3][]{}
\newcommand*\DeclareMicrotypeBabelHook[2]{}
\newcommand\textls[2][]{#2}
\newcommand\lsstyle{}
%</beta>
%    \end{macrocode}
% This command also has a starred version.
%    \begin{macrocode}
\def\DeclareMicrotypeSet{%
  \@ifstar
    {\@ifnextchar[\MT@DeclareSet{\MT@DeclareSet[]}}%
    {\@ifnextchar[\MT@DeclareSet{\MT@DeclareSet[]}}%
}
\def\MT@DeclareSet[#1]#2#3{}
%    \end{macrocode}
% Set declarations are only allowed in the preamble (resp. the main
% configuration file). All the other commands, on the other hand, must be
% allowed in the document, too, since they may be called inside font
% configuration files.
%    \begin{macrocode}
\@onlypreamble{\DeclareMicrotypeSet}
\@onlypreamble{\UseMicrotypeSet}
\@onlypreamble{\DisableLigatures}
%    \end{macrocode}
%\begin{macro}{\MT@old@cmd}
% The old command names had one more hunch.
%\changes{v1.8}{2005/04/28}{renamed commands from \cs{..MicroType..} to
%                           \cs{..Microtype..}}
%    \begin{macrocode}
\def\MT@old@cmd#1#2{%
  \MT@warning{\string#1 is deprecated. Please use\MessageBreak
              \string#2 instead}%
  \let#1#2#2}
%    \end{macrocode}
%\end{macro}
%    \begin{macrocode}
\newcommand*\DeclareMicroTypeSet{%
 \MT@old@cmd\DeclareMicroTypeSet
            \DeclareMicrotypeSet}
\newcommand*\UseMicroTypeSet{%
 \MT@old@cmd\UseMicroTypeSet
            \UseMicrotypeSet}
\newcommand*\DeclareMicroTypeAlias{%
 \MT@old@cmd\DeclareMicroTypeAlias
            \DeclareMicrotypeAlias}
\newcommand*\LoadMicroTypeFile{%
 \MT@old@cmd\LoadMicroTypeFile
            \LoadMicrotypeFile}
%    \end{macrocode}
%\begin{macro}{\MT@error}
%\begin{macro}{\MT@warning}
%\begin{macro}{\MT@warning@nl}
%\begin{macro}{\MT@warn@err}
%\changes{v1.7}{2005/03/16}{new macro: for \opt{verbose}\quotechar=\texttt{errors}}
%\begin{macro}{\MT@info}
%\begin{macro}{\MT@info@nl}
%\begin{macro}{\MT@vinfo}
%\changes{v1.6}{2005/01/06}{new macro: used instead of \cs{ifMT@verbose}}
% Communicate.
%    \begin{macrocode}
\def\MT@error{\PackageError{microtype}}
\def\MT@warning{\PackageWarning{microtype}}
\def\MT@warning@nl#1{\MT@warning{#1\@gobble}}
\def\MT@warn@err#1{\MT@error{#1}{%
  This error message appears because you loaded the `microtype'\MessageBreak
  package with the option `verbose=errors'. Consult the documentation\MessageBreak
  in microtype.(pdf,dvi) to find out what went wrong.}}
\def\MT@info{\PackageInfo{microtype}}
\def\MT@info@nl#1{\MT@info{#1\@gobble}}
%<!debug>\let\MT@vinfo\@gobble
%    \end{macrocode}
%\end{macro}
%\end{macro}
%\end{macro}
%\end{macro}
%\end{macro}
%\end{macro}
%\end{macro}
% Debug. Cases for \cs{tracingmicrotype}:
%\begin{itemize}
%  \item[|0|:] almost none
%  \item[|1|:] + sets \& lists
%  \item[|2|:] + heirs
%  \item[|3|:] + slots
%  \item[|4|:] + factors
%\end{itemize}
%    \begin{macrocode}
%<*debug>
\let\MT@vinfo\MT@info@nl
\newcount\tracingmicrotype
\tracingmicrotype=\tw@
\def\MT@dinfo#1#2{\ifnum\tracingmicrotype<#1\relax\else\MT@info{#2}\fi}
\def\MT@dinfo@nl#1#2{\ifnum\tracingmicrotype<#1\relax\else\MT@info@nl{#2}\fi}
%</debug>
%    \end{macrocode}
%
%\subsection{Requirements}
%
%\begin{macro}{\MT@pdftex@no}
%\changes{v1.6a}{2005/02/02}{new macro}
% \pdftex's features for which we provide an interface here haven't always been
% available, and some specifics have changed over time. Therefore, we have to
% test which \pdftex\ we're using, if any. \cs{MT@pdftex@no} will be used
% throughout the package to respectively do the right thing.
%
% Currently, there are six cases for \pdftex:
%\begin{itemize}
%  \item[|0|:] not running \pdftex
%  \item[|1|:] \pdftex\ (\textless\ 0.14f)
%  \item[|2|:] + micro-typographic extensions (0.14f, 0.14g)
%  \item[|3|:] + protrusion relative to 1em (\textgeq\ 0.14h)
%  \item[|4|:] + automatic font expansion; default \cmd{\efcode}\,=\,1000 (\textgeq\ 1.20)
%\changes{v1.8}{2005/04/15}{case 5: \pdftex\ 1.30}
%  \item[|5|:] + \cmd{\(left\textbar right)marginkern}; \cmd{\pdfnoligatures} (\textgeq\ 1.30)
%\changes{v2.0}{2005/08/20}{case 6: \pdftex\ 1.3x}
%  \item[|6|:] + \cmd{\(knbs\textbar stbs\textbar shbs\textbar knbc\textbar knac)code} (\textgeq\ 1.3x)
%\end{itemize}
%\changes{v1.1}{2004/09/13}{bug fix concerning version check
%                           (reported by Harald Harders)}
%    \begin{macrocode}
\let\MT@pdftex@no\z@
%    \end{macrocode}
%\changes{v1.9}{2005/08/29}{compatibility with \TeX Live hack
%                           (reported by Herbert Vo\ss)} ^^A <Herbert.Voss\at gmx.net>
%                                      ^^A in <MID:deprh8$c69$00$1@news.t-online.com>
% A hack circumventing the \TeX Live 2004 hack which undefines the \pdftex\
% primitives in the format in order to hide the fact that \pdftex\ is being run
% from the user. This has been \emph{fixed} in \TeX Live 2005.
%    \begin{macrocode}
\ifx\normalpdftexversion\@undefined \else
  \let\pdftexversion \normalpdftexversion
  \let\pdftexrevision\normalpdftexrevision
  \let\pdfoutput     \normalpdfoutput
\fi
%    \end{macrocode}
% Old packages might have defined \cmd{\pdftexversion} to \cmd{\relax}.
%    \begin{macrocode}
\ifx\pdftexversion\@undefined \else
  \ifx\pdftexversion\relax \else
%<debug>\MT@dinfo@nl{0}{running pdftex \the\pdftexversion(\pdftexrevision)}
%<!beta>    \def\MT@pdftex@no{5}
%<*beta>
  \ifx\knbccode\@undefined
    \def\MT@pdftex@no{5}
  \else
    \def\MT@pdftex@no{6}
  \fi
%</beta>
    \ifnum\pdftexversion < 130
      \def\MT@pdftex@no{4}
      \ifnum\pdftexversion < 120
        \let\MT@pdftex@no\thr@@
        \ifnum\pdftexversion = 14
          \ifnum \expandafter`\pdftexrevision < `h
            \let\MT@pdftex@no\tw@
            \ifnum \expandafter`\pdftexrevision < `f
              \let\MT@pdftex@no\@ne
            \fi
          \fi
        \else
          \ifnum\pdftexversion < 14
            \let\MT@pdftex@no\@ne
          \fi
        \fi
      \fi
    \fi
  \fi
\fi
%<debug>\MT@dinfo@nl{0}{pdftex no: \number\MT@pdftex@no}
%    \end{macrocode}
%\end{macro}
% If we are not using \pdftex\ or in case it is too old, we disable everything
% and exit here.
%    \begin{macrocode}
\ifnum\MT@pdftex@no<\tw@
  \AtEndOfPackage{\let\@unprocessedoptions\relax}%
  \let\CurrentOption\@empty
  \MT@warning@nl{%
    \ifcase\MT@pdftex@no
      You don't seem to be using pdftex.\MessageBreak
    \or
      You are using a pdftex version older than 0.14f.\MessageBreak
      microtype won't work with such antiquated versions.\MessageBreak
      Please install a newer version of pdftex.\MessageBreak
    \fi
    All micro-typographic features will be disabled}
  \expandafter
  \endinput
\fi
%    \end{macrocode}
% Still there? Then we can begin:
%\begin{macro}{\MT@catcodes}
%\changes{v1.3}{2004/10/27}{check some category codes (compatibility with \pkg{german})}
%\changes{v1.5}{2004/12/03}{reset catcode of `\texttt{\^}' (compatibility with \pkg{chemsym})}
% We have to make sure that the category codes of some characters are correct
% (the \pkg{german} package, for instance, makes |"| active). Probably overly
% cautious. Ceterum censeo: It should be forbidden for packages to change
% catcodes within the preamble.
%    \begin{macrocode}
\def\MT@catcodes{%
  \catcode`\^7 %
  \@makeother\-%
  \@makeother\=%
  \@makeother\*%
  \@makeother\,%
  \@makeother\/%
  \@makeother\`%
  \@makeother\'%
  \@makeother\"%
  \@makeother\|%
}
%    \end{macrocode}
%\end{macro}
%\begin{macro}{\MT@restore@catcodes}
% Polite as we are, we'll restore them afterwards.
%    \begin{macrocode}
\def\MT@restore@catcodes#1{%
  \ifx\relax#1\else
    \noexpand\catcode`\noexpand#1\the\catcode`#1\relax
    \expandafter\MT@restore@catcodes
  \fi
}
\edef\MT@restore@catcodes{%
  \MT@restore@catcodes\^\-\=\*\,\/\`\'\"\|\relax
}
%    \end{macrocode}
%\end{macro}
%    \begin{macrocode}
\MT@catcodes
\AtEndOfPackage{\MT@restore@catcodes}
%    \end{macrocode}
% We need the \pkg{keyval} package, including the new \cmd{\KV@@sp@def}
% implementation.
%    \begin{macrocode}
\RequirePackage{keyval}[1997/11/10]
%    \end{macrocode}
%\begin{macro}{\mt@toks}
% We need a token register.
%\changes{v1.9}{2005/09/28}{no longer use \cmd{\toks@}}
%    \begin{macrocode}
\newtoks\mt@toks
%    \end{macrocode}
%\end{macro}
%\begin{macro}{\ifMT@protrusion} \IndexNewif{MT@protrusion}
%\begin{macro}{\ifMT@expansion}  \IndexNewif{MT@expansion}
%\begin{macro}{\ifMT@auto}       \IndexNewif{MT@auto}
%\begin{macro}{\ifMT@selected}   \IndexNewif{MT@selected}
%\begin{macro}{\ifMT@spacing}    \IndexNewif{MT@spacing}
%\begin{macro}{\ifMT@kerning}    \IndexNewif{MT@kerning}
%\begin{macro}{\ifMT@noligatures}\IndexNewif{MT@noligatures}
%\begin{macro}{\ifMT@DVIoutput}  \IndexNewif{MT@DVIoutput}
%\begin{macro}{\ifMT@draft}      \IndexNewif{MT@draft}
%\begin{macro}{\ifMT@babel}      \IndexNewif{MT@babel}
% These are the global switches~\dots
%    \begin{macrocode}
\newif\ifMT@protrusion
\newif\ifMT@expansion
\newif\ifMT@auto
\newif\ifMT@selected
%<*beta>
\newif\ifMT@spacing
\newif\ifMT@kerning
%</beta>
\newif\ifMT@noligatures
\newif\ifMT@DVIoutput
\newif\ifMT@draft
%<beta>\newif\ifMT@babel
%    \end{macrocode}
%\end{macro}
%\end{macro}
%\end{macro}
%\end{macro}
%\end{macro}
%\end{macro}
%\end{macro}
%\end{macro}
%\end{macro}
%\end{macro}
%\begin{macro}{\MT@pr@level}
%\begin{macro}{\MT@pr@factor}
%\begin{macro}{\MT@pr@unit}
%\begin{macro}{\MT@ex@level}
%\begin{macro}{\MT@ex@factor}
%\begin{macro}{\MT@stretch}
%\begin{macro}{\MT@shrink}
%\begin{macro}{\MT@step}
%\begin{macro}{\MT@sp@factor}
%\begin{macro}{\MT@sp@unit}
%\begin{macro}{\MT@kn@factor}
%\begin{macro}{\MT@kn@unit}
%\begin{macro}{\MT@letterspacing}
% \dots~and numbers.
%    \begin{macrocode}
\let\MT@pr@level\tw@
\let\MT@pr@factor\@m
\let\MT@pr@unit\@empty
\let\MT@ex@level\tw@
\let\MT@ex@factor\@m
\let\MT@stretch\m@ne
\let\MT@shrink \m@ne
\let\MT@step   \m@ne
%<*beta>
\let\MT@sp@factor\@m
\let\MT@kn@factor\@m
%    \end{macrocode}
% Default unit for spacing settings is |space|, default unit for kerning is
% 1em.
%    \begin{macrocode}
\let\MT@sp@unit\m@ne
\def\MT@kn@unit{1em}
\let\MT@letterspacing\m@ne
%</beta>
%    \end{macrocode}
%\end{macro}
%\end{macro}
%\end{macro}
%\end{macro}
%\end{macro}
%\end{macro}
%\end{macro}
%\end{macro}
%\end{macro}
%\end{macro}
%\end{macro}
%\end{macro}
%\end{macro}
%\begin{macro}{\MT@pr@min}
%\begin{macro}{\MT@pr@max}
%\begin{macro}{\MT@ex@min}
%\begin{macro}{\MT@ex@max}
%\begin{macro}{\MT@sp@min}
%\begin{macro}{\MT@sp@max}
%\begin{macro}{\MT@kn@min}
%\begin{macro}{\MT@kn@max}
% Minimum and maximum values allowed by \pdftex.
%    \begin{macrocode}
\def\MT@pr@min{-\@m}
\let\MT@pr@max\@m
\let\MT@ex@min\z@
\let\MT@ex@max\@m
%<*beta>
\def\MT@sp@min{-\@m}
\let\MT@sp@max\@m
\def\MT@kn@min{-\@m}
\let\MT@kn@max\@m
%</beta>
%    \end{macrocode}
%\end{macro}
%\end{macro}
%\end{macro}
%\end{macro}
%\end{macro}
%\end{macro}
%\end{macro}
%\end{macro}
%\begin{macro}{\MT@factor@default}
%\begin{macro}{\MT@stretch@default}
%\begin{macro}{\MT@shrink@default}
%\begin{macro}{\MT@step@default}
% Default values for expansion.
%    \begin{macrocode}
\def\MT@factor@default{1000 }
\def\MT@stretch@default{20 }
\def\MT@shrink@default{20 }
\def\MT@step@default{4 }
%    \end{macrocode}
%\end{macro}
%\end{macro}
%\end{macro}
%\end{macro}
%\begin{macro}{\MT@letterspacing@default}
% Default value for letterspacing (in thousandths of 1em).
%    \begin{macrocode}
%<beta>\def\MT@letterspacing@default{100 }
%    \end{macrocode}
%\end{macro}
%
% \subsection{Compatibility}
%
% For the record, the following \LaTeX\ commands will be modified by \microtype:
%\begin{itemize}
%  \item \cmd{\pickup@font}
%  \item \cmd{\do@subst@correction}
%  \item \cmd{\add@accent}
%  \item \cmd{\showhyphens}
%\end{itemize}
%
%\begin{macro}{\MT@pdfcprot@error}
%\changes{v1.4}{2004/11/04}{check for \pkg{pdfcprot}}
% Our competitor, the \pkg{pdfcprot} package, must not be tolerated!
%    \begin{macrocode}
\def\MT@pdfcprot@error{%
  \MT@error{Detected the `pdfcprot' package!\MessageBreak
            `microtype' and `pdfcprot' may not be used together}{%
The `pdfcprot' package provides an interface to character protrusion.\MessageBreak
So does the `microtype' package. Using both packages at the same\MessageBreak
time will almost certainly lead to undesired results. Have your choice!}%
  \let\MT@pdfcprot@error\relax
}
\@ifpackageloaded{pdfcprot}\MT@pdfcprot@error\relax
%    \end{macrocode}
%\end{macro}
%\begin{macro}{\MT@ledmac@setup}
%\begin{macro}{\MT@led@unhbox@line}
%\begin{macro}{\MT@led@kern}
% The \pkg{ledmac} package first saves each paragraph in a box, from which it
% then splits off the lines one by one. This will destroy character protrusion.
% (There aren't any problems with the \pkg{lineno} package, since it takes a
% different approach.)
%\iffalse
% We issue a warning, so that nobody can say they didn't know.
%\changes{v1.6}{2004/12/29}{warning when using the \pkg{ledmac} package}
% --- This is fixed in \pdftex\ 1.21a, so we no longer warn.
%\changes{v1.6a}{2004/02/02}{no warning when using the \pkg{ledmac} package}
% --- Actually, it was completely broken in 1.21a, but 1.21b provides a work-around.
%\fi
% ---~\dots~---
%\changes{v1.8}{2005/04/15}{character protrusion with \pkg{ledmac}}
% After much to and fro, the situation has finally settled and there is a fix.
% Beginning with \pdftex\ version 1.21b together with \file{ledpatch.sty} as of
% 2005/06/02 (v0.4), character protrusion will work at last.
%
% Peter Wilson was so kind to provide the \cmd{\l@dunhbox@line} hook in
% \pkg{ledmac} to allow for protrusion. \cmd{\leftmarginkern} and
% \cmd{\rightmarginkern} are new primitives of \pdftex\ 1.21b (aka 1.30.0).
%    \begin{macrocode}
\def\MT@ledmac@setup{%
  \ifMT@protrusion
    \ifnum\MT@pdftex@no > 4
      \MT@ifdefined@c\l@dunhbox@line{%
        \MT@info@nl{Patching ledmac to enable character protrusion}%
        \newdimen\MT@led@kern
        \let\MT@led@unhbox@line\l@dunhbox@line
        \renewcommand*{\l@dunhbox@line}[1]{%
          \ifhbox##1%
            \MT@led@kern=\rightmarginkern##1%
            \kern\leftmarginkern##1%
            \MT@led@unhbox@line##1%
            \kern\MT@led@kern
          \fi
        }%
      }{%
        \MT@warning@nl{%
          Character protrusion in paragraphs with line\MessageBreak
          numbering will only work if you update ledmac}%
      }%
    \else
      \MT@warning@nl{%
        The pdftex version you are using does not allow\MessageBreak
        character protrusion in paragraphs with line\MessageBreak
        numbering by the `ledmac' package.\MessageBreak
        Upgrade pdftex to version 1.30 or later}%
    \fi
  \fi
}
%    \end{macrocode}
%\end{macro}
%\end{macro}
%\end{macro}
%\begin{macro}{\MT@setupfont@hook}
% This hook will be executed every time a font is set up (inside a group).
%
% In the preamble, we check for the packages each time a font is set up.
% Thus, it will work regardless when the packages are loaded.
%
% Even for packages that don't activate any characters in the preamble (like
% \pkg{babel} and \pkg{csquotes}), we have to check here, too, in case they were
% loaded before \microtype, and a font is loaded \cmd{\AtBeginDocument}, before
% \microtype.
%    \begin{macrocode}
\def\MT@setupfont@hook{%
%    \end{macrocode}
%^^A\changes{v1.7}{2005/02/06}{compatibility with the \pkg{chemsym} package}
% The \pkg{chemsym} package redefines, among other commands, the Hungarian
% umlaut \cmd{\H} in a way that cannot be parsed by \microtype. As a work-around,
% we restore the usual definition of \cmd{\H} before setting up the font (which
% will be done inside a group).
%^^A\changes{v1.8}{2005/03/24}{remove superfluous \pkg{chemsym} hook}
% --- Since version 1.7, this is no longer needed, since our character parsing
% is robust enough now.
%
%^^A\changes{v1.7}{2005/03/10}{compatibility with the \pkg{statex} package}
% Same for \pkg{statex}.
%^^A\changes{v1.8}{2005/03/24}{remove superfluous \pkg{statex} hook}
% --- No longer needed, either.
%
% Spanish \pkg{babel} modifies the percent character, storing the original
% meaning in \cmd{\percentsign}.
%\changes{v1.8}{2005/03/30}{restore percent character if Spanish \pkg{babel} is loaded}
%    \begin{macrocode}
  \@ifpackagewith{babel}{spanish}{%
    \MT@ifdefined@c\percentsign
      {\let\%\percentsign}\relax
  }\relax
%    \end{macrocode}
%\changes{v1.8}{2005/04/22}{restore \pkg{csquotes}'s active characters}
% Using \cmd{\@disablequotes}, we can restore the original meaning of all
% characters made active by \pkg{csquotes}.
% (It would be doable for older versions, too, but we won't bother.)
%    \begin{macrocode}
  \@ifpackageloaded{csquotes}{%
    \@ifpackagelater{csquotes}{2005/05/11}\@disablequotes\relax
  }\relax
%    \end{macrocode}
% \pkg{hyperref} redefines \cmd{\%} and \cmd{\#} inside a \cmd{\url}. We restore
% the original meanings (which we can only hope are correct).
%\changes{v1.8}{2005/05/20}{restore \cmd{\%} and \cmd{\#} when \pkg{hyperref} is loaded}
%    \begin{macrocode}
  \@ifpackageloaded{hyperref}{%
    \chardef\%`\%
    \chardef\#`\#
  }\relax
}
%    \end{macrocode}
%\end{macro}
% Check again at the end of the preamble.
%    \begin{macrocode}
\AtBeginDocument{%
  \@ifpackageloaded{pdfcprot}\MT@pdfcprot@error\relax
  \@ifpackageloaded{ledmac}\MT@ledmac@setup\relax
%    \end{macrocode}
% We can clean up \cs{MT@setupfont@hook} now.
%    \begin{macrocode}
  \let\MT@setupfont@hook\@empty
  \@ifpackagewith{babel}{spanish}{%
    \g@addto@macro\MT@setupfont@hook{%
      \MT@ifdefined@c\percentsign
        {\let\%\percentsign}\relax}%
  }\relax
  \@ifpackageloaded{csquotes}{%
    \@ifpackagelater{csquotes}{2005/05/11}{%
      \g@addto@macro\MT@setupfont@hook\@disablequotes
    }{%
      \MT@warning@nl{%
        Should you receive warnings about unknown slot\MessageBreak
        numbers, try upgrading the `csquotes' package}%
    }%
  }\relax
  \@ifpackageloaded{hyperref}{%
    \g@addto@macro\MT@setupfont@hook{%
      \chardef\%`\%
      \chardef\#`\#
    }%
%    \end{macrocode}
% We disable \microtype's additions inside \pkg{hyperref}'s \cmd{\pdfstringdef},
% which redefines lots of commands.
%\changes{v1.9}{2005/09/10}{disable \microtype\ setup inside \pkg{hyperref}'s
%                           \cmd{\pdfstringdef} (reported by \thanh)} ^^A private mail, 10/9/05
%    \begin{macrocode}
    \pdfstringdefDisableCommands{%
      \let\pickup@font\MT@orig@pickupfont
%<*beta>
      \let\lsstyle\@empty
      \let\textls\@firstofone
%</beta>
    }%
  }\relax
}
%    \end{macrocode}
%\changes{v1.6}{2005/01/19}{load a font, if none is active}
% We need a font (the \pkg{minimal} class doesn't load one).
%    \begin{macrocode}
\expandafter\ifx\the\font\nullfont\normalfont\fi
%    \end{macrocode}
%
%\subsection{Auxiliary macros}
%
%\begin{macro}{\MT@etex@no}
% Test whether we are using \etex. Cases:
%\begin{itemize}
%  \item[|0|:] not running \etex
%  \item[|1|:] running \etex
%\end{itemize}
%    \begin{macrocode}
\let\MT@etex@no\z@
\ifx\eTeXversion\@undefined \else
  \ifx\eTeXversion\relax \else
    \ifnum\eTeXversion>\z@
      \let\MT@etex@no\@ne
    \fi
  \fi
\fi
%    \end{macrocode}
%\end{macro}
%\changes{v1.4b}{2004/11/21}{optimization: use less \cmd{\csname}s and \cmd{\expandafter}s}
%\begin{macro}{\MT@def@n}
% This is \cmd{\@namedef}.
%    \begin{macrocode}
\def\MT@def@n#1{\expandafter\def\csname #1\endcsname}
%    \end{macrocode}
%\end{macro}
%\begin{macro}{\MT@edef@n}
% Its expanding version.
%    \begin{macrocode}
\def\MT@edef@n#1{\expandafter\edef\csname #1\endcsname}
%    \end{macrocode}
%\end{macro}
%\begin{macro}{\MT@let@nc}
% \cmd{\let} a \cmd{\csname} sequence to a command.
%    \begin{macrocode}
\def\MT@let@nc#1{\expandafter\let\csname #1\endcsname}
%    \end{macrocode}
%\end{macro}
%\begin{macro}{\MT@let@cn}
% \cmd{\let} a command to a \cmd{\csname} sequence.
%    \begin{macrocode}
\def\MT@let@cn#1#2{\expandafter\let\expandafter#1\csname #2\endcsname}
%    \end{macrocode}
%\end{macro}
%\begin{macro}{\MT@let@nn}
% \cmd{\let} a \cmd{\csname} sequence to a \cmd{\csname} sequence.
%    \begin{macrocode}
\def\MT@let@nn#1{\expandafter\MT@let@cn\csname #1\endcsname}
%    \end{macrocode}
%\end{macro}
%\begin{macro}{\MT@exp@string}
% Remove trailing space.
%    \begin{macrocode}
\def\MT@exp@string{\expandafter\string}
%    \end{macrocode}
%\end{macro}
%\begin{macro}{\MT@exp@one@n}
% Expand the second token once and enclose it in braces.
%    \begin{macrocode}
\def\MT@exp@one@n#1#2{\expandafter#1\expandafter{#2}}
%    \end{macrocode}
%\end{macro}
%\begin{macro}{\MT@exp@two@c}
% Expand the next two tokens after \meta{\#1} once.
%    \begin{macrocode}
\def\MT@exp@two@c#1{\expandafter\expandafter\expandafter#1\expandafter}
%    \end{macrocode}
%\end{macro}
%\begin{macro}{\MT@exp@two@n}
%\changes{v1.9}{2005/09/28}{new macros: less \cmd{\expandafter}s}
% Expand the next two tokens after \meta{\#1} once and enclose them in braces.
%    \begin{macrocode}
\def\MT@exp@two@n#1#2#3{\expandafter\expandafter\expandafter
  #1\expandafter\expandafter\expandafter
    {\expandafter#2\expandafter}\expandafter{#3}}
%    \end{macrocode}
%\end{macro}
% You do not wonder why \cmd{\MT@exp@one@c} doesn't exist, do you?
%\begin{macro}{\MT@hop@fi}
%\begin{macro}{\MT@hop@else@fi}
% We need this a couple of lines down to mask \cs{ifcsname}~\dots\ \cs{fi} and
% \cs{ifdefined}~\dots\ \cs{fi}, if \cs{ifcsname} and \cs{ifdefined} aren't
% defined.
%    \begin{macrocode}
\def\MT@hop@fi#1\fi{\fi#1}
\def\MT@hop@else@fi#1\else#2\fi{\fi#1}
%    \end{macrocode}
%\end{macro}
%\end{macro}
%\changes{v1.6}{2005/01/19}{use \etex's \cs{ifcsname} and \cs{ifdefined} if defined}
%\begin{macro}{\MT@ifdefined@c}
%\begin{macro}{\MT@ifdefined@n}
% Wrapper for testing whether command resp. \cmd{\csname} sequence is defined.
%    \begin{macrocode}
\ifcase\MT@etex@no \MT@hop@else@fi{%
%<debug>\MT@dinfo@nl{0}{not running etex}%
  \def\MT@ifdefined@c#1{%
    \ifx#1\@undefined
      \expandafter\@secondoftwo
    \else
      \expandafter\@firstoftwo
    \fi
  }
  \def\MT@ifdefined@n#1{%
    \begingroup\MT@exp@two@c\endgroup
    \ifx\csname #1\endcsname\relax
      \expandafter\@secondoftwo
    \else
      \expandafter\@firstoftwo
    \fi
  }
}\else\MT@hop@fi{%
%    \end{macrocode}
% If we are running \etex, we will use its primitives \cs{ifdefined} and
% \cs{ifcsname}, which decreases memory use substantially.
%    \begin{macrocode}
%<debug>\MT@dinfo@nl{0}{running etex}%
  \def\MT@ifdefined@c#1{%
    \ifdefined#1%
      \expandafter\@firstoftwo
    \else
      \expandafter\@secondoftwo
    \fi
  }
  \def\MT@ifdefined@n#1{%
    \ifcsname#1\endcsname
      \expandafter\@firstoftwo
    \else
      \expandafter\@secondoftwo
    \fi
  }
}\fi
%    \end{macrocode}
%\end{macro}
%\end{macro}
%\begin{macro}{\MT@ifempty}
% Test whether argument is empty.
%\changes{v1.1}{2004/09/15}{bug fix: use category code 12 for the percent character
%                           (reported by Tom Kink)} ^^A <kink\at hia.rwth-aachen.de>
%                                         ^^A in <MID:2qrkp9F134l6lU1@uni-berlin.de>
%    \begin{macrocode}
\begingroup
\catcode`\%=12
\catcode`\&=14
\gdef\MT@ifempty#1{&
  \if %#1%&
    \expandafter\@firstoftwo
  \else
    \expandafter\@secondoftwo
  \fi
}
\endgroup
%    \end{macrocode}
%\end{macro}
%\begin{macro}{\MT@ifnumber}
% Test whether argument is a number [0-9] (using an old trick by Mr. Arseneau).
%    \begin{macrocode}
\def\MT@ifnumber#1{%
  \if!\ifnum9<1#1!\else?\fi
    \expandafter\@firstoftwo
  \else
    \expandafter\@secondoftwo
  \fi
}
%    \end{macrocode}
%\end{macro}
%\begin{macro}{\MT@ifdimen}
% Test whether argument is dimension (or number).
%\changes{v1.4b}{2004/11/22}{don't set \cs{MT@count} globally (save stack problem)}
%    \begin{macrocode}
\def\MT@ifdimen#1{%
  \setbox\z@=\hbox{%
    \MT@count=1#1\relax
    \ifnum\MT@count=\@ne
      \aftergroup\@secondoftwo
    \else
      \aftergroup\@firstoftwo
    \fi}%
}
%    \end{macrocode}
%\end{macro}
%\begin{macro}{\MT@ifgt}
%\begin{macro}{\MT@iflt}
%\begin{macro}{\MT@ifeq}
% Test whether dimensions are smaller, larger or equal.
%    \begin{macrocode}
\def\MT@ifgt#1#2{%
  \ifdim #1\p@ > #2\p@
    \expandafter\@firstoftwo
  \else
    \expandafter\@secondoftwo
  \fi
}
\def\MT@iflt#1#2{%
  \ifdim #1\p@ < #2\p@
    \expandafter\@firstoftwo
  \else
    \expandafter\@secondoftwo
  \fi
}
\def\MT@ifeq#1#2{%
  \ifdim #1\p@ = #2\p@
    \expandafter\@firstoftwo
  \else
    \expandafter\@secondoftwo
  \fi
}
%    \end{macrocode}
%\end{macro}
%\end{macro}
%\end{macro}
%\begin{macro}{\MT@ifstreq}
% Test whether two strings (fully expanded) are equal.
%    \begin{macrocode}
\def\MT@ifstreq#1#2{%
  \edef\x{#1}%
  \edef\y{#2}%
  \ifx\x\y
    \expandafter\@firstoftwo
  \else
    \expandafter\@secondoftwo
  \fi
}
%    \end{macrocode}
%\end{macro}
%\begin{macro}{\MT@xadd}
%\changes{v1.8}{2005/04/17}{simplified}
% Add item to a list.
%    \begin{macrocode}
\def\MT@xadd#1#2{%
  \ifx#1\relax
    \xdef#1{#2}%
  \else
    \xdef#1{#1#2}%
  \fi
}
%    \end{macrocode}
%\end{macro}
%\begin{macro}{\MT@xaddb}
% Add item to the beginning.
%    \begin{macrocode}
\def\MT@xaddb#1#2{%
  \ifx#1\relax
    \xdef#1{#2}%
  \else
    \xdef#1{#2#1}%
  \fi
}
%    \end{macrocode}
%\end{macro}
%\begin{macro}{\MT@map@clist@n}
%\changes{v1.8}{2005/04/17}{new macro: used instead of \cmd{\@for}}
%\begin{macro}{\MT@map@clist@c}
%\begin{macro}{\MT@map@clist@}
%\begin{macro}{\MT@clist@function}
%\begin{macro}{\MT@clist@break}
% Run \meta{\#2} on all elements of the comma list \meta{\#1}. This and the
% following is modelled after \LaTeX3 commands.
%    \begin{macrocode}
\def\MT@map@clist@n#1#2{%
  \ifx\@empty#1\else
    \def\MT@clist@function##1{#2}%
    \expandafter\MT@map@clist@
    \expandafter#1,\@nil,\@nnil,%
  \fi
}
\def\MT@map@clist@c#1{\expandafter\MT@map@clist@n\expandafter{#1}}
\def\MT@map@clist@#1,{%
  \ifx\@nil#1%
    \MT@clist@break
  \else
    \MT@clist@function{#1}%
    \expandafter\MT@map@clist@
  \fi
}
\def\MT@clist@break#1\@nnil,{\fi}
%    \end{macrocode}
%\end{macro}
%\end{macro}
%\end{macro}
%\end{macro}
%\end{macro}
%\begin{macro}{\MT@map@tlist@n}
%\changes{v1.8}{2005/04/16}{new macro: used instead of \cmd{\@tfor}}
%\begin{macro}{\MT@map@tlist@c}
%\begin{macro}{\MT@map@tlist@}
%\begin{macro}{\MT@tlist@break}
% Execute \meta{\#2} on all elements of the token list \meta{\#1}.
% \cs{MT@tlist@break} can be used to jump out of the loop.
%    \begin{macrocode}
\def\MT@map@tlist@n#1#2{%
  \MT@map@tlist@#2#1\@nnil
}
\def\MT@map@tlist@c#1#2{%
  \expandafter\MT@map@tlist@
  \expandafter#2#1\@nnil
}
\def\MT@map@tlist@#1#2{%
  \ifx\@nnil#2\else
    #1{#2}%
    \expandafter\MT@map@tlist@
    \expandafter#1%
  \fi
}
\def\MT@tlist@break#1\@nnil{\fi}
%    \end{macrocode}
%\end{macro}
%\end{macro}
%\end{macro}
%\end{macro}
%\begin{macro}{\ifMT@inlist@}\IndexNewif{MT@inlist@}
%\begin{macro}{\MT@in@clist}
% Test whether item \meta{\#1} is in comma list \meta{\#2}.
%\changes{v1.4b}{2004/11/22}{bug fix: compare with \cmd{\\} instead of \cmd{\relax}
%                          (discovered by Herb Schulz)}  ^^A <herbs\at wideopenwest.com>
%                                      ^^A in <MID:BDBFA967.C01F%herbs@wideopenwest.com>
%    \begin{macrocode}
\newif\ifMT@inlist@
\def\MT@in@clist#1#2{%
  \MT@inlist@false
  \def\x##1#1,##2\@nnil{%
    \ifx\\##2\\\else
      \MT@inlist@true
    \fi
  }%
  \expandafter\x#2,#1,\@nnil
}
%    \end{macrocode}
%\end{macro}
%\end{macro}
%\begin{macro}{\MT@rem@from@list}
% Remove item \meta{\#1} from comma list \meta{\#2}.
%\changes{v1.9}{2005/10/28}{new macro: remove an item from a comma list}
%    \begin{macrocode}
\def\MT@rem@from@list#1#2{%
  \def\x##1#1,##2\@nnil{%
    \ifx\\##2\\\else
      \def\x####1,#1,####2\@nnil{%
        \gdef#2{##1####1}%
      }%
      \x##2,#1,\@nnil
    \fi
  }%
  \expandafter\x#2,#1,\@nnil
}
%    \end{macrocode}
%\end{macro}
%\begin{macro}{\MT@in@tlist}
% Test whether item is in token list.
%    \begin{macrocode}
\def\MT@in@tlist#1#2{%
  \MT@inlist@false
  \def\x{#1}%
  \MT@map@tlist@c#2\MT@in@tlist@
}
\def\MT@in@tlist@#1{%
  \edef\y{#1}%
  \ifx\x\y
    \MT@inlist@true
    \expandafter\MT@tlist@break
  \fi
}
%    \end{macrocode}
%\end{macro}
%\begin{macro}{\MT@in@rlist}
%\changes{v1.8}{2005/04/20}{made recursive}
%\begin{macro}{\MT@in@rlist@}
%\begin{macro}{\MT@in@rlist@@}
%\begin{macro}{\MT@size@name}
% Test whether size \cs{MT@size} is in a list of ranges. Store the name of the
% list in \cs{MT@size@name}
%    \begin{macrocode}
\def\MT@in@rlist#1{%
  \MT@inlist@false
  \MT@map@tlist@c#1\MT@in@rlist@
}
\def\MT@in@rlist@#1{%
  \expandafter\MT@in@rlist@@#1%
}
\def\MT@in@rlist@@#1#2#3{%
  \MT@ifeq{#2}\m@ne{%
    \MT@ifeq{#1}\MT@size
      \MT@inlist@true
      \relax
    }{%
    \MT@iflt\MT@size{#1}\relax{%
      \MT@iflt\MT@size{#2}%
        \MT@inlist@true
        \relax
    }%
  }%
  \ifMT@inlist@
    \def\MT@size@name{#3}%
    \expandafter\MT@tlist@break
  \fi
}
%    \end{macrocode}
%\end{macro}
%\end{macro}
%\end{macro}
%\end{macro}
%\begin{macro}{\MT@loop}
%\changes{v1.2}{2004/09/23}{bug fix: new macro, used instead of \cmd{\loop}}
%\begin{macro}{\MT@iterate}
%\begin{macro}{\MT@repeat}
% This is the same as \LaTeX's \cmd{\loop}, which we mustn't use, since this could
% confuse an outer \cmd{\loop} in the document.
%    \begin{macrocode}
\def\MT@loop#1\MT@repeat{%
  \def\MT@iterate{#1\relax\expandafter\MT@iterate\fi}%
  \MT@iterate
  \let\MT@iterate\relax
}
\let\MT@repeat\fi
%    \end{macrocode}
%\end{macro}
%\end{macro}
%\end{macro}
%\begin{macro}{\MT@while@num}
% Sweetness.
%    \begin{macrocode}
\def\MT@while@num#1#2{\MT@loop #2\ifnum#1\MT@repeat}
%    \end{macrocode}
%\end{macro}
%\changes{v1.4}{2004/11/12}{use one instead of five counters}
%\begin{macro}{\MT@count}
%\begin{macro}{\MT@increment}
% Increment macro \meta{\#1} by one. Saves using up too many counters.
%\changes{v1.7}{2005/03/07}{use \etex's \cmd{\numexpr} if available}
%    \begin{macrocode}
\newcount\MT@count
\ifcase\MT@etex@no
  \def\MT@increment#1{%
    \MT@count=#1\relax
    \advance\MT@count \@ne
    \edef#1{\number\MT@count}%
  }
\else
%    \end{macrocode}
% The \etex\ way is slightly faster.
%    \begin{macrocode}
  \def\MT@increment#1{%
    \edef#1{\number\numexpr #1 + 1\relax}%
  }
\fi
%    \end{macrocode}
%\end{macro}
%\end{macro}
%\begin{macro}{\MT@scale}
%\changes{v1.7}{2005/02/12}{new macro: use \etex's \cs{numexpr} if available}
%\changes{v1.8}{2005/03/30}{bug fix: remove spaces in non-\etex\ variant
%                           (reported by Mark Rossi)} ^^A <mark.rossi\at de.bosch.com>
%                                      ^^A in <d6uhag$qoi$1@ns2.fe.internet.bosch.com>
% Multiply and divide a counter. If we are using \etex, we will use its
% \cmd{\numexpr} primitive. This has the advantage that it is less likely to
% run into arithmetic overflow. The result of the division will be rounded
% instead of truncated. Therefore, we'll get a different (more accurate)
% result in about half of the cases.
%    \begin{macrocode}
\ifcase\MT@etex@no
  \def\MT@scale#1#2#3{%
    \multiply #1 #2\relax
    \ifnum #3 = \z@ \else
      \divide #1 #3\relax
    \fi
  }
\else
  \def\MT@scale#1#2#3{%
    \ifnum #3 = \z@
      #1=\numexpr #1 * #2\relax
    \else
      #1=\numexpr #1 * #2 / #3\relax
    \fi
  }
\fi
%    \end{macrocode}
%\end{macro}
%\begin{macro}{\MT@remove@spaces}
% Remove spaces around \meta{\#1} (\cmd{\KV@@sp@def} is from \pkg{keyval}).
% It will be neutralized in \cs{MT@begin@catcodes}.
%    \begin{macrocode}
\def\MT@remove@spaces#1{\expandafter\KV@@sp@def\expandafter#1\expandafter{#1}}
%    \end{macrocode}
%\end{macro}
%\begin{macro}{\MT@make@string}
%\changes{v1.8}{2005/05/07}{use \cmd{\@onelevel@sanitize}}
% Set the category code of all characters to 12.
%    \begin{macrocode}
\let\MT@make@string\@onelevel@sanitize
%    \end{macrocode}
%\end{macro}
%\changes{v1.7}{2005/03/07}{shorter command names}
%\begin{macro}{\MT@abbr@pr}
%\begin{macro}{\MT@abbr@ex}
%\begin{macro}{\MT@abbr@pr@c}
%\begin{macro}{\MT@abbr@ex@c}
%\begin{macro}{\MT@abbr@pr@inh}
%\begin{macro}{\MT@abbr@ex@inh}
%\begin{macro}{\MT@abbr@nl}
%\begin{macro}{\MT@abbr@sp}
%\begin{macro}{\MT@abbr@sp@c}
%\begin{macro}{\MT@abbr@sp@inh}
%\begin{macro}{\MT@abbr@kn}
%\begin{macro}{\MT@abbr@kn@c}
%\begin{macro}{\MT@abbr@kn@inh}
% Some abbreviations. Thus, we can have short command names but full-length
% log output.
%    \begin{macrocode}
\def\MT@abbr@pr{protrusion}
\def\MT@abbr@ex{expansion}
\def\MT@abbr@pr@c{protrusion codes}
\def\MT@abbr@ex@c{expansion codes}
\def\MT@abbr@pr@inh{protrusion inheritance}
\def\MT@abbr@ex@inh{expansion inheritance}
\def\MT@abbr@nl{noligatures}
%<*beta>
\def\MT@abbr@sp{spacing}
\def\MT@abbr@sp@c{interword spacing codes}
\def\MT@abbr@sp@inh{interword spacing inheritance}
\def\MT@abbr@kn{kerning}
\def\MT@abbr@kn@c{kerning codes}
\def\MT@abbr@kn@inh{kerning inheritance}
%</beta>
%    \end{macrocode}
%\end{macro}
%\end{macro}
%\end{macro}
%\end{macro}
%\end{macro}
%\end{macro}
%\end{macro}
%\end{macro}
%\end{macro}
%\end{macro}
%\end{macro}
%\end{macro}
%\end{macro}
%\begin{macro}{\MT@rbba@protrusion}
%\begin{macro}{\MT@rbba@expansion}
%\begin{macro}{\MT@rbba@spacing}
%\begin{macro}{\MT@rbba@kerning}
% These we also need the other way round.
%    \begin{macrocode}
\def\MT@rbba@protrusion{pr}
\def\MT@rbba@expansion{ex}
%<*beta>
\def\MT@rbba@spacing{sp}
\def\MT@rbba@kerning{kn}
%</beta>
%    \end{macrocode}
%\end{macro}
%\end{macro}
%\end{macro}
%\end{macro}
%
%\subsection{Setting up a font}
%
%\begin{macro}{\MT@setupfont}
% Setting up a font entails checking whether protrusion/expansion is desired for
% the current font (\cs{MT@font@name}), and if so, adjusting \cmd{\lpcode} and
% \cmd{\rpcode} (protrusion) and \cmd{\efcode} (expansion) for each character.
%    \begin{macrocode}
\def\MT@setupfont{%
  \ifx\MT@vinfo\MT@info@nl
    \MT@info{Setting up font `\MT@exp@string\MT@font'}\fi
%    \end{macrocode}
% We might have to disable stuff when used together with adventurous packages.
%    \begin{macrocode}
  \MT@setupfont@hook
%    \end{macrocode}
% The font properties must be extracted from \cs{MT@font@name}, since the current
% value of \cmd{\f@encoding} and friends may be wrong!
%    \begin{macrocode}
  \MT@exp@two@c\MT@split@name\string\MT@font\@nil
%    \end{macrocode}
% Try to find a configuration file for the current font family.
%\changes{v1.2}{2004/09/29}{also search for alias font file}
%    \begin{macrocode}
  \MT@exp@one@n\MT@find@file\MT@family
  \ifx\MT@familyalias\@empty \else
    \MT@exp@one@n\MT@find@file\MT@familyalias\fi
%    \end{macrocode}
%\changes{v1.2}{2004/09/26}{bug fix: call \cmd{\@@enc@update} if necessary}
% We have to make sure that \cmd{\cf@encoding} expands to the correct value (for
% later, in \cs{MT@get@slot}), which isn't the case when \cmd{\selectfont}
% chooses a new encoding (this would be done a second later in
% \cmd{\selectfont}, anyway -- three lines, to be exact).
%    \begin{macrocode}
  \ifx\f@encoding\cf@encoding\else\@@enc@update\fi
%    \end{macrocode}
% Now we can begin setting up the font for all features. The following commands
% are \cmd{\let} to \cmd{\relax} if the respective feature is generally
% disabled.
%
% Protrusion has to be set up first, says Th�nh!
%    \begin{macrocode}
  \MT@protrusion
  \MT@expansion
%    \end{macrocode}
% Interword spacing and kerning.
%    \begin{macrocode}
%<*beta>
  \MT@spacing
  \MT@kerning
%</beta>
%    \end{macrocode}
% Disable ligatures?
%    \begin{macrocode}
  \MT@noligatures
}
%    \end{macrocode}
%\end{macro}
%\begin{macro}{\MT@split@name}
%\changes{v1.7}{2005/02/27}{don't define \cs{MT@encoding} \&c. \cmd{\global}ly}
%\begin{macro}{\MT@encoding}
%\begin{macro}{\MT@family}
%\begin{macro}{\MT@series}
%\begin{macro}{\MT@shape}
%\begin{macro}{\MT@size}
% Split up the font name.
%    \begin{macrocode}
\def\MT@split@name#1/#2/#3/#4/#5\@nil{%
  \def\MT@encoding{#1}%
  \def\MT@family{#2}%
  \def\MT@series{#3}%
  \def\MT@shape{#4}%
  \def\MT@size{#5}%
%    \end{macrocode}
%\begin{macro}{\MT@familyalias}
% Alias family?
%\changes{v1.2}{2004/09/29}{define alias font name as an alternative, not
%                          as a replacement}
%    \begin{macrocode}
  \MT@ifdefined@n{MT@\MT@family @alias}%
    {\MT@let@cn\MT@familyalias{MT@\MT@family @alias}}%
    {\let\MT@familyalias\@empty}%
}
%    \end{macrocode}
%\end{macro}
%\end{macro}
%\end{macro}
%\end{macro}
%\end{macro}
%\end{macro}
%\end{macro}
%\begin{macro}{\ifMT@do}\IndexNewif{MT@do}
%\begin{macro}{\MT@feat}
%\begin{macro}{\MT@maybe@do}
% We check all features of the current font against the lists of the currently
% active font set, and set \cs{ifMT@do} accordingly.
%\changes{v1.2}{2004/09/29}{also check for alias font name}
%\changes{v1.9}{2005/09/28}{redone}
%    \begin{macrocode}
\newif\ifMT@do
\def\MT@maybe@do#1{%
%    \end{macrocode}
% (but only if the feature isn't globally set to false)
%    \begin{macrocode}
  \expandafter\csname ifMT@\csname MT@abbr@#1\endcsname\endcsname
%    \end{macrocode}
% Begin with setting micro-typography to true for this font. The
% \cs{MT@checklist@...} tests will set it to false if the property is not in
% the list. The first non-empty list that does not contain a match will stop us
% (except for |font|).
%    \begin{macrocode}
    \MT@dotrue
    \MT@map@clist@n{font,encoding,family,series,shape,size}{%
      \MT@ifdefined@n{MT@checklist@##1}%
        {\csname MT@checklist@##1\endcsname}%
        {\MT@checklist@{##1}}%
      {#1}%
    }%
  \else
    \MT@dofalse
  \fi
  \ifMT@do
%    \end{macrocode}
% \cs{MT@feat} stores the current feature.
%    \begin{macrocode}
    \def\MT@feat{#1}%
    \csname MT@set@#1@codes\endcsname
  \else
    \MT@vinfo{... No \@nameuse{MT@abbr@#1}}%
  \fi
}
%    \end{macrocode}
%\end{macro}
%\end{macro}
%\end{macro}
%\begin{macro}{\MT@checklist@}
% The generic test.
%    \begin{macrocode}
\def\MT@checklist@#1#2{%
  \edef\@tempa{\csname MT@#2@setname\endcsname}%
  \MT@ifdefined@n{MT@#2list@#1@\@tempa}{%
%    \end{macrocode}
% Begin a \cmd{\expandafter} orgy to test whether the font characteristic is in
% the list.
%    \begin{macrocode}
    \expandafter\expandafter\expandafter
      \MT@in@clist\expandafter\expandafter\expandafter
        {\csname MT@#1\expandafter\endcsname\expandafter}%
         \csname MT@#2list@#1@\@tempa\endcsname
    \ifMT@inlist@
%<debug>\MT@dinfo@nl{1}{\@nameuse{MT@abbr@#2}: #1 `\@nameuse{MT@#1}' in list}%
      \MT@dotrue
    \else
%<debug>\MT@dinfo@nl{1}{\@nameuse{MT@abbr@#2}: #1 `\@nameuse{MT@#1}' not in list}%
      \MT@dofalse
      \expandafter\MT@clist@break
    \fi
  }{%
%    \end{macrocode}
% If no limitations have been specified, i.e. the list for a font characteristic
% has not been defined at all, the font should be expanded resp. protruded.
%    \begin{macrocode}
%<debug>\MT@dinfo@nl{1}{\@nameuse{MT@abbr@#2}: #1 list empty}%
  }%
}
%    \end{macrocode}
%\end{macro}
%\begin{macro}{\MT@checklist@font}
% If the font matches, we skip the rest of the test.
%    \begin{macrocode}
\def\MT@checklist@font#1{%
  \edef\@tempa{\csname MT@#1@setname\endcsname}%
  \MT@ifdefined@n{MT@#1list@font@\@tempa}{%
    \MT@exp@two@n\MT@in@clist
        \MT@font{\csname MT@#1list@font@\@tempa\endcsname}%
    \ifMT@inlist@
%<debug>\MT@dinfo@nl{1}{\@nameuse{MT@abbr@#1}: font `\MT@font' in list}%
      \expandafter\MT@clist@break
    \else
%<debug>\MT@dinfo@nl{1}{\@nameuse{MT@abbr@#1}: font `\MT@font' not in list}%
      \MT@dofalse
    \fi
  }{%
%<debug>\MT@dinfo@nl{1}{\@nameuse{MT@abbr@#1}: font list empty}%
  }%
}
%    \end{macrocode}
%\end{macro}
%\begin{macro}{\MT@checklist@family}
% Also test for the alias font, if the original font is not in the list.
%\changes{v1.4b}{2004/11/22}{bug fix: don't try alias family name if encoding failed}
%\changes{v1.9}{2005/07/12}{bug fix: add two missing \cmd{\expandafter}s}
%    \begin{macrocode}
\def\MT@checklist@family#1{%
  \edef\@tempa{\csname MT@#1@setname\endcsname}%
  \MT@ifdefined@n{MT@#1list@family@\@tempa}{%
    \MT@exp@two@n\MT@in@clist
        \MT@family{\csname MT@#1list@family@\@tempa\endcsname}%
    \ifMT@inlist@
%<debug>\MT@dinfo@nl{1}{\@nameuse{MT@abbr@#1}: family `\@nameuse{MT@family}' in list}%
      \MT@dotrue
    \else
%<debug>\MT@dinfo@nl{1}{\@nameuse{MT@abbr@#1}: family `\@nameuse{MT@family}' not in list}%
      \MT@dofalse
      \ifx\MT@familyalias\@empty \else
        \MT@exp@two@n\MT@in@clist
            \MT@familyalias{\csname MT@#1list@family@\@tempa\endcsname}%
        \ifMT@inlist@
%<debug>  \MT@dinfo@nl{1}{\@nameuse{MT@abbr@#1}: alias `\MT@familyalias' in list}%
          \MT@dotrue
%<debug>\else\MT@dinfo@nl{1}{\@nameuse{MT@abbr@#1}: alias `\MT@familyalias' not in list}%
        \fi
      \fi
    \fi
    \ifMT@do \else
      \expandafter\MT@clist@break
    \fi
  }{%
%<debug>\MT@dinfo@nl{1}{\@nameuse{MT@abbr@#1}: family list empty}%
  }%
}
%    \end{macrocode}
%\end{macro}
%\begin{macro}{\MT@checklist@size}
% Test whether font size is in list of size ranges.
%    \begin{macrocode}
\def\MT@checklist@size#1{%
  \edef\@tempa{\csname MT@#1@setname\endcsname}%
  \MT@ifdefined@n{MT@#1list@size@\@tempa}{%
    \expandafter\MT@in@rlist
        \csname MT@#1list@size@\@tempa\endcsname
    \ifMT@inlist@
%<debug>\MT@dinfo@nl{1}{\@nameuse{MT@abbr@#1}: size `\MT@size' in list}%
      \MT@dotrue
    \else
%<debug>\MT@dinfo@nl{1}{\@nameuse{MT@abbr@#1}: size `\MT@size' not in list}%
      \MT@dofalse
      \expandafter\MT@clist@break
    \fi
  }{%
%<debug>\MT@dinfo@nl{1}{\@nameuse{MT@abbr@#1}: size list empty}%
  }%
}
%    \end{macrocode}
%\end{macro}
%
%\subsubsection{Protrusion}
%
%\begin{macro}{\MT@protrusion}
% Set up for protrusion?
%    \begin{macrocode}
\def\MT@protrusion{\MT@maybe@do{pr}}
%    \end{macrocode}
%\end{macro}
%\begin{macro}{\MT@set@pr@codes}
% This macro is called by \cs{MT@setupfont}, and does all the work for setting
% up a font for protrusion.
%\changes{v1.5}{2004/12/10}{adjust protrusion factors before setting the inheriting
%                           characters}
%\changes{v1.6}{2004/12/18}{introduce \texttt{factor} option}
%    \begin{macrocode}
\def\MT@set@pr@codes{%
  \MT@reset@pr@codes
%    \end{macrocode}
% Check whether and if, which list should be applied to the current font.
%    \begin{macrocode}
  \MT@if@list@exists{%
    \MT@get@dimen@six
    \MT@get@opt
%    \end{macrocode}
% Get the name of the inheritance list and parse it.
%    \begin{macrocode}
    \MT@get@inh@list
%    \end{macrocode}
% Load additional lists?
%    \begin{macrocode}
    \MT@load@list{\MT@pr@c@name}%
%    \end{macrocode}
% Load the main list.
%    \begin{macrocode}
    \edef\MT@curr@list@name{protrusion list `\MT@pr@c@name'}%
    \MT@let@cn\@tempc{MT@pr@c@\MT@pr@c@name}%
    \expandafter\MT@pr@do\@tempc,\relax,%
  }\relax
}
%    \end{macrocode}
%\end{macro}
%\begin{macro}{\MT@set@all@pr}
% Set all protrusion codes of the font.
%    \begin{macrocode}
\def\MT@set@all@pr#1#2{%
%<debug>\MT@dinfo@nl{3}{-- lp/rp: setting all to \number#1/\number#2}%
  \@tempcnta=\z@
  \MT@while@num{\@tempcnta < \@cclvi}{%
    \lpcode\MT@font\@tempcnta=#1\relax
    \rpcode\MT@font\@tempcnta=#2\relax
    \advance\@tempcnta \@ne
  }%
}
%    \end{macrocode}
%\end{macro}
%\begin{macro}{\MT@reset@pr@codes}
%\begin{macro}{\MT@reset@pr@codes@}
% All protrusion codes are zero for new fonts. However, if we have to reload
% the font due to different contexts, we have to reset them.
% This command will be changed by \cs{microtypecontext} if necessary.
%    \begin{macrocode}
\def\MT@reset@pr@codes@{\MT@set@all@pr\z@\z@}
\let\MT@reset@pr@codes\relax
%    \end{macrocode}
%\end{macro}
%\end{macro}
%\begin{macro}{\MT@gobble@settings}
%\begin{macro}{\MT@dimen@six}
%\begin{macro}{\MT@get@dimen@six}
%\changes{v1.8}{2005/05/03}{new macro: test whether \cmd{\fontdimen}6 is defined}
% If \cmd{\fontdimen}\,6 is zero, character protrusion won't work, and we can
% skip the settings (for example, the \pkg{dsfont} fonts don't specify this
% dimension; this is probably a bug).
%    \begin{macrocode}
\def\MT@get@dimen@six{%
  \ifnum\fontdimen6\MT@font=\z@
    \MT@warning@nl{%
      Font `\MT@exp@string\MT@font' does not specify its\MessageBreak
      \@backslashchar fontdimen 6 (width of an `em')! Therefore,\MessageBreak
      \@nameuse{MT@abbr@\MT@feat} will not work with this font}%
    \expandafter\MT@gobble@settings
  \else
    \edef\MT@dimen@six{\number\fontdimen6\MT@font}%
  \fi
}
\def\MT@gobble@settings#1\@tempc,\relax,{}
%    \end{macrocode}
%\end{macro}
%\end{macro}
%\end{macro}
%\begin{macro}{\MT@pr@do}
% Split up the values and set \cmd{\lpcode} and \cmd{\rpcode}.
%    \begin{macrocode}
\def\MT@pr@do#1,{%
  \ifx\relax#1\@empty\else
    \MT@pr@split #1==\relax
    \expandafter\MT@pr@do
  \fi
}
%    \end{macrocode}
%\end{macro}
%\begin{macro}{\MT@pr@split}
%\changes{v1.1}{2004/09/13}{bug fix: allow zero and negative values}
%\changes{v1.8}{2005/05/25}{get character width once only}
% The \pkg{keyval} package would remove spaces here, which we needn't do since
% \cs{SetProtrusion} ignores spaces in the protrusion list anyway.
%    \begin{macrocode}
\def\MT@pr@split#1=#2=#3\relax{%
  \def\@tempa{#1}%
  \ifx\@tempa\@empty \else
    \MT@get@slot
    \ifnum\MT@char > \m@ne
      \MT@get@char@unit
      \MT@pr@split@val#2\relax
    \fi
  \fi
}
%    \end{macrocode}
%\end{macro}
%\begin{macro}{\MT@pr@split@val}
%    \begin{macrocode}
\def\MT@pr@split@val#1,#2\relax{%
  \def\@tempb{#1}%
  \MT@ifempty\@tempb\relax{%
    \MT@scale@to@em
    \lpcode\MT@font\MT@char=\@tempcntb
%<debug>\MT@dinfo@nl{4}{;;; lp (\MT@char): \number\lpcode\MT@font\MT@char}%
  }%
  \def\@tempb{#2}%
  \MT@ifempty\@tempb\relax{%
    \MT@scale@to@em
    \rpcode\MT@font\MT@char=\@tempcntb
%<debug>\MT@dinfo@nl{4}{;;; rp (\MT@char): \number\rpcode\MT@font\MT@char}%
  }%
%    \end{macrocode}
% Now we can set the values for the inheriting characters. Their slot numbers
% are saved in the macro |\MT@inh@|\meta{list name}|@|\meta{slot number}|@|.
%    \begin{macrocode}
  \MT@ifdefined@c\MT@pr@inh@name{%
    \MT@ifdefined@n{MT@inh@\MT@pr@inh@name @\MT@char @}{%
      \expandafter\MT@map@tlist@c
        \csname MT@inh@\MT@pr@inh@name @\MT@char @\endcsname
        \MT@set@pr@heirs
    }\relax
  }\relax
}
%    \end{macrocode}
%\end{macro}
%\begin{macro}{\MT@scale@to@em}
% Since \pdftex\ version 0.14h, we have to adjust the protrusion factors (\ie,
% convert numbers from thousandths of character width to thousandths of an
% \emph{em} of the font). We have to do this \emph{before} setting the
% inheriting characters, so that the latter inherit the absolute value, not the
% relative one if they have a differing width (\eg\ the `ff' ligature).
%\changes{v1.5}{2004/12/10}{don't use \cmd{\lpcode} and \cmd{\rpcode} for the
%                           calculation}
% Unlike \file{protcode.tex} and \pkg{pdfcprot}, we do not calculate with
% \cs{lpcode} resp. \cs{rpcode}, since this would disallow protrusion factors
% larger than the character width (since \cs{[lr]pcode}'s limit is 1000). Now,
% the maximum protrusion is 1em of the font.
%
% The unit is in \cs{MT@count}, the desired factor in \cmd{\@tempb}, and the
% result will be returned in \cmd{\@tempcntb}.
%    \begin{macrocode}
\ifnum\MT@pdftex@no > \tw@
  \def\MT@scale@to@em{%
    \@tempcntb=\MT@count\relax
%    \end{macrocode}
% For really huge fonts (100pt or so), an arithmetic overflow could occur with
% vanilla \TeX. Using \etex, this can't happen, since the intermediate value
% is 64 bit, which could only be reached with a character width larger than
% \cs{maxdimen}.
%    \begin{macrocode}
    \MT@scale\@tempcntb \@tempb \MT@dimen@six
    \ifnum\@tempcntb=\z@ \else
      \MT@scale@factor
    \fi
  }
%    \end{macrocode}
%\end{macro}
%\begin{macro}{\MT@get@charwd}
% Get the width of the character. When using \etex, we can employ
% \cmd{\fontcharwd} instead of building scratch boxes.
%\changes{v1.6}{2005/01/19}{use \etex's \cmd{\fontcharwd}, if available}
%\changes{v1.8}{2005/05/25}{warning for missing (resp. zero-width) characters}
%    \begin{macrocode}
  \ifcase\MT@etex@no
    \def\MT@get@charwd{%
      \setbox\z@=\hbox{\MT@font \char\MT@char}%
      \MT@count=\wd\z@\relax
      \ifnum\MT@count=\z@ \MT@warn@missing@char \fi
    }
  \else
    \def\MT@get@charwd{%
      \MT@count=\number\fontcharwd\MT@font\MT@char\relax
      \ifnum\MT@count=\z@ \MT@warn@missing@char \fi
    }
  \fi
%    \end{macrocode}
%\end{macro}
% No adjustment with versions 0.14f and 0.14g.
%    \begin{macrocode}
\else
  \def\MT@scale@to@em{%
    \MT@count=\@tempb\relax
    \ifnum\MT@count=\z@ \else
      \MT@scale@factor
    \fi
  }
%    \end{macrocode}
% We need this in \cs{MT@warn@code@too@large} (neutralized).
%    \begin{macrocode}
  \def\MT@get@charwd{\MT@count=\MT@dimen@six}
\fi
%    \end{macrocode}
%\begin{macro}{\MT@get@font@dimen}
% For the |space| unit.
%    \begin{macrocode}
\def\MT@get@font@dimen#1{%
  \MT@count=\number\fontdimen#1\MT@font
}
%    \end{macrocode}
%\end{macro}
%\begin{macro}{\MT@warn@missing@char}
% Warning for missing characters, or characters with zero width.
%    \begin{macrocode}
\ifcase\MT@etex@no \MT@hop@else@fi{%
  \def\MT@warn@missing@char{%
    \MT@warning@nl{%
      Character `\the\mt@toks' has a width of 0pt\MessageBreak
      (it's probably missing) in font `\MT@exp@string\MT@font'.\MessageBreak
      It cannot be protruded}%
  }
}\else\MT@hop@fi{%
  \def\MT@warn@missing@char{%
    \MT@warning@nl{Character `\the\mt@toks'
      \iffontchar\MT@font\MT@char has a width of 0pt \else is missing \fi
      in font\MessageBreak `\MT@exp@string\MT@font'. It cannot be protruded}%
  }
}\fi
%    \end{macrocode}
%\end{macro}
%\begin{macro}{\MT@scale@factor}
%\changes{v1.5}{2004/12/10}{warning for factors outside limits}
%\changes{v1.9}{2005/09/28}{generalized}
% Furthermore, we might have to multiply with a factor.
%    \begin{macrocode}
\def\MT@scale@factor{%
  \ifnum\csname MT@\MT@feat @factor@\endcsname=\@m \else
    \expandafter\MT@scale\expandafter
      \@tempcntb \csname MT@\MT@feat @factor@\endcsname \@m
  \fi
  \ifnum\@tempcntb > \csname MT@\MT@feat @max\endcsname\relax
    \@tempcnta=\csname MT@\MT@feat @max\endcsname
    \MT@warn@code@too@large
  \else
    \ifnum\@tempcntb<\csname MT@\MT@feat @min\endcsname\relax
      \@tempcnta=\csname MT@\MT@feat @min\endcsname
      \MT@warn@code@too@large
    \fi
  \fi
}
%    \end{macrocode}
%\end{macro}
%\begin{macro}{\MT@warn@code@too@large}
% Type out a warning if a chosen protrusion factor is too large after the
% conversion. As a special service, we also type out the maximum amount that
% may be specified in the configuration file.
%\changes{v1.7}{2005/02/17}{new macro: type out maximum protrusion factor}
%    \begin{macrocode}
\def\MT@warn@code@too@large{%
  \ifnum\csname MT@\MT@feat @factor@\endcsname=\@m \else
    \expandafter\MT@scale\expandafter\@tempcnta\expandafter\@m
        \csname MT@\MT@feat @factor@\endcsname
  \fi
  \MT@scale\@tempcnta \MT@dimen@six \MT@count
  \MT@warning@nl{The \@nameuse{MT@abbr@\MT@feat} code \@tempb\space
    is too large for character\MessageBreak
    `\the\mt@toks' in \MT@curr@list@name.\MessageBreak
    Setting it to the maximum of \number\@tempcnta}%
  \@tempcntb=\@tempcnta
}
%    \end{macrocode}
%\end{macro}
%\begin{macro}{\MT@get@opt}
% The optional argument to \cs{SetProtrusion}, \cs{SetExtraSpacing} and
% \cs{SetExtraKerning} (\cs{SetExpansion} is being dealt with in
% \cs{MT@get@ex@opt}).
%    \begin{macrocode}
\def\MT@get@opt{%
%    \end{macrocode}
%\begin{macro}{\MT@pr@factor@}
%\begin{macro}{\MT@sp@factor@}
%\begin{macro}{\MT@kn@factor@}
% Apply a factor?
%    \begin{macrocode}
  \MT@ifdefined@n{MT@\MT@feat @c@\csname MT@\MT@feat @c@name\endcsname @factor}{%
    \MT@let@nn{MT@\MT@feat @factor@}
        {MT@\MT@feat @c@\csname MT@\MT@feat @c@name\endcsname @factor}%
    \MT@vinfo{... : Multiplying \@nameuse{MT@abbr@\MT@feat} codes by
                    \number\csname MT@\MT@feat @factor@\endcsname/1000}%
  }{%
    \MT@let@nn{MT@\MT@feat @factor@}{MT@\MT@feat @factor}%
  }%
%    \end{macrocode}
%\end{macro}
%\end{macro}
%\end{macro}
%\begin{macro}{\MT@pr@unit@}
%\begin{macro}{\MT@sp@unit@}
%\begin{macro}{\MT@kn@unit@}
% The |unit| can only be evaluated here, since it might be font-specific.
% If it's \cmd{\@empty}, it's relative to character widths, if it's -1, relative
% to space dimensions.
%    \begin{macrocode}
  \MT@ifdefined@n{MT@\MT@feat @c@\csname MT@\MT@feat @c@name\endcsname @unit}{%
    \MT@let@nn{MT@\MT@feat @unit@}%
        {MT@\MT@feat @c@\csname MT@\MT@feat @c@name\endcsname @unit}%
    \expandafter\ifx\csname MT@\MT@feat @unit@\endcsname\@empty
      \MT@vinfo{... : Setting \@nameuse{MT@abbr@\MT@feat} codes
                      relative to character widths}%
    \else
      \expandafter\ifx\csname MT@\MT@feat @unit@\endcsname\m@ne
        \MT@vinfo{... : Setting \@nameuse{MT@abbr@\MT@feat} codes
                        relative to width of space}%
      \fi
    \fi
  }{%
    \MT@let@nn{MT@\MT@feat @unit@}{MT@\MT@feat @unit}%
  }%
%    \end{macrocode}
%\end{macro}
%\end{macro}
%\end{macro}
%\begin{macro}{\MT@get@space@unit}
%\begin{macro}{\MT@get@char@unit}
% The codes are either relative to character widths, or to a fixed width.
% For spacing and kerning lists, they may also be relative to the width of
% the interword glue.
% Only the setting from the top list will be taken into account.
%    \begin{macrocode}
  \let\MT@get@char@unit\relax
  \let\MT@get@space@unit\@gobble
  \expandafter\ifx\csname MT@\MT@feat @unit@\endcsname\@empty
    \let\MT@get@char@unit\MT@get@charwd
  \else
    \expandafter\ifx\csname MT@\MT@feat @unit@\endcsname\m@ne
      \let\MT@get@space@unit\MT@get@font@dimen
    \else
      \expandafter\MT@get@unit\csname MT@\MT@feat @unit@\endcsname
    \fi
  \fi
%    \end{macrocode}
%\end{macro}
%\end{macro}
% Preset all characters?
%\changes{v1.9}{2005/10/08}{new key `\texttt{preset}' to set all characters to
%                           the specified value before loading the lists}
%    \begin{macrocode}
  \MT@ifdefined@n{MT@\MT@feat @c@\csname MT@\MT@feat @c@name\endcsname @preset}{%
    \csname MT@preset@\MT@feat\endcsname
  }\relax
}
%    \end{macrocode}
%\end{macro}
%\begin{macro}{\MT@get@unit}
%\changes{v1.8}{2005/04/14}{new macro: get unit for codes}
%\begin{macro}{\MT@get@unit@}
% If |unit| contains an |em| or |ex|, we use the corresponding \cmd{\fontdimen}
% to obtain the real size. Simply converting the |em| into points might give a
% wrong result, since the font probably isn't set up yet, so that these
% dimensions haven't been updated, either.
%    \begin{macrocode}
\def\MT@get@unit#1{%
  \expandafter\MT@get@unit@#1 e!\@nil
  \ifx\x\@empty\else\let#1\x\fi
  \@defaultunits\@tempdima#1 pt\relax\@nnil
  \ifdim\@tempdima=\z@
    \MT@warning@nl{%
      Cannot set \@nameuse{MT@abbr@\MT@feat} factors relative to zero\MessageBreak
      width. Setting factors of list `\@nameuse{MT@\MT@feat @c@name}'\MessageBreak
      relative to character widths instead}%
    \let#1\@empty
    \let\MT@get@char@unit\MT@get@charwd
  \else
    \MT@vinfo{... : Setting \@nameuse{MT@abbr@\MT@feat} factors relative
                    to \the\@tempdima}%
    \MT@count=\number\@tempdima\relax
  \fi
}
\def\MT@get@unit@#1e#2#3\@nil{%
  \ifx\\#3\\\let\x\@empty \else
    \if m#2%
      \edef\x{#1\fontdimen6\MT@font}%
    \else
      \if x#2%
        \edef\x{#1\fontdimen5\MT@font}%
      \fi
    \fi
  \fi
}
%    \end{macrocode}
%\end{macro}
%\end{macro}
%\begin{macro}{\MT@set@pr@heirs}
% Set the inheriting characters.
%    \begin{macrocode}
\def\MT@set@pr@heirs#1{%
  \lpcode\MT@font#1=\lpcode\MT@font\MT@char
  \rpcode\MT@font#1=\rpcode\MT@font\MT@char
%<*debug>
  \MT@dinfo@nl{2}{-- heir of \MT@char: #1}%
  \MT@dinfo@nl{4}{;;; lp/rp (#1): \number\lpcode\MT@font\MT@char/%
                                  \number\rpcode\MT@font\MT@char}%
%</debug>
}
%    \end{macrocode}
%\end{macro}
%\begin{macro}{\MT@preset@pr}
%\begin{macro}{\MT@preset@pr@}
%    \begin{macrocode}
\def\MT@preset@pr{%
  \expandafter\expandafter\expandafter\MT@preset@pr@
    \csname MT@pr@c@\MT@pr@c@name @preset\endcsname\@nil
}
\def\MT@preset@pr@#1,#2\@nil{%
  \ifx\MT@pr@unit@\@empty
    \MT@warning@nl{%
      Cannot preset characters relative to their widths\MessageBreak
      for protrusion list `\MT@pr@c@name'. Presetting them\MessageBreak
      relative to 1em instead}%
    \let\MT@preset@aux\MT@preset@aux@factor
  \else
    \let\MT@preset@aux\MT@preset@aux@space
  \fi
  \MT@preset@aux{#1}\@tempa
  \MT@preset@aux{#2}\@tempb
  \MT@set@all@pr\@tempa\@tempb
}
%    \end{macrocode}
%\end{macro}
%\end{macro}
%\begin{macro}{\MT@preset@aux}
%\begin{macro}{\MT@preset@aux@factor}
%\begin{macro}{\MT@preset@aux@space}
% Auxiliary macro for presetting. Store value \meta{\#1} in macro \meta{\#2}.
%    \begin{macrocode}
\def\MT@preset@aux@factor#1#2{%
  \@tempcntb=#1\relax
  \MT@scale@factor
  \edef#2{\number\@tempcntb}%
}
\def\MT@preset@aux@space#1#2{%
  \def\@tempb{#1}%
  \MT@get@space@unit\tw@
  \MT@scale@to@em
  \edef#2{\number\@tempcntb}%
}
%    \end{macrocode}
%\end{macro}
%\end{macro}
%\end{macro}
%
%\subsubsection{Expansion}
%
%\begin{macro}{\MT@expansion}
% Set up for expansion?
%    \begin{macrocode}
\def\MT@expansion{\MT@maybe@do{ex}}
%    \end{macrocode}
%\end{macro}
%\begin{macro}{\MT@set@ex@codes@s}
% Setting up font expansion is a bit different because of the \opt{selected} option.
% There are two versions of this macro.
%
% If \opt{selected}|=true|, we only apply font expansion to those fonts for which a
% list has been declared (\ie, like for protrusion).
%    \begin{macrocode}
\def\MT@set@ex@codes@s{%
  \MT@if@list@exists{%
    \MT@get@ex@opt
    \MT@reset@ef@codes
    \MT@get@inh@list
    \MT@load@list{\MT@ex@c@name}%
    \edef\MT@curr@list@name{expansion list `\MT@ex@c@name'}%
    \MT@let@cn\@tempc{MT@ex@c@\MT@ex@c@name}%
    \expandafter\MT@ex@do\@tempc,\relax,%
    \pdffontexpand\MT@font \MT@stretch@ \MT@shrink@ \MT@step@ \MT@auto@\relax
  }\relax
}
%    \end{macrocode}
%\end{macro}
%
%\begin{macro}{\MT@set@ex@codes@n}
% If, on the other hand, all characters should be expanded by the same amount,
% we only take the first optional argument to \cs{SetExpansion} into account.
%\begin{macro}{\ifMT@nonselected}\IndexNewif{MT@nonselected}
% We need this boolean in \cs{MT@if@list@exists} so that no warning for missing
% lists will be issued.
%    \begin{macrocode}
\newif\ifMT@nonselected
%    \end{macrocode}
%\end{macro}
%    \begin{macrocode}
\def\MT@set@ex@codes@n{%
  \MT@nonselectedtrue
  \MT@if@list@exists
    \MT@get@ex@opt
  {%
    \let\MT@stretch@\MT@stretch
    \let\MT@shrink@\MT@shrink
    \let\MT@step@\MT@step
    \let\MT@auto@\MT@auto
    \let\MT@ex@factor@\MT@ex@factor
  }%
  \MT@reset@ef@codes
  \pdffontexpand\MT@font \MT@stretch@ \MT@shrink@ \MT@step@ \MT@auto@\relax
  \MT@nonselectedfalse
}
%    \end{macrocode}
%\end{macro}
%\begin{macro}{\MT@set@ex@codes}
%\changes{v1.5}{2004/12/02}{allow non-selected font expansion}
%\changes{v1.6}{2004/12/26}{introduce \texttt{factor} option}
%\changes{v1.7}{2005/02/06}{two versions of this macro}
% Default is non-selected. It can be changed in the package options.
%    \begin{macrocode}
\let\MT@set@ex@codes\MT@set@ex@codes@n
%    \end{macrocode}
%\end{macro}
%\begin{macro}{\MT@set@all@ex}
%\begin{macro}{\MT@reset@ef@codes@}
% At first, all expansion factors for the characters will be set to 1000
% (respectively the |factor| of this font).
%    \begin{macrocode}
\def\MT@set@all@ex#1{%
%<debug>\MT@dinfo@nl{3}{-- ex: setting all to \number#1}%
  \@tempcnta=\z@
  \MT@while@num{\@tempcnta < \@cclvi}{%
    \efcode\MT@font\@tempcnta=#1\relax
    \advance\@tempcnta \@ne
  }%
}
\def\MT@reset@ef@codes@{\MT@set@all@ex\MT@ex@factor@}
%    \end{macrocode}
%\end{macro}
%\end{macro}
%\begin{macro}{\MT@reset@ef@codes}
% However, this is only necessary for versions prior to~1.20.
%\changes{v1.6a}{2005/01/30}{only reset \cmd{\efcode}s for older \pdftex\ versions}
%    \begin{macrocode}
\ifnum\MT@pdftex@no < 4
  \let\MT@reset@ef@codes\MT@reset@ef@codes@
\else
  \def\MT@reset@ef@codes{%
    \ifnum\MT@ex@factor@=\@m \else
      \MT@reset@ef@codes@
    \fi
  }
\fi
%    \end{macrocode}
%\end{macro}
%\begin{macro}{\MT@ex@do}
% There's only one number per character.
%    \begin{macrocode}
\def\MT@ex@do#1,{%
  \ifx\relax#1\@empty \else
    \MT@ex@split #1==\relax
    \expandafter\MT@ex@do
  \fi
}
%    \end{macrocode}
%\end{macro}
%\begin{macro}{\MT@ex@split}
%    \begin{macrocode}
\def\MT@ex@split#1=#2=#3\relax{%
  \def\@tempa{#1}%
  \ifx\@tempa\@empty \else
    \MT@get@slot
    \ifnum\MT@char > \m@ne
      \@tempcntb=#2\relax
%    \end{macrocode}
% Take an optional factor into account.
%    \begin{macrocode}
      \ifnum\MT@ex@factor@=\@m \else
        \MT@scale\@tempcntb \MT@ex@factor@ \@m
      \fi
      \ifnum\@tempcntb > \MT@ex@max
        \MT@warn@ex@too@large\MT@ex@max
      \else
        \ifnum\@tempcntb < \MT@ex@min
          \MT@warn@ex@too@large\MT@ex@min
        \fi
      \fi
      \efcode\MT@font\MT@char=\@tempcntb
%<debug> \MT@dinfo@nl{4}{::: ef (\MT@char): \number\efcode\MT@font\MT@char}%
%    \end{macrocode}
% Heirs, heirs, I love thy heirs.
%    \begin{macrocode}
      \MT@ifdefined@c\MT@ex@inh@name{%
        \MT@ifdefined@n{MT@inh@\MT@ex@inh@name @\MT@char @}{%
          \expandafter\MT@map@tlist@c
            \csname MT@inh@\MT@ex@inh@name @\MT@char @\endcsname
            \MT@set@ex@heirs
        }\relax
      }\relax
    \fi
  \fi
}
%    \end{macrocode}
%\end{macro}
%\begin{macro}{\MT@warn@ex@too@large}
%    \begin{macrocode}
\def\MT@warn@ex@too@large#1{%
  \MT@warning@nl{Expansion factor
    \number\@tempcntb\space too large for character\MessageBreak
    `\the\mt@toks' in \MT@curr@list@name.\MessageBreak
    Setting it to the maximum of \number#1}%
  \@tempcntb=#1\relax
}
%    \end{macrocode}
%\end{macro}
%\begin{macro}{\MT@get@ex@opt}
%\begin{macro}{\MT@ex@factor@}
%\begin{macro}{\MT@stretch@}
%\begin{macro}{\MT@shrink@}
%\begin{macro}{\MT@step@}
%\begin{macro}{\MT@auto@}
% Apply different values to this font?
%    \begin{macrocode}
\def\MT@get@ex@opt{%
  \MT@ifdefined@n{MT@ex@c@\MT@ex@c@name @factor}{%
    \MT@let@cn\MT@ex@factor@{MT@ex@c@\MT@ex@c@name @factor}%
    \MT@vinfo{... : Multiplying expansion factors by \number\MT@ex@factor@/1000}%
  }{%
    \let\MT@ex@factor@\MT@ex@factor
  }%
  \MT@ifdefined@n{MT@ex@c@\MT@ex@c@name @stretch}{%
    \MT@let@cn\MT@stretch@{MT@ex@c@\MT@ex@c@name @stretch}%
    \MT@vinfo{... : Setting stretch limit to \number\MT@stretch@}%
  }{%
    \let\MT@stretch@\MT@stretch
  }%
  \MT@ifdefined@n{MT@ex@c@\MT@ex@c@name @shrink}{%
    \MT@let@cn\MT@shrink@{MT@ex@c@\MT@ex@c@name @shrink}%
    \MT@vinfo{... : Setting shrink limit to \number\MT@shrink@}%
  }{%
    \let\MT@shrink@\MT@shrink
  }%
  \MT@ifdefined@n{MT@ex@c@\MT@ex@c@name @step}{%
    \MT@let@cn\MT@step@{MT@ex@c@\MT@ex@c@name @step}%
    \MT@vinfo{... : Setting expansion step to \number\MT@step@}%
  }{%
    \let\MT@step@\MT@step
  }%
  \MT@ifdefined@n{MT@ex@c@\MT@ex@c@name @auto}{%
    \MT@let@cn\MT@auto@{MT@ex@c@\MT@ex@c@name @auto}%
    \def\@tempa{autoexpand}%
    \MT@vinfo{... : \ifx\@tempa\MT@auto@ En\else Dis\fi
                    abling automatic expansion}%
  }{%
    \let\MT@auto@\MT@auto
  }%
}
%    \end{macrocode}
%\end{macro}
%\end{macro}
%\end{macro}
%\end{macro}
%\end{macro}
%\end{macro}
%\begin{macro}{\MT@set@ex@heirs}
%    \begin{macrocode}
\def\MT@set@ex@heirs#1{%
  \efcode\MT@font#1=\efcode\MT@font\MT@char
%<*debug>
  \MT@dinfo@nl{2}{-- heir of \MT@char: #1}%
  \MT@dinfo@nl{4}{::: ef (#1) \number\efcode\MT@font\MT@char}%
%</debug>
}
%    \end{macrocode}
%\end{macro}
%\begin{macro}{\MT@preset@ex}
%    \begin{macrocode}
\def\MT@preset@ex{%
  \@tempcntb=\csname MT@ex@c@\MT@ex@c@name @preset\endcsname\relax
  \MT@scale@factor
  \MT@set@all@ex\@tempcntb
}
%    \end{macrocode}
%\end{macro}
%
%    \begin{macrocode}
%<*beta>
%    \end{macrocode}
%
%\subsubsection{Interword Space (Glue)}
%
%\begin{macro}{\MT@spacing}
% Adjustment of interword spacing? Only for sufficiently new versions of \pdftex.
%    \begin{macrocode}
\ifnum\MT@pdftex@no > 5
  \def\MT@spacing{\MT@maybe@do{sp}}
\else
  \let\MT@spacing\relax
\fi
%    \end{macrocode}
%\end{macro}
%\begin{macro}{\MT@set@sp@codes}
% This is all the same.
%    \begin{macrocode}
\def\MT@set@sp@codes{%
  \MT@reset@sp@codes
  \MT@if@list@exists{%
    \MT@get@dimen@six
    \MT@get@opt
    \MT@get@inh@list
    \MT@load@list{\MT@sp@c@name}%
    \edef\MT@curr@list@name{spacing list `\MT@sp@c@name'}%
    \MT@let@cn\@tempc{MT@sp@c@\MT@sp@c@name}%
    \expandafter\MT@sp@do\@tempc,\relax,%
  }\relax
}
%    \end{macrocode}
%\end{macro}
%\begin{macro}{\MT@sp@do}
%    \begin{macrocode}
\def\MT@sp@do#1,{%
  \ifx\relax#1\@empty \else
    \MT@sp@split #1==\relax
    \expandafter\MT@sp@do
  \fi
}
%    \end{macrocode}
%\end{macro}
%\begin{macro}{\MT@sp@split}
%    \begin{macrocode}
\def\MT@sp@split#1=#2=#3\relax{%
  \def\@tempa{#1}%
  \ifx\@tempa\@empty \else
    \MT@get@slot
    \ifnum\MT@char > \m@ne
      \MT@get@char@unit
      \MT@sp@split@val#2\relax
    \fi
  \fi
}
%    \end{macrocode}
%\end{macro}
%\begin{macro}{\MT@split@val}
% If |unit=space|, \cs{MT@get@space@unit} will be defined to fetch the
% corresponding fontdimen (2 for the first, 3 for the second and 4 for the
% third argument).
%    \begin{macrocode}
\def\MT@sp@split@val#1,#2,#3\relax{%
  \def\@tempb{#1}%
  \MT@ifempty\@tempb\relax{%
    \MT@get@space@unit\tw@
    \MT@scale@to@em
    \knbscode\MT@font\MT@char=\@tempcntb
%<debug>\MT@dinfo@nl{4}{;;; knbs (\MT@char): \number\knbscode\MT@font\MT@char}%
  }%
  \def\@tempb{#2}%
  \MT@ifempty\@tempb\relax{%
    \MT@get@space@unit\thr@@
    \MT@scale@to@em
    \stbscode\MT@font\MT@char=\@tempcntb
%<debug>\MT@dinfo@nl{4}{;;; stbs (\MT@char): \number\stbscode\MT@font\MT@char}%
  }%
  \def\@tempb{#3}%
  \MT@ifempty\@tempb\relax{%
    \MT@get@space@unit4%
    \MT@scale@to@em
    \shbscode\MT@font\MT@char=\@tempcntb
%<debug>\MT@dinfo@nl{4}{;;; shbs (\MT@char): \number\shbscode\MT@font\MT@char}%
  }%
  \MT@ifdefined@c\MT@sp@inh@name{%
    \MT@ifdefined@n{MT@inh@\MT@sp@inh@name @\MT@char @}{%
      \expandafter\MT@map@tlist@c
        \csname MT@inh@\MT@sp@inh@name @\MT@char @\endcsname
        \MT@set@sp@heirs
    }\relax
  }\relax
}
%    \end{macrocode}
%\end{macro}
%\begin{macro}{\MT@set@sp@heirs}
%    \begin{macrocode}
\def\MT@set@sp@heirs#1{%
  \knbscode\MT@font#1=\knbscode\MT@font\MT@char
  \stbscode\MT@font#1=\stbscode\MT@font\MT@char
  \shbscode\MT@font#1=\shbscode\MT@font\MT@char
%<*debug>
  \MT@dinfo@nl{2}{-- heir of \MT@char: #1}%
  \MT@dinfo@nl{4}{;;; knbs/stbs/shbs (#1): \number\knbscode\MT@font\MT@char/%
                                           \number\stbscode\MT@font\MT@char/%
                                           \number\shbscode\MT@font\MT@char}%
%</debug>
}
%    \end{macrocode}
%\end{macro}
%\begin{macro}{\MT@set@all@sp}
%\begin{macro}{\MT@reset@sp@codes}
%\begin{macro}{\MT@reset@sp@codes@}
%    \begin{macrocode}
\def\MT@set@all@sp#1#2#3{%
%<debug>\MT@dinfo@nl{3}{-- knbs/stbs/shbs: setting all to \number#1/\number#2/\number#3}%
  \@tempcnta=\z@
  \MT@while@num{\@tempcnta < \@cclvi}{%
    \knbscode\MT@font\@tempcnta=#1\relax
    \stbscode\MT@font\@tempcnta=#2\relax
    \shbscode\MT@font\@tempcnta=#3\relax
    \advance\@tempcnta \@ne
  }%
}
\def\MT@reset@sp@codes@{\MT@set@all@sp\z@\z@\z@}
\let\MT@reset@sp@codes\relax
%    \end{macrocode}
%\end{macro}
%\end{macro}
%\end{macro}
%\begin{macro}{\MT@preset@sp}
%\begin{macro}{\MT@preset@sp@}
%    \begin{macrocode}
\def\MT@preset@sp{%
  \expandafter\expandafter\expandafter\MT@preset@sp@
    \csname MT@sp@c@\MT@sp@c@name @preset\endcsname\@nil
}
\def\MT@preset@sp@#1,#2,#3\@nil{%
  \ifx\MT@sp@unit@\@empty
    \MT@warning@nl{%
      Cannot preset characters relative to their widths\MessageBreak
      for spacing list `\MT@sp@c@name'. Presetting them\MessageBreak
      relative to 1em instead}%
    \MT@preset@aux@factor{#1}\@tempa
    \MT@preset@aux@factor{#2}\@tempc
    \MT@preset@aux@factor{#3}\@tempb
  \else
    \MT@preset@aux@space{#1}\@tempa
    \def\@tempb{#2}%
    \MT@get@space@unit\thr@@
    \MT@scale@to@em
    \edef\@tempc{\number\@tempcntb}%
    \def\@tempb{#3}%
    \MT@get@space@unit4%
    \MT@scale@to@em
    \edef\@tempb{\number\@tempcntb}%
  \fi
  \MT@set@all@sp\@tempa\@tempc\@tempb
}
%    \end{macrocode}
%\end{macro}
%\end{macro}
%
%\subsubsection{Additional Kerning}
%
%\begin{macro}{\MT@kerning}
% Again, only check for additional kerning for new versions of \pdftex.
%    \begin{macrocode}
\ifnum\MT@pdftex@no > 5
  \def\MT@kerning{\MT@maybe@do{kn}}
\else
  \let\MT@kerning\relax
\fi
%    \end{macrocode}
%\end{macro}
%\begin{macro}{\MT@set@kn@codes}
%    \begin{macrocode}
\def\MT@set@kn@codes{%
  \MT@reset@kn@codes
  \MT@if@list@exists{%
    \MT@get@dimen@six
    \MT@get@opt
    \MT@get@inh@list
    \MT@load@list{\MT@kn@c@name}%
    \edef\MT@curr@list@name{kerning list `\MT@kn@c@name'}%
    \MT@let@cn\@tempc{MT@kn@c@\MT@kn@c@name}%
    \expandafter\MT@kn@do\@tempc,\relax,%
  }\relax
}
%    \end{macrocode}
%\end{macro}
%\begin{macro}{\MT@kn@do}
%    \begin{macrocode}
\def\MT@kn@do#1,{%
  \ifx\relax#1\@empty \else
    \MT@kn@split #1==\relax
    \expandafter\MT@kn@do
  \fi
}
%    \end{macrocode}
%\end{macro}
%\begin{macro}{\MT@kn@split}
%    \begin{macrocode}
\def\MT@kn@split#1=#2=#3\relax{%
  \def\@tempa{#1}%
  \ifx\@tempa\@empty \else
    \MT@get@slot
    \ifnum\MT@char > \m@ne
      \MT@get@char@unit
      \MT@kn@split@val#2\relax
    \fi
  \fi
}
%    \end{macrocode}
%\end{macro}
%\begin{macro}{\MT@kn@split@val}
% Again, the unit may be measured in the space dimension; this time only
% \cmd{\fontdimen}\,2.
%    \begin{macrocode}
\def\MT@kn@split@val#1,#2\relax{%
  \def\@tempb{#1}%
  \MT@ifempty\@tempb\relax{%
    \MT@get@space@unit\tw@
    \MT@scale@to@em
    \knbccode\MT@font\MT@char=\@tempcntb
%<debug>\MT@dinfo@nl{4}{;;; knbc (\MT@char): \number\knbccode\MT@font\MT@char}%
  }%
  \def\@tempb{#2}%
  \MT@ifempty\@tempb\relax{%
    \MT@get@space@unit\tw@
    \MT@scale@to@em
    \knaccode\MT@font\MT@char=\@tempcntb
%<debug>\MT@dinfo@nl{4}{;;; knac (\MT@char): \number\knaccode\MT@font\MT@char}%
  }%
  \MT@ifdefined@c\MT@kn@inh@name{%
    \MT@ifdefined@n{MT@inh@\MT@kn@inh@name @\MT@char @}{%
      \expandafter\MT@map@tlist@c
        \csname MT@inh@\MT@kn@inh@name @\MT@char @\endcsname
        \MT@set@kn@heirs
    }\relax
  }\relax
}
%    \end{macrocode}
%\end{macro}
%\begin{macro}{\MT@set@kn@heirs}
%    \begin{macrocode}
\def\MT@set@kn@heirs#1{%
  \knbccode\MT@font#1=\knbccode\MT@font\MT@char
  \knaccode\MT@font#1=\knaccode\MT@font\MT@char
%<*debug>
  \MT@dinfo@nl{2}{-- heir of \MT@char: #1}%
  \MT@dinfo@nl{4}{;;; knbc (#1): \number\knbccode\MT@font\MT@char/%
                                 \number\knaccode\MT@font\MT@char}%
%</debug>
}
%    \end{macrocode}
%\end{macro}
%\begin{macro}{\MT@set@all@kn}
%\begin{macro}{\MT@reset@kn@codes}
%\begin{macro}{\MT@reset@kn@codes@}
%    \begin{macrocode}
\def\MT@set@all@kn#1#2{%
%<debug>\MT@dinfo@nl{3}{-- knac/knbc: setting all to \number#1/\number#2}%
  \@tempcnta=\z@
  \MT@while@num{\@tempcnta < \@cclvi}{%
    \knbccode\MT@font\@tempcnta=#1\relax
    \knaccode\MT@font\@tempcnta=#2\relax
    \advance\@tempcnta \@ne
  }%
}
\def\MT@reset@kn@codes@{\MT@set@all@kn\z@\z@}
\let\MT@reset@kn@codes\relax
%    \end{macrocode}
%\end{macro}
%\end{macro}
%\end{macro}
%\begin{macro}{\MT@preset@kn}
%\begin{macro}{\MT@preset@kn@}
%    \begin{macrocode}
\def\MT@preset@kn{%
  \expandafter\expandafter\expandafter\MT@preset@kn@
    \csname MT@kn@c@\MT@kn@c@name @preset\endcsname\@nil
}
\def\MT@preset@kn@#1,#2\@nil{%
  \ifx\MT@kn@unit@\@empty
    \MT@warning@nl{%
      Cannot preset characters relative to their widths\MessageBreak
      for kerning list `\MT@kn@c@name'. Presetting them\MessageBreak
      relative to 1em instead}%
    \let\MT@preset@aux\MT@preset@aux@factor
  \else
    \let\MT@preset@aux\MT@preset@aux@space
  \fi
%    \end{macrocode}
% Letterspacing factor.
%    \begin{macrocode}
  \MT@ifstreq\MT@kn@context{letterspacing}{%
    \@tempcnta\MT@kn@factor@\relax
    \MT@scale\@tempcnta \MT@letterspacing@ \@m
    \edef\MT@kn@factor@{\number\@tempcnta}%
  }\relax
  \MT@preset@aux{#1}\@tempa
  \MT@preset@aux{#2}\@tempb
  \MT@set@all@kn\@tempa\@tempb
}
%    \end{macrocode}
%\end{macro}
%\end{macro}
%
%\subsubsection{\lsstyle Letterspacing}
%
%\begin{macro}{\lsstyle}
% Letterspacing is a special case of extra kerning. It will temporarily switch
% kerning on, activate the |all| font set and load the |letterspacing| context.
% The list must have the name |letterspacing|, so that the factor will be
% applied.
%    \begin{macrocode}
\ifnum\MT@pdftex@no > 5
  \DeclareRobustCommand\lsstyle{%
    \ifMT@kerning
%    \end{macrocode}
% We have to add the current font to the active kerning font set, so that the
% letterspacing will be reset. This will fail for font switches inside
% \cs{lsstyle}.
%\todo{fix font switch}
%    \begin{macrocode}
      \begingroup
      \escapechar\m@ne
      \expandafter\MT@exp@two@n\expandafter\MT@in@clist
          \csname\curr@fontshape/\f@size\expandafter\endcsname
          \csname MT@knlist@font@\MT@kn@setname\endcsname
      \ifMT@inlist@ \else
        \expandafter\MT@xadd
          \csname MT@knlist@font@\MT@kn@setname\endcsname
          {\csname\curr@fontshape/\f@size\endcsname,}%
      \fi
      \endgroup
    \fi
    \MT@kerningtrue
    \pdfappendkern\@ne
    \pdfprependkern\@ne
    \def\MT@kn@setname{all}%
    \MT@ifdefined@c\MT@letterspacing@\relax{%
      \let\MT@letterspacing@\MT@letterspacing
    }%
    \microtypecontext{kerning=letterspacing}%
  }
\else
  \DeclareRobustCommand\lsstyle{%
    \MT@warning{Letterspacing only works with pdftex version 1.3x\MessageBreak
      or newer. You might want to use the `soul' package\MessageBreak
      instead}%
    \global\let\lsstyle\relax
  }
\fi
%    \end{macrocode}
%\end{macro}
%\begin{macro}{\textls}
% This command may be used like the other text commands. The optional argument
% may be used to change the letterspacing factor.
%\changes{v2.0}{2005/09/21}{new command: letterspacing}
%    \begin{macrocode}
\DeclareRobustCommand\textls[2][]{%
  \MT@ifempty{#1}{%
    \let\MT@letterspacing@\@undefined
  }{%
    \KV@@sp@def\MT@letterspacing@{#1 }%
  }%
  {\lsstyle #2}%
}
%</beta>
%    \end{macrocode}
%\end{macro}
%
%\subsubsection{Disabling Ligatures}
%
%\begin{macro}{\MT@noligatures}
% The possibility to disable ligatures is a new features of \pdftex\ 1.30.0.
%    \begin{macrocode}
\ifnum\MT@pdftex@no < 5 \MT@hop@else@fi{%
  \let\MT@noligatures\relax
}\else\MT@hop@fi{%
  \def\MT@noligatures{%
    \csname ifMT@\MT@abbr@nl\endcsname
      \MT@dotrue
      \MT@map@clist@n{font,encoding,family,series,shape,size}{%
        \MT@ifdefined@n{MT@checklist@##1}%
          {\csname MT@checklist@##1\endcsname}%
          {\MT@checklist@{##1}}%
        {nl}%
      }%
    \else
      \MT@dofalse
    \fi
    \ifMT@do
      \pdfnoligatures\MT@font
      \MT@vinfo{... Disabling ligatures}%
    \fi
  }
}\fi
%    \end{macrocode}
%\end{macro}
%
%\subsubsection{Loading the Configuration}
%
%\begin{macro}{\MT@load@list}
% Recurse through the lists to be loaded.
%\changes{v1.3}{2004/10/27}{check whether list exists}
%\todo{load more than one list}
%    \begin{macrocode}
\def\MT@load@list#1{%
  \edef\@tempa{#1}%
  \MT@let@cn\@tempb{MT@\MT@feat @c@\@tempa load}%
  \MT@ifstreq\@tempa\@tempb{%
    \MT@warning{\@nameuse{MT@abbr@\MT@feat} list `\@tempa' cannot load itself}%
  }{%
    \ifx\@tempb\relax \else
      \MT@ifdefined@n{MT@\MT@feat @c@\@tempb}{%
        \MT@vinfo{... : First loading \@nameuse{MT@abbr@\MT@feat} list `\@tempb'}%
        \begingroup
          \MT@load@list{\@tempb}%
        \endgroup
        \edef\MT@curr@list@name{\@nameuse{MT@abbr@\MT@feat} list `\@tempb'}%
        \MT@let@cn\@tempc{MT@\MT@feat @c@\@tempb}%
        \expandafter\csname MT@\MT@feat @do\expandafter\endcsname\@tempc,\relax,%
      }{%
        \MT@warning{\@nameuse{MT@abbr@\MT@feat} list `\@tempb' undefined.
                    Cannot load\MessageBreak it from list `\@tempa'}%
      }%
    \fi
  }%
}
%    \end{macrocode}
%\end{macro}
%\changes{v1.1}{2004/09/13}{configuration file names in lowercase
%                           (suggested by Harald Harders)}
%\begin{macro}{\MT@find@file}
%\changes{v1.1}{2004/09/14}{bug fix: also check whether the file for the base
%                           font family has already been loaded}
% Micro-typographic settings may be written into a file |mt-|\meta{font family}|.cfg|.
%\changes{v1.8}{2005/04/16}{no longer wrap names in commands}
%\begin{macro}{\MT@file@list}
% We must also record whether we've already loaded the file.
%    \begin{macrocode}
\let\MT@file@list\@empty
\def\MT@find@file#1{%
%    \end{macrocode}
%\end{macro}
% Check for existence of the file only once.
%    \begin{macrocode}
  \MT@in@clist{#1}\MT@file@list
  \ifMT@inlist@\else
%    \end{macrocode}
% Don't forget that because reading the files takes place inside a group, all
% commands that may be used there have to be defined globally.
%    \begin{macrocode}
    \MT@begin@catcodes
    \let\MT@begin@catcodes\relax
    \let\MT@end@catcodes\relax
    \InputIfFileExists{mt-#1.cfg}{%
      \MT@vinfo{... Loading configuration file mt-#1.cfg}%
      \MT@xadd\MT@file@list{#1,}%
    }{%
      \expandafter\MT@get@basefamily#1\relax\relax\relax
      \MT@exp@one@n\MT@in@clist\@tempa\MT@file@list
      \ifMT@inlist@\else
        \InputIfFileExists{mt-\@tempa.cfg}{%
          \MT@vinfo{... Loading configuration file mt-\@tempa.cfg}%
          \MT@xadd\MT@file@list{\@tempa,#1,}%
        }{%
          \MT@vinfo{... No configuration file mt-#1.cfg}%
          \MT@xadd\MT@file@list{#1,}%
        }%
      \fi
    }%
    \endgroup
  \fi
}
%    \end{macrocode}
%\end{macro}
%\begin{macro}{\MT@begin@catcodes}
% We have to make sure that all characters have the correct category code.
% Especially, new lines and spaces should be ignored, since files might be
% loaded in the middle of the document. This is basically \cmd{\nfss@catcodes}
% (from the \LaTeX\ kernel). I've added:
%\changes{v1.4a}{2004/11/16}{bug fix: reset some more catcodes when reading files
%                          (reported by Michael Hoppe)} ^^A <mh\at michael-hoppe.de>
%                                  ^^A in <MID:p06002000bdbff1f93eb7@[192.168.0.34]>
% |&| (in |tabular|s), |!|, |?|, |;|,
%\changes{v1.7}{2005/02/06}{reset catcode of `\texttt{:}'
%                           (compatibility with \pkg{french}* packages)}
% |:| (|french|), |,|, |$|, |_|, |~|,
%\changes{v1.5}{2004/11/28}{reset catcode of `\texttt{\quotechar=}'
%                           (compatibility with Turkish \pkg{babel})}
% and |=| (Turkish \pkg{babel}).
%
%\changes{v1.8}{2005/03/29}{reset catcodes of the remaining \ascii\ characters}
% OK, now all printable characters up to 127 are `other'. We hope that letters
% are always letters and numbers other.
%
% We leave |^| at catcode 7, so that stuff like `|^^ff|' remains possible.
%
% This will be used before reading the files as well as in the configuration
% commands \cs{Set...}, and \cs{DeclareCharacterInheritance}, so that the
% catcodes are also harmless when these commands are used outside the
% configuration files.
%\changes{v1.7}{2005/02/17}{also use inside configuration commands}
%    \begin{macrocode}
\def\MT@begin@catcodes{%
  \begingroup
  \makeatletter
  \catcode`\^7%
  \catcode`\ 9%
  \catcode`\^^I9%
  \catcode`\^^M9%
  \catcode`\\\z@
  \catcode`\{\@ne
  \catcode`\}\tw@
  \catcode`\#6%
  \catcode`\%14%
  \MT@map@tlist@n
    {\!\"\$\&\'\(\)\*\+\,\-\.\/\:\;\<\=\>\?\[\]\_\`\|\~}%
    \@makeother
%    \end{macrocode}
% Inside the configuration files, we don't have to bother about spaces.
%    \begin{macrocode}
  \def\MT@remove@spaces##1{}%
  \let\KV@@sp@def\def
}
%    \end{macrocode}
%\end{macro}
%\begin{macro}{\MT@end@catcodes}
% End group if outside configuration file (otherwise relax).
%    \begin{macrocode}
\let\MT@end@catcodes\endgroup
%    \end{macrocode}
%\end{macro}
%\begin{macro}{\MT@get@basefamily}
% The family name might have a suffix
%\changes{v1.1}{2004/09/14}{only remove suffix, if it is `x' or `j'}
% for expert or old style number font set
%\changes{v1.2}{2004/09/26}{also remove `w' (swash capitals)}
% or for swash capitals (|x|, |j| or |w|). We mustn't simply remove the last
% letter, as this would make for instance |cms| out of |cmss| \textit{and}
% |cmsy| (OK, |cmex| will still become |cme|~\dots).
%\changes{v1.4b}{2004/11/25}{bug fix: failed for font names of the form \texttt{abczz}
%                           (reported by Georg Verweyen)} ^^A <Georg.Verweyen\at WEB.DE>
%                                                  ^^A in: <MID:41A64DC7.7040404@web.de>
%    \begin{macrocode}
\def\MT@get@basefamily#1#2#3#4\relax{%
  \ifx#2\relax \def\@tempa{#1}\else
    \ifx#3\relax \def\@tempa{#1#2}\else
      \def\@tempa{#1#2#3}%
      \ifx\relax#4\relax \else
        \MT@ifstreq{#4}{\string x}\relax{%
          \MT@ifstreq{#4}{\string j}\relax{%
            \MT@ifstreq{#4}{\string w}\relax{%
              \def\@tempa{#1#2#3#4}}}}\fi\fi\fi
}
%    \end{macrocode}
%\end{macro}
%\begin{macro}{\MT@listname}
%\begin{macro}{\MT@get@listname}
%\begin{macro}{\MT@get@listname@}
% Try all combinations of font family, series, shape and size to get a list for
% the current font.
%\changes{v1.1}{2004/09/15}{don't check for empty characteristics list}
%\changes{v1.2}{2004/09/30}{alternatively check for alias font name}
%\changes{v1.7}{2005/03/15}{use \cmd{\@tfor} (Andreas B\"uhmann's idea)}
%\changes{v1.8}{2005/04/16}{made recursive}
%    \begin{macrocode}
\def\MT@get@listname#1{%
%<debug>\MT@dinfo@nl{1}{trying to find \@nameuse{MT@abbr@#1} list for font \MT@font}%
  \let\MT@listname\@undefined
  \def\@tempb{#1}%
  \MT@map@tlist@c\MT@try@order\MT@get@listname@
}
\def\MT@get@listname@#1{%
  \expandafter\MT@next@listname#1%
  \ifx\MT@listname\@undefined \else
    \expandafter\MT@tlist@break
  \fi
}
%    \end{macrocode}
%\end{macro}
%\end{macro}
%\end{macro}
%\begin{macro}{\MT@try@order}
%\changes{v1.7}{2005/03/11}{bug fix: also check for //\meta{series}/\meta{shape}//
%                          (reported by Andreas B\"uhmann)} ^^A <andreas.buehmann\at web.de>
%                                                           ^^A private mail, 10/3/2005
%\changes{v1.7}{2005/03/11}{always check for size, too
%                          (suggested by Andreas B\"uhmann)} ^^A <andreas.buehmann\at web.de>
%                                                            ^^A private mail, 10/3/2005
%
%\begin{table}\footnotesize
%\caption{Order for matching font attributes}\label{tab:match-order}
%\catcode`\!=13 \let!\match
%\begin{tabular}{@{}L{38pt}*{16}{p{19.5pt}<{\centering}@{}}}
%\ifbooktabs\addlinespace\toprule\addlinespace\else\hline\fi
%          & 1.& 2.& 3.& 4.& 5.& 6.& 7.& 8.& 9.&10.&11.&12.&13.&14.&15.&16.\\
%\ifbooktabs\cmidrule(r){2-2}\cmidrule(r){3-3}\cmidrule(r){4-4}\cmidrule(r){5-5}
%     \cmidrule(r){6-6}\cmidrule(r){7-7}\cmidrule(r){8-8}\cmidrule(r){9-9}
%     \cmidrule(r){10-10}\cmidrule(r){11-11}\cmidrule(r){12-12}\cmidrule(r){13-13}
%     \cmidrule(r){14-14}\cmidrule(r){15-15}\cmidrule(r){16-16}\cmidrule{17-17}
%\else\hline\fi
% Encoding & ! & ! & ! & ! & ! & ! & ! & ! & ! & ! & ! & ! & ! & ! & ! & ! \\
% Family   & ! & ! & ! & ! & ! & ! & ! & ! & - & - & - & - & - & - & - & - \\
% Series   & ! & ! & ! & ! & - & - & - & - & ! & ! & ! & ! & - & - & - & - \\
% Shape    & ! & ! & - & - & ! & ! & - & - & ! & ! & - & - & ! & ! & - & - \\
% Size     & ! & - & ! & - & ! & - & ! & - & ! & - & ! & - & ! & - & ! & - \\
%\ifbooktabs\bottomrule\else\hline\fi
%\end{tabular}
%\end{table}
%
% Beginning with version 1.7, we always check for the font size. Since the
% matching order has become more logical now, it can be described in words, so
% that we don't need table~\ref{tab:match-order} in the documentation part any
% longer and can cast it off here.
%    \begin{macrocode}
\def\MT@try@order{%
  {1111}{1110}{1101}{1100}{1011}{1010}{1001}{1000}%
  {0111}{0110}{0101}{0100}{0011}{0010}{0001}{0000}%
}
%    \end{macrocode}
%\end{macro}
%\begin{macro}{\MT@next@listname}
% The current context is added to the font attributes. That is, the context must
% match.
%    \begin{macrocode}
\def\MT@next@listname#1#2#3#4{%
  \edef\@tempa{\MT@encoding
               /\ifnum#1=\@ne \MT@family\fi
               /\ifnum#2=\@ne \MT@series\fi
               /\ifnum#3=\@ne \MT@shape\fi
               /\ifnum#4=\@ne *\fi
               \MT@context}%
%<debug>\MT@dinfo@nl{1}{trying \@tempa}%
  \MT@ifdefined@n{MT@\@tempb @\@tempa}{%
    \MT@next@listname@#4%
  }{%
%    \end{macrocode}
% Also try with an alias family.
%    \begin{macrocode}
    \ifnum#1=\@ne
      \ifx\MT@familyalias\@empty \else
        \edef\@tempa{\MT@encoding
                    /\MT@familyalias
                    /\ifnum#2=\@ne \MT@series\fi
                    /\ifnum#3=\@ne \MT@shape\fi
                    /\ifnum#4=\@ne *\fi
                    \MT@context}%
%<debug>\MT@dinfo@nl{1}{(alias) \@tempa}%
        \MT@ifdefined@n{MT@\@tempb @\@tempa}{%
          \MT@next@listname@#4%
        }\relax
      \fi
    \fi
  }%
}
%    \end{macrocode}
%\end{macro}
%\begin{macro}{\MT@next@listname@}
% If size is to be evaluated, do that, otherwise use the current list.
%    \begin{macrocode}
\def\MT@next@listname@#1{%
  \ifnum#1=\@ne
    \expandafter\MT@in@rlist\csname MT@\@tempb @\@tempa @sizes\endcsname
    \ifMT@inlist@
      \let\MT@listname\MT@size@name
    \fi
  \else
    \MT@let@cn\MT@listname{MT@\@tempb @\@tempa}%
  \fi
}
%    \end{macrocode}
%\end{macro}
%\begin{macro}{\MT@if@list@exists}
%\changes{v1.7}{2005/02/06}{don't define \cs{MT@\#1@c@name} \cmd{\global}ly,
%                           here and elsewhere}
%\begin{macro}{\MT@context}
%    \begin{macrocode}
\def\MT@if@list@exists{%
  \expandafter\let\expandafter\MT@context\csname MT@\MT@feat @context\endcsname
  \MT@get@listname{\MT@feat @c}%
  \MT@ifdefined@c\MT@listname{%
    \MT@edef@n{MT@\MT@feat @c@name}{\MT@listname}%
    \ifMT@nonselected
      \MT@vinfo{... Applying non-selected expansion (list `\MT@ex@c@name')}%
    \else
      \MT@vinfo{... Loading \@nameuse{MT@abbr@\MT@feat} list
                    `\@nameuse{MT@\MT@feat @c@name}'}%
    \fi
    \@firstoftwo
  }{%
%    \end{macrocode}
% Since the name cannot be \cmd{\@empty}, this is a sound proof that no matching
% list exists.
%    \begin{macrocode}
    \MT@let@nc{MT@\MT@feat @c@name}\@empty
%    \end{macrocode}
% Don't warn if \opt{selected}|=false|.
%    \begin{macrocode}
    \ifMT@nonselected
      \MT@vinfo{... Applying non-selected expansion}%
    \else
      \MT@warning{I cannot find a \@nameuse{MT@abbr@\MT@feat} list
        for font\MessageBreak`\MT@exp@string\MT@font'%
          \ifx\MT@context\@empty\else\space(context: `\MT@context')\fi.
        Switching off\MessageBreak\@nameuse{MT@abbr@\MT@feat} for this font}%
     \fi
    \@secondoftwo
  }%
}
%    \end{macrocode}
%\end{macro}
%\end{macro}
%\begin{macro}{\MT@get@inh@list}
%\changes{v1.6}{2004/12/18}{correct message if \opt{selected} is false}
%\begin{macro}{\MT@context}
% The inheritance lists are global (no context).
%    \begin{macrocode}
\def\MT@get@inh@list{%
  \let\MT@context\@empty
  \MT@get@listname{\MT@feat @inh}%
  \MT@ifdefined@c\MT@listname{%
    \MT@edef@n{MT@\MT@feat @inh@name}{\MT@listname}%
%<*debug>
    \MT@dinfo@nl{1}{... Using \@nameuse{MT@abbr@\MT@feat} inheritance list
                    `\@nameuse{MT@\MT@feat @inh@name}'}%
%</debug>
    \MT@let@cn\@tempc{MT@\MT@feat @inh@\csname MT@\MT@feat @inh@name\endcsname}%
%    \end{macrocode}
% If the list is \cmd{\@empty}, it has already been parsed.
%\changes{v1.2}{2004/09/26}{bug fix: set inheritance list \cmd{\global}ly to \cmd{\@empty}}
%    \begin{macrocode}
    \ifx\@tempc\@empty \else
%<debug>\MT@dinfo@nl{1}{parsing inheritance list ...}%
      \MT@let@cn\MT@inh@name{MT@\MT@feat @inh@name}%
      \def\MT@curr@list@name{inheritance list}%
      \expandafter\MT@inh@do\@tempc,\relax,%
      \global\MT@let@nc{MT@\MT@feat @inh@\csname MT@\MT@feat @inh@name\endcsname}\@empty
    \fi
  }{%
    \MT@let@nc{MT@\MT@feat @inh@name}\@undefined
  }%
}
%    \end{macrocode}
%\end{macro}
%\end{macro}
%
% \subsubsection{Translating Characters}
%
%\changes{v1.4}{2004/11/04}{don't use scratch registers in global definitions}
% Get the slot number of the character in the current encoding.
%
%\begin{macro}{\MT@get@slot}
%\changes{v1.2}{2004/09/26}{bug fix: group must also include \cs{MT@get@composite}}
%\changes{v1.4b}{2004/11/22}{don't define \cs{MT@char} globally (save stack problem)}
%\changes{v1.6a}{2005/01/30}{completely redone, hopefully more robust
%                           (compatible with \pkg{frenchpro};
%                            problem reported by Bernard Gaulle)} ^^A <gaulle\at idris.fr>
%                                                                 ^^A private mail, 28/1/2005
%\changes{v1.7}{2005/03/21}{remove backslash hack}
%\begin{macro}{\MT@char}
%\begin{macro}{\Mt@char}
% There are lots of possibilities how a character may be specified in the
% configuration files, which makes translating them into slot numbers quite
% expensive. Also, we want to have this as robust as possible, so that the user
% does not have to solve a sphinx's riddle if anything goes wrong.
%
% The character is in \cs{@tempa}, we want its slot number in \cs{MT@char}.
%    \begin{macrocode}
\def\MT@get@slot{%
  \escapechar`\\
  \let\Mt@char\m@ne
  \MT@noresttrue
%    \end{macrocode}
%\end{macro}
%\end{macro}
% Save unexpanded string in case we need to issue a warning message.
%    \begin{macrocode}
  \mt@toks=\expandafter\expandafter\expandafter{\expandafter\string\@tempa}%
  \edef\MT@char{\expandafter\meaning\@tempa}%
%    \end{macrocode}
% Now, let's walk through (hopefully all) possible cases.
%\begin{itemize}
% \item It's a letter, a character or a number.
%    \begin{macrocode}
  \expandafter\MT@is@letter\@tempa\relax\relax
  \ifnum\Mt@char < \z@
%    \end{macrocode}
%\changes{v1.8}{2005/03/30}{bug fix: expand active characters}
% \item It might be an active character, \ie, an 8-bit character defined by
%       \pkg{inputenc}.
%    \begin{macrocode}
    \MT@exp@two@c\MT@is@active\string\@tempa\@nil
%    \end{macrocode}
% \item OK, so it must be a macro. We do not allow random commands but only
%       those defined in \LaTeX's idiosyncratic font encoding scheme:
%
%       If |\|\meta{encoding}|\|\meta{command} (that's \emph{one} command) is
%       defined, we try to extract the slot number.
%\changes{v1.7}{2005/02/27}{test whether \cs{}\meta{encoding}\cs{\meta{..}} is defined}
%\changes{v1.8}{2005/04/05}{test whether \cs{}\meta{encoding}\cs{\meta{..}} is defined
%                           made more robust}
%    \begin{macrocode}
    \MT@ifdefined@n{\MT@encoding\MT@detokenize\@tempa}%
      \MT@is@symbol
    {%
%    \end{macrocode}
% \item Now, we'll catch the rest, which hopefully is an accented character
%       (\eg~|\"a|).
%    \begin{macrocode}
      \expandafter\MT@is@composite\@tempa\relax\relax
    }%
    \ifnum\Mt@char < \z@
%    \end{macrocode}
% \item It could also be a \cmd{\chardef}ed command (\eg\ the percent character).
%       This seems the least likely case, so it's last.
%\changes{v1.7}{2005/03/20}{test for \cmd{\chardef}ed commands}
%    \begin{macrocode}
      \MT@exp@two@c\MT@is@char\MT@char\MT@charstring\relax\relax\relax
    \fi
  \fi
%    \end{macrocode}
%\end{itemize}
%    \begin{macrocode}
  \let\MT@char\Mt@char
  \ifnum\MT@char < \z@
    \MT@warn@unknown
  \else
%    \end{macrocode}
% If the user has specified something like `|fi|', or wanted to define a number
% but forgot to use three digits, we'll have something left of the string. In
% this case, we issue a warning and forget the complete string.
%    \begin{macrocode}
    \ifMT@norest \else
      \MT@warn@unknown@i
      \let\MT@char\m@ne
    \fi
  \fi
  \escapechar\m@ne
}
%    \end{macrocode}
%\end{macro}
%\begin{macro}{\ifMT@norest}\IndexNewif{MT@norest}
%\begin{macro}{\MT@testrest}
% Switch and test whether all of the string has been used up.
%    \begin{macrocode}
\newif\ifMT@norest
\def\MT@testrest#1#2{%
  \MT@ifstreq{#1}{#2}\relax\MT@norestfalse
}
%    \end{macrocode}
%\end{macro}
%\end{macro}
%\begin{macro}{\MT@is@letter}
% Input is a letter, a character or a number.
%\changes{v1.8}{2005/04/04}{warning for non-\ascii\ characters}
%\changes{v1.9}{2005/10/22}{using \cmd{\catcode} should be more efficient than
%                           inspecting the \cmd{\meaning}}
%    \begin{macrocode}
\def\MT@is@letter#1#2\relax{%
  \ifcat a\noexpand#1\relax
    \edef\Mt@char{\number`#1}%
    \ifx\\#2\\%
%<debug>\MT@dinfo@nl{3}{> `\the\mt@toks' is a letter (\Mt@char)}%
    \else
      \MT@norestfalse
    \fi
  \else
    \ifcat 1\noexpand#1\relax
      \edef\Mt@char{\number`#1}%
%<debug>\MT@dinfo@nl{3}{> `\the\mt@toks' is a character (\Mt@char)}%
      \ifx\\#2\\%
        \ifnum\Mt@char>127 \Mt@warn@ascii \fi
      \else
        \MT@norestfalse
        \expandafter\MT@is@number#1#2\relax\relax
      \fi
    \fi
  \fi
}
%    \end{macrocode}
%\end{macro}
%\begin{macro}{\MT@is@number}
% Numbers may be specified as a three-digit decimal number (|029|),
%\changes{v1.1}{2004/09/13}{numbers may also be specified in hexadecimal or octal
%                           (suggested by Harald Harders)}
% as a hexadecimal number (prefixed with~|"|: |"1D|) or as a octal number
% (prefixed with~|'|: |'35|). They must consist of at least three characters
% (including the prefix), that is, |"F| is not permitted.
%    \begin{macrocode}
\def\MT@is@number#1#2#3\relax{%
  \ifx\relax#3\relax \else
    \ifx\relax#2\relax \else
      \MT@noresttrue
      \if#1"\relax
        \def\x{\uppercase{\edef\Mt@char{\number#1#2#3}}}\x
%<debug>\MT@dinfo@nl{3}{> ... a hexadecimal number: \Mt@char}%
      \else
        \if#1'\relax
          \def\Mt@char{\number#1#2#3}%
%<debug>  \MT@dinfo@nl{3}{> ... an octal number: \Mt@char}%
        \else
          \MT@ifnumber{#1#2#3}{%
            \def\Mt@char{\number#1#2#3}%
%<debug>    \MT@dinfo@nl{3}{> ... a decimal number: \Mt@char}%
          }\MT@norestfalse
        \fi
      \fi
      \ifnum\Mt@char > \@cclv
        \MT@warn@number@too@large{\noexpand#1\noexpand#2\noexpand#3}%
        \let\Mt@char\m@ne
      \fi
    \fi
  \fi
}
%    \end{macrocode}
%\end{macro}
%\begin{macro}{\MT@is@active}
%\changes{v1.8}{2005/03/30}{new macro: translate \pkg{inputenc}-defined characters}
% Expand an active character. (This was completely broken in v1.7, and only
% worked by chance before.)
%\changes{v1.9}{2005/09/08}{redone: use \cmd{\set@display@protect}}
% We \cmd{\set@display@protect} to translate, \eg, |�| into |\"A|, that is to
% whatever it is defined in the \pkg{inputenc} encoding file.
%
% Unfortunately, the \pkg{inputenc} definitions prefer the protected/generic
% variants (\eg, \cmd{\copyright} instead of \cmd{\textcopyright}), which our
% parser won't be able to understand.
% (I'm fed up now, so you have to complain if you really, really want to be
% able to write `|�|' instead of \cmd{\textcopyright}, thus rendering your
% configuration files unportable.)
%
%\todo{Make \microtype\ Unicode-savvy?}
% Unicode characters (\pkg{inputenc}/|utf8|) are currently not supported.
%    \begin{macrocode}
\def\MT@is@active#1#2\@nil{%
  \ifx\\#2\\%
    \ifnum\catcode`#1 = \active
      \begingroup
        \set@display@protect
        \def\IeC##1{##1}%
%    \end{macrocode}
% The character is undefined in the encoding.
%    \begin{macrocode}
        \def\@inpenc@undefined@##1{undefined^^J%
          (microtype)\@spaces\@spaces\@spaces\@spaces
          in input encoding ``##1''}%
        \edef\x{%
          \def\noexpand\@tempa{\@tempa}%
%    \end{macrocode}
% Append what we think the translation is to the token register we use for the
% log.
%    \begin{macrocode}
          \mt@toks={\the\mt@toks\space (= \@tempa)}%
        }%
      \expandafter\endgroup\x
    \fi
  \fi
}
%    \end{macrocode}
%\end{macro}
%\begin{macro}{\MT@is@symbol}
% The symbol commands might expand to funny stuff, depending on context.
% Instead of simply expanding |\|\meta{command}, we construct the command
% |\|\meta{encoding}|\|\meta{command} and see whether its meaning is
% \cmd{\char}|"|\meta{hex number}, which is the case for everything that has
% been defined with \cs{DeclareTextSymbol} in the encoding definition files.
%    \begin{macrocode}
\def\MT@is@symbol{%
  \edef\@tempa{\expandafter
                 \csname\expandafter
                   \MT@encoding\expandafter
                   \string\@tempa
                 \endcsname}%
  \expandafter\MT@exp@two@c\expandafter\MT@is@char\expandafter
      \meaning\expandafter\@tempa\MT@charstring\relax\relax\relax
  \ifnum\Mt@char < \z@
%    \end{macrocode}
% \dots~or, if it hasn't been defined by \cs{DeclareTextSymbol}, a letter
% (\eg\ \cmd{\i}, when using \pkg{frenchpro}). ^^A as noted by Bernard Gaulle <gaulle\at idris.fr>
%                                              ^^A private mail, 28/1/2005
%    \begin{macrocode}
    \expandafter\MT@is@letter\@tempa\relax\relax
  \fi
}
%    \end{macrocode}
%\end{macro}
%\begin{macro}{\MT@is@char}
%\begin{macro}{\MT@charstring}
% Here we define a helper macro that inspect the \cmd{\meaning} of its
% argument.
%    \begin{macrocode}
%<debug>\def\Mt@info{\MT@dinfo@nl{3}}
\begingroup
  \catcode`\/=0
  /MT@map@tlist@n{/\/C/H/A/R}/@makeother
  /lowercase{%
    /def/x{%
      /def/MT@charstring{\CHAR"}%
      /def/MT@is@char##1\CHAR"##2##3##4/relax{%
        /ifx/relax##1/relax
          /if##3\/relax
            /edef/Mt@char{/number"##2}%
            /MT@testrest/MT@charstring{##3##4}%
          /else
            /edef/Mt@char{/number"##2##3}%
            /MT@testrest/MT@charstring{##4}%
          /fi
%<debug>  /Mt@info{> `/the/mt@toks' is a \char (/Mt@char)}%
        /fi
      }%
    }%
  }
/expandafter/endgroup/x
%    \end{macrocode}
%\end{macro}
%\end{macro}
%\begin{macro}{\MT@is@composite}
% Here, we are dealing with accented characters, specified as two tokens.
%\changes{v1.7}{2005/03/10}{new macro: construct command for composite character;
%                           no uncontrolled expansion}
%    \begin{macrocode}
\def\MT@is@composite#1#2\relax{%
  \ifx\\#2\\\else
%    \end{macrocode}
% Again, we construct a control sequence, this time of the form:
% |\\|\meta{encoding}\hskip0pt|\|\meta{accent}|-|\meta{character}, \eg\
% |\\T1\"-a|, which expands to a letter if it has been defined by
% \cs{DeclareTextComposite}. This should be robust, finally.
%    \begin{macrocode}
    \edef\@tempa{\expandafter
                   \csname\expandafter
                     \string\csname\MT@encoding\endcsname
                     \string#1-%
                     \string#2%
                   \endcsname}%
    \expandafter\MT@is@letter\@tempa\relax\relax
  \fi
}
%    \end{macrocode}
%\end{macro}
%\begin{macro}{\MT@detokenize}
%\changes{v1.9}{2005/08/17}{fix the non-\etex\ version}
% Translate a macro into a token list. We must be cautious not to stumble over
% accented characters consisting of two commands, like \cmd{\`}\cmd{\i} or
% \cmd{\U}\cmd{\CYRI}. With \etex, we can use \cmd{\detokenize} (and
% \cmd{\expandafter}\cmd{\string} to get rid of the trailing space). The
% non-\etex\ version requires some more fiddling.
%    \begin{macrocode}
\ifcase\MT@etex@no
  \def\MT@detokenize#1{\MT@exp@two@c\zap@space\strip@prefix\meaning#1 \@empty}
\else
  \def\MT@detokenize#1{\detokenize
    \expandafter\expandafter\expandafter{\expandafter\string#1}}
\fi
%    \end{macrocode}
%\end{macro}
% Some warning messages, for performance reasons separated here.
%\begin{macro}{\MT@curr@list@name}
%\changes{v1.8}{2005/06/08}{new macro: current list type and name}
% The type and name of the current list, defined at various places.
%    \begin{macrocode}
\let\MT@curr@list@name\@empty
%    \end{macrocode}
%\end{macro}
%\begin{macro}{\Mt@warn@ascii}
% For characters with character code \textgreater\ 127, we issue a warning
% (\pkg{inputenc} probably hasn't been loaded), since correspondence with
% the slot numbers would be purely coincidental.
%    \begin{macrocode}
\def\Mt@warn@ascii{%
  \MT@warning@nl{Character `\the\mt@toks' (= \Mt@char)
    is outside of ASCII range.\MessageBreak
    You must load the `inputenc' package before using\MessageBreak
    8-bit characters in \MT@curr@list@name}%
}
%    \end{macrocode}
%\end{macro}
%\begin{macro}{\MT@warn@number@too@large}
% Number too large.
%    \begin{macrocode}
\def\MT@warn@number@too@large#1{%
  \MT@warning@nl{%
    Number #1 in encoding `\MT@encoding' too large!\MessageBreak
    Ignoring it in \MT@curr@list@name}%
}
%    \end{macrocode}
%\end{macro}
%\begin{macro}{\MT@warn@unknown@i}
% Not all of the string has been parsed.
%    \begin{macrocode}
\def\MT@warn@unknown@i{%
  \MT@warning@nl{%
    Unknown slot number of character `\the\mt@toks' in\MessageBreak
    font encoding `\MT@encoding'. Make sure it's a single\MessageBreak
    character (or a number) in \MT@curr@list@name}%
}
%    \end{macrocode}
%\end{macro}
%\begin{macro}{\MT@warn@unknown}
% No idea what went wrong.
%    \begin{macrocode}
\def\MT@warn@unknown{%
  \MT@warning@nl{%
    Unknown slot number of character `\the\mt@toks' in\MessageBreak
    font encoding `\MT@encoding' in \MT@curr@list@name}%
}
%    \end{macrocode}
%\end{macro}
%
%\subsubsection{Hook into \LaTeX's font selection}
%
% We append \cs{MT@setupfont} to \cmd{\pickup@font}, which is called by \LaTeX\
% every time a font is selected. We then check whether we've already seen this
% font, and if not, set it up for micro-typography.
% This ensures that we will catch all fonts, and that we will not set up fonts
% more than once. The whole package really hangs on this command.
%
% In contrast to the \pkg{pdfcprot} package, there is no need to declare the
% fonts in advance that should benefit from micro-typographic treatment. Also,
% only those fonts that are actually being used will be set up for expansion
% and protrusion.
%
% For my reference:
%\begin{itemize}
%  \item \cmd{\pickup@font} is called by \cmd{\selectfont}, \cmd{\wrong@fontshape},
%     or \\ \cmd{\getanddefine@fonts} (for math).
%  \item \cmd{\pickup@font} calls \cmd{\define@newfont}.
%  \item \cmd{\define@newfont} may call (inside a group!)
%  \begin{itemize}
%     \item \cmd{\wrong@fontshape}, which in turn will call \cmd{\pickup@font},
%        and thus \\ \cmd{\define@newfont} again, or
%     \item \cmd{\extract@font}.
%  \end{itemize}
%  \item \cmd{\get@external@font} is called by \cmd{\extract@font}, by itself,
%        and by the substitution macros
%\end{itemize}
%
%\changes{v1.2}{2004/10/02}{check for packages that might load fonts}
%\changes{v1.4}{2004/11/04}{no need to check for packages that might load fonts anymore}
%\changes{v1.4}{2004/11/04}{use \cmd{\pickup@font} instead of \cmd{\define@newfont}
%                           as the hook for \cs{MT@setupfont}}
% Up to version 1.3 of this package, we were using \cmd{\define@newfont} as the
% hook, which is only called for \emph{new} fonts, and therefore seemed the natural
% choice.
% However, this meant that we had to take special care to catch all fonts: We
% additionally had to set up the default font, the error font (if it wasn't the
% default font), we had to check for some packages that might have been loaded
% before \microtype\ and were loading fonts, e.g. \pkg{jurabib},
% \pkg{ledmac}, \pkg{pifont} (loaded by \pkg{hyperref}), \pkg{tipa}, and
% probably many more. Furthermore, we had to include a hack for the
% \pkg{IEEEtran} class which loads all fonts in the class file itself (to fine
% tune inter-word spacing). Then I learned that even my favorite class, the
% \pkg{memoir} class, loads fonts. To cut this short: It seemed to get out of
% hand, and I decided that it would be better to use \cmd{\pickup@font} and
% decide for ourselves whether we've already seen that font. I hope the
% overhead isn't too large.
%\changes{v1.8}{2005/05/15}{if font substitution has occurred, set up the
%                           substitute font, not the selected one}
%\begin{macro}{\MT@font}
% Additionally, we hook into \cmd{\do@subst@correction}, which is called if
% a substitution has taken place, to record the name of the ersatz font.
% Unfortunately, this will only work for one-level substitutions.
%    \begin{macrocode}
\let\MT@font\@empty
\g@addto@macro\do@subst@correction{%
  \xdef\MT@font{\csname \curr@fontshape/\f@size\endcsname}%
}
%    \end{macrocode}
%\end{macro}
%\begin{macro}{\MT@orig@pickupfont}
% Check whether \cmd{\pickup@font} is defined as expected.
% The warning issued by \cmd{\CheckCommand*} would be a bit too generic.
%    \begin{macrocode}
\def\MT@orig@pickupfont{\expandafter\ifx\font@name\relax\define@newfont\fi}
\ifx\pickup@font\MT@orig@pickupfont \else
  \MT@warning@nl{%
    Command \string\pickup@font\space is not defined as expected.\MessageBreak
    Double-check whether micro-typography is indeed\MessageBreak
    applied to the document.\MessageBreak (Hint: Turn on `verbose' mode)%
  }
\fi
%    \end{macrocode}
%\end{macro}
% Then we append our stuff.
%    \begin{macrocode}
\g@addto@macro\pickup@font{%
  \begingroup
    \escapechar\m@ne
%    \end{macrocode}
% If \cs{MT@font@name} is empty, no substitution has taken place, hence
% \cmd{\font@name} is correct. Otherwise, if they are different,
% \cmd{\font@name} does not describe the font actually used. This test will
% catch one-level substitutions, like |bx| to |b|, but it will still fail if
% the substituting font is itself substituted.
%    \begin{macrocode}
    \ifx\MT@font\@empty
      \let\MT@font\font@name
    \else
      \ifx\MT@font\font@name \else
        \expandafter\MT@xadd
            \csname MT@\MT@curr@contexts font@list\endcsname{\font@name,}%
      \fi
    \fi
%    \end{macrocode}
% Comma-separated lists are used to remember the fonts we've already set up.
% There is one list for each combination of contexts.
%\changes{v1.9}{2005/10/03}{allow context-specific font setup}
%    \begin{macrocode}
    \expandafter\MT@exp@one@n\expandafter\MT@in@clist\expandafter\MT@font
         \csname MT@\MT@curr@contexts font@list\endcsname
    \ifMT@inlist@ \else
      \MT@setupfont
%    \end{macrocode}
% Add the font to the current list and remove it from the other contexts' lists.
%    \begin{macrocode}
      \expandafter\MT@xadd\csname MT@\MT@curr@contexts font@list\endcsname{\MT@font,}%
      \MT@map@tlist@c\MT@doc@contexts\MT@rem@from@lists
    \fi
  \endgroup
  \global\let\MT@font\@empty
}
%    \end{macrocode}
%\begin{macro}{\MT@rem@from@lists}
% Recurse through all context font lists of the document and remove the font,
% unless it's the current context.
%    \begin{macrocode}
\def\MT@rem@from@lists#1{%
  \MT@ifstreq{#1}\MT@curr@contexts\relax{%
    \expandafter\MT@exp@one@n\expandafter\MT@rem@from@list
       \expandafter\MT@font\csname MT@#1font@list\endcsname
  }%
}
%    \end{macrocode}
%\end{macro}
%\begin{macro}{\MT@pickupfont}
% Remember the patched command for later.
%    \begin{macrocode}
\let\MT@pickupfont\pickup@font
%    \end{macrocode}
%\end{macro}
%\begin{macro}{\MT@add@accent}
%\changes{v1.8}{2005/06/14}{bug fix: disable micro-typographic setup inside \cmd{\add@accent}
%                           (reported by Stephan Hennig)}% ^^A <stephanhennig\at arcor.de>
%                      ^^A in <MID: <42adb0e3$0$27784$9b4e6d93@newsread2.arcor-online.net>
% Inside \cmd{\add@accent}, we have to disable \microtype's setup, since the
% grouping in the patched \cmd{\pickup@font} would break the accent if
% different fonts are used for the base character and the accent. Fortunately,
% \LaTeX\ takes care that the fonts used for the \cmd{\accent} are already set
% up, so that we cannot be overlooking them.
% At first, I was going to change \cmd{\hmode@bgroup} only, but that is also
% used in the commands defined by \cmd{\DeclareTextFontCommand}, \ie,
% \cmd{\textit} etc.
%    \begin{macrocode}
\let\MT@add@accent\add@accent
\def\add@accent#1#2{%
  \let\pickup@font\MT@orig@pickupfont
  \MT@add@accent{#1}{#2}%
  \let\pickup@font\MT@pickupfont
}
%    \end{macrocode}
%\end{macro}
%\changes{v1.6}{2004/12/29}{test whether \cmd{\pickup@font} has changed}
% We also test whether our definition has survived (\cmd{\pickup@font} is
% at least redefined by \pkg{CJK}\footnote{
%     Therefore, \microtype\ will probably not work together with \pkg{CJK}.
%     However, I would be glad to be proven wrong.}).
%    \begin{macrocode}
\AtBeginDocument{%
  \ifx\MT@pickupfont\pickup@font \else
    \MT@error{%
      Another package has overwritten the definition\MessageBreak
      of \string\pickup@font. I might not be able to\MessageBreak
      apply any micro-typography. Please find the\MessageBreak
      culprit, and load it before the microtype package
    }{%
The microtype package attaches the micro-typographic setup to\MessageBreak
\string\pickup@font. If the other package has simply overwritten this\MessageBreak
command, nothing will work. If, on the other hand, it has changed\MessageBreak
the command in a cautious way, everything may be fine.\MessageBreak
In either case, please send a report to <w.m.l@gmx.net>.
    }%
  \fi
}
%    \end{macrocode}
%
%
%\subsection{Configuration}
%
%\subsubsection{Font Sets}
%\todo{allow for \texttt{load}ing and extending another set; or using more than one set?}
%
%\begin{macro}{\DeclareMicrotypeSet}
%\begin{macro}{\DeclareMicrotypeSet*}
% Calling this macro will create a comma list for every font characteristic of
% the form: |\MT|\meta{feature}|list@|\meta{characteristic}|@|\meta{set name}.
% If the optional argument is empty, lists for both expansion and protrusion
% will be created.
%
% The third argument must be a list of |key=value| pairs. If a font characteristic
% is not specified, we define the corresponding list to \cmd{\relax}, so that it
% does not constitute a constraint.
%    \begin{macrocode}
\def\DeclareMicrotypeSet{%
  \@ifstar
    {\@ifnextchar[\MT@DeclareSetAndUseIt
                  {\MT@DeclareSetAndUseIt[]}}%
    {\@ifnextchar[\MT@DeclareSet
                  {\MT@DeclareSet[]}}%
}
%    \end{macrocode}
%\end{macro}
%\end{macro}
%\begin{macro}{\MT@DeclareSet}
%\begin{macro}{\MT@DeclareSetAndUseIt}
%    \begin{macrocode}
\def\MT@DeclareSet[#1]{%
  \MT@DeclareSet@{#1}%
}
\def\MT@DeclareSetAndUseIt[#1]#2#3{%
  \MT@DeclareSet@{#1}{#2}{#3}%
  \UseMicrotypeSet[#1]{#2}%
}
%    \end{macrocode}
%\end{macro}
%\end{macro}
%\begin{macro}{\MT@DeclareSet@}
%\changes{v1.1}{2004/09/15}{remove spaces around first argument}
%    \begin{macrocode}
\def\MT@DeclareSet@#1#2#3{%
  \KV@@sp@def\@tempa{#1}%
  \MT@ifempty\@tempa{%
    \MT@declare@sets{pr}{#2}{#3}%
    \MT@declare@sets{ex}{#2}{#3}%
%<*beta>
    \MT@declare@sets{sp}{#2}{#3}%
    \MT@declare@sets{kn}{#2}{#3}%
%</beta>
  }{%
    \MT@map@clist@c\@tempa{%
      {\KV@@sp@def\@tempa{##1}%
       \MT@ifempty\@tempa\relax{%
         \MT@exp@one@n\MT@declare@sets
            {\csname MT@rbba@\@tempa\endcsname}{#2}{#3}}}%
    }%
  }%
}
%    \end{macrocode}
%\end{macro}
%\begin{macro}{\MT@curr@set@name}
% We need to remember the name of the set currently being declared.
%    \begin{macrocode}
\let\MT@curr@set@name\@empty
%    \end{macrocode}
%\end{macro}
%\begin{macro}{\MT@declare@sets}
% Define the current set name and parse the keys.
%\changes{v1.1}{2004/09/20}{remove spaces around set name}
%\changes{v1.8}{2005/05/15}{warning when redefining a set}
%    \begin{macrocode}
\def\MT@declare@sets#1#2#3{%
  \KV@@sp@def\MT@curr@set@name{#2}%
  \MT@ifdefined@n{MT@#1@set@@\MT@curr@set@name}{%
    \MT@warning{Redefining set `\MT@curr@set@name'}%
  }\relax
  \global\MT@let@nc{MT@#1@set@@\MT@curr@set@name}\@empty
%<debug>\MT@dinfo{1}{declaring \@nameuse{MT@abbr@#1} set `\MT@curr@set@name'}%
  \setkeys{MT@#1@set}{#3}%
}
%    \end{macrocode}
%\end{macro}
%\begin{macro}{\MT@define@set@keys}
% Define the \pkg{keyval} keys for expansion and protrusion sets.
%    \begin{macrocode}
\def\MT@define@set@keys#1{%
  \MT@define@set@key@{encoding}{#1}%
  \MT@define@set@key@{family}{#1}%
  \MT@define@set@key@{series}{#1}%
  \MT@define@set@key@{shape}{#1}%
  \MT@define@set@key@size{#1}%
  \MT@define@set@key@font{#1}%
}
%    \end{macrocode}
%\end{macro}
%\begin{macro}{\MT@define@set@key@}
% \meta{\#1} = font axis, \meta{\#2} = feature.
%\changes{v1.8}{2005/04/16}{use comma lists instead of token lists}
%    \begin{macrocode}
\def\MT@define@set@key@#1#2{%
  \csname MT@#2list@#1@\MT@curr@set@name\endcsname
  \define@key{MT@#2@set}{#1}[]{%
    \global\MT@let@nc{MT@#2list@#1@\MT@curr@set@name}\@empty
    \MT@map@clist@n{##1}{%
%    \end{macrocode}
% Get rid of spaces.
%    \begin{macrocode}
      \KV@@sp@def\MT@val{####1}%
      \MT@get@highlevel{#1}%
      \MT@make@string\MT@val
      \expandafter\MT@xadd
        \csname MT@#2list@#1@\MT@curr@set@name\endcsname{\MT@val,}%
    }%
%<debug>\MT@dinfo@nl{1}{-- #1: \@nameuse{MT@#2list@#1@\MT@curr@set@name}}%
  }%
}
%    \end{macrocode}
%\end{macro}
%\begin{macro}{\MT@get@highlevel}
% Saying, for instance, `|family=rm*|' or `|shape=bf*|' will lead to
% \cmd{\rmdefault} resp. \cmd{\bfdefault} being expanded/protruded.
%    \begin{macrocode}
\def\MT@get@highlevel#1{%
  \expandafter\MT@test@ast\MT@val*\@nil{%
%    \end{macrocode}
% And `|family = *|' will become \cmd{\familydefault}.
%    \begin{macrocode}
    \MT@ifempty\@tempa{\def\@tempa{#1}}\relax
    \edef\MT@val{\csname \@tempa default\endcsname}%
%    \end{macrocode}
% Beware that \cmd{\rmdefault} etc. might change after this package has been
% loaded! Therefore, we check (once for each characteristic/list) at the
% beginning of the document.
%\changes{v1.2}{2004/09/26}{check whether defaults have changed}
%\changes{v1.5}{2004/12/02}{don't test defaults if called after begin document}
%    \begin{macrocode}
    \ifx\@nodocument\relax \else
      \expandafter\ifx
          \csname MT@check@\MT@curr@set@name @\@tempa\endcsname\@empty
      \else
        \global\MT@edef@n{MT@\MT@curr@set@name @\@tempa @default}{\MT@val}%
        \edef\x{{\MT@curr@set@name}{\@tempa}}%
        \MT@exp@one@n\AtBeginDocument{%
          \expandafter\MT@check@default\x
        }%
        \global\MT@let@nc{MT@check@\MT@curr@set@name @\@tempa}\@empty
      \fi
    \fi
  }%
}
%    \end{macrocode}
%\end{macro}
%\begin{macro}{\MT@test@ast}
% Test whether last character is an asterisk.
%\changes{v1.7}{2005/03/10}{make it simpler}
%    \begin{macrocode}
\def\MT@test@ast#1*#2\@nil{%
  \def\@tempa{#1}%
  \MT@ifempty{#2}%
    \@gobble
    \@firstofone
}
%    \end{macrocode}
%\end{macro}
%\begin{macro}{\MT@check@default}
%\changes{v1.2}{2004/09/26}{new macro}
%    \begin{macrocode}
\def\MT@check@default#1#2{%
  \MT@let@cn\@tempa{MT@#1@#2@default}%
  \edef\@tempb{\csname #2default\endcsname}%
  \ifx\@tempa\@tempb \else
    \MT@warning@nl{%
      \expandafter\noexpand\csname #2default\endcsname
      has changed (`\@tempa' <> `\@tempb')!\MessageBreak
      This might affect the `#1' font set.\MessageBreak
      Please make all relevant font changes *before*\MessageBreak
      loading the `microtype' package}%
  \fi
}
%    \end{macrocode}
%\end{macro}
%\begin{macro}{\MT@define@set@key@size}
% |size| requires special treatment.
%    \begin{macrocode}
\def\MT@define@set@key@size#1{%
  \define@key{MT@#1@set}{size}[]{%
    \MT@map@clist@n{##1}{%
      \KV@@sp@def\MT@val{####1}%
      \expandafter\MT@get@range\MT@val--\@nil
      \ifx\MT@val\relax \else
        \expandafter\MT@xadd
            \csname MT@#1list@size@\MT@curr@set@name\endcsname
            {{{\MT@lower}{\MT@upper}\relax}}%
      \fi
    }%
%<debug>\MT@dinfo@nl{1}{-- size: \@nameuse{MT@#1list@size@\MT@curr@set@name}}%
  }%
}
%    \end{macrocode}
%\end{macro}
%\changes{v1.7}{2005/03/11}{allow specification of size ranges
%                           (suggested by Andreas B\"uhmann)} ^^A <andreas.buehmann\at web.de>
%                                                             ^^A private mail, 10/3/2005
% Font sizes may also be specified as ranges. This has been requested by Andreas
% B\"uhmann, who has also offered valuable help in implementing this. Now, it
% is for instance possible to set up different lists for fonts with optical
% sizes. (The MinionPro project is trying to do this for the OpenType version
% of Adobe's Minion. See \url{http://developer.berlios.de/projects/minionpro/}.)%
%
%\begin{macro}{\MT@get@range}
%\begin{macro}{\MT@upper}
%\begin{macro}{\MT@lower}
% Ranges will be stored as triples of
% |{|\meta{lower bound}|}{|\meta{upper bound}|}{|\meta{list name}|}|.
% For simple sizes, the upper boundary is -1.
%    \begin{macrocode}
\def\MT@get@range#1-#2-#3\@nil{%
  \MT@ifempty{#1}{%
    \MT@ifempty{#2}{%
      \let\MT@val\relax
    }{%
      \def\MT@lower{0}%
      \def\MT@val{#2}%
      \MT@get@size
      \edef\MT@upper{\MT@val}%
    }%
  }{%
    \def\MT@val{#1}%
    \MT@get@size
    \ifx\MT@val\relax \else
      \edef\MT@lower{\MT@val}%
      \MT@ifempty{#2}{%
        \MT@ifempty{#3}{%
          \def\MT@upper{-1}%
        }{%
          \def\MT@upper{2048}%
        }%
      }{%
        \def\MT@val{#2}%
        \MT@get@size
        \ifx\MT@val\relax \else
          \MT@ifgt\MT@lower\MT@val{%
            \MT@warning{%
              Invalid size range (\MT@lower\space > \MT@val) in font set
              `\MT@curr@set@name'.\MessageBreak Swapping sizes}%
            \edef\MT@upper{\MT@lower}%
            \edef\MT@lower{\MT@val}%
          }{%
            \edef\MT@upper{\MT@val}%
          }%
          \MT@ifeq\MT@lower\MT@upper{%
            \def\MT@upper{-1}%
          }\relax
        \fi
      }%
    \fi
  }%
}
%    \end{macrocode}
%\end{macro}
%\end{macro}
%\end{macro}
%\begin{macro}{\MT@get@size}
% Translate a size selection command and normalize it.
%    \begin{macrocode}
\def\MT@get@size{%
%    \end{macrocode}
% A single star would mean \cs{sizedefault}, which doesn't exist, so we define
% it to be \cmd{\normalsize}.
%    \begin{macrocode}
  \if*\MT@val\relax
    \def\@tempa{\normalsize}%
  \else
    \MT@let@cn\@tempa{\MT@val}%
  \fi
  \ifx\@tempa\relax \else
%    \end{macrocode}
% The \pkg{relsize} solution of parsing \cmd{\@setfontsize} does not work with the
% \pkg{ams*} classes, among others. I hope my hijacking doesn't do any harm.
%\changes{v1.2}{2004/09/26}{hijack \cmd{\set@fontsize} instead of \cmd{\@setfontsize}}
% We redefine \cmd{\set@fontsize}, and not \cmd{\@setfontsize} because some classes
% might define the size selection commands by simply using \cmd{\fontsize}
% (\eg\ the \pkg{a0poster} class).
%    \begin{macrocode}
    \begingroup
      \def\set@fontsize##1##2##3##4\@nil{\gdef\MT@val{##2}}%
      \@tempa\@nil
    \endgroup
  \fi
%    \end{macrocode}
% Test whether we finally got a number or dimension so that we can strip the
% `|pt|' (\cmd{\@defaultunits} and \cmd{\strip@pt} are kernel macros).
%\changes{v1.2}{2004/09/27}{additional magic to catch some errors}
%\changes{v1.7}{2005/03/15}{comparison with 1 to allow size smaller than 1
%                           (suggested by Andreas B\"uhmann)} ^^A <andreas.buehmann\at web.de>
%                                                             ^^A private mail, 15/03/2005
%    \begin{macrocode}
  \MT@ifdimen\MT@val{%
    \@defaultunits\@tempdima\MT@val pt\relax\@nnil
    \edef\MT@val{\strip@pt\@tempdima}%
  }{%
    \MT@warning{Could not parse font size `\MT@val'\MessageBreak
                in font set `\MT@curr@set@name'}%
    \let\MT@val\relax
  }%
}
%    \end{macrocode}
%\end{macro}
%\changes{v1.9}{2005/07/13}{\cs{DeclareMicrotypeSet}: new key: \texttt{font}}
%\begin{macro}{\MT@define@set@key@font}
%    \begin{macrocode}
\def\MT@define@set@key@font#1{%
  \define@key{MT@#1@set}{font}[]{%
    \MT@map@clist@n{##1}{%
      \KV@@sp@def\MT@val{####1}%
      \expandafter\MT@get@font\MT@val/////\@nil
      \expandafter\MT@xadd
        \csname MT@#1list@font@\MT@curr@set@name\endcsname
        {\csname\MT@val\endcsname,}%
    }%
%<debug>\MT@dinfo@nl{1}{-- font: \@nameuse{MT@#1list@font@\MT@curr@set@name}}%
  }%
}
%    \end{macrocode}
%\end{macro}
%\begin{macro}{\MT@get@font}
% Translate any asterisks.
%    \begin{macrocode}
\def\MT@get@font#1/#2/#3/#4/#5/#6\@nil{%
  \MT@ifempty{#1#2#3#4#5}\relax{%
    \let\@tempb\@empty
    \def\MT@temp{#1/#2/#3/#4/#5}%
    \MT@get@axis{encoding}{#1}%
    \MT@get@axis{family}{#2}%
    \MT@get@axis{series}{#3}%
    \MT@get@axis{shape}{#4}%
    \MT@ifempty{#5}{%
      \MT@warning{size axis is empty in font specification\MessageBreak
        `\MT@temp'. Using \string\normalsize\space instead}%
      \def\MT@val{*}%
    }{%
      \def\MT@val{#5}%
    }%
    \MT@get@size
    \ifx\MT@val\relax\def\MT@val{0}\fi
    \edef\MT@val{\expandafter\@gobble\@tempb/\MT@val}%
  }%
}
%    \end{macrocode}
%\end{macro}
%\begin{macro}{\MT@get@axis}
%    \begin{macrocode}
\def\MT@get@axis#1#2{%
  \def\MT@val{#2}%
  \MT@get@highlevel{#1}%
  \MT@ifempty\MT@val{%
    \MT@warning{#1 axis is empty in font specification\MessageBreak
      `\MT@temp'. Using `\csname #1default\endcsname' instead}%
    \edef\@tempb{\@tempb/\csname #1default\endcsname}%
  }{%
    \edef\@tempb{\@tempb/\MT@val}%
  }%
}
%    \end{macrocode}
%\end{macro}
% We have finally assembled all pieces to define \cs{DeclareMicrotypeSet}'s
% keys.
%    \begin{macrocode}
\MT@define@set@keys{pr}
\MT@define@set@keys{ex}
%<*beta>
\MT@define@set@keys{sp}
\MT@define@set@keys{kn}
%</beta>
%    \end{macrocode}
% It is also used for \cs{DisableLigatures}.
%    \begin{macrocode}
\MT@define@set@keys{nl}
%    \end{macrocode}
%\begin{macro}{\UseMicrotypeSet}
% To use a particular set we simply redefine |MT@|\meta{feature}|@setname|.
% If the optional argument is empty, set names for all features will be
% redefined.
%\changes{v1.1}{2004/09/20}{remove spaces around first argument}
%    \begin{macrocode}
\renewcommand*\UseMicrotypeSet[2][]{%
  \KV@@sp@def\@tempa{#1}%
  \MT@ifempty\@tempa{%
    \MT@use@set{pr}{#2}%
    \MT@use@set{ex}{#2}%
%<*beta>
    \MT@use@set{sp}{#2}%
    \MT@use@set{kn}{#2}%
%</beta>
  }{%
    \MT@map@clist@c\@tempa{%
      {\KV@@sp@def\@tempa{##1}%
       \MT@ifempty\@tempa\relax{%
         \MT@exp@one@n\MT@use@set{\csname MT@rbba@\@tempa\endcsname}{#2}}}%
    }%
  }%
}
%    \end{macrocode}
%\end{macro}
%\begin{macro}{\MT@pr@setname}
%\begin{macro}{\MT@ex@setname}
%\begin{macro}{\MT@use@set}
% Only use sets that have been declared.
%\changes{v1.1}{2004/09/20}{remove spaces around set name}
%\changes{v1.4b}{2004/11/25}{don't use undeclared font sets}
%\changes{v1.6}{2005/01/19}{retain current set if new set is undeclared}
%\changes{v1.8}{2005/05/15}{bug fix: remove braces in first line}
%    \begin{macrocode}
\def\MT@use@set#1#2{%
  \KV@@sp@def\@tempa{#2}%
  \MT@ifdefined@n{MT@#1@set@@\@tempa}{%
    \global\MT@edef@n{MT@#1@setname}{\@tempa}%
    \MT@info{Using \@nameuse{MT@abbr@#1} set `\@tempa'}%
  }{%
    \MT@ifdefined@n{MT@#1@setname}\relax{%
      \global\MT@edef@n{MT@#1@setname}{\@nameuse{MT@default@#1@set}}%
    }%
    \MT@warning{%
      The \@nameuse{MT@abbr@#1} set `\@tempa' is undeclared.\MessageBreak
      Using set `\@nameuse{MT@#1@setname}' instead}%
  }%
}
%    \end{macrocode}
%\end{macro}
%\end{macro}
%\end{macro}
%\begin{macro}{\DeclareMicrotypeSetDefault}
%\changes{v1.8}{2005/05/15}{new macro: set fall back font set}
% This command can be used in the main configuration file to declare the
% default font set, in case no set is specified in the package options.
%    \begin{macrocode}
\renewcommand*\DeclareMicrotypeSetDefault[2][]{%
  \KV@@sp@def\@tempa{#1}%
  \MT@ifempty\@tempa{%
    \MT@set@default@set{pr}{#2}%
    \MT@set@default@set{ex}{#2}%
%<*beta>
    \MT@set@default@set{sp}{#2}%
    \MT@set@default@set{kn}{#2}%
%</beta>
  }{%
    \MT@map@clist@c\@tempa{%
      {\KV@@sp@def\@tempa{##1}%
       \MT@ifempty\@tempa\relax{%
         \MT@exp@one@n\MT@set@default@set
            {\csname MT@rbba@\@tempa\endcsname}{#2}}}%
    }%
  }%
}
%    \end{macrocode}
%\end{macro}
%\begin{macro}{\MT@default@pr@set}
%\begin{macro}{\MT@default@ex@set}
%\begin{macro}{\MT@default@kn@set}
%\begin{macro}{\MT@default@sp@set}
%\begin{macro}{\MT@set@default@set}
%    \begin{macrocode}
\def\MT@set@default@set#1#2{%
  \KV@@sp@def\@tempa{#2}%
  \MT@ifdefined@n{MT@#1@set@@\@tempa}{%
%<debug>\MT@dinfo{1}{declaring default \@nameuse{MT@abbr@#1} set `\@tempa'}%
    \global\MT@edef@n{MT@default@#1@set}{\@tempa}%
  }{%
    \MT@warning{%
      The \@nameuse{MT@abbr@#1} set `\@tempa' is not declared.\MessageBreak
      Cannot make it the default set. Using set\MessageBreak `all' instead}%
    \global\MT@edef@n{MT@default@#1@set}{all}%
  }%
}
%    \end{macrocode}
%\end{macro}
%\end{macro}
%\end{macro}
%\end{macro}
%\end{macro}
%\begin{macro}{\DeclareMicrotypeAlias}
% This can be used to set an alias name for a font, so that the file (and the
% settings) for the aliased font will be loaded.
%\changes{v1.5}{2004/12/03}{remove spaces around arguments}
%\changes{v1.8}{2005/05/09}{warning when overriding an alias font}
%    \begin{macrocode}
\renewcommand*\DeclareMicrotypeAlias[2]{%
  \KV@@sp@def\@tempa{#1}%
  \KV@@sp@def\@tempb{#2}%
  \MT@make@string\@tempb
  \MT@ifdefined@n{MT@\@tempa @alias}{%
    \MT@warning{Alias font family `\@tempb' will override
      alias `\@nameuse{MT@\@tempa @alias}'\MessageBreak
      for font family `\@tempa'}}\relax
  \global\MT@edef@n{MT@\@tempa @alias}{\@tempb}%
%    \end{macrocode}
% If we encounter this command while a font is being set up, we also set the
% alias for the current font so that if \cs{DeclareMicrotypeAlias} has been
% issued inside a configuration file, the configuration file for the alias font
% will be loaded, too.
%\changes{v1.7}{2005/03/23}{may also be used inside configuration files}
%    \begin{macrocode}
  \MT@ifdefined@c\MT@family{%
%<debug>\MT@dinfo{1}{Activating alias font `\@tempb' for `\MT@family'}%
    \global\let\MT@familyalias\@tempb
  }\relax
}
%    \end{macrocode}
%\end{macro}
%\begin{macro}{\LoadMicrotypeFile}
%\changes{v1.7}{2005/03/22}{new macro (suggested by Andreas B\"uhmann)}
%                                              ^^A <andreas.buehmann\at web.de>
%                                              ^^A private mail, 21/3/2005
% May be used to load a configuration file manually.
%    \begin{macrocode}
\def\LoadMicrotypeFile#1{%
  \KV@@sp@def\@tempa{#1}%
  \MT@make@string\@tempa
  \MT@exp@one@n\MT@in@clist\@tempa\MT@file@list
  \ifMT@inlist@
    \MT@vinfo{... Configuration file mt-\@tempa.cfg already loaded}%
  \else
    \MT@xadd\MT@file@list{\@tempa,}%
    \MT@begin@catcodes
    \InputIfFileExists{mt-\@tempa.cfg}{%
      \MT@vinfo{... Loading configuration file mt-\@tempa.cfg}%
    }{%
      \MT@warning{... Configuration file mt-\@tempa.cfg\MessageBreak
                      does not exist}%
    }%
    \MT@end@catcodes
  \fi
}
%    \end{macrocode}
%\end{macro}
%\begin{macro}{\DisableLigatures}
%\changes{v1.9}{2005/07/11}{new macro: disable ligatures (requires \pdftex\ 1.30)}
% This is really simple now: We can re-use the set definitions of
% \cs{DeclareMicrotypeSet}; there can only be one set, which we'll call `no
% ligatures'.
%    \begin{macrocode}
\ifnum\MT@pdftex@no > 4
  \renewcommand*\DisableLigatures[1]{%
    \MT@noligaturestrue
    \MT@declare@sets{nl}{no ligatures}{#1}%
    \gdef\MT@nl@setname{no ligatures}%
  }
\else
%    \end{macrocode}
% If \pdftex\ is too old, we issue a warning and neutralize the command.
%    \begin{macrocode}
  \renewcommand*\DisableLigatures[1]{%
    \MT@warning{Disabling ligatures of a font is only possible\MessageBreak
      with pdftex version 1.30 or later.\MessageBreak
      Ignoring \string\DisableLigatures}%
    \let\DisableLigatures\@gobble
  }
\fi
%    \end{macrocode}
%\end{macro}
%
%\subsubsection{Interaction with \pkg{babel}}
%
%\begin{macro}{\DeclareMicrotypeBabelHook}
% Declare the context that should be loaded when a \pkg{babel} language
% is selected.
%\changes{v2.0}{2005/10/05}{new command: interaction with \pkg{babel}}
%    \begin{macrocode}
%<*beta>
\def\DeclareMicrotypeBabelHook#1#2{%
  \MT@map@clist@n{#1}{%
    \KV@@sp@def\@tempa{##1}%
    \global\MT@def@n{MT@babel@\@tempa}{#2}%
  }%
}
\@onlypreamble{\DeclareMicrotypeBabelHook}
%</beta>
%    \end{macrocode}
%\end{macro}
%    \begin{macrocode}
%</package>
%    \end{macrocode}
%
%\subsubsection{Declarations}
%\GeneralChanges{Font Sets}
%\NoIndexing
%
% We now set up some default sets (in the main configuration file).
%\changes{v1.2}{2004/10/02}{new: \texttt{allmath} and \texttt{basicmath}}
%\changes{v1.8}{2005/05/25}{add \U\ encoding to \texttt{allmath}}
%\changes{v1.9}{2005/08/17}{add \otIV\ encoding to text sets}
%\changes{v1.9}{2005/08/17}{add \tV\ encoding to text sets}
%
%    \begin{macrocode}
%<*config&m-t>

%%% ----------------------------------------------------------------------
%%% FONT SETS

\DeclareMicrotypeSet{all}
   { }

\DeclareMicrotypeSet{allmath}
   { encoding = {T1,LY1,OT1,OT4,T5,TS1,OML,OMS,U} }

\DeclareMicrotypeSet{alltext}
   { encoding = {T1,LY1,OT1,OT4,T5,TS1} }

\DeclareMicrotypeSet{basicmath}
   { encoding = {T1,LY1,OT1,OT4,T5,OML,OMS},
     family   = {rm*,sf*},
     series   = {m},
     size     = {normalsize,footnotesize,small,large}
   }

\DeclareMicrotypeSet{basictext}
   { encoding = {T1,LY1,OT1,OT4,T5},
     family   = {rm*,sf*},
     series   = {m},
     size     = {normalsize,footnotesize,small,large}
   }

\DeclareMicrotypeSet{normalfont}
   { font = */*/*/*/* }

%    \end{macrocode}
% The default sets.
%    \begin{macrocode}
%%% ----------------------------------------------------------------------
%%% DEFAULT SETS

\DeclareMicrotypeSetDefault[protrusion]{alltext}
\DeclareMicrotypeSetDefault[expansion] {basictext}
%<*beta>
\DeclareMicrotypeSetDefault[spacing]   {basictext}
\DeclareMicrotypeSetDefault[kerning]   {alltext}
%</beta>

%    \end{macrocode}
% The Latin Modern fonts, the virtual fonts from the \pkg{ae} and \pkg{zefonts},
% and the \pkg{eco} and \pkg{hfoldsty} packages (oldstyle numerals) all inherit
% the (basic) settings from Computer Modern Roman. Some of them are in part
% overwritten later.
%\changes{v1.2}{2004/10/03}{declare \texttt{cmr} as an alias for \texttt{cmor}}
%\changes{v1.3}{2004/10/22}{declare \texttt{cmr} as an alias for
%                          \texttt{aer}, \texttt{zer} and \texttt{hfor}}
% The packages \pkg{pxfonts} and \pkg{txfonts} fonts inherit Palatino and Times
% settings respectively.
%\changes{v1.8}{2005/04/19}{declare \texttt{ppl} and \texttt{ptm} as alias fonts
%                           for \texttt{pxr} resp. \texttt{txr}}
%\changes{v1.9}{2005/08/16}{declare \texttt{qpl} and \texttt{qtm} (\pkg{qfonts})
%                           as alias fonts for \texttt{ppl} resp. \texttt{ptm}}
%    \begin{macrocode}
%%% ----------------------------------------------------------------------
%%% FONT ALIASES

\DeclareMicrotypeAlias{lmr} {cmr}  % lmodern
\DeclareMicrotypeAlias{aer} {cmr}  % ae
\DeclareMicrotypeAlias{zer} {cmr}  % zefonts
\DeclareMicrotypeAlias{cmor}{cmr}  % eco
\DeclareMicrotypeAlias{hfor}{cmr}  % hfoldsty
\DeclareMicrotypeAlias{pxr} {ppl}  % pxfonts
\DeclareMicrotypeAlias{qpl} {ppl}  % qfonts/QuasiPalatino
\DeclareMicrotypeAlias{txr} {ptm}  % txfonts
\DeclareMicrotypeAlias{qtm} {ptm}  % qfonts/QuasiTimes

%    \end{macrocode}
%\todo{check Times variants}
% More Times variants, to be checked: |pns|, |mns| (TimesNewRomanPS); |mnt|
% (TimesNewRomanMT), |mntx| (TimesNRExpertMT); |mtm| (TimesSmallTextMT); |pte|
% (Times\-Europa); |ptt|, |pttj| (TimesTen); TimesEighteen; TimesModernEF.
%    \begin{macrocode}
%<*beta>
%    \end{macrocode}
% Interaction with \pkg{babel}.
%    \begin{macrocode}
%%% ----------------------------------------------------------------------
%%% INTERACTION WITH THE `babel' PACKAGE

\DeclareMicrotypeBabelHook
   {french,francais}
   {kerning=french, spacing=}

\DeclareMicrotypeBabelHook
   {english,american,USenglish,british,UKenglish}
   {kerning=, spacing=nonfrench}

\DeclareMicrotypeBabelHook
   {turkish}
   {kerning=turkish, spacing=}

%</beta>
%</config&m-t>
%    \end{macrocode}
%
%\GeneralChanges!
%\Indexing
%
%\subsubsection{Fine Tuning}
%
% The macros \cs{SetExpansion} and \cs{SetProtrusion} provide a similar
% interface for setting the character protrusion resp. expansion factors for a
% set of fonts.
%
%\begin{macro}{\SetProtrusion}
%\begin{macro}{\MT@pr@c@name}
%\begin{macro}{\MT@load}
%\begin{macro}{\MT@extra@factor}
%\begin{macro}{\MT@extra@preset}
%\begin{macro}{\MT@extra@unit}
%\begin{macro}{\MT@extra@context}
% This macro accepts three arguments: [options,] set of font characteristics and
% list of character protrusion factors.
%
% A new macro called |\MT@pr@c@|\meta{name} will be defined to be \meta{\#3}
% (i.e. the list of characters, not expanded).
%    \begin{macrocode}
%<*package>
\renewcommand*\SetProtrusion[2][]{%
  \let\MT@pr@c@name\@undefined
  \let\MT@load\@undefined
  \let\MT@extra@factor\@undefined
  \let\MT@extra@unit\@undefined
  \let\MT@extra@preset\@undefined
  \let\MT@extra@context\@empty
%    \end{macrocode}
% Parse the optional first argument:
%    \begin{macrocode}
  \setkeys{MT@pr@c}{#1}%
%    \end{macrocode}
% If the user hasn't specified a name, we will create one.
%    \begin{macrocode}
  \MT@get@codes@name{pr}%
  \MT@set@pr@opt
%<debug>\MT@dinfo{1}{creating protrusion list `\MT@pr@c@name'}%
  \def\MT@permutelist{pr@c}%
  \setkeys{MT@pr@c}{#2}%
%    \end{macrocode}
%\begin{macro}{\MT@permutelist}
% We have parsed the second argument, and can now define macros for all
% permutations of the font characteristics to point to
% |\MT@pr@c@|\meta{name},~\dots
%    \begin{macrocode}
  \MT@permute
%    \end{macrocode}
% \dots~which we can now define to be \meta{\#3}.
% We want the catcodes to be correct even if this is called in the preamble.
%    \begin{macrocode}
  \MT@begin@catcodes
  \MT@set@pr@list
}
%    \end{macrocode}
%\end{macro}
%\end{macro}
%\end{macro}
%\end{macro}
%\end{macro}
%\end{macro}
%\end{macro}
%\end{macro}
%\begin{macro}{\MT@set@pr@list}
% Here, as elsewhere, we have to make the definitions global, since they will
% occur inside a group.
%    \begin{macrocode}
\def\MT@set@pr@list#1{%
  \global\MT@def@n{MT@pr@c@\MT@pr@c@name}{#1}%
  \MT@end@catcodes
}
%    \end{macrocode}
%\end{macro}
%\begin{macro}{\SetExpansion}
%\begin{macro}{\MT@ex@c@name}
%\begin{macro}{\MT@load}
%\begin{macro}{\MT@extra@factor}
%\begin{macro}{\MT@extra@stretch}
%\begin{macro}{\MT@extra@shrink}
%\begin{macro}{\MT@extra@step}
%\begin{macro}{\MT@extra@auto}
%\begin{macro}{\MT@extra@preset}
%\begin{macro}{\MT@extra@context}
% \cs{SetExpansion} only differs in that it allows some extra options
% (|stretch|, |shrink|, |step|, |auto|).
%    \begin{macrocode}
\renewcommand*\SetExpansion[2][]{%
  \let\MT@ex@c@name\@undefined
  \let\MT@load\@undefined
  \let\MT@extra@factor\@undefined
  \let\MT@extra@stretch\@undefined
  \let\MT@extra@shrink\@undefined
  \let\MT@extra@step\@undefined
  \let\MT@extra@auto\@undefined
  \let\MT@extra@preset\@undefined
  \let\MT@extra@context\@empty
  \setkeys{MT@ex@c}{#1}%
  \MT@get@codes@name{ex}%
  \MT@set@ex@opt
%<debug>\MT@dinfo{1}{creating expansion list `\MT@ex@c@name'}%
  \def\MT@permutelist{ex@c}%
  \setkeys{MT@ex@c}{#2}%
  \MT@permute
  \MT@begin@catcodes
  \MT@set@ex@list
}
%    \end{macrocode}
%\end{macro}
%\end{macro}
%\end{macro}
%\end{macro}
%\end{macro}
%\end{macro}
%\end{macro}
%\end{macro}
%\end{macro}
%\end{macro}
%\begin{macro}{\MT@set@ex@list}
% Same story.
%    \begin{macrocode}
\def\MT@set@ex@list#1{%
  \global\MT@def@n{MT@ex@c@\MT@ex@c@name}{#1}%
  \MT@end@catcodes
}
%    \end{macrocode}
%\end{macro}
%    \begin{macrocode}
%<*beta>
%    \end{macrocode}
%\begin{macro}{\SetExtraSpacing}
%\begin{macro}{\MT@sp@c@name}
%\begin{macro}{\MT@load}
%\begin{macro}{\MT@extra@factor}
%\begin{macro}{\MT@extra@unit}
%\begin{macro}{\MT@extra@preset}
%\begin{macro}{\MT@extra@context}
%    \begin{macrocode}
\renewcommand*\SetExtraSpacing[2][]{%
  \let\MT@sp@c@name\@undefined
  \let\MT@load\@undefined
  \let\MT@extra@factor\@undefined
  \let\MT@extra@unit\@undefined
  \let\MT@extra@preset\@undefined
  \let\MT@extra@context\@empty
  \setkeys{MT@sp@c}{#1}%
  \MT@get@codes@name{sp}%
  \MT@set@sp@opt
%<debug>\MT@dinfo{1}{creating space list `\MT@sp@c@name'}%
  \def\MT@permutelist{sp@c}%
  \setkeys{MT@sp@c}{#2}%
  \MT@permute
  \MT@begin@catcodes
  \MT@set@sp@list
}
%    \end{macrocode}
%\end{macro}
%\end{macro}
%\end{macro}
%\end{macro}
%\end{macro}
%\end{macro}
%\end{macro}
%\begin{macro}{\MT@set@sp@list}
%    \begin{macrocode}
\def\MT@set@sp@list#1{%
  \global\MT@def@n{MT@sp@c@\MT@sp@c@name}{#1}%
  \MT@end@catcodes
}
%    \end{macrocode}
%\end{macro}
%\begin{macro}{\SetExtraKerning}
%\begin{macro}{\MT@kn@c@name}
%\begin{macro}{\MT@load}
%\begin{macro}{\MT@extra@factor}
%\begin{macro}{\MT@extra@unit}
%\begin{macro}{\MT@extra@preset}
%\begin{macro}{\MT@extra@context}
%    \begin{macrocode}
\renewcommand*\SetExtraKerning[2][]{%
  \let\MT@kn@c@name\@undefined
  \let\MT@load\@undefined
  \let\MT@extra@factor\@undefined
  \let\MT@extra@unit\@undefined
  \let\MT@extra@preset\@undefined
  \let\MT@extra@context\@empty
  \setkeys{MT@kn@c}{#1}%
  \MT@get@codes@name{kn}%
  \MT@set@kn@opt
%<debug>\MT@dinfo{1}{creating kerning list `\MT@kn@c@name'}%
  \def\MT@permutelist{kn@c}%
  \setkeys{MT@kn@c}{#2}%
  \MT@permute
  \MT@begin@catcodes
  \MT@set@kn@list
}
%    \end{macrocode}
%\end{macro}
%\end{macro}
%\end{macro}
%\end{macro}
%\end{macro}
%\end{macro}
%\end{macro}
%\begin{macro}{\MT@set@kn@list}
%    \begin{macrocode}
\def\MT@set@kn@list#1{%
  \global\MT@def@n{MT@kn@c@\MT@kn@c@name}{#1}%
  \MT@end@catcodes
}
%    \end{macrocode}
%\end{macro}
%    \begin{macrocode}
%</beta>
%    \end{macrocode}
%\begin{macro}{\MT@get@codes@name}
% Simply use a roman number as the list name if the user didn't bother creating
% one.
%    \begin{macrocode}
\def\MT@get@codes@name#1{%
  \MT@ifdefined@n{MT@#1@c@name}{%
    \MT@ifdefined@n{MT@#1@c@\csname MT@#1@c@name\endcsname}{%
      \MT@warning{Redefining list `\@nameuse{MT@#1@c@name}'}%
    }\relax
  }{%
    \@tempcnta=\@ne
    \MT@while@num{\@tempcnta > \z@}{%
      \MT@ifdefined@n{MT@#1@c@#1-\romannumeral\@tempcnta}{%
        \advance \@tempcnta \@ne
      }{%
        \MT@edef@n{MT@#1@c@name}{#1-\romannumeral\@tempcnta}%
        \@tempcnta=\z@
      }%
    }%
  }%
  \MT@let@cn\MT@curr@set@name{MT@#1@c@name}%
%    \end{macrocode}
%\changes{v1.3}{2004/10/27}{bug fix: specifying \texttt{load} option does no
%                           longer require to give a \texttt{name}, too}
% Now that we know the name, we can cater for any set to be loaded by this list.
%    \begin{macrocode}
  \MT@ifdefined@c\MT@load{%
    \global\MT@let@nc{MT@#1@c@\MT@curr@set@name load}\MT@load
  }\relax
}
%    \end{macrocode}
%\end{macro}
%\begin{macro}{\MT@set@pr@opt}
% Three extra option for protrusion: |factor|, |unit| and |preset|.
%    \begin{macrocode}
\def\MT@set@pr@opt{%
  \MT@set@opt@{pr}{factor}%
  \MT@set@opt@{pr}{unit}%
  \MT@set@opt@{pr}{preset}%
}
%    \end{macrocode}
%\end{macro}
%\begin{macro}{\MT@set@ex@opt}
% The extra options to \cs{SetExpansion} also have to be dealt with only after
% we know the name.
%\changes{v1.4}{2004/11/10}{bug fix: specifying extra options does no
%                           longer require to give a \texttt{name}, too}
%    \begin{macrocode}
\def\MT@set@ex@opt{%
  \MT@ifdefined@c\MT@extra@factor{%
    \ifnum\MT@extra@factor>\@m
      \MT@warning@nl{Expansion factor \number\MT@extra@factor\space too
        large in list\MessageBreak `\MT@ex@c@name'. Setting it to the
        maximum of 1000}%
      \let\MT@extra@factor\@m
    \fi
    \global\MT@let@nc{MT@ex@c@\MT@ex@c@name @factor}\MT@extra@factor
  }\relax
  \MT@set@opt@{ex}{stretch}%
  \MT@set@opt@{ex}{shrink}%
  \MT@set@opt@{ex}{step}%
  \MT@set@opt@{ex}{auto}%
  \MT@set@opt@{ex}{preset}%
}
%    \end{macrocode}
%\end{macro}
%    \begin{macrocode}
%<*beta>
%    \end{macrocode}
%\begin{macro}{\MT@set@sp@opt}
%    \begin{macrocode}
\def\MT@set@sp@opt{%
  \MT@set@opt@{sp}{factor}%
  \MT@set@opt@{sp}{unit}%
  \MT@set@opt@{sp}{preset}%
}
%    \end{macrocode}
%\end{macro}
%\begin{macro}{\MT@set@kn@opt}
%    \begin{macrocode}
\def\MT@set@kn@opt{%
  \MT@set@opt@{kn}{factor}%
  \MT@set@opt@{kn}{unit}%
  \MT@set@opt@{kn}{preset}%
}
%    \end{macrocode}
%\end{macro}
%    \begin{macrocode}
%</beta>
%    \end{macrocode}
%\begin{macro}{\MT@set@opt@}
%    \begin{macrocode}
\def\MT@set@opt@#1#2{%
  \MT@ifdefined@n{MT@extra@#2}{%
    \global\MT@let@nn{MT@#1@c@\csname MT@#1@c@name\endcsname @#2}{MT@extra@#2}%
  }\relax
}
%    \end{macrocode}
%\end{macro}
%\begin{macro}{\MT@define@code@key}
% Define the keys for expansion and protrusion character code lists.
%    \begin{macrocode}
\def\MT@define@code@key#1#2{%
  \define@key{MT@#2}{#1}[]{%
    \@tempcnta=\@ne
    \MT@map@clist@n{##1}{%
      \KV@@sp@def\MT@val{####1}%
%    \end{macrocode}
% Here, too, we allow for something like `|bf*|'. It will be expanded
% immediately.
%    \begin{macrocode}
      \MT@get@highlevel{#1}%
      \MT@edef@n{MT@temp#1\romannumeral\@tempcnta}{\MT@val}%
      \advance\@tempcnta \@ne
    }%
  }%
}
%    \end{macrocode}
%\end{macro}
%\begin{macro}{\MT@define@code@key@size}
%    \begin{macrocode}
\def\MT@define@code@key@size#1{%
  \define@key{MT@#1}{size}[]{%
    \MT@map@clist@n{##1}{%
      \KV@@sp@def\MT@val{####1}%
      \expandafter\MT@get@range\MT@val--\@nil
      \ifx\MT@val\relax \else
        \expandafter\MT@xadd
           \csname MT@tempsize\endcsname
           {{{\MT@lower}{\MT@upper}{\csname MT@#1@name\endcsname}}}%
      \fi
    }%
  }%
}
%    \end{macrocode}
%\end{macro}
%\changes{v1.9}{2005/07/13}{\cs{SetProtrusion} and \cs{SetExpansion}: new key: \texttt{font}}
%\begin{macro}{\MT@define@code@key@font}
%    \begin{macrocode}
\def\MT@define@code@key@font#1{%
  \define@key{MT@#1}{font}[]{%
    \MT@map@clist@n{##1}{%
      \KV@@sp@def\MT@val{####1}%
      \expandafter\MT@get@font@and@size\MT@val/////\@nil
      \global\MT@edef@n{MT@\MT@permutelist @\@tempb}%
        {\csname MT@\MT@permutelist @name\endcsname}%
      \expandafter\MT@xaddb
        \csname MT@\MT@permutelist @\@tempb @sizes\endcsname
        {{{\MT@val}{\m@ne}{\csname MT@#1@name\endcsname}}}%
    }%
  }%
}
%    \end{macrocode}
%\end{macro}
%\begin{macro}{\MT@get@font@and@size}
% Translate any asterisks and split off the size.
%    \begin{macrocode}
\def\MT@get@font@and@size#1/#2/#3/#4/#5/#6\@nil{%
  \MT@ifempty{#1#2#3#4#5}\relax{%
    \let\@tempb\@empty
    \def\MT@temp{#1/#2/#3/#4/#5}%
    \MT@get@axis{encoding}{#1}%
    \MT@get@axis{family}{#2}%
    \MT@get@axis{series}{#3}%
    \MT@get@axis{shape}{#4}%
%    \end{macrocode}
% Remove leading slash and append an asterisk for the size.
%    \begin{macrocode}
    \edef\@tempb{\expandafter\@gobble\@tempb/*}%
    \MT@ifempty{#5}{%
      \MT@warning{size axis is empty in font specification\MessageBreak
        `\MT@temp'. Using \string\normalsize\space instead}%
      \def\MT@val{*}%
    }{%
      \def\MT@val{#5}%
    }%
    \MT@get@size
  }%
}
%    \end{macrocode}
%\end{macro}
%\begin{macro}{\MT@declare@codes}
%    \begin{macrocode}
\def\MT@declare@codes#1{%
  \define@key{MT@#1@c}{name}[]{%
    \MT@ifempty{##1}\relax{%
      \MT@def@n{MT@#1@c@name}{##1}%
    }%
  }%
  \define@key{MT@#1@c}{load}[]{%
    \MT@ifempty{##1}\relax{%
      \def\MT@load{##1}%
    }%
  }%
  \define@key{MT@#1@c}{factor}[]{%
    \MT@ifempty{##1}\relax{%
      \def\MT@extra@factor{##1 }%
    }%
  }%
  \MT@define@code@key{encoding}{#1@c}%
  \MT@define@code@key{family}{#1@c}%
  \MT@define@code@key{series}{#1@c}%
  \MT@define@code@key{shape}{#1@c}%
  \MT@define@code@key@size{#1@c}%
  \MT@define@code@key@font{#1@c}%
  \define@key{MT@#1@c}{preset}[]{%
    \MT@ifempty{##1}\relax{%
      \def\MT@extra@preset{##1}%
    }%
  }%
%    \end{macrocode}
% Only one context is allowed. This might change in the future.
%    \begin{macrocode}
  \define@key{MT@#1@c}{context}[]{%
    \MT@ifempty{##1}\relax{%
      \def\MT@extra@context{##1}%
    }%
  }%
}
%    \end{macrocode}
%\end{macro}
%    \begin{macrocode}
\MT@declare@codes{pr}
\MT@declare@codes{ex}
%<*beta>
\MT@declare@codes{sp}
\MT@declare@codes{kn}
%</beta>
%    \end{macrocode}
% Protrusion codes may be relative to character width, or to any dimension.
%\changes{v1.8}{2005/04/14}{\cs{SetProtrusion}: new option: \texttt{unit}}
%\changes{v1.9}{2005/09/28}{\cs{SetProtrusion}: value `\texttt{relative}' renamed
%                           to `\texttt{character}' for option \texttt{unit}}
%    \begin{macrocode}
\define@key{MT@pr@c}{unit}[character]{%
  \let\MT@extra@unit\@empty
  \KV@@sp@def\@tempa{#1}%
  \MT@ifstreq\@tempa{relative}{%
    \MT@warning{Value `relative' for key `unit' is deprecated.\MessageBreak
      Use `unit=character' instead. For now, I'll do it\MessageBreak
      for you}%
    \def\@tempa{character}%
  }\relax
  \MT@ifstreq\@tempa{character}\relax{%
%    \end{macrocode}
% Test whether it's a dimension, but do not translate it into its final
% form here, since it may be font-specific.
%    \begin{macrocode}
    \MT@ifdimen\@tempa{%
      \let\MT@extra@unit\@tempa
    }{%
      \MT@warning{`\@tempa' is not a dimension.\MessageBreak
        Ignoring it and setting values relative to\MessageBreak
        character widths}%
    }%
  }%
}
%    \end{macrocode}
%\changes{v2.0}{2005/09/28}{\cs{SetExtraKerning}/\cs{SetExtraSpacing}:
%                           new key `\texttt{space}' for option \texttt{unit}}
%\begin{macro}{\MT@define@key@unit}
% Spacing and kerning codes may additionally be relative to space dimensions.
%    \begin{macrocode}
%<*beta>
\def\MT@define@key@unit#1{%
  \define@key{MT@#1@c}{unit}[space]{%
    \let\MT@extra@unit\@empty
    \KV@@sp@def\@tempa{##1}%
    \MT@ifstreq\@tempa{relative}{%
      \MT@warning{Value `relative' for key `unit' is deprecated.\MessageBreak
        Use `unit=character' instead. For now, I'll do it\MessageBreak
        for you}%
      \def\@tempa{character}%
    }\relax
    \MT@ifstreq\@tempa{character}\relax{%
      \let\MT@extra@unit\m@ne
      \MT@ifstreq\@tempa{space}\relax{%
        \MT@ifdimen\@tempa{%
          \let\MT@extra@unit\@tempa
        }{%
          \MT@warning{`\@tempa' is not a dimension.\MessageBreak
            Ignoring it and setting values relative to\MessageBreak
            width of space}%
        }%
      }%
    }%
  }%
}
%    \end{macrocode}
%\end{macro}
%    \begin{macrocode}
\MT@define@key@unit{sp}
\MT@define@key@unit{kn}
%</beta>
%    \end{macrocode}
%\begin{macro}{\MT@define@ex@c@key}
% The first argument to \cs{SetExpansion} accepts some more options.
%    \begin{macrocode}
\def\MT@define@ex@c@key#1{%
  \define@key{MT@ex@c}{#1}[]{%
    \MT@ifempty{##1}\relax{%
      \MT@ifnumber{##1}{%
%    \end{macrocode}
% A space terminates the number.
%    \begin{macrocode}
        \MT@def@n{MT@extra@#1}{##1 }%
      }{%
        \MT@warning{%
          Value `##1' for option `#1' is not a number.\MessageBreak
          Ignoring it}%
      }%
    }%
  }%
}
%    \end{macrocode}
%\end{macro}
%    \begin{macrocode}
\MT@define@ex@c@key{stretch}
\MT@define@ex@c@key{shrink}
\MT@define@ex@c@key{step}
\define@key{MT@ex@c}{auto}[true]{%
  \KV@@sp@def\@tempa{#1}%
  \csname if\@tempa\endcsname
%    \end{macrocode}
% Don't alter \cs{MT@extra@auto} for \pdftex\ version older than 1.20.
%\changes{v1.7}{2005/03/07}{\cs{SetExpansion}: bug fix: remove
%                           space after \texttt{autoexpand}}
%\changes{v1.7}{2005/03/07}{\cs{SetExpansion}: don't allow automatic
%                           expansion for old \pdftex\ versions}
%    \begin{macrocode}
    \ifnum\MT@pdftex@no > \thr@@
      \def\MT@extra@auto{autoexpand}%
    \else
      \MT@warning{pdfTeX too old for automatic font expansion}%
    \fi
  \else
    \ifnum\MT@pdftex@no > \thr@@
      \let\MT@extra@auto\@empty
    \fi
  \fi
}
%    \end{macrocode}
%
%\subsubsection{Character Inheritance}
%
%\begin{macro}{\DeclareCharacterInheritance}
%\changes{v1.1}{2004/09/15}{new macro: possibility to specify character inheritance}
% This macro may be used in the configuration files to declare characters that
% should inherit protrusion resp. expansion values from other characters. Thus,
% there is no need to define all accented characters (\eg\ |\`a|, |\'a|,
% |\^a|, |\~a|, |\"a|, |\r{a}|, |\k{a}|, |\u{a}|), which will make the
% configuration files look much nicer and easier to maintain. If a single
% character of an inheritance list should have a different value, one can
% simply override it.
%    \begin{macrocode}
\renewcommand*\DeclareCharacterInheritance[1][]{%
  \KV@@sp@def\@tempa{#1}%
  \MT@begin@catcodes
  \MT@set@inh@list
}
%    \end{macrocode}
%\end{macro}
%\begin{macro}{\MT@set@inh@list}
% Safe category codes.
%    \begin{macrocode}
\def\MT@set@inh@list#1#2{%
  \MT@ifempty\@tempa{%
     \MT@declare@char@inh{pr}{#1}{#2}%
     \MT@declare@char@inh{ex}{#1}{#2}%
%<*beta>
     \MT@declare@char@inh{sp}{#1}{#2}%
     \MT@declare@char@inh{kn}{#1}{#2}%
%</beta>
  }{%
    \MT@map@clist@c\@tempa{%
      {\KV@@sp@def\@tempa{##1}%
       \MT@ifempty\@tempa\relax{%
         \MT@exp@one@n\MT@declare@char@inh
            {\csname MT@rbba@\@tempa\endcsname}{#1}{#2}}}%
    }%
  }%
  \MT@end@catcodes
}
%    \end{macrocode}
%\end{macro}
%\begin{macro}{\MT@declare@char@inh}
% The optional argument may be used to restrict the inheritance list to
% protrusion or expansion.
%    \begin{macrocode}
\def\MT@declare@char@inh#1#2#3{%
  \MT@let@nc{MT@#1@inh@name}\@undefined
  \MT@get@inh@name{#1}%
%<debug>\MT@dinfo{1}{creating inheritance list `\@nameuse{MT@#1@inh@name}'}%
  \global\MT@def@n{MT@#1@inh@\csname MT@#1@inh@name\endcsname}{#3}%
  \def\MT@permutelist{#1@inh}%
  \setkeys{MT@#1@inh}{#2}%
  \MT@permute
}
%    \end{macrocode}
%\end{macro}
%\begin{macro}{\MT@get@inh@name}
% The inheritance lists cannot be named by the user.
%    \begin{macrocode}
\def\MT@get@inh@name#1{%
  \@tempcnta=\@ne
  \MT@while@num{\@tempcnta > \z@}{%
    \MT@ifdefined@n{MT@#1@inh@#1-inh-\romannumeral\@tempcnta}{%
      \advance \@tempcnta \@ne
    }{%
      \MT@edef@n{MT@#1@inh@name}{#1-inh-\romannumeral\@tempcnta}%
      \@tempcnta=\z@
    }%
  }%
}
%    \end{macrocode}
%\end{macro}
%\begin{macro}{\MT@define@inh@key@encoding}
% Parse the first argument. \cs{DeclareCharacterInheritance} may also be set
% up for various combinations.
%\changes{v1.2}{2004/09/29}{check whether only one encoding specified}
%    \begin{macrocode}
\def\MT@define@inh@key@encoding#1{%
  \define@key{MT@#1}{encoding}[]{%
    \def\MT@val{##1}%
    \expandafter\MT@encoding@check\MT@val,\@nil
    \MT@get@highlevel{encoding}%
    \MT@edef@n{MT@tempencoding\romannumeral1}{\MT@val}%
  }%
}
%    \end{macrocode}
%\end{macro}
%\begin{macro}{\MT@encoding@check}
% But we only allow \emph{one} encoding.
%    \begin{macrocode}
\def\MT@encoding@check#1,#2\@nil{%
  \MT@ifempty{#2}\relax{%
    \edef\MT@val{#1}%
    \MT@warning{You may only specify one encoding for character\MessageBreak
                inheritance lists. Ignoring encoding(s) #2}%
  }%
}
%    \end{macrocode}
%\end{macro}
%\begin{macro}{\MT@define@inh@keys}
%    \begin{macrocode}
\def\MT@define@inh@keys#1{%
  \MT@define@inh@key@encoding{#1@inh}%
%    \end{macrocode}
% For the rest, we can reuse the key setup from \cs{SetProtrusion} resp.
% \cs{SetExpansion}.
%    \begin{macrocode}
  \MT@define@code@key{family}{#1@inh}%
  \MT@define@code@key{series}{#1@inh}%
  \MT@define@code@key{shape}{#1@inh}%
  \MT@define@code@key@size{#1@inh}%
  \MT@define@code@key@font{#1@inh}%
}
\MT@define@inh@keys{pr}
\MT@define@inh@keys{ex}
%<*beta>
\MT@define@inh@keys{sp}
\MT@define@inh@keys{kn}
%</beta>
%    \end{macrocode}
%\end{macro}
%\begin{macro}{\MT@inh@do}
% Parse the second argument, the inheritance lists. We define the commands
% |\MT@inh@|\meta{name}|@|\meta{slot}|@|, containing the inheriting characters.
% They will also be translated to slot numbers here, to save some time. The
% following will be executed only once, namely the first time this inheritance
% list is encountered (in \cs{MT@set@pr@codes} resp. \cs{MT@set@ex@codes}).
%    \begin{macrocode}
\def\MT@inh@do#1,{%
  \ifx\relax#1\@empty \else
    \MT@inh@split #1==\relax
    \expandafter\MT@inh@do
  \fi
}
%    \end{macrocode}
%\end{macro}
%\begin{macro}{\MT@inh@split}
% Only gather the inheriting characters here. Their codes will actually be set
% in |MT@set@|\meta{feature}|@codes|),
%    \begin{macrocode}
\def\MT@inh@split#1=#2=#3\relax{%
  \def\@tempa{#1}%
  \ifx\@tempa\@empty \else
    \MT@get@slot
    \ifnum\MT@char > \m@ne
      \let\MT@val\MT@char
      \MT@map@clist@n{#2}{%
        \def\@tempa{##1}%
        \ifx\@tempa\@empty \else
          \MT@get@slot
          \ifnum\MT@char > \m@ne
            \expandafter\MT@xadd
              \csname MT@inh@\MT@inh@name @\MT@val @\endcsname
              {{\MT@char}}%
          \fi
        \fi
      }%
%<*debug>
      \MT@dinfo@nl{2}{children of #1 (\MT@val):
                      \@nameuse{MT@inh@\MT@inh@name @\MT@val @}}%
%</debug>
    \fi
  \fi
}
%    \end{macrocode}
%\end{macro}
%
%\subsubsection{Permutation}
%
%\begin{macro}{\MT@permute}
%\changes{v1.1}{2004/09/15}{don't use sets for empty encoding}
%\begin{macro}{\MT@permute@}
%\begin{macro}{\MT@permute@@}
%\begin{macro}{\MT@permute@@@}
%\begin{macro}{\MT@permute@@@@}
%\begin{macro}{\MT@permute@@@@@}
% Calling \cs{MT@permute} will define commands for all permutations
% of the specified font characteristics of the form
% |\MT@|\meta{list type}|@/|\meta{encoding}|/|\meta{family}|/|^^A
%       \meta{series}|/|\hskip0pt\meta{shape}|/|\meta{\textbar*}
% to be the expansion of |\MT@|\meta{list type}|@name|, \ie, the name of
% the currently defined list. Size ranges are held in a separate macro called
% |\MT@|\meta{list type}|@/|\meta{font axes}|@sizes|, which in turn contains
% the respective \meta{list name}s attached to the ranges.
%    \begin{macrocode}
\def\MT@permute{%
  \let\MT@cnt@encoding\@ne
  \MT@permute@
%    \end{macrocode}
% Undefine commands for the next round.
%    \begin{macrocode}
  \MT@permute@reset
}
\def\MT@permute@{%
  \let\MT@cnt@family\@ne
  \MT@permute@@
  \MT@increment\MT@cnt@encoding
  \MT@ifdefined@n{MT@tempencoding\romannumeral\MT@cnt@encoding}%
    \MT@permute@
    \relax
}
\def\MT@permute@@{%
  \let\MT@cnt@series\@ne
  \MT@permute@@@
  \MT@increment\MT@cnt@family
  \MT@ifdefined@n{MT@tempfamily\romannumeral\MT@cnt@family}%
    \MT@permute@@
    \relax
}
\def\MT@permute@@@{%
  \let\MT@cnt@shape\@ne
  \MT@permute@@@@
  \MT@increment\MT@cnt@series
  \MT@ifdefined@n{MT@tempseries\romannumeral\MT@cnt@series}%
    \MT@permute@@@
    \relax
}
\def\MT@permute@@@@{%
  \MT@permute@@@@@
  \MT@increment\MT@cnt@shape
  \MT@ifdefined@n{MT@tempshape\romannumeral\MT@cnt@shape}%
    \MT@permute@@@@
    \relax
}
\def\MT@permute@@@@@{%
  \MT@permute@define{encoding}%
  \MT@permute@define{family}%
  \MT@permute@define{series}%
  \MT@permute@define{shape}%
  \edef\@tempa{\MT@tempencoding
              /\MT@tempfamily
              /\MT@tempseries
              /\MT@tempshape
              /\MT@ifdefined@c\MT@tempsize *\@empty}%
%    \end{macrocode}
%\changes{v1.2}{2004/09/29}{more sanity checks for \cs{SetProtrusion} and
%                          \cs{SetExpansion}}
% Some sanity checks: An encoding must be specified (unless nothing else is).
%    \begin{macrocode}
  \def\@tempb{////}%
  \ifx\@tempa\@tempb \else
    \ifx\MT@tempencoding\@empty
      \MT@warning{%
        You have to specify an encoding for\MessageBreak
        \@nameuse{MT@abbr@\MT@permutelist} list
        `\@nameuse{MT@\MT@permutelist @name}'.\MessageBreak
        Ignoring it}%
    \else
      \MT@ifdefined@c\MT@tempsize{%
%    \end{macrocode}
% Add the list of ranges to the beginning of the current combination, after
% checking for conflicts.
%\changes{v1.8}{2005/04/20}{add ranges to the beginning of the lists}
%    \begin{macrocode}
        \MT@ifdefined@n{MT@\MT@permutelist @\@tempa\MT@extra@context @sizes}{%
          \MT@map@tlist@c
            \MT@tempsize
            \MT@check@rlist
        }\relax
        \expandafter\MT@xaddb
          \csname MT@\MT@permutelist @\@tempa\MT@extra@context @sizes\endcsname
          \MT@tempsize
%<*debug>
          \MT@dinfo@nl{1}{initializing: use list for font \@tempa,\MessageBreak
                   sizes: \csname MT@\MT@permutelist @\@tempa\MT@extra@context
                                  @sizes\endcsname}%
%</debug>
      }{%
%    \end{macrocode}
% Only one list should apply to a given combination.
%    \begin{macrocode}
        \MT@ifdefined@n{MT@\MT@permutelist @\@tempa\MT@extra@context}{%
          \MT@warning{\@nameuse{MT@abbr@\MT@permutelist} list
            `\@nameuse{MT@\MT@permutelist @name}' will override list\MessageBreak
            `\@nameuse{MT@\MT@permutelist @\@tempa\MT@extra@context}' for font `\@tempa'}%
        }\relax
%<*debug>
        \MT@dinfo@nl{1}{initializing: use list for font \@tempa
                        \ifx\MT@extra@context\@empty\else\MessageBreak
                          (context: \MT@extra@context)\fi}%
%</debug>
      }%
      \global\MT@edef@n{MT@\MT@permutelist @\@tempa\MT@extra@context}%
          {\csname MT@\MT@permutelist @name\endcsname}%
    \fi
  \fi
}
%    \end{macrocode}
%\end{macro}
%\end{macro}
%\end{macro}
%\end{macro}
%\end{macro}
%\end{macro}
%\begin{macro}{\MT@permute@define}
% Define the commands.
%    \begin{macrocode}
\def\MT@permute@define#1{%
  \expandafter\@tempcnta=\csname MT@cnt@#1\endcsname\relax
  \MT@ifdefined@n{MT@temp#1\romannumeral\@tempcnta}%
    {\MT@edef@n{MT@temp#1}{\csname MT@temp#1\romannumeral\@tempcnta\endcsname}}%
    {\MT@let@nc{MT@temp#1}\@empty}%
}
%    \end{macrocode}
%\end{macro}
%\begin{macro}{\MT@permute@reset}
% Reset the commands.
%    \begin{macrocode}
\def\MT@permute@reset{%
  \MT@permute@reset@{encoding}%
  \MT@permute@reset@{family}%
  \MT@permute@reset@{series}%
  \MT@permute@reset@{shape}%
  \let\MT@tempsize\@undefined
}
%    \end{macrocode}
%\end{macro}
%\begin{macro}{\MT@permute@reset@}
%    \begin{macrocode}
\def\MT@permute@reset@#1{%
  \@tempcnta=\@ne
  \MT@loop
    \MT@let@nc{MT@temp#1\romannumeral\@tempcnta}\@undefined
    \advance\@tempcnta\@ne
    \MT@ifdefined@n{MT@temp#1\romannumeral\@tempcnta}%
      \iftrue
      \iffalse
  \MT@repeat
}
%    \end{macrocode}
%\end{macro}
%\begin{macro}{\MT@check@rlist}
%\changes{v1.8}{2005/04/20}{made recursive}
% For every new range item in \cs{MT@tempsize}, check whether it overlaps with
% ranges in the existing list.
%    \begin{macrocode}
\def\MT@check@rlist#1{%
  \expandafter\MT@check@rlist@#1%
}
%    \end{macrocode}
%\end{macro}
%\begin{macro}{\MT@check@rlist@}
% Define the current new range and \dots
%    \begin{macrocode}
\def\MT@check@rlist@#1#2#3{%
  \def\@tempb{#1}%
  \def\@tempc{#2}%
  \@tempswafalse
  \expandafter\MT@map@tlist@c
    \csname MT@\MT@permutelist @\@tempa\MT@extra@context @sizes\endcsname
    \MT@check@range
}
%    \end{macrocode}
%\end{macro}
%\begin{macro}{\MT@check@range}
% \dots~recurse through the list of existing ranges.
%    \begin{macrocode}
\def\MT@check@range#1{%
  \expandafter\MT@check@range@#1%
}
%    \end{macrocode}
%\end{macro}
%\begin{macro}{\MT@check@range@}
% \cmd{\@tempb} and \cmd{\@tempc} are lower resp. upper bound of the new range,
% \meta{\#2} and \meta{\#3} those of the existing range.
%    \begin{macrocode}
\def\MT@check@range@#1#2#3{%
  \MT@ifeq{#2}\m@ne{%
    \MT@ifeq\@tempc\m@ne{%
%    \end{macrocode}
%\begin{itemize}
% \item Both items are simple sizes.
%    \begin{macrocode}
      \MT@ifeq\@tempb{#1}\@tempswatrue\relax
    }{%
%    \end{macrocode}
% \item Item in list is a simple size, new item is a range.
%    \begin{macrocode}
      \MT@ifgt\@tempb{#1}\relax{%
        \MT@ifgt\@tempc{#1}{%
          \@tempswatrue
          \edef\@tempb{#1 (with range: \@tempb\space to \@tempc)}%
        }\relax
      }%
    }%
  }{%
    \MT@ifeq\@tempc\m@ne{%
%    \end{macrocode}
% \item Item in list is a range, new item is a simple size.
%    \begin{macrocode}
      \MT@iflt\@tempb{#2}{%
        \MT@iflt\@tempb{#1}\relax\@tempswatrue
      }\relax
    }{%
%    \end{macrocode}
% \item Both items are ranges.
%    \begin{macrocode}
      \MT@iflt\@tempb{#2}{%
        \MT@ifgt\@tempc{#1}{%
          \@tempswatrue
          \edef\@tempb{#1 to #2 (with range: \@tempb\space to \@tempc)}%
        }\relax
      }\relax
    }%
  }%
  \if@tempswa
    \MT@warning{\@nameuse{MT@abbr@\MT@permutelist} list
      `\@nameuse{MT@\MT@permutelist @name}' will override\MessageBreak
      list `#3' for font \@tempa,\MessageBreak size \@tempb}%
%    \end{macrocode}
% If we've already found a conflict with this item, we can skip the rest of the
% list.
%    \begin{macrocode}
    \expandafter\MT@tlist@break
  \fi
}
%    \end{macrocode}
%\end{itemize}
%\end{macro}
%
%\subsection{User Command}
%
%\begin{macro}{\microtypesetup}
%\changes{v1.4}{2004/11/04}{bug fix: set the correct levels, and remember them;
%                           warning when enabling an option disabled in package options}
%\changes{v1.7}{2005/02/17}{bug fix: warning also when setting to \texttt{(no)compatibility}}
%\begin{macro}{\MT@define@optionX}
% This command may be used anywhere in the document. It accepts the options:
% |protrusion|, |expansion| and |activate|, and |spacing| and |kerning|.
% Specifying font sets is not allowed.
%    \begin{macrocode}
\def\microtypesetup{\setkeys{MTX}}
\def\MT@define@optionX#1#2{%
  \define@key{MTX}{#1}[true]{%
    \KV@@sp@def\@tempb{#1}%
    \MT@map@clist@n{##1}{%
      \KV@@sp@def\MT@val{####1}%
      \edef\@tempb{\csname MT@rbba@\@tempb\endcsname}%
      \MT@ifempty\MT@val\relax{%
        \@tempcnta=\m@ne
        \MT@ifstreq\MT@val{true}{%
%    \end{macrocode}
% Enabling micro-typography in the middle of the document is not allowed if it
% has been disabled in the package options since fonts might already have been
% loaded and hence wouldn't be set up.
%    \begin{macrocode}
          \MT@checksetup\@tempb{%
            \expandafter\@tempcnta=\csname MT@\@tempb @level\endcsname
            \MT@info{Enabling #1
                    (level \number\csname MT@\@tempb @level\endcsname)}%
          }%
        }{%
          \MT@ifstreq\MT@val{false}{%
            \@tempcnta=\z@
            \MT@info{Disabling #1}%
          }{%
            \MT@ifstreq\MT@val{compatibility}{%
              \MT@checksetup\@tempb{%
                \@tempcnta=\@ne
                \MT@let@nc{MT@\@tempb @level}\@ne
                \MT@info{Setting #1 to level 1}%
              }%
            }{%
              \MT@ifstreq\MT@val{nocompatibility}{%
                \MT@checksetup\@tempb{%
                  \@tempcnta=\tw@
                  \MT@let@nc{MT@\@tempb @level}\tw@
                  \MT@info{Setting #1 to level 2}%
                }%
              }{%
                \MT@warning{%
                  Value `\MT@val' for key `#1' not recognized.\MessageBreak
                  Use any of `true', `false', `compatibility' or\MessageBreak
                  `nocompatibility'}%
              }%
            }%
          }%
        }%
        \ifnum\@tempcnta>\m@ne
          #2\@tempcnta\relax
        \fi
      }%
    }%
  }%
}
%    \end{macrocode}
%\end{macro}
%\end{macro}
%\begin{macro}{\MT@checksetup}
% Test whether the feature wasn't disabled in the package options.
%    \begin{macrocode}
\def\MT@checksetup#1{%
  \expandafter\csname ifMT@\csname MT@abbr@#1\endcsname\endcsname
    \expandafter\@firstofone
  \else
    \MT@warning{%
      You cannot enable \@nameuse{MT@abbr@#1} if it was disabled\MessageBreak
      in the package options,}%
    \expandafter\@gobble
  \fi
}
%    \end{macrocode}
%\end{macro}
%    \begin{macrocode}
\MT@define@optionX{protrusion}\pdfprotrudechars
\MT@define@optionX{expansion}\pdfadjustspacing
%<*beta>
%    \end{macrocode}
%\begin{macro}{\MT@define@optionX@}
% The same for |spacing| and |kerning|, which do not have a |nocompatibility| level.
%    \begin{macrocode}
\def\MT@define@optionX@#1#2{%
  \define@key{MTX}{#1}[true]{%
    \KV@@sp@def\@tempb{#1}%
    \MT@map@clist@n{##1}{%
      \KV@@sp@def\MT@val{####1}%
      \edef\@tempb{\csname MT@rbba@\@tempb\endcsname}%
      \MT@ifempty\MT@val\relax{%
        \@tempcnta=\m@ne
        \MT@ifstreq\MT@val{true}{%
          \MT@checksetup\@tempb{%
            \@tempcnta=\@ne
            \MT@info{Enabling #1}%
          }%
        }{%
          \MT@ifstreq\MT@val{false}{%
            \@tempcnta=\z@
            \MT@info{Disabling #1}%
          }{%
            \MT@warning{%
              Value `\MT@val' for key `#1' not recognized.\MessageBreak
              Use either `true' or `false'}%
          }%
        }%
        \ifnum\@tempcnta>\m@ne
          #2\relax
        \fi
      }%
    }%
  }%
}
%    \end{macrocode}
%\end{macro}
%    \begin{macrocode}
\MT@define@optionX@{spacing}{\pdfadjustinterwordglue\@tempcnta}
\MT@define@optionX@{kerning}{\pdfprependkern\@tempcnta
                             \pdfappendkern \@tempcnta}
%</beta>
\define@key{MTX}{activate}[]{%
  \setkeys{MTX}{protrusion={#1}}%
  \setkeys{MTX}{expansion={#1}}%
}
%    \end{macrocode}
%\begin{macro}{\microtypecontext}
%\changes{v1.9}{2005/10/03}{new macro: change setup context in the document}
% The user may now also change the context, so that different setups are
% possible. This is especially useful for multi-lingual documents.
%    \begin{macrocode}
\def\microtypecontext#1{%
  \setkeys{MTC}{#1}%
  \edef\MT@curr@contexts{\MT@pr@context|%
                         \MT@ex@context|%
                         \MT@sp@context|%
                         \MT@kn@context}%
%<debug>\MT@dinfo{2}{>>> current context: \MT@curr@contexts}%
%    \end{macrocode}
% Keep track of all contexts in the document.
%    \begin{macrocode}
  \MT@exp@one@n\MT@in@tlist\MT@curr@contexts\MT@doc@contexts
  \ifMT@inlist@ \else
    \MT@xadd\MT@doc@contexts{{\MT@curr@contexts}}%
%<debug>\MT@dinfo{2}{>>> document contexts: \MT@doc@contexts}%
  \fi
  \selectfont
  \aftergroup\MT@reset@context
}
%    \end{macrocode}
%\end{macro}
%\begin{macro}{\MT@reset@context}
% We have to reset the font at the end of the group.
%    \begin{macrocode}
\def\MT@reset@context{%
  \MT@vinfo{Resetting contexts on line \the\inputlineno}%
  \selectfont
}
%    \end{macrocode}
%\end{macro}
%\begin{macro}{\MT@define@context}
%    \begin{macrocode}
\def\MT@define@context#1{%
  \define@key{MTC}{#1}[]{%
    \KV@@sp@def\@tempb{#1}%
    \edef\@tempb{\@nameuse{MT@rbba@\@tempb}}%
    \KV@@sp@def\MT@val{##1}%
    \MT@vinfo{--- Changing #1 context to `\MT@val'}%
    \MT@edef@n{MT@\@tempb @context}{\MT@val}%
%    \end{macrocode}
% We have to reset \emph{all} factors the next time we see the font.
%    \begin{macrocode}
    \MT@ifempty\MT@val\relax{%
      \global\MT@let@nn{MT@reset@\@tempb @codes}{MT@reset@\@tempb @codes@}%
    }%
  }%
}
%    \end{macrocode}
%\end{macro}
%    \begin{macrocode}
\MT@define@context{protrusion}
\MT@define@context{expansion}
%<*beta>
\MT@define@context{spacing}
\MT@define@context{kerning}
%</beta>
%    \end{macrocode}
%\begin{macro}{\MT@pr@context}
%\begin{macro}{\MT@ex@context}
%\begin{macro}{\MT@sp@context}
%\begin{macro}{\MT@kn@context}
%\begin{macro}{\MT@curr@contexts}
%\begin{macro}{\MT@doc@contexts}
%\begin{macro}{\MT@extra@context}
% Initialize the contexts.
%    \begin{macrocode}
\let\MT@pr@context\@empty
\let\MT@ex@context\@empty
%<*beta>
\let\MT@sp@context\@empty
\let\MT@kn@context\@empty
%</beta>
\def\MT@curr@contexts{|||}
\def\MT@doc@contexts{{|||}}
\let\MT@extra@context\@empty
%    \end{macrocode}
%\end{macro}
%\end{macro}
%\end{macro}
%\end{macro}
%\end{macro}
%\end{macro}
%\end{macro}
%    \begin{macrocode}
%    \end{macrocode}
% Disable everything -- may be used as a work-around in case setting up fonts
% doesn't work in certain environments. \emph{(Undocumented.)}
%    \begin{macrocode}
\def\MT@gobblethree#1#2#3{}
\let\MT@saved@setupfont\MT@setupfont
\define@key{MTX}{disable}[]{%
  \MT@info{Inactivate microtype package}%
  \let\MT@setupfont\MT@gobblethree
}
\define@key{MTX}{enable}[]{%
  \MT@info{Reactivate microtype package}%
  \let\MT@setupfont\MT@saved@setupfont
}
%    \end{macrocode}
%
%\subsection{Package Options}
%
%\subsubsection{Declaring the Options}
%
%\begin{macro}{\ifMT@opt@expansion}\IndexNewif{MT@opt@expansion}
%\begin{macro}{\ifMT@opt@auto}\IndexNewif{MT@opt@auto}
% Keep track of whether the user explicitly set these options.
%    \begin{macrocode}
\newif\ifMT@opt@expansion
\newif\ifMT@opt@auto
%    \end{macrocode}
%\end{macro}
%\end{macro}
%\begin{macro}{\MT@define@option}
% \opt{expansion} and \opt{protrusion} may be |true|, |false|, |compatibility|,
% |nocompatibility| and/or a \meta{set name}.
%\changes{v1.9}{2005/10/06}{fix: use \texttt{true} as the default value}
%    \begin{macrocode}
\def\MT@define@option#1{%
  \define@key{MT}{#1}[true]{%
    \csname MT@opt@#1true\endcsname
    \MT@map@clist@n{##1}{%
      \KV@@sp@def\MT@val{####1}%
      \MT@ifempty\MT@val\relax{%
        \csname MT@#1true\endcsname
        \edef\@tempb{\csname MT@rbba@#1\endcsname}%
        \MT@ifstreq\MT@val{true}\relax
        {%
          \MT@ifstreq\MT@val{false}{%
            \csname MT@#1false\endcsname
          }{%
            \MT@ifstreq\MT@val{compatibility}{%
              \MT@let@nc{MT@\@tempb @level}\@ne
            }{%
              \MT@ifstreq\MT@val{nocompatibility}{%
                \MT@let@nc{MT@\@tempb @level}\tw@
              }{%
%    \end{macrocode}
% If everything failed, it should be a set name.
%    \begin{macrocode}
                \MT@ifdefined@n{MT@\@tempb @set@@\MT@val}{%
                  \global\MT@edef@n{MT@\@tempb @setname}{\MT@val}%
                }{%
                  \global\MT@edef@n{MT@\@tempb @setname}%
                    {\@nameuse{MT@default@\@tempb @set}}%
                  \MT@warning@nl{%
                    The #1 set `\MT@val' is undeclared.\MessageBreak
                    Using set `\@nameuse{MT@\@tempb @setname}' instead}%
                }%
              }%
            }%
          }%
        }%
      }%
    }%
  }%
}
%    \end{macrocode}
%\end{macro}
%    \begin{macrocode}
\MT@define@option{protrusion}
\MT@define@option{expansion}
%    \end{macrocode}
% \opt{activate} is a shortcut for \opt{protrusion} and \opt{expansion} (and
% \opt{spacing}?).
%\todo{add spacing to activate, when the feature has stabilized in \pdftex}
%    \begin{macrocode}
\define@key{MT}{activate}[]{%
  \setkeys{MT}{protrusion={#1}}%
  \setkeys{MT}{expansion={#1}}%
}
%<*beta>
%    \end{macrocode}
%\begin{macro}{\MT@define@option@}
%    \begin{macrocode}
\def\MT@define@option@#1{%
  \define@key{MT}{#1}[true]{%
    \csname MT@opt@#1true\endcsname
    \MT@map@clist@n{##1}{%
      \KV@@sp@def\MT@val{####1}%
      \MT@ifempty\MT@val\relax{%
        \csname MT@#1true\endcsname
        \edef\@tempb{\csname MT@rbba@#1\endcsname}%
        \MT@ifstreq\MT@val{true}\relax
        {%
          \MT@ifstreq\MT@val{false}{%
            \csname MT@#1false\endcsname
          }{%
            \MT@ifdefined@n{MT@\@tempb @set@@\MT@val}{%
              \global\MT@edef@n{MT@\@tempb @setname}{\MT@val}%
            }{%
              \global\MT@edef@n{MT@\@tempb @setname}%
                {\@nameuse{MT@default@\@tempb @set}}%
              \MT@warning@nl{%
                The #1 set `\MT@val' is undeclared.\MessageBreak
                Using set `\@nameuse{MT@\@tempb @setname}' instead}%
            }%
          }%
        }%
      }%
    }%
  }%
}
%    \end{macrocode}
%\end{macro}
%    \begin{macrocode}
\MT@define@option@{spacing}
\MT@define@option@{kerning}
%</beta>
%    \end{macrocode}
%\changes{v1.5}{2004/12/02}{new option: \opt{selected}, by default false
%                           (suggested by \thanh)} ^^A <hanthethanh\at gmx.net>
%                                                  ^^A private mail, 30/11/2004
%\changes{v2.0}{2005/10/04}{new option: \opt{babel}, by default false
%                           (language-dependant setup suggested by Ulrich Dirr)}
%                                                       ^^A <ud\at satz-art.de>
%\begin{macro}{\MT@def@bool@opt}
% The |true|/|false| options:
% \opt{draft} (may be inherited from the class options),
% \opt{DVIoutput},
% \opt{auto},
% \opt{selected},
% \opt{babel}.
%    \begin{macrocode}
\def\MT@def@bool@opt#1{%
  \define@key{MT}{#1}[]{%
    \MT@ifempty{##1}%
      {\def\@tempa{true}}%
      {\def\@tempa{##1}}%
    \MT@ifstreq\@tempa{true}\relax{%
      \MT@ifstreq\@tempa{false}\relax{%
        \MT@warning@nl{%
          `##1' is not an admissible value for option\MessageBreak
          `#1'. Assuming `false'}%
        \def\@tempa{false}%
      }%
    }%
    \csname MT@#1\@tempa\endcsname
  }%
}
%    \end{macrocode}
%\end{macro}
%    \begin{macrocode}
\MT@map@tlist@n{{draft}{DVIoutput}{auto}{selected}%
%<beta>{babel}%
}\MT@def@bool@opt
%    \end{macrocode}
% \opt{final} is the opposite to \opt{draft}.
%\changes{v1.4a}{2004/11/16}{new option: \opt{final}}
%    \begin{macrocode}
\define@key{MT}{final}[]{%
  \MT@draftfalse
  \MT@ifempty{#1}%
    {\def\@tempa{true}}%
    {\def\@tempa{#1}}%
  \MT@ifstreq\@tempa{true}\relax{%
    \MT@ifstreq\@tempa{false}%
      \MT@drafttrue
    {%
      \MT@warning@nl{%
        `#1' is not an admissible value for option\MessageBreak
        `final'. Assuming `true'}%
      \MT@draftfalse
    }%
  }%
}
%    \end{macrocode}
% For \opt{verbose} output, we simply redefine \cs{MT@vinfo}.
%    \begin{macrocode}
\define@key{MT}{verbose}[]{%
  \let\MT@vinfo\MT@info@nl
  \MT@ifempty{#1}%
    {\def\@tempa{true}}%
    {\def\@tempa{#1}}%
  \MT@ifstreq\@tempa{true}\relax{%
%    \end{macrocode}
%\changes{v1.7}{2005/03/16}{new value for \opt{verbose} option: \texttt{errors}}
% Take problems seriously.
%    \begin{macrocode}
    \MT@ifstreq\@tempa{errors}{%
      \let\MT@warning\MT@warn@err
      \let\MT@warning@nl\MT@warn@err
    }{%
      \let\MT@vinfo\@gobble
      \MT@ifstreq\@tempa{false}\relax{%
        \MT@warning@nl{%
          `#1' is not an admissible value for option\MessageBreak
          `verbose'. Assuming `false'}%
      }%
    }%
  }%
}
%    \end{macrocode}
%\changes{v1.5}{2004/12/02}{defaults: \opt{step}: 4
%                           (suggested by \thanh)} ^^A <hanthethanh\at gmx.net>
%                                                  ^^A private mail, 30/11/2004
%\changes{v1.6}{2004/12/18}{new option: \opt{factor}, by default 1000}
%\changes{v2.0}{2005/09/21}{new option: \opt{letterspacing}, by default 100}
%\begin{macro}{\MT@def@num@opt}
% Options with numerical keys:
% \opt{factor},
% \opt{stretch},
% \opt{shrink},
% \opt{step},
% \opt{letterspacing}.
%    \begin{macrocode}
\def\MT@def@num@opt#1{%
  \define@key{MT}{#1}[]{%
    \MT@ifempty{##1}%
      {\MT@let@cn\@tempa{MT@#1@default}}%
      {\def\@tempa{##1 }}%
%    \end{macrocode}
% No nonsense in \cs{MT@factor} et al.?
%\changes{v1.6}{2005/01/06}{test whether numeric options receive a number}
% A space terminates the number.
%    \begin{macrocode}
    \MT@ifnumber\@tempa{%
      \MT@edef@n{MT@#1}{\@tempa}%
    }{\MT@warning@nl{%
        Value `##1' for option `#1' is not a number.\MessageBreak
        Using default value of \number\@nameuse{MT@#1@default}}%
    }%
  }%
}
%    \end{macrocode}
%\end{macro}
%    \begin{macrocode}
\MT@map@tlist@n{{stretch}{shrink}{step}%
%<beta>{letterspacing}%
}\MT@def@num@opt
%    \end{macrocode}
% \opt{factor} will define the protrusion factor only.
%\todo{factor option for which extensions?}
%    \begin{macrocode}
\define@key{MT}{factor}[]{%
  \MT@ifempty{#1}%
    {\let\@tempa\MT@factor@default}%
    {\def\@tempa{#1 }}%
  \MT@ifnumber\@tempa{%
    \MT@edef@n{MT@pr@factor}{\@tempa}%
  }{\MT@warning@nl{%
      Value `#1' for option `factor' is not a number.\MessageBreak
      Using default value of \number\MT@factor@default}%
  }%
}
%    \end{macrocode}
% Unit for codes.
%\changes{v1.8}{2005/04/14}{new option: \opt{unit}, by default \texttt{character}}
%\changes{v1.9}{2005/09/28}{option \opt{unit}: rename value \texttt{relative} to \texttt{character}}
%    \begin{macrocode}
\define@key{MT}{unit}[]{%
  \MT@ifempty{#1}%
    {\def\@tempa{character}}%
    {\KV@@sp@def\@tempa{#1}}%
  \MT@ifstreq\@tempa{relative}{%
    \MT@warning{Value `relative' for option `unit' is deprecated.\MessageBreak
      Use `unit=character' instead. For now, I'll do it\MessageBreak
      for you}%
    \def\@tempa{character}%
  }\relax
  \MT@ifstreq\@tempa{character}\relax{%
    \MT@ifdimen\@tempa{%
      \let\MT@pr@unit\@tempa
    }{%
      \MT@warning@nl{`\@tempa' is not a dimension. Ignoring it and\MessageBreak
                     setting values relative to character widths}%
    }%
  }%
}
%    \end{macrocode}
% The package should just work if called without any options. Therefore,
% expansion will be switched off by default if output is \dvi, since it isn't
% likely that expanded fonts are available. (This grows more important as \TeX\
% systems are switching to the \pdftex\ engine even for \dvi\ output, so that
% the user might not even be aware of the fact that she's running \pdftex.)
%    \begin{macrocode}
\MT@protrusiontrue
\ifnum\pdfoutput=\z@ \else
%    \end{macrocode}
% Also, we only enable expansion by default if \pdftex\ can expand the fonts
% automatically.
%\changes{v1.6}{2004/12/23}{defaults: turn off expansion for old \pdftex\ versions}
%\todo{should spacing be enabled by default?}
%    \begin{macrocode}
  \ifnum\MT@pdftex@no > \thr@@
    \MT@expansiontrue
    \MT@autotrue
  \fi
\fi
%    \end{macrocode}
% The main configuration file will be loaded before processing the package
% options.
%\changes{v1.8}{2005/05/15}{new option: \opt{config} to load a different main
%                           configuration file}
%\begin{macro}{\MT@config@file}
%\begin{macro}{\MT@get@config}
% However, the \opt{config} option must of course be evaluated beforehand.
% We also have to define a no-op for the regular option processing later.
%    \begin{macrocode}
\define@key{MT}{config}[]{\relax}
\def\MT@get@config#1config=#2,#3\@nil{%
  \MT@ifempty{#2}%
    {\def\MT@config@file{microtype.cfg}}%
    {\KV@@sp@def\MT@config@file{#2.cfg}}%
}
\expandafter\expandafter\expandafter\MT@get@config
  \csname opt@\@currname.\@currext\endcsname,config=,\@nil
%    \end{macrocode}
%\end{macro}
%\end{macro}
%\changes{v1.4b}{2004/11/25}{bug fix: set catcodes before reading global configuration file
%                           (reported by Christoph Bier)} ^^A <christoph.bier\at web.de>
%                                            ^^A in: <MID:30k2tqF30v483U1@uni-berlin.de>
% Load the file.
%    \begin{macrocode}
\IfFileExists{\MT@config@file}{%
  \MT@info@nl{Loading configuration file \MT@config@file}%
  \MT@begin@catcodes
  \let\MT@begin@catcodes\relax
  \let\MT@end@catcodes\relax
  \input{\MT@config@file}%
  \endgroup
}{%
  \MT@warning@nl{%
    Could not find configuration file `\MT@config@file'!\MessageBreak
    This will almost certainly cause undesired results.\MessageBreak
    Please fix your installation}%
}
%    \end{macrocode}
% If no default font set has been declared in the main configuration file,
% we use the (empty, possibly non-existent) `|all|' set.
%\changes{v1.9}{2005/08/05}{disable \cs{DeclareMicrotypeSetDefault} after
%                           configuration file has been loaded}
% We also disable the command.
%    \begin{macrocode}
\MT@ifdefined@c\MT@default@pr@set\relax{\gdef\MT@default@pr@set{all}}
\MT@ifdefined@c\MT@default@ex@set\relax{\gdef\MT@default@ex@set{all}}
%<*beta>
\MT@ifdefined@c\MT@default@sp@set\relax{\gdef\MT@default@sp@set{all}}
\MT@ifdefined@c\MT@default@kn@set\relax{\gdef\MT@default@kn@set{all}}
%</beta>
\renewcommand*\DeclareMicrotypeSetDefault[2][]{%
  \MT@warning{%
    The command \string\DeclareMicrotypeSetDefault\space may only\MessageBreak
    be used inside the main configuration file.\MessageBreak
    Ignoring it}%
}
%    \end{macrocode}
%
%
%\subsubsection{Hook for Other Packages}\label{sub:hook}
%
%\makeatletter
% {\let\special@index\index\SpecialIndex@{\Microtype@Hook}{\encapchar usage}}
%\makeatother
%\begin{macro}{\Microtype@Hook}
%\changes{v1.7}{2005/03/22}{new macro for font package authors}
% This hook may be used by font package authors, \eg\ to declare alias fonts. If
% it is defined, it will be executed here, \ie, after the main configuration
% file has been loaded, and before the package options are evaluated.
%
% This hook overcomes the situation that (1) the \microtype\ package should be
% loaded after all font defaults have been set up (hence, using
% \cmd{\@ifpackageloaded} in the font package is not viable), and (2) checking
% \cmd{\AtBeginDocument} can be too late, since fonts might already have been
% loaded, and consequently set up, in the preamble.
%
% Package authors should check whether the command is already defined so that
% existing definitions by other packages aren't overwritten. Example:
%\begin{verbatim}
%\def\MinionPro@MT@Hook
%  {\DeclareMicrotypeAlias{MinionPro-LF}{MinionPro}}
%\@ifundefined{Microtype@Hook}
%  {\let\Microtype@Hook\MinionPro@MT@Hook}
%  {\g@addto@macro\Microtype@Hook{\MinionPro@MT@Hook}}
%\end{verbatim}
% \cs{MicroType@Hook} with a capital |T| is provided for compatibility reasons.
% At some point in the future, it will no longer be available, hence it should
% not be used.
%    \begin{macrocode}
\MT@ifdefined@c\MicroType@Hook{%
  \MT@warning@nl{%
    Command \string\MicroType@Hook\space is deprecated.\MessageBreak
    Use \string\Microtype@Hook\space instead}\MicroType@Hook}\relax
\MT@ifdefined@c\Microtype@Hook\Microtype@Hook\relax
%    \end{macrocode}
%\end{macro}
%
%\subsubsection{Processing the Options}
%
%\begin{macro}{\MT@ProcessOptionsWithKV}
% Parse options.
%    \begin{macrocode}
\def\MT@ProcessOptionsWithKV#1{%
  \let\@tempc\relax
  \let\KVo@tempa\@empty
  \MT@map@clist@c\@classoptionslist{%
    \def\CurrentOption{##1}%
    \MT@ifdefined@n{KV@#1@\CurrentOption}{%
      \edef\KVo@tempa{\KVo@tempa,\CurrentOption,}%
      \@expandtwoargs\@removeelement\CurrentOption
        \@unusedoptionlist\@unusedoptionlist
    }\relax
  }%
  \edef\KVo@tempa{%
    \noexpand\setkeys{#1}{%
      \KVo@tempa\@ptionlist{\@currname.\@currext}%
    }%
  }%
  \KVo@tempa
  \AtEndOfPackage{\let\@unprocessedoptions\relax}%
  \let\CurrentOption\@empty
}
%    \end{macrocode}
%\end{macro}
%    \begin{macrocode}
\MT@ProcessOptionsWithKV{MT}
%    \end{macrocode}
% Now we can take the appropriate actions:
%
% \pdftex\ can create \dvi\ output, too. However, both the \dvi\ viewer and
% dvips need to find actual fonts. Therefore, expansion will only work if the
% fonts for different degrees of expansion are readily available.
%
%\changes{v1.4b}{2004/11/19}{new message if \cmd{\pdfoutput} is changed}
% Some packages depend on the value of \cmd{\pdfoutput} and will get confused if
% it is changed after they have been loaded. These packages are, among others:
% \pkg{color}, \pkg{graphics}, \pkg{hyperref}, \pkg{crop}, \pkg{contour},
% \pkg{pstricks} and, as a matter of course, \pkg{ifpdf}. Instead of testing
% for each package (that's not our job), we issue a different message if
% \cmd{\pdfoutput} is actually changed by \opt{DVIoutput}. That must be
% sufficient!
%    \begin{macrocode}
\ifMT@DVIoutput
  \ifnum\pdfoutput=\z@
    \MT@info@nl{Generating DVI output}
  \else
    \pdfoutput\z@
    \MT@info@nl{Changing output mode to DVI}
%    \end{macrocode}
% For \dvi\ output, the user must have explicitly passed the \opt{expansion} option
% to the package.
%\changes{v1.5}{2004/12/02}{defaults: turn off expansion for \dvi\ output}
%    \begin{macrocode}
    \ifMT@opt@expansion \else
      \MT@expansionfalse
    \fi
  \fi
\else
  \MT@info@nl{Generating \ifnum\pdfoutput=\z@ DVI \else PDF \fi output}
\fi
%    \end{macrocode}
% Tell the log file which options the user has chosen (in case it's interested).
% We disable most of what we've just defined in the \arabic{CodelineNo} lines
% above if we are running in draft mode.
%\changes{v1.7}{2005/02/06}{warning when running in draft mode}
%    \begin{macrocode}
\ifMT@draft
  \MT@warning@nl{`draft' option active.\MessageBreak
                 Disabling all micro-typographic extensions.\MessageBreak
                 This might lead to different line and page breaks}
  \MT@protrusionfalse
  \MT@expansionfalse
%<*beta>
  \MT@spacingfalse
  \MT@kerningfalse
%</beta>
  \let\MT@setupfont\relax
  \def\DeclareMicrotypeSet{%
    \@ifstar
      {\@ifnextchar[\MT@DeclareSet{\MT@DeclareSet[]}}%
      {\@ifnextchar[\MT@DeclareSet{\MT@DeclareSet[]}}%
  }
  \def\MT@DeclareSet[#1]#2#3{}
  \renewcommand*\UseMicrotypeSet[2][]{}
  \renewcommand*\SetProtrusion[3][]{}
  \renewcommand*\SetExpansion[3][]{}
%<*beta>
  \renewcommand*\SetExtraSpacing[3][]{}
  \renewcommand*\SetExtraKerning[3][]{}
%</beta>
  \renewcommand*\DeclareCharacterInheritance[3][]{}
  \renewcommand*\DeclareMicrotypeAlias[2]{}
  \renewcommand*\LoadMicrotypeFile[1]{}
  \renewcommand*\microtypesetup[1]{}
  \renewcommand*\microtypecontext[1]{}
  \expandafter
  \endinput
\fi
\ifMT@protrusion
  \pdfprotrudechars\MT@pr@level
  \MT@info@nl{Character protrusion enabled (level \number\MT@pr@level)%
    \ifnum\MT@pr@factor=\MT@factor@default \else,\MessageBreak
      factor: \number\MT@pr@factor\fi
    \ifx\MT@pr@unit\@empty \else,\MessageBreak unit: \MT@pr@unit\fi}
%    \end{macrocode}
% We have to make sure that font sets are active at the end of the package. If
% the user didn't activate any in the package options, we use those sets declared
% by \cs{DeclareMicrotypeSetDefault}. They can still be overridden later on, of
% course.
%    \begin{macrocode}
  \MT@ifdefined@c\MT@pr@setname{%
    \MT@info@nl{Using protrusion set `\MT@pr@setname'}%
  }{%
    \global\let\MT@pr@setname\MT@default@pr@set
    \MT@info@nl{Using default protrusion set `\MT@pr@setname'}%
  }
\else
  \let\MT@protrusion\relax
  \MT@info@nl{No character protrusion}
\fi
\ifMT@expansion
%    \end{macrocode}
% Set up the values for font expansion: If \opt{stretch} has not been specified, we
% take the default value of 20.
%    \begin{macrocode}
  \ifnum\MT@stretch=\m@ne
    \let\MT@stretch\MT@stretch@default
  \fi
%    \end{macrocode}
% If \opt{shrink} has not been specified, it will inherit the value from \opt{stretch}.
%    \begin{macrocode}
  \ifnum\MT@shrink=\m@ne
    \ifnum\MT@stretch>\z@
      \let\MT@shrink\MT@stretch
    \else
      \let\MT@shrink\MT@shrink@default
    \fi
  \fi
%    \end{macrocode}
% If \opt{step} has not been specified, we will set it to min(\opt{stretch},\opt{shrink})/5,
% rounded off, minimum value 1.
%\changes{v1.5}{2004/12/02}{defaults: calculate \opt{step} as
%                           min(\texttt{stretch},\texttt{shrink})/5}
%\changes{v1.9}{2005/07/08}{warning if user requested zero \opt{step}}
%    \begin{macrocode}
  \ifnum\MT@step=\m@ne
    \ifnum\MT@stretch>\MT@shrink
      \ifnum\MT@shrink=\z@
        \@tempcnta=\MT@stretch
      \else
        \@tempcnta=\MT@shrink
      \fi
    \else
      \ifnum\MT@stretch=\z@
        \@tempcnta=\MT@shrink
      \else
        \@tempcnta=\MT@stretch
      \fi
    \fi
    \divide\@tempcnta 5\relax
  \else
    \@tempcnta=\MT@step
    \ifnum\@tempcnta=\z@
      \MT@warning@nl{The expansion step cannot be set to zero.\MessageBreak
        Setting it to one}
    \fi
  \fi
  \ifnum\@tempcnta=\z@ \@tempcnta=\@ne \fi
  \edef\MT@step{\number\@tempcnta\space}
%    \end{macrocode}
%\begin{macro}{\MT@auto}
% Automatic expansion of the font? This new feature of \pdftex\ 1.20 makes the
% \textit{hz}-algorithm really usable. It must be either `|autoexpand|' or
% empty (or `|1000|' for older versions of \pdftex).
%    \begin{macrocode}
  \let\MT@auto\@empty
  \ifMT@auto
    \ifnum\MT@pdftex@no > \thr@@
%    \end{macrocode}
%\end{macro}
% We turn off automatic expansion if output mode is \dvi.
%\changes{v1.5}{2004/12/02}{disable automatic expansion for \dvi\ output}
%    \begin{macrocode}
      \ifnum\pdfoutput=\z@
        \ifMT@opt@auto
          \MT@warning@nl{%
            Automatic font expansion only works for PDF output.\MessageBreak
            However, you are creating a DVI file. I will switch\MessageBreak
            automatic font expansion off and hope that expanded\MessageBreak
            fonts are available}
        \fi
        \MT@autofalse
      \else
        \def\MT@auto{autoexpand}
      \fi
%    \end{macrocode}
% Also, if \pdftex\ is too old.
%\changes{v1.1}{2004/09/13}{issue an error instead of a warning, when \pdftex\
%                            version is too old for \texttt{autoexpand}}
%\changes{v1.6}{2004/12/23}{disable automatic expansion for old \pdftex\ versions}
%    \begin{macrocode}
    \else
      \ifMT@opt@auto
        \MT@warning@nl{%
          The pdftex you are using is too old for automatic\MessageBreak
          font expansion. I will switch it off and hope that\MessageBreak
          expanded fonts are available on your system.\MessageBreak
          Install pdftex version 1.20 or newer}
      \fi
      \MT@autofalse
      \def\MT@auto{1000 }
    \fi
%    \end{macrocode}
% No automatic expansion.
%    \begin{macrocode}
  \else
    \ifnum\MT@pdftex@no < 4
      \def\MT@auto{1000 }
    \fi
  \fi
%    \end{macrocode}
% Choose the appropriate macro for selected expansion.
%    \begin{macrocode}
  \ifMT@selected
    \let\MT@set@ex@codes\MT@set@ex@codes@s
  \else
    \let\MT@set@ex@codes\MT@set@ex@codes@n
  \fi
%    \end{macrocode}
% Filter out |stretch=0,shrink=0|, since it would result in an \pdftex\ error.
%\changes{v1.9}{2005/07/08}{disable expansion if both \opt{step} and \opt{shrink} are zero}
%    \begin{macrocode}
  \ifnum\MT@stretch=\z@
    \ifnum\MT@shrink=\z@
      \MT@warning@nl{%
        Both the stretch and shrink limit are set to zero.\MessageBreak
        Disabling font expansion}
      \MT@expansionfalse
    \fi
  \fi
\fi
\ifMT@expansion
  \pdfadjustspacing\MT@ex@level
  \MT@info@nl{\ifMT@auto Automatic f\else F\fi ont expansion enabled
              (level \number\MT@ex@level),\MessageBreak
              stretch: \number\MT@stretch, shrink: \number\MT@shrink,
              step: \number\MT@step, \ifMT@selected\else non-\fi selected}
  \MT@ifdefined@c\MT@ex@setname{%
    \MT@info@nl{Using expansion set `\MT@ex@setname'}%
  }{%
    \global\let\MT@ex@setname\MT@default@ex@set
    \MT@info@nl{Using default expansion set `\MT@ex@setname'}%
  }
%    \end{macrocode}
%\changes{v1.7}{2005/03/02}{modify \cmd{\showhyphens}}
% Inside \cmd{\showhyphens}, font expansion should be disabled.
%    \begin{macrocode}
  \CheckCommand*{\showhyphens}[1]{%
    \setbox0\vbox{\color@begingroup\everypar{}\parfillskip\z@skip
    \hsize\maxdimen\normalfont\pretolerance\m@ne\tolerance\m@ne
    \hbadness\z@\showboxdepth\z@\ #1\color@endgroup}}
%    \end{macrocode}
%\begin{macro}{\showhyphens}
% I wonder why it's defined globally (in \file{ltfssbas.dtx})?
%    \begin{macrocode}
  \gdef\showhyphens#1{%
    \setbox0\vbox{%
      \color@begingroup
      \pdfadjustspacing\z@
      \everypar{}%
      \parfillskip\z@skip\hsize\maxdimen
      \normalfont
      \pretolerance\m@ne\tolerance\m@ne\hbadness\z@\showboxdepth\z@\ #1%
      \color@endgroup}}
\else
  \let\MT@expansion\relax
  \MT@info@nl{No font expansion}
\fi
%<*beta>
\ifnum\MT@pdftex@no > 5
  \ifMT@spacing
    \pdfadjustinterwordglue\@ne
%    \end{macrocode}
% Warning if \cmd{\nonfrenchspacing} is active, since space factors will be
% ignored with \cmd{\pdfadjustinterwordglue}\,|>|\,0. Why 1500? Because some
% packages redefine \cmd{\frenchspacing}. See the |c.t.t| thread `\cmd{\frenchspacing}
% with AMS packages and babel', started by this message from Philipp Lehman:
% \nolinkurl{<ddtbaj$rob$1@online.de>} on August 16, 2005.
%    \begin{macrocode}
    \AtBeginDocument{%
      \ifnum\sfcode`\. > 1500
        \MT@ifstreq\MT@sp@context{nonfrench}\relax{%
          \MT@warning@nl{%
            \string\nonfrenchspacing\space is active. Adjustment of\MessageBreak
            interword spacing will disable it. You might want\MessageBreak
            to add `\string\microtypecontext{spacing=nonfrench}'\MessageBreak
            to your preamble}%
        }%
      \fi
    }
    \MT@info@nl{Adjustment of interword spacing enabled}
    \MT@ifdefined@c\MT@sp@setname{%
      \MT@info@nl{Using spacing set `\MT@sp@setname'}%
    }{%
      \global\let\MT@sp@setname\MT@default@sp@set
      \MT@info@nl{Using default spacing set `\MT@sp@setname'}%
    }
  \else
    \let\MT@spacing\relax
    \MT@info@nl{No adjustment of interword spacing}
  \fi
  \ifMT@kerning
    \pdfprependkern\@ne
    \pdfappendkern\@ne
    \MT@info@nl{Adjustment of character kerning enabled}
    \MT@ifdefined@c\MT@kn@setname{%
      \MT@info@nl{Using kerning set `\MT@kn@setname'}%
    }{%
      \global\let\MT@kn@setname\MT@default@kn@set
      \MT@info@nl{Using default kerning set `\MT@kn@setname'}%
    }
  \else
%    \end{macrocode}
% We do \emph{not} set \cs{MT@kerning} to \cmd{\relax} since it is also used by
% \cs{textls}.
%    \begin{macrocode}
    \MT@info@nl{No adjustment of character kerning}
  \fi
  \ifnum\MT@letterspacing=\m@ne
    \let\MT@letterspacing\MT@letterspacing@default
  \fi
%    \end{macrocode}
% If \pdftex\ is too old, we disable everything.
%    \begin{macrocode}
\else
  \ifMT@spacing
    \MT@warning@nl{Adjustment of interword spacing only works with\MessageBreak
      pdftex version 1.3x or newer. Switching it off}%
  \else
    \MT@info@nl{No adjustment of interword spacing}
  \fi
  \MT@spacingfalse
  \let\MT@spacing\relax
  \ifMT@kerning
    \MT@warning@nl{Character kerning only works with\MessageBreak
      pdftex version 1.3x or newer. Switching it off}%
  \else
    \MT@info@nl{No adjustment of character kerning}
  \fi
  \MT@kerningfalse
  \let\MT@kerning\relax
\fi
%    \end{macrocode}
% Interaction with \pkg{babel}. We hook into the language switching commands
% to enable language-dependent setup.
%    \begin{macrocode}
\ifMT@babel
  \MT@info@nl{Redefining babel's language switching commands}
  \let\MT@orig@select@language\select@language
  \def\select@language#1{%
    \MT@orig@select@language{#1}%
    \MT@ifdefined@n{MT@babel@#1}{%
      \MT@vinfo{Changing to language `#1' on line \the\inputlineno}%
      \expandafter\MT@exp@one@n\expandafter\microtypecontext
          \csname MT@babel@#1\endcsname
    }{%
      \microtypecontext{protrusion=,expansion=,spacing=,kerning=}%
    }%
  }
  \let\MT@orig@foreign@language\foreign@language
  \def\foreign@language#1{%
    \MT@orig@foreign@language{#1}%
    \MT@ifdefined@n{MT@babel@#1}{%
      \MT@vinfo{Changing to context `#1' on line \the\inputlineno}%
      \expandafter\MT@exp@one@n\expandafter\microtypecontext
          \csname MT@babel@#1\endcsname
    }{%
      \microtypecontext{protrusion=,expansion=,spacing=,kerning=}%
    }%
  }
%    \end{macrocode}
% Disable \pkg{babel}s active characters.
%    \begin{macrocode}
  \ifMT@kerning
    \AtBeginDocument{%
      \@ifpackageloaded{babel}{%
        \@tempswafalse
        \@ifpackagewith{babel}{french}\@tempswatrue\relax
        \@ifpackagewith{babel}{frenchb}\@tempswatrue\relax
        \@ifpackagewith{babel}{francais}\@tempswatrue\relax
        \if@tempswa
          \NoAutoSpaceBeforeFDP
          \MT@warning@nl{Switching off French babel's active punctuation characters}%
        \fi
      }{}%
    }%
  \fi
\fi
%</beta>
%</package>
%    \end{macrocode}
%\end{macro}
% That was that.
%
% ^^A -------------------------------------------------------------------------
% ^^A From now on, don't bother indexing:
%\NoIndexing
%
%\section{Configuration Files}
%\changes{v1.6}{2005/01/24}{restructure \file{dtx} file}
%
% Let's now write the font configuration files.
%    \begin{macrocode}
%<*config>

%    \end{macrocode}
% The following characters must not appear verbatim:
%
%\begin{tabular}{l@{\quad:\quad}l}
% |\| & \cmd{\textbackslash}\\
% |{| & \cmd{\textbraceleft}\\
% |}| & \cmd{\textbraceright}\\
% |^| & \cmd{\textasciicircum}\\
% |%| & \cs{\%}\\
% |#| & \cs{\#}
%\end{tabular}
%
%\noindent
% Comma and equal sign must be guarded with braces (`|{,}|', `|{=}|').
% Of course, numerical identification is possible in any case.
% Ligatures and \cmd{\mathchardef}ed symbols always have to be specified
% numerically.
%
%\changes{v1.1}{2004/09/14}{remove 8-bit characters from the configuration files
%                           (suggested by Harald Harders)}
%
%\subsection{Character Inheritance}
%\GeneralChanges{Inheritance}
%
% First the lists of inheriting characters. We only declare those characters
% that are the same on \emph{both} sides, \ie, not \OE\ for O.
%    \begin{macrocode}
%<*m-t>
%%% ----------------------------------------------------------------------
%%% CHARACTER INHERITANCE

%    \end{macrocode}
%
%\subsubsection{\otI}
% Glyphs that should possibly inherit settings on one side only:
% |012|~(`fi' ligature), |013|~(`fl'), |014|~(`ffi'), |015|~(`ffl'),
% \AE, \ae, \OE, \oe.
%    \begin{macrocode}
\DeclareCharacterInheritance
   { encoding = OT1 }
   { f = {011}, % ff
     i = {\i},
     j = {\j},
     O = {\O},
     o = {\o},
   }

%    \end{macrocode}
%
%\subsubsection{\tI}
% Candidates here: |028|~(`fi'), |029|~(`fl'), |030|~(`ffi'), |031|~(`ffl'),
% |156| (`IJ'~ligature), |188|~(`ij'), \AE, \ae, \OE, \oe.
%\changes{v1.5}{2004/12/11}{remove \cmd{\ss} from \tI\ list, add \cmd{\DJ}}
%\changes{v1.8}{2005/04/26}{remove \cmd{\DJ} from \tI\ list (it's the same as \cmd{\DH})}
%    \begin{macrocode}
\DeclareCharacterInheritance
   { encoding = T1 }
   { A = {\`A,\'A,\^A,\~A,\"A,\r{A},\k{A},\u{A}},
     a = {\`a,\'a,\^a,\~a,\"a,\r{a},\k{a},\u{a}},
     C = {\'C,\c{C},\v{C}},
     c = {\'c,\c{c},\v{c}},
     D = {\v{D},\DH},
     d = {\v{d},\dj},
     E = {\`E,\'E,\^E,\"E,\k{E},\v{E}},
     e = {\`e,\'e,\^e,\"e,\k{e},\v{e}},
     f = {027}, % ff
     G = {\u{G}},
     g = {\u{g}},
     I = {\`I,\'I,\^I,\"I,\.I},
     i = {\`i,\'i,\^i,\"i,\i},
     j = {\j},
     L = {\L,\'L,\v{L}},
     l = {\l,\'l,\v{l}},
     N = {\'N,\~N,\v{N}},
     n = {\'n,\~n,\v{n}},
     O = {\O,\`O,\'O,\^O,\~O,\"O,\H{O}},
     o = {\o,\`o,\'o,\^o,\~o,\"o,\H{o}},
     R = {\'R,\v{R}},
     r = {\'r,\v{r}},
     S = {\'S,\c{S},\v{S},\SS},
     s = {\'s,\c{s},\v{s}},
     T = {\c{T},\v{T}},
     t = {\c{t},\v{t}},
     U = {\`U,\'U,\^U,\"U,\H{U},\r{U}},
     u = {\`u,\'u,\^u,\"u,\H{u},\r{u}},
     Y = {\'Y,\"Y},
     y = {\'y,\"y},
     Z = {\'Z,\.Z,\v{Z}},
     z = {\'z,\.z,\v{z}},
     - = {127},
   }

%    \end{macrocode}
%
%\subsubsection{\lyI}
% More characters: |008|~(`fl'), |012|~(`fi'), |014|~(`ffi'), |015|~(`ffl'),
% \AE, \ae, \OE, \oe.
%    \begin{macrocode}
\DeclareCharacterInheritance
   { encoding = LY1 }
   { A = {\`A,\'A,\^A,\~A,\"A,\r{A}},
     a = {\`a,\'a,\^a,\~a,\"a,\r{a}},
     C = {\c{C}},
     c = {\c{c}},
     D = {\DH},
     E = {\`E,\'E,\^E,\"E},
     e = {\`e,\'e,\^e,\"e},
     f = {011}, % ff
     I = {\`I,\'I,\^I,\"I},
     i = {\`i,\'i,\^i,\"i,\i},
     L = {\L},
     l = {\l},
     N = {\~N},
     n = {\~n},
     O = {\`O,\'O,\^O,\~O,\"O,\O},
     o = {\`o,\'o,\^o,\~o,\"o,\o},
     S = {\v{S}},
     s = {\v{s}},
     U = {\`U,\'U,\^U,\"U},
     u = {\`u,\'u,\^u,\"u},
     Y = {\'Y,\"Y},
     y = {\'y,\"y},
     Z = {\v{Z}},
     z = {\v{z}},
   }

%    \end{macrocode}
%
%\subsubsection{\otIV}
%\changes{v1.9}{2005/08/16}{add list for \otIV}
% The Polish \otI\ extension. More interesting characters here:
% |009|~(`fk'), |012|~(`fi'), |013|~(`fl'), |014|~(`ffi'), |015|~(`ffl'),
% \AE, \ae, \OE, \oe.
%    \begin{macrocode}
\DeclareCharacterInheritance
   { encoding = OT4 }
   { A = {\k{A}},
     a = {\k{a}},
     C = {\'C},
     c = {\'c},
     E = {\k{E}},
     e = {\k{e}},
     f = {011}, % ff
     i = {\i},
     j = {\j},
     L = {\L},
     l = {\l},
     N = {\'N},
     n = {\'n},
     O = {\O,\'O},
     o = {\o,\'o},
     S = {\'S},
     s = {\'s},
     Z = {\'Z,\.Z},
     z = {\'z,\.z},
   }

%    \end{macrocode}
%
%\subsubsection{\tV}
%\changes{v1.9}{2005/08/17}{add list for \tV\ (requested by \thanh)} ^^A private email, 16/8/2005
% The Vietnamese encoding \tV. It is so crowded with accented and double-accented
% characters that there is no room for any ligatures.
%    \begin{macrocode}
\DeclareCharacterInheritance
   { encoding = T5 }
   { A = {\`A,\'A,\~A,\h{A},\d{A},\^A,\u{A},
          \`\Acircumflex,\'\Acircumflex,\~\Acircumflex,\h\Acircumflex,\d\Acircumflex,
          \`\Abreve,\'\Abreve,\~\Abreve,\h\Abreve,\d\Abreve},
     a = {\`a,\'a,\~a,\h{a},\d{a},\^a,\u{a},
          \`\acircumflex,\'\acircumflex,\~\acircumflex,\h\acircumflex,\d\acircumflex,
          \`\abreve,\'\abreve,\~\abreve,\h\abreve,\d\abreve},
     D = {\DJ},
     d = {\dj},
     E = {\`E,\'E,\~E,\h{E},\d{E},\^E,
          \`\Ecircumflex,\'\Ecircumflex,\~\Ecircumflex,\h\Ecircumflex,\d\Ecircumflex},
     e = {\`e,\'e,\~e,\h{e},\d{e},\^e,
          \`\ecircumflex,\'\ecircumflex,\~\ecircumflex,\h\ecircumflex,\d\ecircumflex},
     I = {\`I,\'I,\~I,\h{I},\d{I}},
     i = {\`i,\'i,\~i,\h{i},\d{i},\i},
     O = {\`O,\'O,\~O,\h{O},\d{O},\^O,\horn{O},
          \`\Ocircumflex,\'\Ocircumflex,\~\Ocircumflex,\h\Ocircumflex,\d\Ocircumflex,
          \`\Ohorn,\'\Ohorn,\~\Ohorn,\h\Ohorn,\d\Ohorn},
     o = {\`o,\'o,\~o,\h{o},\d{o},\^o,\horn{o},
          \`\ocircumflex,\'\ocircumflex,\~\ocircumflex,\h\ocircumflex,\d\ocircumflex,
          \`\ohorn,\'\ohorn,\~\ohorn,\h\ohorn,\d\ohorn},
     U = {\`U,\'U,\~U,\h{U},\d{U},\horn{U},
          \`\Uhorn,\'\Uhorn,\~\Uhorn,\h\Uhorn,\d\Uhorn},
     u = {\`u,\'u,\~u,\h{u},\d{u},\horn{u},
          \`\uhorn,\'\uhorn,\~\uhorn,\h\uhorn,\d\uhorn},
     Y = {\`Y,\'Y,\~Y,\h{Y},\d{Y}},
     y = {\`y,\'y,\~y,\h{y},\d{y}},
   }

%</m-t>
%    \end{macrocode}
%
%\subsection{Font Expansion}
%\GeneralChanges{Expansion}
%
% These are \thanh's original expansion settings. They are used for all fonts
% (until somebody shows mercy and creates font-specific settings).
%    \begin{macrocode}
%<*m-t>
%%% ----------------------------------------------------------------------
%%% EXPANSION SETTINGS

\SetExpansion
%<m-t>   [ name     = default      ]
%<bch>   [ name     = bch-default  ]
%<cmr>   [ name     = cmr-default  ]
%<pad>   [ name     = pad-default  ]
%<pmn>   [ name     = pmn-default  ]
%<ppl>   [ name     = ppl-default  ]
%<ptm>   [ name     = ptm-default  ]
%<m-t>   { encoding = {OT1,OT4,T1,LY1} }
%<!m-t>   { encoding = {OT1,T1,LY1},
%<bch>     family   = bch }
%<cmr>     family   = cmr }
%<pad>     family   = {pad,padx,padj} }
%<pmn>     family   = {pmn,pmnx,pmnj} }
%<ppl>     family   = {ppl,pplx,pplj} }
%<ptm>     family   = {ptm,ptmx,ptmj} }
   {
     A = 500,     a = 700,
   \AE = 500,   \ae = 700,
     B = 700,     b = 700,
     C = 700,     c = 700,
     D = 500,     d = 700,
     E = 700,     e = 700,
     F = 700,
     G = 500,     g = 700,
     H = 700,     h = 700,
     K = 700,     k = 700,
     M = 700,     m = 700,
     N = 700,     n = 700,
     O = 500,     o = 700,
   \OE = 500,   \oe = 700,
     P = 700,     p = 700,
     Q = 500,     q = 700,
     R = 700,
     S = 700,     s = 700,
     U = 700,     u = 700,
     W = 700,     w = 700,
     Z = 700,     z = 700,
     2 = 700,
     3 = 700,
     6 = 700,
     8 = 700,
     9 = 700,
   }

%    \end{macrocode}
% \tV\ encoding does not contain \cmd{\AE}, \cmd{\ae}, \cmd{\OE} and \cmd{\oe}.
%    \begin{macrocode}
\SetExpansion
   [ name     = T5 ]
   { encoding = T5 }
   {
     A = 500,     a = 700,
     B = 700,     b = 700,
     C = 700,     c = 700,
     D = 500,     d = 700,
     E = 700,     e = 700,
     F = 700,
     G = 500,     g = 700,
     H = 700,     h = 700,
     K = 700,     k = 700,
     M = 700,     m = 700,
     N = 700,     n = 700,
     O = 500,     o = 700,
     P = 700,     p = 700,
     Q = 500,     q = 700,
     R = 700,
     S = 700,     s = 700,
     U = 700,     u = 700,
     W = 700,     w = 700,
     Z = 700,     z = 700,
     2 = 700,
     3 = 700,
     6 = 700,
     8 = 700,
     9 = 700,
   }

%</m-t>
%    \end{macrocode}
%
%\subsection{Character Protrusion}
%\GeneralChanges{Protrusion}
%
%\changes{v1.1}{2004/09/14}{add factors for some more characters}
%\changes{v1.4b}{2004/11/19}{harmonize dashes in upshape and italic
%                           (\texttt{cmr}, \texttt{pad}, \texttt{ppl})}
%\changes{v1.9}{2005/08/16}{settings for \otIV\ encoding (Computer Modern Roman,
%                           Palatino, Times)}
%\changes{v1.9}{2005/08/17}{settings for \tV\ encoding (Computer Modern Roman)}
%
%    \begin{macrocode}
%%% ----------------------------------------------------------------------
%%% PROTRUSION SETTINGS

%    \end{macrocode}
% For future historians, \thanh's original settings (from \file{protcode.tex},
% converted to \microtype\ notation).
%\begin{verbatim}
%\SetProtrusion
%   [ name     = thanh ]
%   { encoding = OT1 }
%   {
%     A = {50,50},
%     F = {  ,50},
%     J = {50,  },
%     K = {  ,50},
%     L = {  ,50},
%     T = {50,50},
%     V = {50,50},
%     W = {50,50},
%     X = {50,50},
%     Y = {50,50},
%     k = {  ,50},
%     r = {  ,50},
%     t = {  ,50},
%     v = {50,50},
%     w = {50,50},
%     x = {50,50},
%     y = {50,50},
%     . = { ,700},    {,}= { ,700},
%     : = { ,500},     ; = { ,500},
%     ! = { ,200},     ? = { ,200},
%     ( = {50,  },     ) = {  ,50},
%     - = { ,700},
%     \textendash        = { ,300},     \textemdash        = { ,200},
%     \textquoteleft     = {700, },     \textquoteright    = { ,700},
%     \textquotedblleft  = {500, },     \textquotedblright = { ,500},
%   }
%\end{verbatim}
%
% We also create configuration files for the fonts
%\changes{v1.5}{2004/11/28}{settings for Bitstream Charter}
% Bitstream Charter (\nfss\ code |bch|),
% Computer Modern Roman (|cmr|),
% Palatino (|ppl|, |pplx|, |pplj|),
% Times (|ptm|, |ptmx|, |ptmj|),
% Adobe Garamond (|pad|, |padx|, |padj|) and
%\changes{v1.1}{2004/09/14}{settings for Adobe Minion
%                           (contributed by Harald Harders)}
% Minion\footnote{
%     Contributed by Harald Harders (\mailto{h.harders@tu-bs.de})}
% (|pmnx|, |pmnj|),
%\changes{v1.8}{2005/06/01}{settings for \ams\ math fonts}
% and for the \ams\ math fonts (|msa|, |msb|, |euf|, |eus|).
%
%\subsubsection{Default}
%
% The default settings always use the most moderate value.
%    \begin{macrocode}
%<*!cfg-u>
\SetProtrusion
%<m-t>   [ name     = default ]
%<bch>   [ name     = bch-default ]
%<cmr>   [ name     = cmr-default ]
%<pad>   [ name     = pad-default ]
%<pmn>   [ name     = pmnj-default ]
%<ppl>   [ name     = ppl-default ]
%<ptm>   [ name     = ptm-default ]
%<m-t>   { encoding = OT1     }
%<cmr>   { }
%<bch|pad|pmn>   { encoding = OT1,
%<ppl|ptm>   { encoding = {OT1,OT4},
%<bch>     family   = bch }
%<pad>     family   = {pad,padx,padj} }
%<pmn>     family   = pmnj }
%<ppl>     family   = {ppl,pplx,pplj} }
%<ptm>     family   = {ptm,ptmx,ptmj} }
   {
     A = {50,50},
%<m-t|pad|ptm>   \AE = {50,  },
%<bch|pad|pmn>     C = {50,  },
%<bch|pad|pmn>     D = {  ,50},
%<m-t|bch|cmr|pad|pmn|ptm>     F = {  ,50},
%<bch|pad|pmn>     G = {50,  },
%<m-t|cmr|pad|pmn|ppl|ptm>     J = {50,  },
%<bch>     J = {100,  },
     K = {  ,50},
%<m-t|bch|cmr|pad|pmn|ppl>     L = {  ,50},
%<ptm>     L = {  ,80},
%<bch|pad|pmn>     O = {50,50},
%<bch|pad|pmn>   \OE = {50,  },
%<bch|pad|pmn>     Q = {50,70},
%<bch>     R = {  ,50},
     T = {50,50},
     V = {50,50},
     W = {50,50},
     X = {50,50},
%<m-t|bch|cmr|pad|pmn|ppl>     Y = {50,50},
%<ptm>     Y = {80,80},
     k = {  ,50},
%<pmn>     l = {  ,-50},
%<pad|ppl>     p = {50,50},
%<pad|ppl>     q = {50,  },
     r = {  ,50},
%<cmr|pad|pmn>     t = {  ,70},
%<bch>     t = {  ,50},
     v = {50,50},
     w = {50,50},
     x = {50,50},
%<m-t|bch|pad|pmn>     y = {  ,50},
%<cmr|ppl|ptm>     y = {50,70},
%    \end{macrocode}
%\changes{v1.6}{2005/01/11}{improve settings for numbers
%                          (pointed out by Peter Muthesius)} ^^A <newsname\at gmx.de>
%                                        ^^A in: <MID:34b1h7F48s44tU1@individual.net>
%    \begin{macrocode}
%<cmr>     0 = {  ,50},
%<m-t>     1 = {50,50},
%<bch|pad|ptm>     1 = {150,150},
%<cmr>     1 = {100,200},
%<pmn>     1 = {  ,50},
%<ppl>     1 = {100,100},
%<bch|cmr|pad>     2 = {50,50},
%<cmr|pad>     3 = {50,50},
%<bch|pmn>     3 = {50,  },
%<m-t|pad>     4 = {50,50},
%<bch>     4 = {100,50},
%<cmr>     4 = {70,70},
%<pmn>     4 = {50,  },
%<ptm>     4 = {70,  },
%<cmr>     5 = {  ,50},
%<pad>     5 = {50,50},
%<bch>     6 = {50,  },
%<cmr>     6 = {  ,50},
%<pad>     6 = {50,50},
%<m-t>     7 = {50,50},
%<bch|pad|pmn>     7 = {50,80},
%<cmr|ptm>     7 = {50,100},
%<ppl>     7 = { ,50},
%<cmr>     8 = {  ,50},
%<bch|pad>     9 = {50,50},
%<cmr>     9 = {  ,50},
%<m-t|cmr|pad|pmn|ppl|ptm>     . = { ,700},
%<bch>     . = { ,600},
    {,}= { ,500},
%<m-t|cmr|pad|pmn|ppl|ptm>     : = { ,500},
%<bch>     : = { ,400},
%<m-t|bch|pad|pmn|ptm>     ; = { ,300},
%<cmr|ppl>     ; = { ,500},
     ! = { ,100},
%<m-t|pad|pmn|ptm>     ? = { ,100},
%<bch|cmr|ppl>     ? = { ,200},
%<pmn>     " = {300,300},
%<m-t|bch|cmr|pad|pmn|ppl>     @ = {50,50},
%<ptm>     @ = {100,100},
     ~ = {200,250},
%<m-t|bch|pad|pmn|ppl|ptm>     _ = {100,100},
%<cmr>     _ = {200,200},
%<pad|ppl|ptm>     & = {50,100},
%<m-t|cmr|pad|pmn>    \% = {50,50},
%<bch>    \% = {  ,50},
%<ppl|ptm>    \% = {100,100},
%<m-t|ppl|ptm>     * = {200,200},
%<bch|pmn>     * = {200,300},
%<cmr|pad>     * = {300,300},
%<m-t|cmr|ppl|ptm>     + = {250,250},
%<bch>     + = {150,250},
%<pad>     + = {300,300},
%<pmn>     + = {150,200},
%<m-t|pad|pmn|ptm>     ( = {100,   },    ) = {   ,200},
%<bch>     ( = {200,   },    ) = {   ,200},
%<cmr|ppl>     ( = {100,   },    ) = {   ,300},
%<bch|pmn>     [ = {100,   },    ] = {   ,100},
%    \end{macrocode}
%\changes{v1.7}{2005/03/15}{fix: remove \textbackslash\ from \otI,
%                           add \cmd{\textbackslash} to \tI\ encoding}
%    \begin{macrocode}
%<m-t|pad|pmn|ptm>     / = {100,200},
%<bch>     / = { ,200},
%<cmr|ppl>     / = {200,300},
%<m-t|ptm>     - = {500,500},
%<bch|cmr|ppl>     - = {400,500},
%<pad>     - = {300,500},
%<pmn>     - = {200,400},
%<m-t|pmn>     \textendash       = {200,200},   \textemdash        = {150,150},
%<bch>     \textendash       = {200,300},   \textemdash        = {150,250},
%<cmr>     \textendash       = {400,300},   \textemdash        = {300,200},
%<pad|ppl|ptm>     \textendash       = {300,300},   \textemdash        = {200,200},
%    \end{macrocode}
% Why settings for left \emph{and} right quotes? Because in some languages they
% might be used like that (see the \pkg{csquotes} package for examples).
%    \begin{macrocode}
%<m-t|bch|pmn>     \textquoteleft    = {300,400},   \textquoteright    = {300,400},
%<cmr>     \textquoteleft    = {500,700},   \textquoteright    = {500,600},
%<pad|ppl>     \textquoteleft    = {500,700},   \textquoteright    = {500,700},
%<ptm>     \textquoteleft    = {500,500},   \textquoteright    = {300,500},
%<m-t|bch|pmn>     \textquotedblleft = {300,300},   \textquotedblright = {300,300},
%<cmr>     \textquotedblleft = {500,300},   \textquotedblright = {200,600},
%<pad|ppl|ptm>     \textquotedblleft = {300,400},   \textquotedblright = {300,400},
   }

%    \end{macrocode}
% Greek uppercase letters are in \otI\ encoding only.
%\changes{v1.9}{2005/07/10}{fix: remove uppercase Greek letters from \tI\ encoded CMR}
%    \begin{macrocode}
%<*cmr>
\SetProtrusion
   [ name     = cmr-OT1,
     load     = cmr-default ]
   { encoding = {OT1,OT4},
     family   = cmr  }
   {
     \AE = { 50,   },
     "00 = {   ,150}, % \Gamma
     "01 = {100,100}, % \Delta
     "02 = { 50, 50}, % \Theta
     "03 = {100,100}, % \Lambda
%    "04 = {   ,   }, % \Xi
%    "05 = {   ,   }, % \Pi
     "06 = { 50, 50}, % \Sigma
     "07 = {100,100}, % \Upsilon
     "08 = { 50, 50}, % \Phi
     "09 = { 50, 50}, % \Psi
%    "0A = {   ,   }, % \Omega
   }

%</cmr>
%    \end{macrocode}
% \tI\ and \lyI\ encodings contain some more characters. The default list
% will be loaded first.
%    \begin{macrocode}
\SetProtrusion
%<m-t>   [ name     = T1-default,
%<bch>   [ name     = bch-T1,
%<cmr>   [ name     = cmr-T1,
%<pad>   [ name     = pad-T1,
%<pmn>   [ name     = pmnj-T1,
%<ppl>   [ name     = ppl-T1,
%<ptm>   [ name     = ptm-T1,
%<m-t>     load     = default     ]
%<bch>     load     = bch-default ]
%<cmr>     load     = cmr-default ]
%<pad>     load     = pad-default ]
%<pmn>     load     = pmnj-default ]
%<ppl>     load     = ppl-default ]
%<ptm>     load     = ptm-default ]
%<m-t>   { encoding = {T1,LY1}    }
%<bch|cmr|pad|pmn|ppl>   { encoding = {T1,LY1},
%<ptm>   { encoding = {T1},
%<bch>     family   = bch }
%<cmr>     family   = cmr }
%<pad>     family   = {pad,padx,padj} }
%<pmn>     family   = pmnj }
%<ppl>     family   = {ppl,pplx,pplj} }
%<ptm>     family   = {ptm,ptmx,ptmj} }
   {
%<cmr>     \AE = {50,  },
%<pmn>     \TH = {  ,50},
%<m-t|pad|pmn|ptm>     \textbackslash    = {100,200},
%<bch>     \textbackslash    = {150,200},
%<cmr|ppl>     \textbackslash    = {200,300},
%<cmr>     \textquotedblleft = {200,600},
%<cmr>     \textquotedbl     = {300,300},
%    \end{macrocode}
%\changes{v1.4}{2004/10/30}{tweak quote characters for \texttt{cmr}
%                           variants (\otI, \tI, \texttt{lmr})}
% The EC fonts do something weird: They insert an implicit kern between quote
% and boundary character. Therefore, we must override the settings from \otI.
%    \begin{macrocode}
%<m-t|cmr|pad|ppl|ptm>     \quotesinglbase   = {400,400},   \quotedblbase      = {400,400},
%<bch|pmn>     \quotesinglbase   = {400,400},   \quotedblbase      = {300,300},
%<m-t|bch|pmn>     \guilsinglleft    = {400,300},   \guilsinglright    = {300,400},
%<cmr|pad|ppl|ptm>     \guilsinglleft    = {400,400},   \guilsinglright    = {300,500},
%<m-t>     \guillemotleft    = {200,200},   \guillemotright    = {200,200},
%<cmr>     \guillemotleft    = {300,200},   \guillemotright    = {100,400},
%<bch|pmn>     \guillemotleft    = {200,200},   \guillemotright    = {150,300},
%<pad>     \guillemotleft    = {300,300},   \guillemotright    = {200,400},
%<ppl|ptm>     \guillemotleft    = {300,300},   \guillemotright    = {200,400},
%<m-t|bch|cmr|pad|pmn|ppl>     \textexclamdown   = {100,   },   \textquestiondown  = {100,   },
%<ptm>     \textexclamdown   = {200,   },   \textquestiondown  = {200,   },
%<m-t|cmr|pad|ppl|ptm>     \textbraceleft    = {400,200},   \textbraceright    = {200,400},
%<bch|pmn>     \textbraceleft    = {200,   },   \textbraceright    = {   ,300},
%<m-t|bch|cmr|pad|ppl|ptm>     \textless         = {200,100},   \textgreater       = {100,200},
%<pmn>     \textless         = {100,   },   \textgreater       = {   ,100},
%<pmn>     \textvisiblespace = {100,100}, % not in LY1
%   \dh = {  ,  },
%   \th = {  ,  },
%   \NG = {  ,  },
%   \ng = {  ,  },
%   \textasciicircum = {  ,  },
%   \textbar = {  ,  },
%   \textsterling = {  ,  }, % also in TS1
%   \textsection = {  ,  }, % also in TS1
   }

%<*cmr>
%    \end{macrocode}
% \tV\ is based on \otI; it shares some but not all extra characters of \tI.
% All accented characters are already taken care of by the inheritance list.
%    \begin{macrocode}
\SetProtrusion
   [ name = cmr-T5,
     load = cmr-default ]
   { encoding = T5,
     family   = cmr }
   {
     \textbackslash    = {200,300},
     \textquotedblleft = {200,600},
     \textquotedbl     = {300,300},
     \quotesinglbase   = {400,400},   \quotedblbase      = {400,400},
     \guilsinglleft    = {400,400},   \guilsinglright    = {300,500},
     \guillemotleft    = {300,200},   \guillemotright    = {100,400},
     \textbraceleft    = {400,200},   \textbraceright    = {200,400},
     \textless         = {200,100},   \textgreater       = {100,200},
   }

%    \end{macrocode}
% The \pkg{lmodern} fonts, on the other hand, restore the original kerning from
% the \otI\ fonts, and so do we. Silly, isn't it?
%    \begin{macrocode}
\SetProtrusion
   [ name     = lmr-T1,
     load     = cmr-T1   ]
   { encoding = {T1,LY1},
     family   = lmr      }
   {
     \textquotedblleft = {500,300},
     \quotedblbase     = {500,300},
   }

%</cmr>
%<*pmn>
\SetProtrusion
   [ name     = pmnx-OT1,
     load     = pmnj-default ]
   { encoding = OT1,
     family   = pmnx }
   {
     1 = {230,180},
   }

\SetProtrusion
   [ name     = pmnx-T1,
     load     = pmnj-T1  ]
   { encoding = {T1,LY1},
     family   = pmnx     }
   {
     1 = {230,180},
   }

%</pmn>
%    \end{macrocode}
% Times is the default font for \lyI, therefore we provide settings for the
% additional characters in this encoding, too.
%\changes{v1.8}{2005/04/26}{add \lyI\ characters for Times}
%    \begin{macrocode}
%<*ptm>
 \SetProtrusion
   [ name     = ptm-LY1,
     load     = ptm-T1 ]
   { encoding = LY1,
     family   = {ptm,ptmx,ptmj} }
   {
     \texttrademark            = {100,100},
     \textregistered           = {100,100},
     \textcopyright            = {100,100},
     \textdegree               = {300,300},
     \textminus                = {200,200},
     \textellipsis             = {100,100},
     \texteuro                 = {   ,   }, % ?
     \textcent                 = {100,100},
     \textquotesingle          = {500,500},
     \textflorin               = { 50, 70},
     \textdagger               = {150,150},
     \textdaggerdbl            = {100,100},
     \textperthousand          = {   , 50},
     \textbullet               = {150,150},
     \textonesuperior          = {100,100},
     \texttwosuperior          = { 50, 50},
     \textthreesuperior        = { 50, 50},
     \textperiodcentered       = {300,300},
     \textplusminus            = { 50, 80},
     \textmultiply             = {100,100},
     \textdivide               = { 50,150},
%    \textbrokenbar            = {   ,   },
%    \textyen                  = {   ,   },
%    \textfractionsolidus      = {   ,   },
%    \textordfeminine          = {   ,   },
%    \textordmasculine         = {   ,   },
%    \textmu                   = {   ,   },
%    \textparagraph            = {   ,   },
%    \textonequarter           = {   ,   },
%    \textonehalf              = {   ,   },
%    \textthreequarters        = {   ,   },
   }

%</ptm>
%    \end{macrocode}
%
%\subsubsection{Italics}
%
% To find default settings for italic is difficult, since the character shapes
% and their behaviour at the beginning or end of line may be wildly different
% for different fonts. Therefore, we leave the letters away, and only set up the
% punctuation characters.
%\changes{v1.4b}{2004/11/22}{slanted like italics}
%\changes{v1.6}{2004/12/26}{add italic uppercase Greek letters}
%    \begin{macrocode}
\SetProtrusion
%<m-t>   [ name     = OT1-it   ]
%<bch>   [ name     = bch-it   ]
%<cmr>   [ name     = cmr-it   ]
%<pad>   [ name     = pad-it   ]
%<pmn>   [ name     = pmnj-it  ]
%<ppl>   [ name     = ppl-it   ]
%<ptm>   [ name     = ptm-it   ]
%<m-t|bch|pad|pmn>   { encoding = OT1,
%<cmr>   { }
%<ppl|ptm>   { encoding = {OT1,OT4},
%<bch>     family   = bch,
%<pad>     family   = {pad,padx,padj},
%<pmn>     family   = pmnj,
%<ppl>     family   = {ppl,pplx,pplj},
%<ptm>     family   = {ptm,ptmx,ptmj},
%<!cmr>     shape    = {it,sl}  }
   {
%<cmr|ptm>     A = {100,50},
%<pad|pmn>     A = {50,  },
%<ppl>     A = {50,50},
%<ptm>   \AE = {100,  },
%<pad|ppl>   \AE = {50,  },
%<pmn>   \AE = {  ,-50},
%<cmr|pad|ppl|ptm>     B = {50,  },
%<pmn>     B = {20,-50},
%<bch|ppl|ptm>     C = {50,  },
%<cmr|pad>     C = {100, },
%<pmn>     C = {50,-50},
%<cmr|pad|ppl|ptm>     D = {50,50},
%<pmn>     D = {20,  },
%<cmr|pad|ppl|ptm>     E = {50,  },
%<pmn>     E = {20,-50},
%<cmr|pad|ptm>     F = {100, },
%<pmn>     F = {10,  },
%<ppl>     F = {50,  },
%<bch|ppl|ptm>     G = {50,  },
%<cmr|pad>     G = {100, },
%<pmn>     G = {50,-50},
%<cmr|pad|ppl|ptm>     H = {50,  },
%<cmr|pad|ptm>     I = {50,  },
%<pmn>     I = {20,-50},
%<cmr|ptm>     J = {100, },
%<pad>     J = {50,  },
%<pmn>     J = {20,  },
%<cmr|pad|ppl|ptm>     K = {50,  },
%<pmn>     K = {20,  },
%<cmr|pad|ppl|ptm>     L = {50,  },
%<pmn>     L = {20,50},
%<cmr|ptm>     M = {50,  },
%<pmn>     M = {  ,-30},
%<cmr|ptm>     N = {50,  },
%<pmn>     N = {  ,-30},
%<bch|pmn|ppl|ptm>     O = {50,  },
%<cmr|pad>     O = {100, },
%<bch|pmn|ppl|ptm>   \OE = {50,  },
%<pad>   \OE = {100, },
%<cmr|pad|ppl|ptm>     P = {50,  },
%<pmn>     P = {20,-50},
%<bch|pmn|ppl|ptm>     Q = {50,  },
%<cmr|pad>     Q = {100, },
%<cmr|pad|ppl|ptm>     R = {50,  },
%<pmn>     R = {20,  },
%<bch|cmr|pad|ppl|ptm>     S = {50,  },
%<pmn>     S = {20,-30},
%<bch|cmr|pad|ppl|ptm>     $ = {50,  },
%<pmn>     $ = {20,-30},
%<bch|pmn>     T = {70,  },
%<cmr|pad|ppl|ptm>     T = {100, },
%<cmr|pad|ppl|ptm>     U = {50,  },
%<pmn>     U = {50,-50},
%<cmr|pad|pmn>     V = {100, },
%<ppl|ptm>     V = {100,50},
%<cmr|pad|pmn>     W = {100, },
%<ppl>     W = {50,  },
%<ptm>     W = {100,50},
%<cmr|ppl|ptm>     X = {50,  },
%<cmr|ptm>     Y = {100, },
%<pmn>     Y = {50,  },
%<ppl>     Y = {100,50},
%<pmn>     Z = {  ,-50},
%<pmn>     d = {  ,-50},
%<pad|pmn>     f = { ,-100},
%<pmn>     i = {  ,-30},
%<pmn>     j = {  ,-30},
%<pmn>     l = {  ,-100},
%<bch>     o = {50,50},
%<bch>     p = {  ,50},
%<pmn>     p = {-50,  },
%<bch>     q = {50,  },
%<pmn>     r = {  ,50},
%<bch>     t = {  ,50},
%<pmn>     v = {50,  },
%<bch>     w = {  ,50},
%<pmn>     w = {50,  },
%<bch>     y = {  ,50},
%<cmr>     0 = {100, },
%<bch|ptm>     1 = {150,100},
%<cmr>     1 = {200,50},
%<pad>     1 = {150, },
%<pmn>     1 = {50,  },
%<ppl>     1 = {100, },
%<cmr>     2 = {100,-100},
%<pad|ppl|ptm>     2 = {50,  },
%<pmn>     2 = {-50,  },
%<bch>     3 = {50,  },
%<cmr>     3 = {100,-100},
%<pmn>     3 = {-100, },
%<ptm>     3 = {100,50},
%<bch>     4 = {100, },
%<cmr|pad>     4 = {150, },
%<ppl|ptm>     4 = {50,  },
%<cmr>     5 = {100, },
%<ptm>     5 = {50,  },
%<bch>     6 = {50,  },
%<cmr>     6 = {100, },
%<bch|pad|ptm>     7 = {100, },
%<cmr>     7 = {200,-150},
%<pmn>     7 = {20,  },
%<ppl>     7 = {50,  },
%<cmr>     8 = {50,-50},
%<cmr>     9 = {100,-100},
%<m-t|cmr|pad|pmn|ppl>     . = { ,500},
%<bch|ptm>     . = { ,700},
%<m-t|cmr|pad|pmn|ppl>    {,}= { ,500},
%<bch>    {,}= { ,600},
%<ptm>    {,}= { ,700},
%<m-t|cmr|pad|ppl>     : = { ,300},
%<bch>     : = { ,400},
%<pmn>     : = { ,200},
%<ptm>     : = { ,500},
%<m-t|cmr|pad|ppl>     ; = { ,300},
%<bch>     ; = { ,400},
%<pmn>     ; = { ,200},
%<ptm>     ; = { ,500},
%<ptm>     ! = { ,100},
%<bch>     ? = { ,200},
%<ptm>     ? = { ,100},
%<ppl>     ? = { ,300},
%<pmn>     " = {400,200},
%<m-t|bch|pmn>     _ = {  ,100},
%<cmr>     _ = {100,200},
%<pad|ppl|ptm>     _ = {100,100},
%<m-t|pad|pmn|ppl|ptm>     & = {50,50},
%<bch>     & = {  ,80},
%<cmr>     & = {100,50},
%<m-t|cmr|pad|pmn>    \% = {100, },
%<bch>    \% = {50,50},
%<ppl|ptm>    \% = {100,100},
%<m-t|pmn|ppl>     * = {200,200},
%<bch>     * = {300,200},
%<cmr>     * = {400,100},
%<pad>     * = {500,100},
%<ptm>     * = {400,200},
%<m-t|cmr|pmn|ppl>     + = {150,200},
%<bch>     + = {250,250},
%<pad|ptm>     + = {250,200},
%<m-t|pad|pmn|ppl>     @ = {50,50},
%<bch>     @ = {80,50},
%<cmr>     @ = {200,50},
%<ptm>     @ = {150,150},
%<m-t|bch>     ~ = {150,150},
%<cmr|pad|pmn|ppl|ptm>     ~ = {200,150},
     ( = {200, },    ) = {  ,200},
%<m-t|cmr|pad|ppl|ptm>     / = {100,200},
%<bch>     / = {  ,150},
%<pmn>     / = {100,150},
%<m-t>     - = {300,300},
%<bch|pad>     - = {300,400},
%<pmn>     - = {200,300},
%<cmr>     - = {500,300},
%<ppl>     - = {300,500},
%<ptm>     - = {500,500},
%<m-t|pmn>     \textendash       = {200,200},   \textemdash        = {150,150},
%<bch>     \textendash       = {200,300},   \textemdash        = {150,200},
%<cmr>     \textendash       = {500,300},   \textemdash        = {400,200},
%<pad|ppl|ptm>     \textendash       = {300,300},   \textemdash        = {200,200},
%<m-t|bch|pmn>     \textquoteleft    = {400,200},   \textquoteright    = {400,200},
%<cmr|pad>     \textquoteleft    = {800,200},   \textquoteright    = {800,200},
%<ppl>     \textquoteleft    = {700,400},   \textquoteright    = {700,400},
%<ptm>     \textquoteleft    = {800,500},   \textquoteright    = {800,500},
%<m-t|bch|pmn>     \textquotedblleft = {400,200},   \textquotedblright = {400,200},
%<cmr>     \textquotedblleft = {700,100},   \textquotedblright = {500,300},
%<pad>     \textquotedblleft = {700,200},   \textquotedblright = {700,200},
%<ppl>     \textquotedblleft = {500,300},   \textquotedblright = {500,300},
%<ptm>     \textquotedblleft = {700,400},   \textquotedblright = {700,400},
   }

%<*cmr>
\SetProtrusion
   [ name     = cmr-it-OT1,
     load     = cmr-it   ]
   { encoding = {OT1,OT4},
     family   = cmr,
     shape    = it       }
   {
     \AE = {100,   },
     \OE = {100,   },
     "00 = {200,150}, % \Gamma
     "01 = {150,100}, % \Delta
     "02 = {150, 50}, % \Theta
     "03 = {150, 50}, % \Lambda
     "04 = {100,100}, % \Xi
     "05 = {100,100}, % \Pi
     "06 = {100, 50}, % \Sigma
     "07 = {200,150}, % \Upsilon
     "08 = {150, 50}, % \Phi
     "09 = {150,100}, % \Psi
     "0A = { 50, 50}, % \Omega
   }

%</cmr>
\SetProtrusion
%<m-t>   [ name     = T1-it-default,
%<bch>   [ name     = bch-it-T1,
%<cmr>   [ name     = cmr-it-T1,
%<pad>   [ name     = pad-it-T1,
%<pmn>   [ name     = pmnj-it-T1,
%<ppl>   [ name     = ppl-it-T1,
%<ptm>   [ name     = ptm-it-T1,
%<m-t>     load     = OT1-it   ]
%<bch>     load     = bch-it   ]
%<cmr>     load     = cmr-it   ]
%<pmn>     load     = pmnj-it  ]
%<pad>     load     = pad-it   ]
%<ppl>     load     = ppl-it   ]
%<ptm>     load     = ptm-it   ]
%<m-t|bch|cmr|pad|pmn|ppl>   { encoding = {T1,LY1},
%<ptm>   { encoding = {T1},
%<bch>     family   = bch,
%<cmr>     family   = cmr,
%<pmn>     family   = pmnj,
%<pad>     family   = {pad,padx,padj},
%<ppl>     family   = {ppl,pplx,pplj},
%<ptm>     family   = {ptm,ptmx,ptmj},
%<!cmr>     shape    = {it,sl}  }
%<cmr>     shape    = it       }
   {
%<cmr>     \AE = {100,   },
%<cmr>     \OE = {100,   },
%<m-t|pad|ppl|ptm>     \textbackslash    = {100,200},
%<cmr>     \textbackslash    = {300,300},
%<bch>     \textbackslash    = {150,150},
%<pmn>     \textbackslash    = {100,150},
%<pmn>     031 = { ,-100}, % ffl
%<cmr|ptm>     156 = {100, },  % IJ
%<pad>     156 = {50,  },  % IJ
%<pmn>     156 = {20,  },  % IJ
%<pmn>     188 = {  ,-30}, % ij
%<pmn>   \v{t} = { ,100},
%<cmr>     \textquotedblleft = {500,300},
%<m-t|ptm>     \quotesinglbase   = {300,700},   \quotedblbase      = {400,500},
%<cmr>     \quotesinglbase   = {300,700},   \quotedblbase      = {200,600},
%<bch|pmn>     \quotesinglbase   = {200,500},   \quotedblbase      = {150,500},
%<pad|ppl>     \quotesinglbase   = {500,500},   \quotedblbase      = {400,400},
%<m-t|ppl|ptm>     \guilsinglleft    = {400,400},   \guilsinglright    = {300,500},
%<bch|pmn>     \guilsinglleft    = {300,400},   \guilsinglright    = {200,500},
%<cmr>     \guilsinglleft    = {500,300},   \guilsinglright    = {400,400},
%<pad>     \guilsinglleft    = {500,400},   \guilsinglright    = {300,500},
%<m-t|ppl>     \guillemotleft    = {300,300},   \guillemotright    = {300,300},
%<bch|pmn>     \guillemotleft    = {200,300},   \guillemotright    = {150,400},
%<cmr>     \guillemotleft    = {400,100},   \guillemotright    = {200,300},
%<pad>     \guillemotleft    = {300,300},   \guillemotright    = {200,400},
%<ptm>     \guillemotleft    = {300,400},   \guillemotright    = {200,400},
%<m-t|pad|ppl>     \textexclamdown   = {100,   },   \textquestiondown  = {200,   },
%<cmr|ptm>     \textexclamdown   = {200,   },   \textquestiondown  = {200,   },
%<pmn>     \textexclamdown   = {-50,   },   \textquestiondown  = {-50,   },
%<m-t|ppl>     \textbraceleft    = {200,100},   \textbraceright    = {200,200},
%<bch|pmn>     \textbraceleft    = {200,   },   \textbraceright    = {   ,200},
%<cmr|pad|ptm>     \textbraceleft    = {400,100},   \textbraceright    = {200,200},
%<bch|pmn>     \textless         = {100,   },   \textgreater       = {   ,100},
%<cmr|pad|ppl|ptm>     \textless         = {300,100},   \textgreater       = {200,100},
%<pmn>     \textvisiblespace = {100,100},
  }

%<*cmr>
\SetProtrusion
   [ name = cmr-it-T5,
     load = cmr-it ]
   { encoding = T5,
     family   = cmr,
     shape    = it }
   {
     \textbackslash    = {300,300},
     \quotesinglbase   = {300,700},   \quotedblbase      = {200,600},
     \guilsinglleft    = {500,300},   \guilsinglright    = {400,400},
     \guillemotleft    = {400,100},   \guillemotright    = {200,300},
     \textbraceleft    = {400,100},   \textbraceright    = {200,200},
     \textless         = {300,100},   \textgreater       = {200,100},
  }

%    \end{macrocode}
%\changes{v1.8}{2005/06/01}{verified settings for slanted Computer Modern Roman}
% Slanted is very similar to italic.
%    \begin{macrocode}
\SetProtrusion
   [ name     = cmr-sl,
     load     = cmr-it-OT1 ]
   { encoding = {OT1,OT4},
     family   = cmr,
     shape    = sl  }
   {
      L = {  ,50},
      f = { ,-50},
      - = {300, },
     \textendash = {400,  },  \textemdash = {300,  },
   }

\SetProtrusion
   [ name     = cmr-sl-T1,
     load     = cmr-it-T1 ]
   { encoding = {T1,LY1},
     family   = cmr,
     shape    = sl  }
   {
      L = {  ,50},
      f = { ,-50},
      - = {300, },
     \textendash = {400,  },  \textemdash = {300,  },
   }

\SetProtrusion
   [ name     = cmr-sl-T5,
     load     = cmr-it-T5 ]
   { encoding = T5,
     family   = cmr,
     shape    = sl  }
   {
      L = {  ,50},
      f = { ,-50},
      - = {300, },
     \textendash = {400,  },  \textemdash = {300,  },
   }

\SetProtrusion
   [ name     = lmr-it-T1,
     load     = cmr-it-T1 ]
   { encoding = {T1,LY1},
     family   = lmr,
     shape    = {it,sl} }
   {
     \textquotedblleft = {700,100},
     \quotedblbase     = {600,300},
   }

%    \end{macrocode}
% Oldstyle numerals are slightly different.
%    \begin{macrocode}
\SetProtrusion
   [ name = cmr(oldstyle)-it,
     load = cmr-it-T1 ]
   { encoding = T1,
     family   = {hfor,cmor},
     shape    = {it,sl}  }
   {
     1 = {250, 50},
     2 = {150,-100},
     3 = {100,-50},
     4 = {150,150},
     6 = {200,   },
     7 = {200, 50},
     8 = {150,-50},
     9 = {100, 50},
   }

%</cmr>
%<*pmn>
\SetProtrusion
   [ name     = pmnx-it,
     load     = pmnj-it ]
   { encoding = OT1,
     family   = pmnx,
     shape    = {it,sl} }
   {
     1 = {100,150},
   }

\SetProtrusion
   [ name     = pmnx-it-T1,
     load     = pmnj-it-T1 ]
   { encoding = {T1,LY1},
     family   = pmnx,
     shape    = {it,sl} }
   {
     1 = {100,150},
   }

%</pmn>
%<*ptm>
\SetProtrusion
   [ name     = ptm-it-LY1,
     load     = ptm-it-T1   ]
   { encoding = {LY1},
     family   = {ptm,ptmx,ptmj},
     shape    = {it,sl}  }
   {
     \texttrademark            = {100,100},
     \textregistered           = {100,100},
     \textcopyright            = {100,100},
     \textdegree               = {300,100},
     \textminus                = {200,200},
     \textellipsis             = {100,150},
     \texteuro                 = {   ,   },
     \textcent                 = {100,100},
     \textquotesingle          = {500,   },
     \textflorin               = {100, 70},
     \textdagger               = {150,150},
     \textdaggerdbl            = {100,100},
     \textbullet               = {150,150},
     \textonesuperior          = {150,100},
     \texttwosuperior          = {150, 50},
     \textthreesuperior        = {150, 50},
     \textparagraph            = {100,   },
     \textperiodcentered       = {500,300},
     \textonequarter           = { 50,   },
     \textonehalf              = { 50,   },
     \textplusminus            = {100,100},
     \textmultiply             = {150,150},
     \textdivide               = {150,150},
   }

%</ptm>
%    \end{macrocode}
%
%\subsubsection{Small caps}
%
% Small caps should inherit the values from their big brothers. Since values are
% relative to character width, we don't need to adjust them any further (but
% we have to reset some characters).
%    \begin{macrocode}
\SetProtrusion
%<m-t>   [ name     = OT1-sc,
%<bch>   [ name     = bch-sc,
%<cmr>   [ name     = cmr-sc-OT1,
%<pad>   [ name     = pad-sc,
%<pmn>   [ name     = pmnj-sc,
%<ppl>   [ name     = ppl-sc,
%<ptm>   [ name     = ptm-sc,
%<m-t>     load     = default ]
%<bch>     load     = bch-default ]
%<cmr>     load     = cmr-OT1 ]
%<pad>     load     = pad-default ]
%<pmn>     load     = pmnj-default ]
%<ppl>     load     = ppl-default ]
%<ptm>     load     = ptm-default ]
%<m-t|bch|pad|pmn>   { encoding = OT1,
%<cmr|ppl|ptm>   { encoding = {OT1,OT4},
%<bch>     family   = bch,
%<cmr>     family   = cmr,
%<pad>     family   = {pad,padx,padj},
%<pmn>     family   = pmnj,
%<ppl>     family   = {ppl,pplx,pplj},
%<ptm>     family   = {ptm,ptmx,ptmj},
     shape    = sc }
   {
     a = {50,50},
%<cmr|pad|ppl|ptm>   \ae = {50,  },
%<bch|pmn>     c = {50,  },
%<bch|pad|pmn>     d = {  ,50},
%<m-t|bch|cmr|pad|pmn|ptm>     f = {  ,50},
%<bch|pad|pmn>     g = {50,  },
%<m-t|cmr|pad|pmn|ppl|ptm>     j = {50,  },
%<bch>     j = {100,  },
%<m-t|bch|cmr|pad|pmn|ppl>     l = {  ,50},
%<ptm>     l = {  ,80},
%<m-t|bch|cmr|pad|pmn|ppl>   013 = {  ,50}, % fl
%<ptm>   013 = {  ,80}, % fl
%<bch|pad|pmn>     o = {50,50},
%<bch|pad|pmn>   \oe = {50,  },
%<ppl>     p = { 0, 0},
%<bch|pad|pmn>     q = {50,70},
%<ppl>     q = { 0,  },
%<m-t|cmr|pad|pmn|ppl|ptm>     r = {  , 0},
     t = {50,50},
%<m-t|bch|cmr|pad|pmn|ppl>     y = {50,50},
%<ptm>     y = {80,80},
   }

\SetProtrusion
%<m-t>   [ name     = T1-sc,
%<bch>   [ name     = bch-sc-T1,
%<cmr>   [ name     = cmr-sc-T1,
%<pad>   [ name     = pad-sc-T1,
%<pmn>   [ name     = pmnj-sc-T1,
%<ppl>   [ name     = ppl-sc-T1,
%<ptm>   [ name     = ptm-sc-T1,
%<m-t>     load     = T1-default ]
%<bch>     load     = bch-T1     ]
%<cmr>     load     = cmr-T1     ]
%<pad>     load     = pad-T1     ]
%<pmn>     load     = pmnj-T1    ]
%<ppl>     load     = ppl-T1     ]
%<ptm>     load     = ptm-T1     ]
   { encoding = {T1,LY1},
%<bch>     family   = bch,
%<cmr>     family   = cmr,
%<pad>     family   = {pad,padx,padj},
%<pmn>     family   = pmnj,
%<ppl>     family   = {ppl,pplx,pplj},
%<ptm>     family   = {ptm,ptmx,ptmj},
     shape    = sc }
   {
     a = {50,50},
%<cmr|pad|ppl|ptm>   \ae = {50,  },
%<bch|pmn>     c = {50,  },
%<bch|pad|pmn>     d = {  ,50},
%<m-t|bch|cmr|pad|pmn|ptm>     f = {  ,50},
%<bch|pad|pmn>     g = {50,  },
%<m-t|cmr|pad|pmn|ppl|ptm>     j = {50,  },
%<bch>     j = {100,  },
%<m-t|bch|cmr|pad|pmn|ppl>     l = {  ,50},
%<ptm>     l = {  ,80},
%<m-t|bch|cmr|pad|pmn|ppl>   029 = {  ,50}, % fl
%<ptm>   029 = {  ,80}, % fl
%<bch|pad|pmn>     o = {50,50},
%<bch|pad|pmn>   \oe = {50,  },
%<ppl>     p = { 0, 0},
%<bch|pad|pmn>     q = {50,70},
%<ppl>     q = { 0,  },
%<m-t|cmr|pad|pmn|ppl|ptm>     r = {  , 0},
     t = {50,50},
%<m-t|bch|cmr|pad|pmn|ppl>     y = {50,50},
%<ptm>     y = {80,80},
   }

%<*cmr>
\SetProtrusion
   [ name     = cmr-sc-T5,
     load     = cmr-T5 ]
   { encoding = T5,
     family   = cmr,
     shape    = sc }
   {
     a = {50,50},
     f = {  ,50},
     j = {50,  },
     l = {  ,50},
     r = {  , 0},
     t = {50,50},
     y = {50,50},
   }

%</cmr>
%<*pmn>
\SetProtrusion
   [ name     = pmnx-sc,
     load     = pmnj-sc ]
   { encoding = OT1,
     family   = pmnx,
     shape    = sc }
   {
     1 = {230,180},
   }

\SetProtrusion
   [ name     = pmnx-sc-T1,
     load     = pmnj-sc-T1 ]
   { encoding = {T1,LY1},
     family   = pmnx,
     shape    = sc }
   {
     1 = {230,180},
   }

%    \end{macrocode}
%
%\subsubsection{Italic Small Caps}
% Minion provides real small caps in italics.
% The \pkg{slantsc} package calls them |scit|, Philipp Lehman's
% \pkg{fontinstallationguide} suggests |si|.
%    \begin{macrocode}
\SetProtrusion
   [ name     = pmnj-scit,
     load     = pmnj-it   ]
   { encoding = OT1,
     family   = pmnj,
     shape    = {scit,si} }
   {
     a = {50,  },
   \ae = {  ,-50},
     b = {20,-50},
     c = {50,-50},
     d = {20, 0},
     e = {20,-50},
     f = {10, 0},
   012 = {10,-50}, % fi
   013 = {10,-50}, % fl
   014 = {10,-50}, % ffi
   015 = {10,-50}, % ffl
     g = {50,-50},
     i = {20,-50},
     j = {20, 0},
     k = {20,  },
     l = {20,50},
     m = {  ,-30},
     n = {  ,-30},
     o = {50,  },
   \oe = {50,-50},
     p = {20,-50},
     q = {50,  },
     r = {20, 0},
     s = {20,-30},
     t = {70,  },
     u = {50,-50},
     v = {100,  },
     w = {100,  },
     y = {50,  },
     z = {  ,-50},
   }

\SetProtrusion
   [ name     = pmnj-scit-T1,
     load     = pmnj-it-T1   ]
   { encoding = {T1,LY1},
     family   = pmnj,
     shape    = {scit,si}    }
   {
     a = {50,  },
   \ae = {  ,-50},
     b = {20,-50},
     c = {50,-50},
     d = {20, 0},
     e = {20,-50},
     f = {10, 0},
   028 = {10,-50}, % fi
   029 = {10,-50}, % fl
   030 = {10,-50}, % ffi
   031 = {10,-50}, % ffl
     g = {50,-50},
     i = {20,-50},
   188 = {20, 0},  % ij
     j = {20, 0},
     k = {20,  },
     l = {20,50},
     m = {  ,-30},
     n = {  ,-30},
     o = {50,  },
   \oe = {50,-50},
     p = {20,-50},
     q = {50,  },
     r = {20, 0},
     s = {20,-30},
     t = {70,  },
     u = {50,-50},
     v = {100,  },
     w = {100,  },
     y = {50,  },
     z = {  ,-50},
   }

\SetProtrusion
   [ name     = pmnx-scit,
     load     = pmnj-scit ]
   { encoding = OT1,
     family   = pmnx,
     shape    = {scit,si} }
   {
     1 = {100,150},
   }

\SetProtrusion
   [ name     = pmnx-scit-T1,
     load     = pmnj-scit-T1 ]
   { encoding = {T1,LY1},
     family   = pmnx,
     shape    = {scit,si}    }
   {
     1 = {100,150},
   }

%</pmn>
%    \end{macrocode}
%
%\subsubsection{\texttt{textcomp}}
%
% Finally the \tsI\ encoding.
% Still quite incomplete for Times and especially Palatino. Anybody?
%\changes{v1.2}{2004/09/28}{add settings for Adobe Garamond and
%                           Computer Modern Roman in \tsI\ encoding}
%    \begin{macrocode}
\SetProtrusion
%<m-t>   [ name     = textcomp ]
%<bch>   [ name     = bch-textcomp ]
%<cmr>   [ name     = cmr-textcomp ]
%<pad>   [ name     = pad-textcomp ]
%<pmn>   [ name     = pmn-textcomp ]
%<ppl>   [ name     = ppl-textcomp ]
%<ptm>   [ name     = ptm-textcomp ]
%<m-t>   { encoding = TS1      }
%<!m-t>   { encoding = TS1,
%<bch>     family   = bch }
%<cmr>     family   = cmr }
%<pad>     family   = {pad,padx,padj} }
%<pmn>     family   = {pmnx,pmnj} }
%<ppl>     family   = {ppl,pplx,pplj} }
%<ptm>     family   = {ptm,ptmx,ptmj} }
   {
%<cmr>     \textquotestraightbase    = {300,300},
%<pad|pmn>     \textquotestraightbase    = {400,400},
%<cmr|pmn>     \textquotestraightdblbase = {300,300},
%<pad>     \textquotestraightdblbase = {400,400},
%<bch|cmr|pad|pmn>     \texttwelveudash          = {200,200},
%<bch|cmr|pad|pmn>     \textthreequartersemdash  = {150,150},
%<cmr|pmn>     \textquotesingle          = {300,400},
%<pad>     \textquotesingle          = {400,500},
%<ptm>     \textquotesingle          = {500,500},
%<bch|cmr|pmn>     \textasteriskcentered     = {200,300},
%<pad>     \textasteriskcentered     = {300,300},
%<pmn>     \textfractionsolidus      = {-200,-200},
%<cmr>     \textoneoldstyle          = {100,100},
%<pmn>     \textoneoldstyle          = {   , 50},
%<cmr>     \textthreeoldstyle        = {   , 50},
%<pad|pmn>     \textthreeoldstyle        = { 50,   },
%<cmr>     \textfouroldstyle         = { 50, 50},
%<pad|pmn>     \textfouroldstyle         = { 50,   },
%<cmr|pad|pmn>     \textsevenoldstyle        = { 50, 80},
%<cmr>     \textlangle               = {400,   },
%<cmr>     \textrangle               = {   ,400},
%<m-t|bch|pmn|ptm>     \textminus                = {200,200},
%<cmr|pad|ppl>     \textminus                = {300,300},
%<bch|pad|pmn>     \textlbrackdbl            = {100,   },
%<bch|pad|pmn>     \textrbrackdbl            = {   ,100},
%<pmn>     \textasciigrave           = {200,500},
%<bch|cmr|pad|pmn>     \texttildelow             = {200,250},
%<pmn>     \textasciibreve           = {300,400},
%<pmn>     \textasciicaron           = {300,400},
%<pmn>     \textacutedbl             = {200,300},
%<pmn>     \textgravedbl             = {150,300},
%<bch|pmn>     \textdagger               = { 80, 80},
%<cmr|pad>     \textdagger               = {100,100},
%<ptm>     \textdagger               = {150,150},
%<cmr|pad|pmn>     \textdaggerdbl            = { 80, 80},
%<ptm>     \textdaggerdbl            = {100,100},
%<bch>     \textbardbl               = {100,100},
%<bch>     \textbullet               = {200,200},
%<cmr|pad|pmn>     \textbullet               = {   ,100},
%<ptm>     \textbullet               = {150,150},
%<bch|cmr|pmn>     \textcelsius              = { 50,   },
%<pad>     \textcelsius              = { 80,   },
%<bch>     \textflorin               = { 50, 50},
%<pad>     \textflorin               = {   ,100},
%<pmn>     \textflorin               = { 50,100},
%<ptm>     \textflorin               = { 50, 70},
%<cmr>     \textcolonmonetary        = {   , 50},
%<pad|pmn>     \textcolonmonetary        = { 50,   },
%<pmn>     \textinterrobang          = {   ,100},
%<pmn>     \textinterrobangdown      = {100,   },
%<m-t|pad|ptm>     \texttrademark            = {100,100},
%<bch>     \texttrademark            = {150,150},
%<cmr|ppl>     \texttrademark            = {200,200},
%<pmn>     \texttrademark            = { 50, 50},
%<bch>     \textcent                 = { 50,   },
%<ptm>     \textcent                 = {100,100},
%<bch>     \textsterling             = { 50,   },
%<bch>     \textbrokenbar            = {200,200},
%<pmn>     \textasciidieresis        = {300,400},
%<m-t|bch|cmr|pad|ptm>     \textcopyright            = {100,100},
%<pmn>     \textcopyright            = {100,150},
%<ppl>     \textcopyright            = {200,200},
%<bch|cmr>     \textordfeminine          = {100,200},
%<pad|pmn>     \textordfeminine          = {200,200},
%<bch|cmr|pad|pmn>     \textlnot                 = {200,   },
%<m-t|bch|cmr|pad|ptm>     \textregistered           = {100,100},
%<pmn>     \textregistered           = { 50,150},
%<ppl>     \textregistered           = {200,200},
%<pmn>     \textasciimacron          = {150,200},
%<m-t|ppl|ptm>     \textdegree               = {300,300},
%<bch>     \textdegree               = {150,200},
%<cmr|pad>     \textdegree               = {400,400},
%<pmn>     \textdegree               = {150,400},
%<bch|cmr|pad|pmn>     \textpm                   = {150,200},
%<ptm>     \textpm                   = { 50, 80},
%<bch>     \texttwosuperior          = {100,200},
%<cmr>     \texttwosuperior          = { 50,100},
%<pad|pmn>     \texttwosuperior          = {200,200},
%<ptm>     \texttwosuperior          = { 50, 50},
%<bch>     \textthreesuperior        = {100,200},
%<cmr>     \textthreesuperior        = { 50,100},
%<pad|pmn>     \textthreesuperior        = {200,200},
%<ptm>     \textthreesuperior        = { 50, 50},
%<pmn>     \textasciiacute           = {300,400},
%<bch>     \textmu                   = {   ,100},
%<bch|pad|pmn>     \textparagraph            = {  ,100},
%<bch|cmr|pad|pmn>     \textperiodcentered       = {300,400},
%<ptm>     \textperiodcentered       = {300,300},
%<bch>     \textonesuperior          = {200,300},
%<cmr|pad|pmn>     \textonesuperior          = {200,200},
%<ptm>     \textonesuperior          = {100,100},
%<bch|pad|pmn>     \textordmasculine         = {200,200},
%<cmr>     \textordmasculine         = {100,200},
%<bch|cmr|pmn>     \texteuro                 = {100,   },
%<pad>     \texteuro                 = { 50,100},
%<bch|ptm>     \texttimes                = {100,100},
%<cmr>     \texttimes                = {150,250},
%<pad>     \texttimes                = {100,150},
%<pmn>     \texttimes                = { 70,100},
%<bch|pad|pmn>     \textdiv                  = {150,200},
%<cmr>     \textdiv                  = {150,250},
%<ptm>     \textdiv                  = { 50,100},
%<ptm>     \textperthousand          = {    ,50},
%    \end{macrocode}
% All remaining characters can be found in the source.
%\iffalse
%^^A ... namely, here:
%    \textsection                 = {  ,  },
%    \textyen                     = {  ,  },
%    \textonehalf                 = {  ,  },
%    \textonequarter              = {  ,  },
%    \textthreequarters           = {  ,  },
%    \texttwooldstyle             = {  ,  },
%    \textfiveoldstyle            = {  ,  },
%    \textsixoldstyle             = {  ,  },
%    \texteightoldstyle           = {  ,  },
%    \textnineoldstyle            = {  ,  },
%    \textleftarrow               = {  ,  },
%    \textrightarrow              = {  ,  },
%    \textblank                   = {  ,  },
%    \textdollar                  = {  ,  },
%    \textdblhyphen               = {  ,  },
%    \textdblhyphenchar           = {  ,  },
%    \textohm                     = {  ,  },
%    \textmho                     = {  ,  },
%    \textbigcircle               = {  ,  },
%    \textuparrow                 = {  ,  },
%    \textdownarrow               = {  ,  },
%    \textborn                    = {  ,  },
%    \textdied                    = {  ,  },
%    \textmarried                 = {  ,  },
%    \textdivorced                = {  ,  },
%    \textleaf                    = {  ,  },
%    \textmusicalnote             = {  ,  },
%    \textdollaroldstyle          = {  ,  },
%    \textcentoldstyle            = {  ,  },
%    \textwon                     = {  ,  },
%    \textnaira                   = {  ,  },
%    \textguarani                 = {  ,  },
%    \textpeso                    = {  ,  },
%    \textlira                    = {  ,  },
%    \textrecipe                  = {  ,  },
%    \textdong                    = {  ,  },
%    \textpertenthousand          = {  ,  },
%    \textpilcrow                 = {  ,  },
%    \textbaht                    = {  ,  },
%    \textnumero                  = {  ,  },
%    \textdiscount                = {  ,  },
%    \textestimated               = {  ,  },
%    \textopenbullet              = {  ,  },
%    \textservicemark             = {  ,  },
%    \textlquill                  = {  ,  },
%    \textrquill                  = {  ,  },
%    \textcopyleft                = {  ,  },
%    \textcircledP                = {  ,  },
%    \textreferencemark           = {  ,  },
%    \textsurd                    = {  ,  },
%\fi
%    \begin{macrocode}
   }

%<*cmr|pad|pmn>
\SetProtrusion
%<cmr>   [ name     = cmr-textcomp-it ]
%<pad>   [ name     = pad-textcomp-it ]
%<pmn>   [ name     = pmn-textcomp-it ]
   { encoding = TS1,
%<cmr>     family   = cmr,
%<pad>     family   = {pad,padx,padj},
%<pmn>     family   = {pmnx,pmnj},
     shape    = {it,sl} }
   {
%<cmr>     \textquotestraightbase    = {300,600},
%<pad|pmn>     \textquotestraightbase    = {400,400},
%<cmr>     \textquotestraightdblbase = {300,600},
%<pad>     \textquotestraightdblbase = {300,400},
%<pmn>     \textquotestraightdblbase = {300,300},
     \texttwelveudash          = {200,200},
     \textthreequartersemdash  = {150,150},
%<cmr>     \textquotesingle          = {600,300},
%<pad>     \textquotesingle          = {800,100},
%<pmn>     \textquotesingle          = {300,200},
%<cmr>     \textasteriskcentered     = {300,200},
%<pad>     \textasteriskcentered     = {500,100},
%<pmn>     \textasteriskcentered     = {200,300},
%<pmn>     \textfractionsolidus      = {-200,-200},
%<cmr>     \textoneoldstyle          = {100, 50},
%<pad>     \textoneoldstyle          = {100,   },
%<pmn>     \textoneoldstyle          = { 50,   },
%<pad>     \texttwooldstyle          = { 50,   },
%<pmn>     \texttwooldstyle          = {-50,   },
%<cmr>     \textthreeoldstyle        = {100, 50},
%<pmn>     \textthreeoldstyle        = {-100,  },
%<cmr>     \textfouroldstyle         = { 50, 50},
%<pad>     \textfouroldstyle         = { 50,100},
%<cmr>     \textsevenoldstyle        = { 50, 80},
%<pad>     \textsevenoldstyle        = { 50,   },
%<pmn>     \textsevenoldstyle        = { 20,   },
%<cmr>     \textlangle               = {400,   },
%<cmr>     \textrangle               = {   ,400},
%<cmr|pad>     \textminus                = {300,300},
%<pmn>     \textminus                = {200,200},
%<pad|pmn>     \textlbrackdbl            = {100,   },
%<pad|pmn>     \textrbrackdbl            = {   ,100},
%<pmn>     \textasciigrave           = {300,300},
     \texttildelow             = {200,250},
%<pmn>     \textasciibreve           = {300,300},
%<pmn>     \textasciicaron           = {300,300},
%<pmn>     \textacutedbl             = {200,300},
%<pmn>     \textgravedbl             = {150,300},
%<cmr>     \textdagger               = {100,100},
%<pad>     \textdagger               = {200,100},
%<pmn>     \textdagger               = { 80, 50},
%<cmr|pad>     \textdaggerdbl            = { 80, 80},
%<pmn>     \textdaggerdbl            = { 80, 50},
%<cmr>     \textbullet               = {200,100},
%<pad>     \textbullet               = {300,   },
%<pmn>     \textbullet               = { 30, 70},
%<cmr>     \textcelsius              = {100,   },
%<pad>     \textcelsius              = {200,   },
%<pmn>     \textcelsius              = { 50,-50},
%<pad>     \textflorin               = {100,   },
%<pmn>     \textflorin               = { 50,100},
%<cmr>     \textcolonmonetary        = {150,   },
%<pad>     \textcolonmonetary        = {100,   },
%<pmn>     \textcolonmonetary        = { 50,-50},
%<cmr|pad>     \texttrademark            = {200,   },
%<pmn>     \texttrademark            = { 50,100},
%<pmn>     \textasciidieresis        = {300,200},
%<cmr>     \textcopyright            = {100,   },
%<pad>     \textcopyright            = {200,100},
%<pmn>     \textcopyright            = {100,150},
%<cmr>     \textordfeminine          = {100,100},
%<pmn>     \textordfeminine          = {200,200},
%<cmr|pad>     \textlnot                 = {300,   },
%<pmn>     \textlnot                 = {200,   },
%<cmr>     \textregistered           = {100,   },
%<pad>     \textregistered           = {200,100},
%<pmn>     \textregistered           = { 50,150},
%<pmn>     \textasciimacron          = {150,200},
%<cmr|pad>     \textdegree               = {500,100},
%<pmn>     \textdegree               = {150,150},
%<cmr>     \textpm                   = {150,100},
%<pad>     \textpm                   = {200,150},
%<pmn>     \textpm                   = {150,200},
%<cmr>     \textonesuperior          = {400,   },
%<pad>     \textonesuperior          = {300,100},
%<pmn>     \textonesuperior          = {200,100},
%<cmr>     \texttwosuperior          = {400,   },
%<pad>     \texttwosuperior          = {300,   },
%<pmn>     \texttwosuperior          = {200,100},
%<cmr>     \textthreesuperior        = {400,   },
%<pad>     \textthreesuperior        = {300,   },
%<pmn>     \textthreesuperior        = {200,100},
%<pmn>     \textasciiacute           = {300,200},
%<cmr>     \textparagraph            = {200,   },
%<pmn>     \textparagraph            = {   ,100},
%<cmr>     \textperiodcentered       = {500,500},
%<pad|pmn>     \textperiodcentered       = {300,400},
%<cmr>     \textordmasculine         = {100,100},
%<pmn>     \textordmasculine         = {200,200},
%<cmr>     \texteuro                 = {200,   },
%<pad>     \texteuro                 = {100,   },
%<pmn>     \texteuro                 = {100,-50},
%<cmr>     \texttimes                = {200,200},
%<pad>     \texttimes                = {200,100},
%<pmn>     \texttimes                = { 70,100},
%<cmr|pad>     \textdiv                  = {200,200},
%<pmn>     \textdiv                  = {150,200},
   }

%</cmr|pad|pmn>
%    \end{macrocode}
%
%\subsubsection{Math}
%
% Now to the math symbols for Computer Modern Roman. Definitions have been
% extracted from \file{fontmath.ltx}.
%\changes{v1.2}{2004/10/02}{add settings for Computer Modern Roman math symbols}
% I did not spend too much time fiddling with these settings, so they can surely
% be improved.
%
% The math font `|operators|' (also used for the \cmd{\mathrm} and
% \cmd{\mathbf} alphabets) is |OT1/cmr|, which we've already set up above.
% It's declared as:
%\begin{verbatim}
%\DeclareSymbolFont{operators}   {OT1}{cmr} {m}{n}
%\SetSymbolFont{operators}{bold}{OT1}{cmr} {bx}{n}
%\end{verbatim}
%
% \cmd{\mathit} (|OT1/cmr/m/it|) is also already set up.
%
% There are (for the moment) no settings for \cmd{\mathsf} and \cmd{\mathtt}.
%
% Math font `|letters|' (also used as \cmd{\mathnormal}) is declared as:
%\begin{verbatim}
%\DeclareSymbolFont{letters}     {OML}{cmm} {m}{it}
%\SetSymbolFont{letters}  {bold}{OML}{cmm} {b}{it}
%\end{verbatim}
%\changes{v1.6}{2004/12/26}{tune CMR math letters (OML encoding)}
%    \begin{macrocode}
%<*cmr>
\SetProtrusion
   [ name     = cmr-math-letters ]
   { encoding = OML,
     family   = cmm,
     series   = {m,b},
     shape    = it  }
   {
       A = {100, 50}, % \mathnormal
       B = { 50,   },
       C = { 50,   },
       D = { 50, 50},
       E = { 50,   },
       F = {100, 50},
       G = { 50, 50},
       H = { 50, 50},
       I = { 50, 50},
       J = {150, 50},
       K = { 50,100},
       L = { 50, 50},
       M = { 50,   },
       N = { 50,   },
       O = { 50,   },
       P = { 50,   },
       Q = { 50, 50},
       R = { 50,   },
       S = { 50,   },
       T = { 50,100},
       U = { 50, 50},
       V = {100,100},
       W = { 50,100},
       X = { 50,100},
       Y = {100,100},
       f = {100,100},
       h = {   ,100},
       i = {   , 50},
       j = {   , 50},
       k = {   , 50},
       r = {   , 50},
       v = {   , 50},
       w = {   , 50},
       x = {   , 50},
     "0B = { 50,100}, % \alpha
     "0C = { 50, 50}, % \beta
     "0D = {200,150}, % \gamma
     "0E = { 50, 50}, % \delta
     "0F = { 50, 50}, % \epsilon
     "10 = { 50,150}, % \zeta
%    "11 = {   ,   }, % \eta
     "12 = { 50,   }, % \theta
     "13 = {   ,100}, % \iota
     "14 = {   ,100}, % \kappa
     "15 = {100, 50}, % \lambda
     "16 = {   , 50}, % \mu
     "17 = {   , 50}, % \nu
     "18 = {   , 50}, % \xi
     "19 = { 50,100}, % \pi
     "1A = { 50, 50}, % \rho
     "1B = {   ,150}, % \sigma
     "1C = { 50,150}, % \tau
     "1D = { 50, 50}, % \upsilon
%    "1E = {   ,   }, % \phi
     "1F = { 50,100}, % \chi
     "20 = { 50, 50}, % \psi
     "21 = {   , 50}, % \omega
     "22 = {   , 50}, % \varepsilon
     "23 = {   , 50}, % \vartheta
     "24 = {   , 50}, % \varpi
     "25 = {100,   }, % \varrho
     "26 = {100,100}, % \varsigma
     "27 = { 50, 50}, % \varphi
     "28 = {100,100}, % \leftharpoonup
     "29 = {100,100}, % \leftharpoondown
     "2A = {100,100}, % \rightharpoonup
     "2B = {100,100}, % \rightharpoondown
     "2C = {300,200}, % \lhook
     "2D = {200,300}, % \rhook
     "2E = {   ,100}, % \triangleright
     "2F = {100,   }, % \triangleleft
     % 0 - 9
     "3A = {   ,500}, % ., \ldotp
     "3B = {   ,500}, % ,
     "3C = {200,100}, % <
     "3D = {300,400}, % /
     "3E = {100,200}, % >
     "3F = {200,200}, % \star
%    "40 = {   ,   }, % \partial
     "5B = {   ,100}, % \flat
%    "5C = {   ,   }, % \natural
%    "5D = {   ,   }, % \sharp
     "5E = {200,200}, % \smile
     "5F = {200,200}, % \frown
%    "60 = {   ,   }, % \ell
%    "7B = {   ,   }, % \imath
     "7C = {100,   }, % \jmath
     "7D = {   ,100}, % \wp
   }

%    \end{macrocode}
% Math font `|symbols|' (also used for the \cmd{\mathcal} alphabet) is declared
% as:
%\begin{verbatim}
%\DeclareSymbolFont{symbols}     {OMS}{cmsy}{m}{n}
%\SetSymbolFont{symbols}  {bold}{OMS}{cmsy}{b}{n}
%\end{verbatim}
%    \begin{macrocode}
\SetProtrusion
   [ name     = cmr-math-symbols ]
   { encoding = OMS,
     family   = cmsy,
     series   = {m,b},
     shape    = n  }
   {
       A = {150, 50}, % \mathcal
       C = {   ,100},
       D = {   , 50},
       F = { 50,150},
       I = {   ,100},
       J = {100,150},
       K = {   ,100},
       L = {100,   },
       M = { 50, 50},
       N = { 50,100},
       P = {   , 50},
       Q = { 50,   },
       R = {   , 50},
       T = { 50,150},
       V = { 50, 50},
       W = {   , 50},
       X = {100,100},
       Y = {100,   },
       Z = {100,150},
     "00 = {300,300}, % -
     "01 = {   ,700}, % \cdot, \cdotp
     "02 = {150,250}, % \times
     "03 = {150,250}, % *, \ast
     "04 = {200,300}, % \div
     "05 = {150,250}, % \diamond
     "06 = {200,200}, % \pm
     "07 = {200,200}, % \mp
     "08 = {100,100}, % \oplus
     "09 = {100,100}, % \ominus
     "0A = {100,100}, % \otimes
     "0B = {100,100}, % \oslash
     "0C = {100,100}, % \odot
     "0D = {100,100}, % \bigcirc
     "0E = {100,100}, % \circ
     "0F = {100,100}, % \bullet
     "10 = {100,100}, % \asymp
     "11 = {100,100}, % \equiv
     "12 = {200,100}, % \subseteq
     "13 = {100,200}, % \supseteq
     "14 = {200,100}, % \leq
     "15 = {100,200}, % \geq
     "16 = {200,100}, % \preceq
     "17 = {100,200}, % \succeq
     "18 = {200,200}, % \sim
     "19 = {150,150}, % \approx
     "1A = {200,100}, % \subset
     "1B = {100,200}, % \supset
     "1C = {200,100}, % \ll
     "1D = {100,200}, % \gg
     "1E = {300,100}, % \prec
     "1F = {100,300}, % \succ
     "20 = {100,200}, % \leftarrow
     "21 = {200,100}, % \rightarrow
     "22 = {100,100}, % \uparrow
     "23 = {100,100}, % \downarrow
     "24 = {100,100}, % \leftrightarrow
     "25 = {100,100}, % \nearrow
     "26 = {100,100}, % \searrow
     "27 = {100,100}, % \simeq
     "28 = {100,100}, % \Leftarrow
     "29 = {100,100}, % \Rightarrow
     "2A = {100,100}, % \Uparrow
     "2B = {100,100}, % \Downarrow
     "2C = {100,100}, % \Leftrightarrow
     "2D = {100,100}, % \nwarrow
     "2E = {100,100}, % \swarrow
     "2F = {   ,100}, % \propto
     "30 = {   ,400}, % \prime
     "31 = {100,100}, % \infty
     "32 = {150,100}, % \in
     "33 = {100,150}, % \ni
     "34 = {100,100}, % \triangle, \bigtriangleup
     "35 = {100,100}, % \bigtriangledown
%    "36 = {   ,   }, % \not
%    "37 = {   ,   }, % \mapstochar
     "38 = {   ,100}, % \forall
     "39 = {100,   }, % \exists
     "3A = {200,   }, % \neg
%    "3B = {   ,   }, % \emptyset
%    "3C = {   ,   }, % \Re
%    "3D = {   ,   }, % \Im
     "3E = {200,200}, % \top
     "3F = {200,200}, % \bot, \perp
%    "40 = {   ,   }, % \aleph
%    "5B = {   ,   }, % \cup
%    "5C = {   ,   }, % \cap
%    "5D = {   ,   }, % \uplus
     "5E = {100,200}, % \wedge
     "5F = {100,200}, % \vee
     "60 = {   ,300}, % \vdash
     "61 = {300,   }, % \dashv
     "62 = {100,100}, % \lfloor
     "63 = {100,100}, % \rfloor
     "64 = {100,100}, % \lceil
     "65 = {100,100}, % \rceil
     "66 = {150,   }, % \lbrace
     "67 = {   ,150}, % \rbrace
     "68 = {400,   }, % \langle
     "69 = {   ,400}, % \rangle
%    "6A = {   ,   }, % \arrowvert, \mid, \vert, |
%    "6B = {   ,   }, % \Arrowvert, \parallel, \Vert
     "6C = {100,100}, % \updownarrow
     "6D = {100,100}, % \Updownarrow
     "6E = {100,300}, % \, \backslash, \setminus
%    "6F = {   ,   }, % \wr
%    "70 = {   ,   }, % \sqrtsign
%    "71 = {   ,   }, % \amalg
     "72 = {100,100}, % \nabla
%    "73 = {   ,   }, % \smallint
%    "74 = {   ,   }, % \sqcup
%    "75 = {   ,   }, % \sqcap
%    "76 = {   ,   }, % \sqsubseteq
%    "77 = {   ,   }, % \sqsupseteq
%    "78 = {   ,   }, % \mathsection
     "79 = {200,200}, % \dagger
     "7A = {100,100}, % \ddagger
     "7B = {100,   }, % \mathparagraph
     "7C = {100,100}, % \clubsuit
     "7D = {100,100}, % \diamondsuit
     "7E = {100,100}, % \heartsuit
     "7F = {100,100}, % \spadesuit
   }

%    \end{macrocode}
% We don't bother about `|largesymbols|', since it will only be used in display
% math, where protrusion doesn't work anyway. It's declared as:
%\begin{verbatim}
%\DeclareSymbolFont{largesymbols}{OMX}{cmex}{m}{n}
%\end{verbatim}
%    \begin{macrocode}
%\SetProtrusion
%   [ name     = cmr-math-largesymbols ]
%   { encoding = OMX,
%     family   = {cmex,cmr} }
%   {
%     "00 % (
%     "01 % )
%     "02 % [
%     "03 % ]
%     "04 % \lfloor
%     "05 % \rfloor
%     "06 % \lceil
%     "07 % \rceil
%     "08 % \lbrace
%     "09 % \rbrace
%     "0A % <, \langle
%     "0B % >, \rangle
%     "0C % \vert, |
%     "0D % \Vert
%     "0E % /
%     "0F % \backslash
%     "3A % \lgroup
%     "3B % \rgroup
%     "3C % \arrowvert
%     "3D % \Arrowvert
%     "3E % \bracevert
%     "3F % \updownarrow
%     "40 % \lmoustache
%     "41 % \rmoustache
%     "46 % \bigsqcup
%     "48 % \ointop
%     "4A % \bigodot
%     "4C % \bigoplus
%     "4E % \bigotimes
%     "50 % \sum
%     "51 % \prod
%     "52 % \intop
%     "53 % \bigcup
%     "54 % \bigcap
%     "55 % \biguplus
%     "56 % \bigwedge
%     "57 % \bigvee
%     "60 % \coprod
%     "70 % \sqrtsign
%     "77 % \Updownarrow
%     "78 % \uparrow
%     "79 % \downarrow
%     "7A % \braceld, \lmoustache
%     "7B % \bracerd, \rmoustache
%     "7C % \bracelu
%     "7D % \braceru
%     "7E % \Uparrow
%     "7F % \Downarrow
%  }
%</cmr>
%</!cfg-u>
%    \end{macrocode}
%
% \subsubsection{\ams\ fonts}
%
%    \begin{macrocode}
%<*cfg-u>
%    \end{macrocode}
% Symbol font `a', defined in \pkg{amssymb}.
%
%\begin{verbatim}
%\DeclareSymbolFont{AMSa}{U}{msa}{m}{n}
%\end{verbatim}
%    \begin{macrocode}
%<*msa>
\SetProtrusion
   [ name     = AMSa ]
   { encoding = U,
     family   = msa  }
   {
%    "00  =  {   ,   },  % \boxdot
%    "01  =  {   ,   },  % \boxplus
%    "02  =  {   ,   },  % \boxtimes
%    "03  =  {   ,   },  % \square
%    "04  =  {   ,   },  % \blacksquare
     "05  =  {150,250},  % \centerdot
     "06  =  {100,100},  % \lozenge
     "07  =  { 50, 50},  % \blacklozenge
     "08  =  { 50, 50},  % \circlearrowright
     "09  =  { 50, 50},  % \circlearrowleft
     "0A  =  { 50, 50},  % \rightleftharpoons
     "0B  =  { 50, 50},  % \leftrightharpoons
%    "0C  =  {   ,   },  % \boxminus
     "0D  =  {-50,150},  % \Vdash
     "0E  =  {-50,150},  % \Vvdash
     "0F  =  {-70,150},  % \vDash
     "10  =  {100,100},  % \twoheadrightarrow
     "11  =  { 50,150},  % \twoheadleftarrow
     "12  =  { 50,100},  % \leftleftarrows
     "13  =  { 50, 80},  % \rightrightarrows
     "14  =  {120,120},  % \upuparrows
     "15  =  {120,120},  % \downdownarrows
     "16  =  {200,200},  % \upharpoonright
     "17  =  {200,200},  % \downharpoonright
     "18  =  {200,200},  % \upharpoonleft
     "19  =  {200,200},  % \downharpoonleft
     "1A  =  { 50, 80},  % \rightarrowtail
     "1B  =  { 50, 80},  % \leftarrowtail
     "1C  =  { 50, 50},  % \leftrightarrows
     "1D  =  { 50, 50},  % \rightleftarrows
     "1E  =  {150,   },  % \Lsh
     "1F  =  {   ,150},  % \Rsh
     "20  =  { 50,100},  % \rightsquigarrow
     "21  =  { 50, 50},  % \leftrightsquigarrow
     "22  =  { 50, 50},  % \looparrowleft
     "23  =  { 50, 50},  % \looparrowright
     "24  =  {   , 50},  % \circeq
     "25  =  {   ,100},  % \succsim
     "26  =  {   ,100},  % \gtrsim
     "27  =  {   ,100},  % \gtrapprox
     "28  =  { 50, 50},  % \multimap
%    "29  =  {   ,   },  % \therefore
%    "2A  =  {   ,   },  % \because
     "2B  =  { 50, 50},  % \doteqdot
     "2C  =  { 50, 50},  % \triangleq
     "2D  =  {100, 50},  % \precsim
     "2E  =  {100, 50},  % \lesssim
     "2F  =  { 50, 50},  % \lessapprox
     "30  =  {100, 50},  % \eqslantless
     "31  =  { 50, 50},  % \eqslantgtr
     "32  =  {100, 50},  % \curlyeqprec
     "33  =  { 50,100},  % \curlyeqsucc
     "34  =  {100, 50},  % \preccurlyeq
%    "35  =  {   ,   },  % \leqq
     "36  =  { 50,   },  % \leqslant
%    "37  =  {   ,   },  % \lessgtr
     "38  =  {   , 50},  % \backprime
     "39  =  {200,200},  % \dabar@ : the dash bar in \dash(left,right)arrow
%    "3A  =  {   ,   },  % \risingdotseq
%    "3B  =  {   ,   },  % \fallingdotseq
     "3C  =  { 50,100},  % \succcurlyeq
%    "3D  =  {   ,   },  % \geqq
     "3E  =  {   , 50},  % \geqslant
%    "3F  =  {   ,   },  % \gtrless
     "40  =  {   , 50},  % \sqsubset
     "41  =  { 50,   },  % \sqsupset
     "42  =  {   ,150},  % \vartriangleright, \rhd
     "43  =  {150,   },  % \vartriangleleft, \lhd
     "44  =  {   ,100},  % \trianglerighteq, \unrhd
     "45  =  {100,   },  % \trianglelefteq, \unlhd
     "46  =  {100,100},  % \bigstar
%    "47  =  {   ,   },  % \between
     "48  =  { 50, 50},  % \blacktriangledown
     "49  =  {   ,100},  % \blacktriangleright
     "4A  =  {100,   },  % \blacktriangleleft
     "4B  =  {   ,150},  % \dashrightarrow (the arrow)
     "4C  =  {150,   },  % \dashleftarrow
     "4D  =  { 50, 50},  % \vartriangle
     "4E  =  { 50, 50},  % \blacktriangle
     "4F  =  { 50, 50},  % \triangledown
     "50  =  { 50, 50},  % \eqcirc
%    "51  =  {   ,   },  % \lesseqgtr
%    "52  =  {   ,   },  % \gtreqless
%    "53  =  {   ,   },  % \lesseqqgtr
%    "54  =  {   ,   },  % \gtreqqless
%    "55  =  {   ,   },  % \yen
     "56  =  {   ,150},  % \Rrightarrow
     "57  =  {150,   },  % \Lleftarrow
     "58  =  {100,200},  % \checkmark
%    "59  =  {   ,   },  % \veebar
%    "5A  =  {   ,   },  % \barwedge
%    "5B  =  {   ,   },  % \doublebarwedge
     "5C  =  { 50, 50},  % \angle
     "5D  =  { 50, 50},  % \measuredangle
     "5E  =  { 50, 50},  % \sphericalangle
     "5F  =  {   , 50},  % \varpropto
     "60  =  {100,100},  % \smallsmile
     "61  =  {100,100},  % \smallfrown
     "62  =  { 50,   },  % \Subset
     "63  =  {   , 50},  % \Supset
%    "64  =  {   ,   },  % \Cup
%    "65  =  {   ,   },  % \Cap
     "66  =  {100,100},  % \curlywedge
     "67  =  {100,100},  % \curlyvee
     "68  =  { 50,100},  % \leftthreetimes
     "69  =  {100, 50},  % \rightthreetimes
%    "6A  =  {   ,   },  % \subseteqq
%    "6B  =  {   ,   },  % \supseteqq
     "6C  =  { 50, 50},  % \bumpeq
     "6D  =  { 50, 50},  % \Bumpeq
     "6E  =  {100,   },  % \lll
     "6F  =  {   ,100},  % \ggg
     "70  =  { 50,100},  % \ulcorner
     "71  =  {100, 50},  % \urcorner
%    "72  =  {   ,   },  % \circledR
%    "73  =  {   ,   },  % \circledS
%    "74  =  {   ,   },  % \pitchfork
     "75  =  {100,150},  % \dotplus
     "76  =  { 50,100},  % \backsim
%    "77  =  {   ,   },  % \backsimeq
     "78  =  { 50,100},  % \llcorner
     "79  =  {100, 50},  % \lrcorner
%    "7A  =  {   ,   },  % \maltese
%    "7B  =  {   ,   },  % \complement
     "7C  =  {100,100},  % \intercal
     "7D  =  { 50, 50},  % \circledcirc
     "7E  =  { 50, 50},  % \circledast
     "7F  =  { 50, 50},  % \circleddash
   }

%</msa>
%    \end{macrocode}
% Symbol font `b'.
%
%\begin{verbatim}
%\DeclareSymbolFont{AMSb}{U}{msb}{m}{n}
%\DeclareSymbolFontAlphabet{\mathbb}{AMSb}
%\end{verbatim}
%    \begin{macrocode}
%<*msb>
\SetProtrusion
   [ name     = AMSb ]
   { encoding = U,
     family   = msb  }
   {
       A  =  { 50, 50},  % \mathbb
       C  =  { 50, 50},
       G  =  {   , 50},
       L  =  {   , 50},
       P  =  {   , 50},
       R  =  {   , 50},
       T  =  {   , 50},
       V  =  { 50, 50},
       X  =  { 50, 50},
       Y  =  { 50, 50},
     "00  =  { 50, 50},  % \lvertneqq
     "01  =  { 50, 50},  % \gvertneqq
     "02  =  { 50, 50},  % \nleq
     "03  =  { 50, 50},  % \ngeq
     "04  =  {100, 50},  % \nless
     "05  =  { 50,150},  % \ngtr
     "06  =  {100, 50},  % \nprec
     "07  =  { 50,150},  % \nsucc
     "08  =  { 50, 50},  % \lneqq
     "09  =  { 50, 50},  % \gneqq
     "0A  =  {100,100},  % \nleqslant
     "0B  =  {100,100},  % \ngeqslant
     "0C  =  {100, 50},  % \lneq
     "0D  =  { 50,100},  % \gneq
     "0E  =  {100, 50},  % \npreceq
     "0F  =  { 50,100},  % \nsucceq
     "10  =  { 50,   },  % \precnsim
     "11  =  { 50, 50},  % \succnsim
     "12  =  { 50, 50},  % \lnsim
     "13  =  { 50, 50},  % \gnsim
     "14  =  { 50, 50},  % \nleqq
     "15  =  { 50, 50},  % \ngeqq
     "16  =  { 50, 50},  % \precneqq
     "17  =  { 50, 50},  % \succneqq
     "18  =  { 50, 50},  % \precnapprox
     "19  =  { 50, 50},  % \succnapprox
     "1A  =  { 50, 50},  % \lnapprox
     "1B  =  { 50, 50},  % \gnapprox
     "1C  =  {150,200},  % \nsim
     "1D  =  { 50, 50},  % \ncong
     "1E  =  {100,150},  % \diagup
     "1F  =  {100,150},  % \diagdown
     "20  =  {100, 50},  % \varsubsetneq
     "21  =  { 50,100},  % \varsupsetneq
     "22  =  {100, 50},  % \nsubseteqq
     "23  =  { 50,100},  % \nsupseteqq
     "24  =  {100, 50},  % \subsetneqq
     "25  =  { 50,100},  % \supsetneqq
     "26  =  {100, 50},  % \varsubsetneqq
     "27  =  { 50,100},  % \varsupsetneqq
     "28  =  {100, 50},  % \subsetneq
     "29  =  { 50,100},  % \supsetneq
     "2A  =  {100, 50},  % \nsubseteq
     "2B  =  { 50,100},  % \nsupseteq
     "2C  =  { 50,100},  % \nparallel
     "2D  =  {100,150},  % \nmid
     "2E  =  {150,150},  % \nshortmid
     "2F  =  {100,100},  % \nshortparallel
     "30  =  {   ,150},  % \nvdash
     "31  =  {   ,150},  % \nVdash
     "32  =  {   ,100},  % \nvDash
     "33  =  {   ,100},  % \nVDash
     "34  =  {   ,100},  % \ntrianglerighteq
     "35  =  {100,   },  % \ntrianglelefteq
     "36  =  {100,   },  % \ntriangleleft
     "37  =  {   ,100},  % \ntriangleright
     "38  =  {100,200},  % \nleftarrow
     "39  =  {100,200},  % \nrightarrow
     "3A  =  {100,100},  % \nLeftarrow
     "3B  =  { 50,100},  % \nRightarrow
     "3C  =  {100,100},  % \nLeftrightarrow
     "3D  =  {100,200},  % \nleftrightarrow
     "3E  =  { 50, 50},  % \divideontimes
     "3F  =  { 50, 50},  % \varnothing
%    "40  =  {   ,   },  % \nexists
     "60  =  {200,   },  % \Finv
     "61  =  {   , 50},  % \Game
%    "66  =  {   ,   },  % \mho
%    "67  =  {   ,   },  % \eth
     "68  =  {100,100},  % \eqsim
     "69  =  { 50,   },  % \beth
     "6A  =  { 50,   },  % \gimel
     "6B  =  {150,   },  % \daleth
     "6C  =  {200,   },  % \lessdot
     "6D  =  {   ,200},  % \gtrdot
     "6E  =  {100,200},  % \ltimes
     "6F  =  {150,100},  % \rtimes
     "70  =  { 50,100},  % \shortmid
     "71  =  { 50, 50},  % \shortparallel
     "72  =  {200,300},  % \smallsetminus
     "73  =  {100,200},  % \thicksim
     "74  =  { 50,100},  % \thickapprox
     "75  =  { 50, 50},  % \approxeq
     "76  =  { 50,100},  % \succapprox
     "77  =  { 50, 50},  % \precapprox
     "78  =  {100,100},  % \curvearrowleft
     "79  =  { 50,150},  % \curvearrowright
     "7A  =  { 50,200},  % \digamma
     "7B  =  {100, 50},  % \varkappa
%    "7C  =  {   ,   },  % \Bbbk
%    "7D  =  {   ,   },  % \hslash
%    "7E  =  {   ,   },  % \hbar
     "7F  =  {200,   },  % \backepsilon
   }

%</msb>
%    \end{macrocode}
% Euler Fraktur font (\pkg{eufrak}).
%
%\begin{verbatim}
%\DeclareMathAlphabet{\mathfrak}{U}{euf}{m}{n}
%\SetMathAlphabet{\mathfrak}{bold}{U}{euf}{b}{n}
%\end{verbatim}
%    \begin{macrocode}
%<*euf>
\SetProtrusion
   [ name     = mathfrak ]
   { encoding = U,
     family   = euf  }
   {
       A  =  {   , 50},
       B  =  {   , 50},
       C  =  { 50, 50},
       D  =  {   , 80},
       E  =  { 50,   },
       G  =  {   , 50},
       L  =  {   , 80},
       O  =  {   , 50},
       T  =  {   , 80},
       X  =  { 80, 50},
       Z  =  { 80, 50},
       b  =  {   , 50},
       c  =  {   , 50},
       k  =  {   , 50},
       p  =  {   , 50},
       q  =  { 50,   },
       v  =  {   , 50},
       w  =  {   , 50},
       x  =  {   , 50},
       1  =  {100,100},
       2  =  { 80, 80},
       3  =  { 80, 50},
       4  =  { 80, 50},
       7  =  { 50, 50},
   }

%</euf>
%    \end{macrocode}
% Euler script font (\pkg{eucal}).
%
%\begin{verbatim}
%\DeclareMathAlphabet\EuScript{U}{eus}{m}{n}
%\SetMathAlphabet\EuScript{bold}{U}{eus}{b}{n}
%\end{verbatim}
%    \begin{macrocode}
%<*eus>
\SetProtrusion
   [ name     = euscript ]
   { encoding = U,
     family   = eus  }
   {
       A  =  {100,100},
       B  =  { 50,100},
       C  =  { 50, 50},
       D  =  { 50,100},
       E  =  { 50,100},
       F  =  { 50,   },
       G  =  { 50,   },
       H  =  {   ,100},
       K  =  {   , 50},
       L  =  {   ,150},
       M  =  {   , 50},
       N  =  {   , 50},
       O  =  { 50, 50},
       P  =  { 50, 50},
       T  =  {   ,100},
       U  =  {   , 50},
       V  =  { 50, 50},
       W  =  { 50, 50},
       X  =  { 50, 50},
       Z  =  { 50,100},
   }

%</eus>
%</cfg-u>
%<*beta>
%    \end{macrocode}
%
% \subsection{Interword Spacing}\label{sub:conf-spacing}
%
% Default unit is space.
%
%    \begin{macrocode}
%%% ----------------------------------------------------------------------
%%% INTERWORD SPACING SETTINGS

\SetExtraSpacing
   [ name = default ]
   { encoding = {OT1,T1,LY1,OT4,T5} }
   {
%    \end{macrocode}
% These settings are only a first approximation. The following reasoning is from
% a mail from Ulrich Dirr. ^^A private mail, 14/06/2005
% I do not claim to have coped with the task.
%
%\begin{quote}
% `The idea is -- analog to the tables for expansion and protrusion -- to
% have tables for optical reduction/expansion of spaces in dependence of the
% actual character so that the distance between words is optically equal.
%
% When reducing distances the (weighting) order is:
%
% \begin{itemize}
%   \item after commas
%    \begin{macrocode}
     {,} = { ,-500,500},
%    \end{macrocode}
% \item in front of capitals which have optical more room on their left
%       side, \eg, `A', `J', `T', `V', `W', and `Y'
%       [this is not yet possible -- RS]
% \item in front of capitals which have circle/oval shapes on their left
%       side, \eg, `C', `G', `O', and `Q'
%       [ditto -- RS]
% \item after `r' (because of the bigger optical room on the righthand side)
%    \begin{macrocode}
      r  = { ,-300,300},
%    \end{macrocode}
% \item before or after lowercase characters with ascenders
%    \begin{macrocode}
      b  = { ,-200,200},
      d  = { ,-200,200},
      f  = { ,-200,200},
      h  = { ,-200,200},
      k  = { ,-200,200},
      l  = { ,-200,200},
      t  = { ,-200,200},
%    \end{macrocode}
% \item before of after lowercase characters with x-heigth plus descender
%       with additional optical space, \eg, `v', or `w'
%    \begin{macrocode}
      c  = { ,-100,100},
      p  = { ,-100,100},
      v  = { ,-100,100},
      w  = { ,-100,100},
      z  = { ,-100,100},
      x  = { ,-100,100},
      y  = { ,-100,100}, % ?
%    \end{macrocode}
% \item before of after lowercase characters with x-heigth plus descender
%       without additional optical space
%    \begin{macrocode}
      i  = { , 50, -50},
      m  = { , 50, -50},
      n  = { , 50, -50},
      u  = { , 50, -50},
%    \end{macrocode}
% \item after colon and semicolon
%    \begin{macrocode}
      :  = { ,200,-200},
      ;  = { ,200,-200},
%    \end{macrocode}
% \item after punctuation which ends a sentence, \eg, period, exclamation
%       mark, question mark
%    \begin{macrocode}
      .  = { ,250,-250},
      !  = { ,250,-250},
      ?  = { ,250,-250},
%    \end{macrocode}
%\end{itemize}
% The order has to be reversed when enlarging is needed.'
%\end{quote}
%    \begin{macrocode}
   }

%    \end{macrocode}
% Questions are:
%\begin{itemize}
%  \item Is the result really better?
%  \item Is it overdone? (Try with a |factor| \textless\ 1000.)
%  \item Should the first parameter also be used? (Probably.)
%\end{itemize}
%
% The following settings simulate \cmd{\nonfrenchspacing} (since space factors
% will be ignored when spacing adjustment is in effect). They may be used for
% English contexts.
%
% From the \TeX book:
%\begin{quote}
% `If the space factor \textit{f} is different from 1000, the interword glue is
% computed as follows:  Take the normal space glue for the current font, and
% add the extra space if \textit{f}\,\textgeq\,2000.
% [...]
% Then the stretch component is multiplied by \textit{f}\,/\,1000, while the
% shrink component is multiplied by 1000\,/\,\textit{f}.'
%\end{quote}
% The `extra space' (\cmd{\fontdimen}7) for Computer Modern Roman is a third
% of \cmd{\fontdimen}\,2, \ie, 333.
%    \begin{macrocode}
\SetExtraSpacing
   [ name     = nonfrench-cmr,
     load     = default,
     context  = nonfrench ]
   { encoding = {OT1,T1,LY1,OT4,T5},
     family   = cmr }
   {
%    \end{macrocode}
% \file{latex.ltx} has:
%\begin{verbatim}
%\def\nonfrenchspacing{
%  \sfcode`\. 3000
%\end{verbatim}
%    \begin{macrocode}
     . = {333,2000,-667},
%    \end{macrocode}
%\begin{verbatim}
%  \sfcode`\? 3000
%\end{verbatim}
%    \begin{macrocode}
     ? = {333,2000,-667},
%    \end{macrocode}
%\begin{verbatim}
%  \sfcode`\! 3000
%\end{verbatim}
%    \begin{macrocode}
     ! = {333,2000,-667},
%    \end{macrocode}
%\begin{verbatim}
%  \sfcode`\: 2000
%\end{verbatim}
%    \begin{macrocode}
     : = {333,1000,-500},
%    \end{macrocode}
%\begin{verbatim}
%  \sfcode`\; 1500
%\end{verbatim}
%    \begin{macrocode}
     ; = {   , 500,-333},
%    \end{macrocode}
%\begin{verbatim}
%  \sfcode`\, 1250
%\end{verbatim}
%    \begin{macrocode}
    {,}= {   , 250,-200},
%    \end{macrocode}
%\begin{verbatim}
%}
%\end{verbatim}
%    \begin{macrocode}
   }

%    \end{macrocode}
% \pkg{fontinst}, however, which is also used to create the \pkg{PSNFSS} font
% metrics, sets it to 240 by default. Therefore, the fallback settings use this
% value for the first component.
%    \begin{macrocode}
\SetExtraSpacing
   [ name     = nonfrench-default,
     load     = default,
     context  = nonfrench ]
   { encoding = {OT1,T1,LY1,OT4,T5} }
   {
     . = {240,2000,-667},
     ? = {240,2000,-667},
     ! = {240,2000,-667},
     : = {240,1000,-500},
     ; = {   , 500,-333},
    {,}= {   , 250,-200},
   }

%    \end{macrocode}
%
% \subsection{Additional Kerning}
%
% Default unit is 1em.
%
%    \begin{macrocode}
%%% ----------------------------------------------------------------------
%%% ADDITIONAL KERNING

%    \end{macrocode}
% A dummy list to be loaded when no context is active.
%    \begin{macrocode}
\SetExtraKerning
   [ name = empty ]
   { encoding = {OT1,T1,LY1,OT4,T5,TS1} }
   { }

\SetExtraKerning
   [ name     = french-default,
     context  = french,
     unit     = space   ]
   { encoding = {OT1,T1,LY1} }
   {
     :  = {1000,}, % = \fontdimen2
     ;  = {500, }, % ~ \thinspace
     !  = {500, },
     ?  = {500, },
   }

%    \end{macrocode}
% This has the disadvantage that the word following a left guillemot will not be
% hyphenated. This might be fixed in \pdftex.
%    \begin{macrocode}
\SetExtraKerning
   [ name     = french-guillemets,
     context  = french-guillemets,
     load     = french-default,
     unit     = space   ]
   { encoding = {OT1,T1,LY1} }
   {
    \guillemotleft  = { ,800}, % = 0.8\fontdimen2
    \guillemotright = {800, },
   }

\SetExtraKerning
   [ name     = turkish,
     context  = turkish,
     unit     = space  ]
   { encoding = {OT1,T1,LY1} }
   {
     :  = {500, }, % ~ \thinspace
     !  = {500, },
    {=} = {500, },
   }

%    \end{macrocode}
% The settings with the `|letterspacing|' context will be loaded whenever the
% command \cs{textls} resp. \cs{lsstyle} are used.
%    \begin{macrocode}
%%% The settings for the commands \lsstyle and \textls.
\SetExtraKerning
   [ name     = letterspacing-default,
     context  = letterspacing,
     unit     = 1em,
     preset   = {1000,1000} ]
   { encoding = {OT1,OT4}   }
   {
%    \end{macrocode}
% The full stop and quotation marks should be spaced out less.
% Numbers are not spaced out, according to \pkg{soul}.
%    \begin{macrocode}
     .  = {0, },
     0  = {0,0},
     1  = {0,0},
     2  = {0,0},
     3  = {0,0},
     4  = {0,0},
     5  = {0,0},
     6  = {0,0},
     7  = {0,0},
     8  = {0,0},
     9  = {0,0},
     \textquoteleft    = {0,0},   \textquoteright    = {0,0},
     \textquotedblleft = {0,0},   \textquotedblright = {0,0},
   }

\SetExtraKerning
   [ name     = letterspacing-T1,
     load     = letterspacing-default,
     context  = letterspacing,
     unit     = 1em,
     preset   = {1000,1000} ]
   { encoding = {T1,LY1,T5} }
   {
     \quotesinglbase   = {0,0},   \quotedblbase      = {0,0},
     \guilsinglleft    = {0,0},   \guilsinglright    = {0,0},
     \guillemotleft    = {0,0},   \guillemotright    = {0,0},
   }

%</beta>
%</config>
%    \end{macrocode}
%\GeneralChanges!
%
%\section{Auxiliary File for Micro Fine Tuning}
%
% This file can be used to test protrusion and expansion settings.
%    \begin{macrocode}
%<*test>
\documentclass{article}

%% Here you can set the font you want to test, using
%% the commands \fontfamily, \fontseries, and \fontshape.
%% Make sure to end all lines with a comment character!
\newcommand*{\TestFont}{%
  \fontfamily{ppl}%
%%  \fontseries{b}%
%%  \fontshape{it}% sc, sl
}

\usepackage{ifthen}
\usepackage[T1]{fontenc}
%%\usepackage[latin1]{inputenc}
\usepackage[verbose,expansion=alltext,stretch=50]{microtype}

\pagestyle{empty}
\setlength{\parindent}{0pt}
\newcommand*{\crulefill}{\cleaders\hbox{$\mkern-2mu\smash-\mkern-2mu$}\hfill}
\newcommand*{\testprotrusion}[2][]{%
  \ifthenelse{\equal{#1}{r}}{}{#2}%
  lorem ipsum dolor sit amet,
    \ifthenelse{\equal{#1}{r}}{\crulefill}{\leftarrowfill} #2
    \ifthenelse{\equal{#1}{l}}{\crulefill}{\rightarrowfill}
  you know the rest%
  \ifthenelse{\equal{#1}{l}}{}{#2}%
  \linebreak
  {\fontencoding{\encodingdefault}%
  \fontseries{\seriesdefault}%
  \fontshape{\shapedefault}%
  \selectfont
  Here is the beginning of a line, \dotfill and here is its end}\linebreak
}
\newcommand*{\showTestFont}{\expandafter\stripprefix\meaning\TestFont}
\def\stripprefix#1>{}
\newcount\charcount
\begin{document}

\microtypesetup{expansion=false}

{\centering The font in this document is called by:\\
 \texttt{\showTestFont}\par}\bigskip

\TestFont\selectfont
 This line intentionally left empty\linebreak
%% A -- Z
\charcount=65
\loop
  \testprotrusion{\char\charcount}
  \advance\charcount 1
  \ifnum\charcount < 91 \repeat
%% a -- z
\charcount=97
\loop
  \testprotrusion{\char\charcount}
  \advance\charcount 1
  \ifnum\charcount < 123 \repeat
%% 0 -- 9
\charcount=48
\loop
  \testprotrusion{\char\charcount}
  \advance\charcount 1
  \ifnum\charcount < 58 \repeat
%%
 \testprotrusion[r]{,}
 \testprotrusion[r]{.}
 \testprotrusion[r]{;}
 \testprotrusion[r]{:}
 \testprotrusion[r]{?}
 \testprotrusion[r]{!}
 \testprotrusion[l]{\textexclamdown}
 \testprotrusion[l]{\textquestiondown}
 \testprotrusion[r]{)}
 \testprotrusion[l]{(}
 \testprotrusion{/}
 \testprotrusion{\char`\\}
 \testprotrusion{-}
 \testprotrusion{\textendash}
 \testprotrusion{\textemdash}
 \testprotrusion{\textquoteleft}
 \testprotrusion{\textquoteright}
 \testprotrusion{\textquotedblleft}
 \testprotrusion{\textquotedblright}
 \testprotrusion{\quotesinglbase}
 \testprotrusion{\quotedblbase}
 \testprotrusion{\guilsinglleft}
 \testprotrusion{\guilsinglright}
 \testprotrusion{\guillemotleft}
 \testprotrusion{\guillemotright}

\bigskip
The following displays the current font stretched by 5\%,
normal, and shrunk by 5\%:

\microtypesetup{expansion=true}

\bigskip
\newlength{\MTln}
\newcommand*{\teststring}
  {ABCDEFGHIJKLMNOPQRSTUVWXYZabcdefghijklmnopqrstuvwxyz0123456789}
\settowidth{\MTln}{\teststring}

\parbox{1.05\MTln}{\teststring\linebreak\\
                   \teststring}\par\bigskip
\parbox{0.95\MTln}{\teststring}
\end{document}
%</test>
%    \end{macrocode}
%
% ^^A -------------------------------------------------------------------------
% Needless to say that things may always be improved. For suggestions, mail to
% \mailtoRS.
%
% ^^A -------------------------------------------------------------------------
%
% \CharacterTable
%  {Upper-case    \A\B\C\D\E\F\G\H\I\J\K\L\M\N\O\P\Q\R\S\T\U\V\W\X\Y\Z
%   Lower-case    \a\b\c\d\e\f\g\h\i\j\k\l\m\n\o\p\q\r\s\t\u\v\w\x\y\z
%   Digits        \0\1\2\3\4\5\6\7\8\9
%   Exclamation   \!     Double quote  \"     Hash (number) \#
%   Dollar        \$     Percent       \%     Ampersand     \&
%   Acute accent  \'     Left paren    \(     Right paren   \)
%   Asterisk      \*     Plus          \+     Comma         \,
%   Minus         \-     Point         \.     Solidus       \/
%   Colon         \:     Semicolon     \;     Less than     \<
%   Equals        \=     Greater than  \>     Question mark \?
%   Commercial at \@     Left bracket  \[     Backslash     \\
%   Right bracket \]     Circumflex    \^     Underscore    \_
%   Grave accent  \`     Left brace    \{     Vertical bar  \|
%   Right brace   \}     Tilde         \~}
%
% \CheckSum{7591}
%
% \Finale
%
\endinput
%
